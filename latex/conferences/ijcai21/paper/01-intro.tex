% !TeX root = ./main.tex

Building on old ideas to explain domain-specific propagations performed by constraint solvers  \cite{sqalli1996inference,freuder2001explanation}, we recently introduced a 
method that takes as input a satisfiable constraint program and explains the solution-finding process in a human-understandable way  \cite{ecai/BogaertsGCG20}. 
Explanations in that work are sequences of simple inference steps, involving as few constraints and facts as possible. 
The explanation-generation algorithms presented in that work rely heavily on calls for  \emph{Minimal Unsatisfiable Subsets} (MUS) \cite{marques2010minimal} of a derived program, exploiting a one-to-one correspondence between so-called \emph{non-redundant explanations} and MUSs.
The explanation steps in the seminal work are heuristically optimized with respect to a given cost function that should approximate human-understandability, e.g., taking the number of constraints and facts into account, as well as a valuation of their complexity (or priority). 
The algorithm developed in that work has two main weaknesses: first, it provides no guarantees on the quality of the produced explanations due to internally relying on the computation of $\subseteq$-minimal unsatisfiable subsets, which are often suboptimal with respect to the given cost function. 
Secondly, it suffers from performance problems: the lack of optimality is partly overcome by calling a MUS algorithm on increasingly larger subsets of constraints for each candidate implied fact.
However, using multiple MUS calls per literal in each iterations quickly causes efficiency problems, causing the explanation generation process to take several hours.


Motivated by these observations, we develop algorithms that aid explaining CSPs and improve the state-of-the-art in the following ways: 
\begin{itemize}
 \item We develop algorithms that compute (cost-)\textbf{Optimal} Unsatisfiable Subsets (from now on called OUSs) based on the well-known hitting-set duality that is also used for computing cardinality-minimal MUSs \cite{ignatiev2015smallest,DBLP:conf/kr/SaikkoWJ16}.
\item We observe that many of the individual calls for MUSs (or OUSs) can actually be replaced by a single call that searches for an optimal unsatisfiable subset \textbf{among subsets satisfying certain structural constraints}. In other words, we introduce the \emph{Optimal \textbf{Constrained} Unsatisfiable Subsets (OCUS)} problem and we show how $O(n^2)$ calls to MUS/OUS can be replaced by $O(n)$ calls to an OCUS oracle, where $n$ denotes the number of facts to explain. 
\item Finally, we develop techniques for \textbf{optimizing} the OCUS algorithms further, exploiting domain-specific information coming from the fact that we are in the  \emph{explanation-generation context}. One such optimization is the development of methods for \textbf{information re-use} between consecutive OCUS calls.
\end{itemize}

In this paper, we apply our OCUS algorithms to generate \emph{step-wise} explanations of satisfaction problems. However, MUSs have been used in a variety of contexts, and in particular lie at the foundations of several explanation techniques \cite{junker2001quickxplain,ignatiev2019abduction,schotten}. We conjecture that OCUS can also prove useful in those settings, to take more fine-grained control over which MUSs, and eventually, which explanations are produced.

The rest of this paper is structured as follows.
We discuss background on the hitting-set duality in \cref{sec:background}. \cref{sec:motviation} motivates our work, while \cref{sec:ocus} introduces the OCUS problem and a generic \hitsetbased algorithm for computing OCUSs. In \cref{sec:ocusEx} we show how to optimize this computation in the context of explanations and in  
\cref{sec:experiments}  we experimentally validate the approach.
We discuss related work in  \cref{sec:related} and conclude in \cref{sec:conclusion}.
