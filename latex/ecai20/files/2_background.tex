\paragraph{Logic grid puzzles}
Our proposed method is applicable to constraint satisfaction problem \tias{with a unique solution?}. We will use \textit{logic grid puzzles} as example domain, as it requires no expert knowledge to understand.

A logic grid puzzle (also known as a Zebra puzzle or Einstein puzzle) consists of a set of natural language sentences (from hereon referred to as ``clues'') and (optionally) a set of \emph{entities} occurring in those sentences. 
For instance, our running example contains a clue ``The person who chose arrabiata sauce is either Angie or Elisa'' and (among others) entities ``arrabiata sauce'', ``Angie'' and ``Elisa''. 

The set of entities is sometimes left implicit if it can be derived from the clues, but often it is given in the form of a grid. Furthermore, in such a puzzle the set of entities is partitioned into equally sized groups (corresponding to \emph{types}); in our example, at least ``person'' and ``sauce'' are such types. 

The goal of the puzzle is to find relations between each two types such that
\begin{compactitem}
	\item each clue is respected, 
	\item each entity of one type is matched with exactly one entity of the second type, e.g., each person chose exactly one sauce (this type of constraint will be referred to as \emph{bijectivity}), and 
	\item the relations are logically linked, e.g., if Angie chose arrabiata sauce and arrabiata sauce was paired with farfalle, then Angie must also have eaten farfalle (from now on called \emph{transitivity}). 
\end{compactitem}
In section Sec~\ref{sec:holistic} we explain how we obtain a vocabulary and first-order theory in a mostly automated way from the clues. The result is a vocabulary with types corresponding to the groups of entities in the clues, and the names and types of the binary relations to find (e.g \textit{chose(person, sauce)}, \textit{paired(sauce, pasta)}, \textit{eaten(person, pasta)});
%We furthermore assume that the interpretation of the types is fixed (and hence all interpretations agree on this). 
as well as first-order sentences corresponding to the clues, and the bijectivity and transitivity constraints. Let $T_P$ be a theory containing all of these sentences for a given puzzle $P$.

\paragraph{Typed first-order logic}
Our reasoning method is based on \emph{typed first-order logic}. %, with links to \emph{typed second-order logic}.
Part of the input is a logical vocabulary consisting of a set of type symbols, typed constant symbols, and relation symbols with associated type signature (i.e., each relation symbol is typed $T_1\times \dots \times T_n$ with $T_i$ types).\footnote{We here omit function symbols since they are not used in this paper.} For example, type \textit{person} with constant symbol \textit{Angie} of type \textit{person} and a relation \textit{choseSauce(.,.)} with signature \textit{person $\times$ sauce}.


A \emph{first-order theory} is a set of sentences (well-formed variable-free first-order formulas in which each quantified variable has an associated type). 
Given a logical vocabulary $V$, a \emph{\textbf{partial} interpretation} $I$ assigns to each type symbol $T$ (e.g. \textit{person}) a finite set $I(T)$ and to each 
relation symbol $P$ with type signature $T_1\times \dots \times T_n$ a function 
\[I(P): I(T_1)\times \dots \times I(T_n)\to \{\ltrue,\lunkn,\lfalse\},\] 
where $\ltrue$ stands for true, $\lunkn$ for unknown, and $\lfalse$ for false. In case nothing \tias{nothing of what?} is mapped to $\lunkn$, we call this an \emph{interpretation}. \tias{'full' interpretation to discriminate with partial? I need that distinction in a later section...}
A partial interpretation $I_1$ is \emph{more precise} than partial interpretation $I_2$ (notation $I_1\geqp I_2$) if $I_1$ and $I_2$ agree on everything not unknown in $I_2$.

% 
% In a second-order theory also quantifiers over relations have an associated type signature. 
% \bart{Clear or do I make it more formal}
