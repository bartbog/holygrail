% \todo{The macro \allconstraints should be changed to jsut $T$, not $\allconstraints$ since $\allconstraints$ refers to a Puzzle P. This should be done carefully so that no ambiguity is created. Also in the section on logic grid puzzles $\allconstraints$ shoudl be used. }

The overarching goal of this paper is to generate a sequence of small reasoning steps, each with an interpretable explanation. 
We first introduce the concept of an explanation of a single reasoning step, after which we introduce a cost function as a proxy for the interpretability of a reasoning step, and the cost of a sequence of such steps. 

\subsection{Explanation of reasoning steps}
We assume that a theory $\allconstraints$ and an initial partial interpretation $I_0$ are given and fixed. 

\begin{definition}
We define the \textbf{maximal consequence} of a theory $\allconstraints$ and partial interpretation $I$ (denoted $max(I,T)$) as the precision-maximal partial interpretation $J$ such that  $I \wedge \allconstraints \entails J$. 
\end{definition}

Phrased differently, $max(I,T)$ is the intersection of all models of $T$ that are more precise than $I$; this is also known as the set of \emph{cautious consequences} of $T$ and $I$ and corresponds to ensuring \emph{global consistency} in constraint solving.
%\emilio{why is it the intersection of all models of T more precise than I?} -> no, intersection of 'all models more precise'
% if J \entails T and K \entails T; J \geq I and K \geq I, their intersection is not necessarly 
% } 
Algorithms for computing cautious consequences without explicitly enumerating all models exist, such as for instance the ones implemented in clasp \cite{DBLP:conf/lpnmr/GebserKS09} or \idp \cite{IDP} (in the latter system the task of computing all cautious consequences is called \emph{optimal-propagate} since it performs the strongest propagation possible).

Weaker levels of propagation consistency can be used as well, leading to a potentially smaller maximal consequence interpretation $max_{other-consistency}(I,T)$. 
The rest of this paper assumes we want to construct a sequence that starts at $I_0$ and ends at $max(I_0,\allconstraints)$ for some consistency algorithm, i.e., that can explain all computable consequences of $\allconstraints$ and $I_0$. 
\begin{definition}
A \textbf{sequence of incremental partial interpretations} of a theory $\allconstraints$ with initial partial interpretation $I_0$ is a sequence $\langle I_0, I_1, \ldots, I_n  = max(I_0,\allconstraints)\rangle$ where $\forall i>0, I_{i-1} \leqp I_{i}$ (i.e., the sequence is precision-increasing).
\end{definition} 

The goal of our work is not just to obtain a sequence of incremental partial interpretations, but also for each incremental step $\langle I_{i-1}, I_i \rangle$ we want an explanation $(E_i,S_i)$ that justifies the newly derived information $N_i = I_i \setminus I_{i-1}$. When visualized, such as in Figure~\ref{fig:zebrascreen}, it will show the user precisely which information and constraints were used to derive a new piece of information.

\begin{definition}
 Let $I_{i-1}$ and $I_i$ be partial interpretations such that $I_{i-1}\land \allconstraints \models I_i$.
 We say that $(E_i,S_i,N_i)$ \emph{explains} the derivation of $I_{i}$ from $I_{i-1}$ if the following hold:
\begin{itemize}
    \item $N_i= I_i \setminus I_{i-1}$ (i.e., $N_i$ consists of all newly defined facts), 
	\item $E_i\subseteq I_{i - 1}$ (i.e., the explaining facts are a subset of what was previously derived),
	\item $S_i \subseteq \allconstraints$ (i.e., a subset of the constraints used), and 
	\item $S_i \cup E_i \entails N_i$ (i.e., all newly derived information indeed follows from this explanation).
\end{itemize}
\end{definition}

The problem of simply checking whether $(E_i,S_i,N_i)$ explains the derivation of $I_{i}$ from $I_{i-1}$ is in co-NP since this problem can be performed by verifying that $S_i \land \lnot N_i$ has no models more precise than $E_i$. It is hence an instance of the negation of a first-order model expansion problem \cite{DBLP:conf/lpar/KolokolovaLMT10}.

Part of our goal of finding easy to interpret explanations is to avoid redundancy. 
That is, we want a non-redundant explanation $(E_i,S_i,N_i)$ where none of the facts in $E_i$ or constraints in $S_i$ can be removed while still explaining the derivation of $I_i$ from $I_{i-1}$; in other word: the explanation must be \textit{subset-minimal}. 
\begin{definition}
 We call $(E_i,S_i,N_i)$ a \emph{non-redundant explanation of  the derivation of $I_i$ from $I_{i-1}$} if it explains this derivation and whenever $E'\subseteq E_i; S'\subseteq S_i$ while $(E',S',N_i)$ also explains this derivation, it must be that $E_i=E', S_i=S'$. 
\end{definition}

\begin{definition} \label{def:nonred}
A \textbf{non-redundant explanation sequence} is a sequence 
\[\langle(I_0,(\emptyset,\emptyset,\emptyset)), (I_1,(E_1,S_1,N_i)), \dots ,(I_n,(E_n,S_n,N_n))\rangle\]
%\[(I_0,(\emptyset,\emptyset,\emptyset)), (I_1,(E_1,S_1,N_i)), \dots ,(I_n,(E_n,S_n,N_n))\]
such that $(I_i)_{i\leq n}$ is sequence of incremental partial interpretations and each $(E_i,S_i,N_i)$ explains the derivation of $I_i$ from $I_{i-1}$.
\end{definition} 

\subsection{Interpretability of reasoning steps}
While subset-minimality ensures that an explanation is non-redundant, it does not quantify how \textit{interpretable} a explanation is. 
This quantification is problem-specific and is often designed in a subjective manner. 
For this, we will assume the existence of a cost function $f(E_i,S_i,N_i)$ that quantifies the interpretability of a single explanation. 

In line with the goal of ``simple enough for a person to understand'' and Occam's Razor, we reason that smaller explanations are easier to interpret than explanations that use a larger number of facts or constraints.
In Section~\ref{sec:cost} we provide a size-based cost function for use in our logic grid puzzle tool, though others can be used as well.
Developing generic methods to find a \emph{good} cost function that correctly captures perceived complexity is an orthogonal research topic we are not concerned with in the current paper. 

\subsection{Interpretability of a sequence of reasoning steps}
In its most general form, we would like to optimize the understandability of the entire sequence of explanations. 
While quantifying the interpretability of a single step can be hard, doing so for a sequence of explanations is even harder. For example, is it related to the most difficult step or the average difficulty, and how important is the ordering within the sequence?
As a starting point, we here consider the total cost to be an aggregation of the costs of the individual explanations, e.g. the average or maximum cost.

\begin{definition}
Given a theory $\allconstraints$ and initial partial interpretation $I_0$, the \textbf{explanation-production problem} consist of finding a non-redundant explanation sequence
\[\langle(I_0,(\emptyset,\emptyset,\emptyset)), (I_1,(E_1,S_1,N_1)), \dots ,(I_n,(E_n,S_n,N_n))\rangle\]
such that a predefined aggregate over the sequence $\left(f(E_i,S_i,N_i)\right)_{i\leq n}$ is minimised.
\end{definition} 
% \tias{check that $I_n$ etc definition matches the one defined higher up}

Example aggregation operators are $max()$ and $average()$, which each have their peculiarities: the $max()$ aggregation operator will minimize the cost of the most complicated reasoning step, but does not capture whether there is one such step used, or multiple. Likewise, the $average()$ aggregation operator will favour many simple steps, including splitting up trivial steps into many small components if the constraint abstraction allows this.
% 
Even for a fixed aggregation operator, the problem of holistically optimizing a sequence of explanation steps is much harder than optimizing the cost of a single reasoning step, since there are exponentially more sequences. 
