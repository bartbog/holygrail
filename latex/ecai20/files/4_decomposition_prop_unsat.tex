In this section, we tackle the goal of searching for a sequence of $I' \wedge C' \rightarrow A$ explanations that is as simple to understand as possible and such that the sequence explains the steps to a complete solution (a two-valued interpretation). \textbf{TODO: adapt to formalism/language used in previous section} 
% 
\tias{Tias: Perhaps, I should explain the scoring first...} \bart{partly done in previous section... Enough to understand what is going on I think}
% 
Our general approach is to iteratively find the reasoning step with the best scoring explanation, and to apply that. This is a greedy approach to the problem of finding the best scoring \textit{sequence} of explanations, formalized in Algorithm \ref{alg:main}.

\tias{TODO: check that 'full interpretation' is valid wording}\bart{I would use two-valued or total} 
\bart{I removed ``all constraints'' I propose we give that set of constraints a name; here I used \allconstraints but other proposals are fine;}
\begin{algorithm}
%  \begin{algorithmic}
$C \gets \allconstraints$\;
Seq $\gets$ empty sequence\;
$I \gets \{\}$\;
\While{$I$ not two-valued}{
  $(I', C', A) \gets $smallest-explanation$(I, C)$\;
  append $(I', C', A)$ to Seq\;
  $I \gets I \cup A$
}
% \end{algorithmic}
\caption{High-level greedy sequence-generating algorithm.}
\label{alg:main}
\end{algorithm}

The main challenge is finding the best scoring explanation, among all reasoning steps that can be applied for a given partial interpretation $I$.

We use propagation to get the set of new facts derived from a given partial interpretation $I$ and a set of constraints $C$. Each new fact can be obtained through a reasoning step, and we seek to find a small explanation for each of these reasoning steps.

\bart{Terminology: 
* I would replace optimal-propagate by propagate (abstractly it does not matter WHICH propagation mechanism we use. 
* MUS by core? or unsat-core (to be a bit more verbose?)}

The code below shows our proposed algorithm. The key part of the algorithm is on line \ref{line:mus} where we find an explanation of a single new fact $a$ by searching for a \textit{minimal unsatisfiable core} that includes $\neg a$. We do this over the literals of $I$ as well as over newly introduced \textit{reified} literals for every constraint in $C$. The result will be a minimal unsatisfiable core $\neg a \wedge I' \wedge C'$ where $I' \subseteq I$ and $C' \subseteq C$. The latter can in fact be extracted by observing which of the reified literals are set to \textit{true} in the MUS. From $\neg a \wedge I' \wedge C'$ we know that $I' \wedge C' \rightarrow a$. As $I' \wedge C'$ may imply more than just $a$, we also compute the other literals on line \textbf{...}. Note that those other derived facts in $A \setminus \{a\}$ may have simpler explanations hence our for loop searches for each fact regardless if it is also derived in another explanation.


\bart{TODO PUT ALGO BAKC}
% % \comment{
\begin{algorithm}
% 
\SetKwInOut{Input}{input}\SetKwInOut{Output}{output}
\SetKwComment{command}{/*}{*/}

 \Input{A partial interpretation $I$ and a set of constraints $C$}
% \end{algorithm}
% 
% \Require{$I = $ partial interpretation, $C =$ a set of constraints}
% \Fn(\tcc*[h]{algorithm as a recursive function}){BLA{some args}}{

% \Function{candidate-explanations}{I, C}
  Candidates $\gets \{\}$\;
  $J \gets$ optimal-propagate$(I \wedge C)$\;
  \For{$a \in J \setminus I$}{ 
  \tcp{Find minimal expl. for each new fact}
    $X \gets MUS(\neg a \wedge I \wedge reify(C))$ \label{line:mus}\;
    $I' \gets I \cap X$\;
    $C' \gets C \cap X$\;
    $A \gets$ optimal-propagate$(I' \wedge C')$\;
    \tcp{Compute implied facts:}
    add $(I', C', A)$ to Candidates
  }
  \Return{Candidates}
% \EndFunction
\caption{candidate-explanations$(I,C)$}

\label{alg:cand}
\end{algorithm}

% }

\bart{Instead of ``minimal'' vs minimum, I would go for ``subset-minimal'', ``size-minimal'', ``cost-minimal'', ...  I.e., make the criteria explicit.} 
An important observation is that Minimal Unsatisfiable Core extraction methods extract \textit{an} unsatisfiable core that is \textit{minimal}, but not \textit{minimum}. That is, the unsat core can not be reduced further, but there could be another minimal unsat core whose size is smaller. Finding a MUS can be computationally expensive, hence finding the minimum one even more so as it will need to search the space of all possible MUS's.


%If we want to find the best ordering (TODO), we need the absolute minimal MUS, which is typically only a few constraints.
%This relates to the objective function...

\bart{I like the way the algorithm is built-up here.} 
Instead, we use the observation that typically a small (1 to a few) number of constraints is sufficient, and also preferred in terms of easy to understand explanations. We will hence not call \textit{candidate-explanations} with the full set of constraints, but will iteratively grow the number of constraints used. 
%\bart{ In practice this means that we are doing some form of prioritized explanatoin. 
%First priority: keep the number of constraints small. Second priority, size (which also takes structure into acccount}

\paragraph{Smallest explanations and cost functions}
\bart{Do we have examples where $A$ is to be taken into account for the size! } 
In the following, we assume that we have a cost function $f(I', C', A)$ that returns a score for every possible explanation. The goal will be to find the lowest scoring explanation. We make one further assumption, namely that we have an optimistic estimate $g(C')$ computed on only the constraint part of the explanation, and that $\forall I', A, g(C') \leq f(I', C', A)$. This is for example the case if $f$ is an additive function, such as $f(I', C', A) = f_1(I') + f_2(C') + f_3(A)$ where $g(C') = f_2(C')$ assuming $f_1$ and $f_3$ are always positive.

We can then search for the smallest explanation among the candidates found, by searching among increasingly worse scoring $C'$ as shown in the code below, which makes use of the candidate-explanations function defined above.

% \bart{TODO PUT ALGO BACK}
\begin{algorithm}
\SetKwInOut{Input}{input}\SetKwInOut{Output}{output}
\SetKwComment{command}{/*}{*/}
\SetKw{Break}{break}


 \Input{A partial interpretation $I$ and a set of constraints $C$}
  Candidates $\gets \{\}$\;
  $J \gets$ optimal-propagate$(I \wedge C)$\;
  \For{$C' \subseteq C$ ordered by $g(C')$}{ 
    \If{$g(C') < min(f($Candidates$))$}{
        \Break\;}
     cand $\gets$ candidate-explanations$(I, C')$\; 
     add to Candidates all cand$_i$ with corresp. value $f($cand$_i)$\;
     }  
          \Return{smallest scoring candidate of Candidates}
 % \EndFunction
\caption{smallest-explanations$(I,C)$}
\label{alg:cand}
\end{algorithm}

%   \State{Candidates $\gets$ empty stack}
%   \For{}
%     \If{ }
%       \State{Break}
%     \EndIf
%     \State{}
%     \State{}
%   \EndFor
%   \State{Return }
% \EndFunction
% \end{algorithmic}

\textbf{That is it for now... more optimisations are possible, basically in the way we search over the C', like, cache Candidates so that its minimum value is maintained (and do an 'update' to it before starting to potentially clean it); do not search for $C'' \supset C'$ where C' has some explanations; perhaps we can find out that at some point we no longer need to optimalprop some $C'$? (e.g. a 'fully used' clue?)}

\paragraph{Rest is braindump part}
We assume every constraint $c \in C$ has a weight $w(c)$, and we assume every literal has a weight when used as previously derived fact $w_l(f)$, and when it is a newly derived fact $w_r(f)$. The total cost of an explanation is then:
$$ cost(I', C', A) = w_1*(\sum_{i \in I'} w(i)) + w_2*(\sum_{c \in C'} w(c)) + w_3*(\sum_{a \in A} w(a))$$

The $w_1, w_2, w_3$ can be used to trade-off weights of constraints and facts globally, e.g. to mimic a lexicographic ordering.

In this work, we assume that the weights are manually set based on domain knowledge. For the logic grid puzzles, we prefer small numbers of constraints primarily, so $w_2$ has a high value ($w_2=100$). Among constraints, clues get a weight of $1$ + $0.01$ times their index in the list, such that clues higher in the list get preference. Implicit constraints are even more preferred with a weight of $0.5$ for transitivity constraints and $-1$ for bijection constraints, as users typically complete the latter immediately.
Previously used facts are uniformily weighted $w_1=1$ and $w(i)=1, \forall i$. The number of newly derived facts matter less, hence $w_3=0.1$ though with a slight preference for positive literals which get a weight of $w(a)=0.1$ when $a$ is positive, else $w(a)=1$.

Based on this cost function, it is clear that small $C'$ sets should be preferred, but the MUS search does not ensure to find the smallest possible MUS, just that the MUS is minimal. Hence, we will do a 'level-wise' search for increasing constraint set cost $w_2*(\sum_{c \in C'} w(c))$.

\textbf{Should we do something with checking whether a (set of) constraints is implied, so that we never have to check it again?}


%
%
%	
%
%\begin{verbatim}
%C /\ B -> A
%
%given(I facts, C cons, D impl cons):
%
%store = {}
%for c in Cons:
%  I' <- optimal_propagate(I, c /\ D) # is this needed, with D added?
%  let A = I \setminus I'
%  if |A| != 0:
%    # refine 'c'... because?
%    forall a in A:
%      X = MUS(not a /\ I /\ c /\ D)
%      if c \notin X:
%        # some special case...? should be avoided
%      else:
%        let B' is I \intersect X
%        let D' is D \intersect X
%        # find other implications of D' /\ c
%        A' = optimal_propagate(B', c /\ D')
%        add to store (c /\ D', B', A') as candidate explanation
%
%if |store| == 0:
%  # TODO: more constraints at same time.
%  # perhaps binary search makes sense?
%  cut C in C1 and C2
%  I'1 <- optimal_propagate(I, C1 /\ D)
%  if not empty, recursively refine
%  if recursive children were empty:
%    no... can not recursively split, may be combination across split
%\end{verbatim}

