%% Indien je niet vertrouwd ben met Latex:
%%  Maak een .pdf als volgt:
%%  - Vul alles in 
%%  - Doe: pdflatex verslag.tex (dit produceert de .pdf)

\documentclass[12pt]{report}
%\usepackage{a4wide}

\setlength{\parindent}{0cm}

\begin{document}
\pagestyle{myheadings}
\markright{Tussentijds verslag November -  Student: Jens Claes}
{\bf Titel eindwerk:} {\em Natural Language processing for Knowledge Representation}

\vspace{0.5cm}
{\bf Promotor(s):} Marc Denecker


\vspace{0.5cm}
{\bf Begeleider(s):} Laurent Janssens

\vspace{1cm}
{\bf Korte situering en Doelstelling: } In dit eindwerk wordt onderzocht of we een formele taal kunnen ontwerpen die toegankelijk is voor domein experts (die geen kennis hebben van formele talen) maar toch nog rijk genoeg is voor praktische problemen en bruikbaar is binnen het Knowledge Base Systems paradigma.

\vspace{1cm}
{\bf Belangrijkste bestudeerde literatuur:}
\begin{itemize}
\item V. Ambriola and V. Gervasi, “Processing natural language requirements,” Proc. 12th IEEE Int. Conf. Autom. Softw. Eng., pp. 36–45, 1997.
\item N. E. Fuchs, K. Kaljurand, and T. Kuhn, “Attempto controlled english for knowledge representation,” Lect. Notes Comput. Sci. (including Subser. Lect. Notes Artif. Intell. Lect. Notes Bioinformatics), vol. 5224 LNCS, pp. 104–124, 2008.
\item R. Schwitter, A. Ljungberg, and D. Hood, “ECOLE–A Look-ahead Editor for a Controlled Language,” Eamt-Claw03, pp. 141–150, 2003.
\item S. Flake, W. Müller, and J. Ruf, “Structured English for Model Checking Specification,” Methoden und Beschreibungssprachen zur Model. und Verif. von Schaltungen und Syst., no. February, pp. 99–108, 2002.
\end{itemize}

\vspace{1cm}
{\bf Geleverd werk (inclusief tijdsrapportering):}
Ik heb onderzoek gedaan naar reeds bestaande natuurlijke formele talen en de afbakening van de probleemstelling. Ik heb hieraan ongeveer 35 uur besteed.
Ik heb ook reeds onderzoek gedaan naar een aantal kleine problemen en een basis vertaling hiervan opgesteld. Dit zowel in een natuurlijke formele taal als in IDP. Hieraan heb ik ongeveer 20u besteed.

\vspace{1cm}
{\bf Belangrijkste resultaten:}
Ik heb een mooi overzicht van wat er al eerder is gebeurd in de literatuur rond natuurlijke formele talen.
Ik heb ook al een idee hoe de natuurlijke formele taal er zou kunnen uitzien en een aantal concepten die moeilijker uit te drukken zijn.

\vspace{1cm}
{\bf Belangrijkste moeilijkheden:}
De bestaande literatuur vermeld vaak de resultaten van hun talen maar de structuur van de talen zelf is minder goed gedocumenteerd. Daardoor zal ik grotendeels van nul moeten beginnen bij het opstellen van mijn grammatica.

\vspace{1cm}
{\bf Gepland werk:} 
We zullen beginnen met het opstellen van een grammatica om kleinere problemen op te lossen (zoals een Zebra Puzzel). Deze kleine problemen zijn namelijk makkelijk te begrijpen maar zijn reeds moeilijk om uit te drukken in een meer natuurlijke taal. Daarna zullen we deze problemen uitbreiden om meer concepten van FO(.) te introduceren en deze dan ook in de nieuwe formele taal toevoegen. Uiteindelijk kunnen we een groter probleem bekijken en zien hoeveel procent we kunnen uitdrukken in de nieuwe taal.


\vspace{1cm}
{\bf Als ik verder werk zoals ik tot nu toe deed, dan denk ik 15/20 te
    verdienen op het einde.}

{\bf Ik plan mijn eindwerk af te geven in juni}


\end{document}
