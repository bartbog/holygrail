\section{A semantical framework}
Blackburn and Bos constructed a framework~\cite{Blackburn2005, Blackburn2006} to capture the meaning of natural language by translating it into a logic. It consists of four parts: the lexicon, the grammar, the semantics of the lexicon and the semantics of the grammar. The framework is based on the $\lambda$-calculus and Frege's compositionality principle. Every word has a $\lambda$-expression as its meaning. The meaning of a group of words is a combination of the meaning of the words that are part of the group. In this framework, $\lambda$-application is used to combine the meaning of words.

The grammar defines which sentences are grammatically correct (and thus allowed) and which are not. E.g. a correct sentence consists of a noun phrase and a verb phrase that agree in number. This process results in a parse tree of every grammatically correct sentence, in which the words are the leaves. The framework then combines the meaning of the leaves (the words) upwards in the tree to the meaning of the root node (the sentence).

\begin{dcg}{A simple grammar}{dcg:frameworkIllustration}
s([]) -->
  np([num:Num]),
  vp([num:Num]).
np([num:sg]) -->
  pn([]).
np([num:Num]) -->
  det([]),
  n([num:Num]).
vp([num:Num, sem:Sem]) -->
  iv([num:Num, sem:Sem]).
vp([num:Num, sem:Sem]) -->
  tv([num:Num, sem:]),
  np([num:_]).
\end{dcg}

The lexicon consists of the enumeration of the different words that can be used along with some linguistic features. E.g. ``loves'' is a transitive verb in the present tense, ``love'' is a transitive verb and an infinitive. A lot of words also have a feature \textit{Symbol} which is used in the translation as the name of a constant or predicate.

The goal of the framework is to lexicalize as much of the meaning as possible. The semantics of a grammar rule should be limited to $\lambda$-applications, if possible. Table~\ref{tbl:grammar-sem} gives the semantics for grammar~\ref{dcg:frameworkIllustration}.

\begin{table}[h]
  \begin{tabular}{@{}ll}
    \hline
    \textbf{Grammatical rule} & \textbf{Semantics} \\
    \hline
    S (line 1-3) & $\sem{s} = \app{\sem{vp}}{\sem{np}}$ \\
    NP1 (line 4-5) & $\sem{np} = \sem{pn}$ \\
    NP2 (line 6-8) & $\sem{np} = \app{\sem{det}}{\sem{n}}$ \\
    VP1 (line 9-10) & $\sem{vp} = \sem{iv}$ \\
    VP2 (line 11-13) & $\sem{vp} = \app{\sem{tv}}{\sem{np}}$\\
    \hline
  \end{tabular}
  \centering
  \caption{The semantics of grammar~\ref{dcg:frameworkIllustration}}
  \label{tbl:grammar-sem}
\end{table}

The meaning of the words themselves is the only thing still missing. Blackburn and Bos assume that the meaning of a word is only dependent on its linguistic features, most importantly its lexical category. Constructing these $\lambda$-expressions can be hard. Therefore, it can help to analyze the signature of these expressions. Blackburn and Bos suggest a $\lambda$-calculus with two types for this: $e$ represents entities and $t$ truth-values. The signature of a noun is $\tau(n) = e \rightarrow t$, given an entity, the noun says whether or not the entity can be described with the noun. E.g. $\sem{n_{man}} = \lambdaf{x}{man(x)}$

A noun phrase represents one or more entities. E.g. ``a man'', ``every woman''. A sentence is true if enough entities of the noun phrase satisfy a certain property of the verb phrase. The signature of a noun phrase is thus $\tau(np) = (e \rightarrow t) \rightarrow t$. Given a property $P$ with signature $e \rightarrow t$, the semantics of the noun phrase tell us if the noun phrase satisfies the property $P$ of the verb phrase. The simplest noun phrase is a proper noun like John. Its semantics is $\sem{pn_{John}} = \lambdaf{P}{\app{P}{John}}$. I.e. when we say something about ``John'', the property should hold for $John$. The meaning of the noun phrase ``every man'' is $\sem{np_{every\ man}} = \lambdaf{P}{\forall x \cdot man(x) \Rightarrow P(x)}$. The property $P$ should hold for all men for the sentence to be true. From this, we conclude the meaning of ``every'' as $\sem{det_{universal}} = \lambdaf{R}{\lambdaf{P}{\forall x \cdot R(x) \Rightarrow P(x)}}$. We call $R$ the restriction by the noun. We can generalize this to other determiners like ``two'': $\sem{det_{binary}} = \lambdaf{R}{\lambdaf{P}{\exists_{>2} x \cdot R(x) \land P(x)}}$.

A verb phrase should say whether the sentence is true or not, given a noun phrase as its subject. An intransitive verb like ``sleeps'' is the simplest verb phrase. Its meaning is $\sem{iv_{sleeps}} = \lambdaf{S}{\app{S}{\lambdaf{x}{sleeps(x)}}}$. The sentence is true if the subject $S$ holds the property $\lambdaf{x}{sleeps(x)}$. In other words, the sentence is true, if the semantics of the subject $S$ (a function from a property with signature $e \to t$ to a truth value) evaluates to $true$ for the property $\lambdaf{x}{sleeps(x)}$.

The meaning of the sentence ``John sleeps'' is then $$\sem{s} = \app{\sem{vp}}{\sem{np}} =$$$$ \appH{\lambdaf{S}{\app{S}{\lambdaf{x}{sleeps(x)}}}}{\lambdaf{P}{\app{P}{John}}}$$$$= sleeps(John)$$

% \item Een overgankelijk werkwoord (\texttt{tv}) is gelijkaardig maar krijgt twee naamwoordgroepen als argument. Het eerste argument is het lijdend voorwerp $L$, het tweede het onderwerp $O$. Er zijn meerdere vertaling mogelijk. Neem bijvoorbeeld de zin ``Every man loves a woman''. Is er één vrouw waarvan alle mannen houden of kan dit een verschillende vrouw zijn voor elke man? Dit wordt een \textit{Quantifier Scope Ambiguïteit} genoemd omdat de ambiguïteit ligt in de volgorde van de kwantoren. Zo worden de twee lezingen respectievelijk $\exists w \cdot woman(w) \land (\forall m \cdot man(m) \Rightarrow loves(m, w))$ en $\forall m \cdot man(m) \Rightarrow (\exists w \cdot woman(w) \land loves(m, w))$. Blackburn en Bos lossen deze ambiguïteiten op door de kwantoren in de vertaling in dezelfde volgorde te laten als in de natuurlijke taal\footnote{Ze leggen daarnaast ook uit hoe men de andere lezingen kan verkrijgen. We verwijzen nar hoofdstuk 3 uit hun eerste boek \cite{Blackburn2005} voor de details.}. Voor het overgankelijke werkwoord wordt dit $$\sem{tv} = \lambdaf{L}{\lambdaf{O}{\app{O}{\lambdaf{x_o}{\app{L}{\lambdaf{x_l}{\drs{}{\textit{Symbool}(x_o, x_l)}}}}}}}$$
%Een entiteit $x_o$ omschreven door het onderwerp voldoet aan de verbale constituent als voor die $x_o$ het lijdend voorwerp voldoet aan de eigenschap $\lambdaf{x_l}{\drs{}{\textit{Symbool}(x_o, x_l)}}$.
%We passen dit toe op ``Every man loves a woman''. Een man $x_o$ voldoet aan de verbale constituent (``loves a woman'') als er voor die man $x_o$ een vrouw $x_l$ is zodat $\drs{}{\textit{loves}(x_o, x_l)}$ waar is. Het onderwerp (als geheel) voldoet aan de verbale constituent als elke man $x_o$ voldoet aan bovenstaande eigenschap. Als er dus voor elke man $x_o$ een vrouw $x_l$ bestaat waarvan hij houdt. Voor elke man kan er dit een andere vrouw $x_l$ zijn.

Finally, the meaning of a transitive verb like ``loves'' is $$\sem{tv_{loves}} = \lambdaf{O}{\lambdaf{S}{\app{S}{\lambdaf{x_S}{\app{O}{\lambdaf{x_O}{loves(x_S, x_O)}}}}}}$$ An entity $x_S$ from the subject $S$ satisfies the property if there are enough corresponding $x_O$'s from the object $O$ such that $loves(x_S, x_O)$.

For example for the sentence ``Every man loves a woman'', a man $x_S$ satisfies the verb phrase (``loves a woman'') if for this man $x_S$, there is a woman $x_O$ such that $loves(x_S, x_O)$. The subject (as a whole) satisfies the verb phrase if \textit{every} man $x_S$ satisfies the verb phrase, i.e. if there is a woman $x_O$ for every man $x_S$ such that the man loves the woman. This woman can be different for every man.

This translation therefore decides how quantifier scope ambiguities are resolved, namely the translation follows the order of the natural language. This is how all quantifier scope ambiguities are resolved within this paper. Blackburn and Bos explain in their books how all readings can be obtained.

