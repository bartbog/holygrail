%%%% ijcai21.tex

% These are the instructions for authors for IJCAI-21.

\documentclass{article}
\pdfpagewidth=8.5in
\pdfpageheight=11in
% The file ijcai21.sty is NOT the same than previous years'
\usepackage{ijcai21}

% Use the postscript times font!
\usepackage{times}
\usepackage{soul}
\usepackage{url}
\usepackage[hidelinks]{hyperref}
\usepackage[utf8]{inputenc}
\usepackage[small]{caption}
\usepackage{graphicx}
\usepackage{amsmath,amssymb}
\usepackage{amsthm}
\usepackage{booktabs}
\usepackage{adjustbox}
\usepackage{multirow,array}
\urlstyle{same}
\newcommand\citet[1]{\citeauthor{#1}~[\citeyear{#1}]}


\usepackage[dvipsnames]{xcolor}
\usepackage[ruled,linesnumbered]{algorithm2e}
\usepackage{xspace} 
\usepackage{paralist}
\setdefaultenum{\bfseries(i)}{}{}{}
\usepackage[11pt]{moresize}
\usepackage[normalem]{ulem}

\newcommand\m[1]{\ensuremath{\mathcal{#1}}}
\newcommand\ignore[1]{}
\usepackage[capitalise]{cleveref}

\newcommand\voc{\ensuremath{\Sigma}\xspace}
\newtheorem{thm}{Theorem}
\newtheorem{definition}[thm]{Definition}
\newtheorem{prop}{Property} 
\newtheorem{property}[prop]{Property}
\newtheorem{lem}{Lemma}
\newtheorem{lemma}[lem]{Lemma}
\newtheorem{proposition}[thm]{Proposition} 
\newtheorem{theorem}[thm]{Theorem}
\newtheorem{ex}{Example}
\newtheorem{example}[ex]{Example}

\newcommand\muses[1]{\ensuremath{\mathit{MUSs}(#1)}\xspace}
\newcommand\mcses[1]{\ensuremath{\mathit{MCSs}(#1)}\xspace}

\newcommand\setstohit{\ensuremath{\m{H}}\xspace}
\newcommand\F{\ensuremath{\m{F} }\xspace}
% \newcommand\ohs{\ensuremath{\m{OHS} }\xspace}
\newcommand\setstohitall{\ensuremath{\m{H}_\mathit{all} }\xspace}
\newcommand\Iend{\ensuremath{I_\mathit{end} }\xspace}
\newcommand\formula{\ensuremath{\m{F} }\xspace}
\newcommand\formulac{\ensuremath{\m{C} }\xspace}
\newcommand\formulag{\ensuremath{\m{G} }\xspace}
\newcommand\mm[1]{\ensuremath{#1}\xspace}
\newcommand\nat{\mm{\mathbb{N}}}
\newcommand\ltrue{\mm{\textbf{t}}}
\newcommand\lfalse{\mm{\textbf{f}}}
\newcommand\uservars{\ensuremath{\m{U} }\xspace}

\newcommand\satsets{\mm{\mathbf{SSs}}}
\newcommand\fall{\mm{\formula_{\mathit{all}}}}

\newcommand\call[1]{\mm{\textsc{#1}}}
\newcommand\geths{\mm{\call{GetHittingSet}}}
\newcommand\ohs{\mm{\call{OptHittingSet}}}
\newcommand\ghs{\mm{\call{GreedyHittingSet}}}
\newcommand\ihs{\mm{\call{IncrementalHittingSet}}}
\newcommand\cohs{\mm{\call{CondOptHittingSet}}}
\newcommand\chs{\mm{\call{CondHittingSet}}}
\newcommand\sat{\mm{\call{sat}}}
\newcommand\grow{\mm{\call{Grow}}}
\newcommand\omus{\mm{\call{OUS}}}
\newcommand\comus{\mm{\call{OCUS}}}
\newcommand\omusinc{\mm{\call{OUS-Inc}}}
\newcommand\store{\mm{\call{Store}}}
\newcommand\optprop{\mm{\call{MaxConsequence}}}
\newcommand\initsat{\mm{\call{InitSat}}}
\newcommand\hitsetbased{hitting set--based\xspace} %en-dash!

\SetKwInOut{Input}{Input}
\SetKwInOut{OptInput}{Optional}
\SetKwInOut{Output}{Output}
\SetKwInOut{State}{State}
\SetKwInOut{Ext.}{Ext}
\SetKwComment{command}{/*}{*/}

\newcommand\negset[1]{\mm{\overline{#1}}}
\newcommand\maxsat{MaxSAT\xspace}
\newcommand\comment[1]{\marginpar{\tiny #1}}
\renewcommand\comment[1]{#1}
\newcommand{\emilio}[1]{{\comment{\color{red}\textsc{EG:}#1}}}
\newcommand{\bart}[1]{{\comment{\color{blue}\textsc{BB:}#1}}}
\newcommand{\tias}[1]{{\comment{\color{orange}\textsc{TG:}#1}}}
%PDF Info Is REQUIRED.
\pdfinfo{
/Title (Efficiently Explaining CSPs with Unsatisfiable Subset Optimization)
/Author (Emilio Gamba, Bart Bogaerts and Tias Guns)
/TemplateVersion (IJCAI.2021.0)
}

\title{Efficiently Explaining CSPs with Unsatisfiable Subset Optimization}

% Single author syntax
\author{
	Emilio Gamba$^1$ \and Bart Bogaerts$^1$ \And Tias Guns$^{1,2}$
	\affiliations
	$^1$Vrije Universiteit Brussel, Belgium\\
	$^2$KU Leuven, Belgium\\
	\emails
	emilio.gamba@vub.be, bart.bogaerts@vub.be,
	tias.guns@kuleuven.be
}

\usepackage{etoolbox}

\newcommand\setcitation[2]{%
  \csdef{mycommoncitation#1}{#2}}
\newcommand\getcitation[1]{%
  \csuse{mycommoncitation#1}}

\setcitation{IDP}{WarrenBook/DeCatBBD14}
\setcitation{idp}{WarrenBook/DeCatBBD14}
\setcitation{fodot}{tocl/DeneckerT08}
\setcitation{foid}{tocl/DeneckerT08}
\setcitation{FOID}{tocl/DeneckerT08}
\setcitation{cplogic}{journal/tplp/VennekensDB10}
\setcitation{CPlogic}{journal/tplp/VennekensDB10}
\setcitation{CPLogic}{journal/tplp/VennekensDB10}
\setcitation{CP}{fai/Rossi06}
\setcitation{cp}{fai/Rossi06}
\setcitation{EZCSP}{lpnmr/Balduccini11}
\setcitation{KR}{Baral:2003}
\setcitation{ASPComp2}{lpnmr/DeneckerVBGT09}
\setcitation{ASPComp3}{journals/tplp/CalimeriIR14}
\setcitation{ASPComp4}{conf/lpnmr/AlvianoCCDDIKKOPPRRSSSWX13}
\setcitation{ASPComp5}{journals/ai/CalimeriGMR16}
\setcitation{ASPComp6}{jair/GebserMR17}
\setcitation{ASPComp7}{tplp/GebserMR20}
\setcitation{CPSupport}{ictai/DeCat13}
\setcitation{CPsupport}{ictai/DeCat13}
\setcitation{functionDetection}{iclp/DeCatB13}
\setcitation{FunctionDetection}{iclp/DeCatB13}
\setcitation{fodot2asp}{DeneckerLTV12} %TODO replace by journal publication if one is publishedr
\setcitation{Tarskian}{DeneckerLTV12} %Same as the one above 
\setcitation{TarskianSemanticsASP}{DeneckerLTV12} %Same as the one above 
\setcitation{Inca}{iclp/DrescherW12}
\setcitation{csp2asp}{ijcai/DrescherW11}
\setcitation{DPLLT}{cav/GanzingerHNOT04}
\setcitation{AspInPractice}{synthesis/2012Gebser}
\setcitation{ASPInPractice}{synthesis/2012Gebser}
\setcitation{clasp}{ai/GebserKS12}
\setcitation{oclingo}{kr/GebserGKOSS12}
\setcitation{clingo}{iclp/GebserKKOSW16}
\setcitation{gringo}{lpnmr/GebserST07}
\setcitation{cmodels}{aaai/GiunchigliaLM04}
\setcitation{inputster}{tplp/Jansen13}
\setcitation{DLV}{tocl/LeonePFEGPS06}
\setcitation{LearningPaper}{TPLP/BruynoogheBBDDJLRDV} %TODO replace by published version
\setcitation{clog}{iclp/BogaertsVDV14} %TODO replace by journal publication if one is published
\setcitation{foc}{iclp/BogaertsVDV14}
\setcitation{FOC}{iclp/BogaertsVDV14}
\setcitation{inferenceClog}{ecai/BogaertsVDV14}
\setcitation{examplesClog}{nmr/BogaertsVDV14b} %TODO replace by better publication if possible
\setcitation{AFT}{DeneckerMT00}
\setcitation{KBS}{iclp/DeneckerV08}
\setcitation{KBS-invitedtalk}{jelia/Denecker16}
\setcitation{KBPE}{inap/DePooterWD11}
\setcitation{lazyGrounding}{jair/CatDBS15} 
\setcitation{LazyGrounding}{jair/CatDBS15} 
\setcitation{lazygrounding}{jair/CatDBS15} 
\setcitation{lazygroundingASP}{ijcai/BogaertsW18} 
\setcitation{justifications}{lpnmr/DeneckerBS15} 
\setcitation{justificationsAlpha}{ijcai/BogaertsW18} 
\setcitation{ASP}{marek99stable}
\setcitation{satid}{sat/MarienWDB08}
\setcitation{lazyclausegeneration}{constraints/OhrimenkoSC09}
\setcitation{FP}{ACMCS/Hudak89}
\setcitation{GroundingWithBounds}{jair/WittocxMD10}
\setcitation{GroundWithBounds}{jair/WittocxMD10}
\setcitation{SAT}{faia/SilvaLM09}
\setcitation{HandbookOfSAT}{faia/2009-185}
\setcitation{LTC}{iclp/Bogaerts14}
\setcitation{SPSAT}{ictai/DevriendtBMDD12}
\setcitation{BreakID}{sat/DevriendtBBD16}
\setcitation{breakid}{sat/DevriendtBBD16}
\setcitation{LCG}{stuckeyLCG}
\setcitation{MiniZinc}{conf/cp/NethercoteSBBDT07}
\setcitation{minizinc}{conf/cp/NethercoteSBBDT07}
\setcitation{amadini}{cpaior/AmadiniGM13}
\setcitation{bootstrapping}{ngc/BogaertsJDJBD16}
\setcitation{Bootstrapping}{ngc/BogaertsJDJBD16}
\setcitation{GroundedFixpoints}{ai/BogaertsVD15}
\setcitation{PartialGroundedFixpoints}{ijcai/BogaertsVD15}
\setcitation{LogicBlox}{datalog/GreenAK12}
\setcitation{proB}{journals/sttt/LeuschelB08}
\setcitation{NaturalInductions}{KR/DeneckerV14} %TODO replace by journal publication if one is published
\setcitation{LP}{jacm/EmdenK76}
\setcitation{SMT}{faia/BarrettSST09}
\setcitation{AF}{ai/Dung95}
\setcitation{ADF}{kr/BrewkaW10}
\setcitation{af}{ai/Dung95}
\setcitation{adf}{kr/BrewkaW10}
\setcitation{ADFRevisited}{ijcai/BrewkaSEWW13}
\setcitation{adfrevisited}{ijcai/BrewkaSEWW13}
\setcitation{DefaultLogic}{ai/Reiter80}
\setcitation{DL}{ai/Reiter80}
\setcitation{AEL}{mo85}
\setcitation{minisat}{sat/EenS03}
\setcitation{completion}{adbt/Clark78}
\setcitation{ClarkCompletion}{adbt/Clark78}
\setcitation{wasp}{lpnmr/AlvianoDFLR13}
\setcitation{minisatid}{ictai/DeCat13}
\setcitation{lcg}{stuckeyLCG}
\setcitation{CEGAR}{jacm/ClarkeGJLV03}
\setcitation{cegar}{jacm/ClarkeGJLV03}
\setcitation{CuttingPlane}{or/DantzigFJ54}
\setcitation{kodkod}{tacas/TorlakJ07}
\setcitation{cdcl}{Marques-SilvaS99}
\setcitation{CDCL}{Marques-SilvaS99}
\setcitation{1UIP}{iccad/ZhangMMM01}
\setcitation{relevance}{ijcai/JansenBDJD16}
\setcitation{relevance-implementation}{aspocp/JansenBDJD16}
\setcitation{WFS}{GelderRS91}
\setcitation{wfs}{GelderRS91}
\setcitation{UnfoundedSet}{GelderRS91}
\setcitation{UFS}{GelderRS91}
\setcitation{stablesemantics}{iclp/GelfondL88}
\setcitation{StableSemantics}{iclp/GelfondL88}
\setcitation{shatter}{Shatter}
\setcitation{sbass}{drtiwa11a}
\setcitation{lparsemanual}{url:lparse_manual}
\setcitation{AIC}{ppdp/FlescaGZ04}
\setcitation{templates}{tplp/DassevilleHJD15}
\setcitation{templates2}{iclp/DassevilleHBJD16}
\setcitation{sat-to-sat}{aaai/JanhunenTT16}
\setcitation{sat-to-sat-qbf}{bnp/BogaertsJT16}
\setcitation{sat-to-sat-QBF}{bnp/BogaertsJT16}
\setcitation{sat-to-sat-SO}{kr/BogaertsJT16}
\setcitation{XSB}{SwiW12}
\setcitation{KCmap}{jair/DarwicheM02}
\setcitation{TLA}{DBLP:books/aw/Lamport2002}
\setcitation{EventB}{BookAbrial2010}
\setcitation{MX}{MitchellT05}
\setcitation{MIP}{Sierksma96}
\setcitation{perefectmodel}{minker88/Przymusinski88}
\setcitation{SafeInductions}{ijcai/BogaertsVD17}
\setcitation{AIC}{ppdp/FlescaGZ04}
\setcitation{aic}{ppdp/FlescaGZ04}
\setcitation{alpha}{lpnmr/Weinzierl17}
\setcitation{omiga}{jelia/Dao-TranEFWW12}
\setcitation{gasp}{fuin/PaluDPR09}
\setcitation{asperix}{lpnmr/LefevreN09a}
\setcitation{CTL}{lop/ClarkeE81}
\setcitation{AFT-AIC}{ai/BogaertsC18}
\setcitation{UltimateApproximator}{DeneckerMT04}
\setcitation{KripkeKleene}{Fitting85}
\setcitation{AFT-HO}{corr/CharalambidisRS18} %TODO replace
\setcitation{HereThere}{Heyting30}
\setcitation{dAEL}{ijcai/HertumCBD16}
\setcitation{SDD}{ijcai/Darwiche11}
\setcitation{HEX}{ijcai/EiterIST05}
\setcitation{wADF}{aaai/BrewkaSWW18}
\setcitation{wADFfix}{corr/BrewkaSWW18}
\setcitation{TransitionSystems}{jacm/NieuwenhuisOT06}
\setcitation{galliwasp}{lopstr/MarpleG12}
\setcitation{GalliWasp}{lopstr/MarpleG12}
\setcitation{clingcon}{tplp/BanbaraKOS17}
\setcitation{lp2sat}{birthday/JanhunenN11}
\setcitation{lp2mip}{LIU12}
\setcitation{lp2diff}{lpnmr/JanhunenNS09}
\setcitation{lp2acyc}{ecai/GebserJR14}
\setcitation{PB}{faia/RousselM09}
\setcitation{pb}{faia/RousselM09}
\setcitation{CuttingPlanes}{dam/CookCT87}
\setcitation{RoundingSAT}{ijcai/ElffersN18}
\setcitation{PRS}{aaai/DixonG02}
\setcitation{sat4j}{jsat/BerreP10}
\setcitation{SAT4J}{jsat/BerreP10}
\setcitation{mingo}{LIU12}
\setcitation{pbmodels}{lpnmr/LiuT05}
\setcitation{HEF-LP}{lpnmr/GebserLL07}
\setcitation{HCF-LP}{amai/Ben-EliyahuD94}
\setcitation{aspcore2}{AspCore2}
\setcitation{}{}
\setcitation{}{}
\setcitation{}{}
\setcitation{}{}
\setcitation{}{}
\setcitation{}{}
\setcitation{}{}
\setcitation{}{}
\setcitation{}{}
\setcitation{}{}
\setcitation{}{}
\setcitation{}{}
\setcitation{}{}
\setcitation{}{}
\setcitation{}{}
\setcitation{}{}
\setcitation{}{}
\setcitation{}{}
\setcitation{}{}
\setcitation{}{}
\setcitation{}{}
\setcitation{}{}
\setcitation{}{}
  
 %Command for in case you want multiple citations in one, e.g., \cite{\refto{fodot},\refto{idp}}
 %Warning: no safety checks
\newcommand\refto[1]{%
      \ifcsname mycommoncitation#1\endcsname%
      \getcitation{#1}%
      \else%
      #1%
      \fi%
      }
      
 %usage: \mycite{key}, e.g., \mycite{fodot} results in \cite{tocl/DeneckerT08}
\newcommand\mycite[1]{%
      \ifcsname mycommoncitation#1\endcsname%
   \cite{\getcitation{#1}}%
  \else%
    \cite{#1}%
  \fi%
}	
  
   %usage: \mycite{key}, e.g., \mycite{fodot} results in \cite{tocl/DeneckerT08}
\newcommand\mycitet[1]{%
      \ifcsname mycommoncitation#1\endcsname%
   \citet{\getcitation{#1}}%
  \else%
    \citet{#1}
  \fi%
}	
  


\begin{document}
 
\maketitle

\begin{abstract}
We build on a recently proposed method for explaining solutions of constraint satisfaction problems.
An explanation here is a \textit{sequence} of simple inference steps, where the simplicity of an inference step is measured by the number and types of constraints and facts used, and where the sequence explains all logical consequences of the problem. 
We build on these formal foundations and tackle two emerging questions, namely how to generate explanations that are provably optimal (with respect to the given cost metric) and how to generate them efficiently. 
To answer these questions, we develop 1) an implicit hitting set algorithm for finding \textit{optimal} unsatisfiable subsets; 2) a method to reduce multiple calls for (optimal) unsatisfiable subsets to a single call that takes \emph{constraints} on the subset into account, and 3) a method for re-using relevant information over multiple calls to these algorithms. 
The method is also applicable to other problems that require finding cost-optimal unsatisfiable subsets.
We specifically show that this approach can be used to effectively find sequences of \textit{optimal} explanation steps for constraint satisfaction problems like logic grid puzzles.
\end{abstract}

\section{Introduction}
% !TeX root = ./main.tex

Building on old ideas to explain domain-specific propagations performed by constraint solvers  \cite{sqalli1996inference,freuder2001explanation}, we recently introduced a 
method that takes as input a satisfiable constraint program and explains the solution-finding process in a human-understandable way  \cite{ecai/BogaertsGCG20}. 
Explanations in that work are sequences of simple inference steps, involving as few constraints and facts as possible. 
The explanation-generation algorithms presented in that work rely heavily on calls for  \emph{Minimal Unsatisfiable Subsets} (MUS) \cite{marques2010minimal} of a derived program, exploiting a one-to-one correspondence between so-called \emph{non-redundant explanations} and MUSs.
The explanation steps in the seminal work are heuristically optimized with respect to a given cost function that should approximate human-understandability, e.g., taking the number of constraints and facts into account, as well as a valuation of their complexity (or priority). 
The algorithm developed in that work has two main weaknesses: first, it provides no guarantees on the quality of the produced explanations due to internally relying on the computation of $\subseteq$-minimal unsatisfiable subsets, which are often suboptimal with respect to the given cost function. 
Secondly, it suffers from performance problems: the lack of optimality is partly overcome by calling a MUS algorithm on increasingly larger subsets of constraints for each candidate implied fact.
However, using multiple MUS calls per literal in each iterations quickly causes efficiency problems, causing the explanation generation process to take several hours.


Motivated by these observations, we develop algorithms that aid explaining CSPs and improve the state-of-the-art in the following ways: 
\begin{itemize}
 \item We develop algorithms that compute (cost-)\textbf{Optimal} Unsatisfiable Subsets (from now on called OUSs) based on the well-known hitting-set duality that is also used for computing cardinality-minimal MUSs \cite{ignatiev2015smallest,DBLP:conf/kr/SaikkoWJ16}.
\item We observe that many of the individual calls for MUSs (or OUSs) can actually be replaced by a single call that searches for an optimal unsatisfiable subset \textbf{among subsets satisfying certain structural constraints}. In other words, we introduce the \emph{Optimal \textbf{Constrained} Unsatisfiable Subsets (OCUS)} problem and we show how $O(n^2)$ calls to MUS/OUS can be replaced by $O(n)$ calls to an OCUS oracle, where $n$ denotes the number of facts to explain. 
\item Finally, we develop techniques for \textbf{optimizing} the OCUS algorithms further, exploiting domain-specific information coming from the fact that we are in the  \emph{explanation-generation context}. One such optimization is the development of methods for \textbf{information re-use} between consecutive OCUS calls.
\end{itemize}

In this paper, we apply our OCUS algorithms to generate \emph{step-wise} explanations of satisfaction problems. However, MUSs have been used in a variety of contexts, and in particular lie at the foundations of several explanation techniques \cite{junker2001quickxplain,ignatiev2019abduction,schotten}. We conjecture that OCUS can also prove useful in those settings, to take more fine-grained control over which MUSs, and eventually, which explanations are produced.

The rest of this paper is structured as follows.
We discuss background on the hitting-set duality in \cref{sec:background}. \cref{sec:motviation} motivates our work, while \cref{sec:ocus} introduces the OCUS problem and a generic \hitsetbased algorithm for computing OCUSs. In \cref{sec:ocusEx} we show how to optimize this computation in the context of explanations and in  
\cref{sec:experiments}  we experimentally validate the approach.
We discuss related work in  \cref{sec:related} and conclude in \cref{sec:conclusion}.


\section{Background}\label{sec:backgr}\label{sec:background}
\begin{itemize}
    \item Explanation sequence : CSP journal paper ?
    \item concept of OMUS
    % \item Logic grid puzzles (depends on experiments)
    % \item Nested explanation still applicable?
\end{itemize}

\section{Motivation}\label{sec:motivation}\label{sec:motviation}
Our work is motivated by the problem of explaining satisfaction problems through a sequence of simple explanation steps~\citet{ecai/BogaertsGCG20}. This can be used to teach people problem-solving skills, to compare the difficulty of related satisfaction problems (through the number and complexity of steps needed), and in human-computer solving assistants.

\ignore{

\paragraph{Using Constraints to Encode Domain Knowledge}

% {\color{OliveGreen} OLD TEXT TO BE REWRITTEN
\bart{Not efficient -> Begin paper zeggen. }
The constraints on OCUSs can not only be used to restrict the set of solutions, but also to improve the solver performance by encoding domain knowledge.
Indeed, if we know that all ``good'' OCUSs will satisfy certain constraints, or if we know that it suffices to search for OCUSs satisfying certain constraints (because each OCUS can easily be extended to one such OCUS),  we can also encode that knowledge in $p$, thereby restricting the possible options of the hitting set solver, aiming to improve overall performance of the algorithm. 

In the explanation application, we encountered this phenomenon as follows. 
The clues to be used in explanations were high-level (first-order) constraints. They were translated into clauses, using among other, a Tseitin transformation.
Hence, in the end the transformation of a single high-level clue consists of several clauses, of which some are definitions of newly introduced variables. 
Now, the associated cost function was only concerned with the issue ``\emph{was a certain clue used or not?}'', which translates at the lower level to ``\emph{does the OCUS contain at least one clause from the translation of the clue?}''.
Using such a cost function means that to compute the cost of an OCUS, it does not matter if a single, or if all clauses corresponding to a given clue are used. As such, we might as well include all of them, which can be encoded in $p$ as well.  

An alternative view on the same property is that we can \emph{reify} the high level constraint by considering an indicator variable defining satisfaction of the entire constraint. 
We can then add the property to $p$ that all reified constraints are \emph{hard constraints}, in the sense that they have to be included in each OCUS (and thus each hitting set). With that, only the truth/falsity of the single indicator variable is considered to be a clause of $\formulac$ that can be enabled/disabled by the hitting set algorithm. 
% This variable then represent whether or not the high level constraint is active.
It is easy to see that there is a one to one correspondence between the OCUSs produced by the two approaches. In our implementation, we opted for the latter because of its simplicity. 
%\tias{is this really to $p$? higher up we argued that we push $p$ into the MIP, but all hard clauses are kept outside of the MIP... I guess saying that har dlcauses are 'always included' is somehow doing that? it also means they are 'constant' in the MIP objective and hence can be removed from it, that is perhaps a more pragmatic view on it...}
%\emilio{phrases are loooong.}
}

Assumed given is a constraint satisfaction problem $\formulac$, a partial interpretation $I$ and a set of atoms (variables) $U$ whose literals need explaining; more specifically the cautious consequence of $\formulac \wedge I$ projected onto $U$. %A single explanation step is an inference step $I' \implies N$ where $I' \subseteq I$ and $atoms(I') \subseteq U$.
\tias{Optional, alg can move to Apdx:} Alg.~\ref{alg:explainCSP2} shows the basic explanation sequence generation algorithm of~\cite{bogaerts2020framework}.



\newcommand\onestep{\ensuremath{\call{explain-One-Step}}\xspace}

\begin{algorithm}[t]
    %\Input{${\cal C}$,  \textit{a CNF ${\cal C}$ over a vocabulary $V$} }
    %\Input{$U$, \textit{a user vocabulary $U \subseteq V$} }
    %\Input{$f$, \textit{a cost function $f : 2^{lits({U})} \rightarrow  \mathbb{N} $.}}
    %\Input{$I$, \textit{a partial interpretation over $U$}}
    %\Output{$E$, \textit{a sequence of explanation steps as implications $I_{expl} \implies N_{expl}$}}
    % \vspace*{0.01cm}
    \DontPrintSemicolon
    \caption{$\call{ExplainCSP}(\formulac, f, I, U)$}
    \label{alg:explainCSP}
    %$\sat \gets \initsat({\cal C}$) \;
        % \tcp{Hyp: f}
    $I_{end} \gets U \cap \optprop(\formulac \wedge I)$ \;
    $E \gets \langle \rangle$\;
    %$U \gets U \cap I_{end}$ \;
    %$I_{expl} = \{i \in I_{end} | f(i) < inf \wedge f(-i) < inf\}$ \;
    % \algemilio{bart: What's a better way to get the initial interpr.?}
    % $I \gets \{l \in I_{end} | f(\lnot l) = inf\}$ \;
    \While{$(I \cap U) \neq I_{end}$}{
        %$E \gets \call{bestStep}({\cal C},U, f,\Iend, I)$\;
        $X \gets \onestep({\cal C},f,I,\Iend)$\;
        %$I_{\mathit{best}} \gets I \cap X$\;
        %$\formulac_{\mathit{best}}\gets \formulac\cap X$\;
        $N \gets  (\Iend \setminus I) \cap \optprop(X \cap I)$\;
        %add $\{I_{\mathit{best}} \wedge \formulac_{\mathit{best}} \implies N_{\mathit{best}}\}$ to $E$\;
        add $\{(X \cap I) \implies N\}$ to $E$\;
        $I \gets I \cup N$\;
    }
    \Return{E}\;
\end{algorithm}


\begin{algorithm}[t]
  \caption{$\onestep(\formulac,f,I,\Iend)$}
  \label{alg:oneStep}
$X_{best} \gets \mathit{nil}$\;
\For{$l \in \{\Iend \setminus I\}$}{
    $X \gets \call{MUS}{(\formulac \land I \land \neg l)}$\;
    \If{$f(X)<f(X_{best})$}{
        $X_{best} \gets X$\;
    }
}
\Return{$X_{best}$} 
\end{algorithm}


The goal is to find a sequence of \textit{simple} explanation steps, where the simplicity of a step is measured by a cost function $f$. 
An explanation step is an implication $I' \wedge \formulac' \implies N$ where $I'$ is a set of already derived literals, $\formulac'$ is a subset of constraints of the input formula $\formulac$, and $N$ is a set of literals entailed by $I'$ and $\formulac'$ that is not yet explained.
To obtain a sequence of such steps, they iteratively search for the best (least costly) explanation step and add its consequence to the partial interpretation $I$.
When aiming to explain satisfaction problems in terms of the subset of constraints and literals needed to derive a new literal, the initial interpretation $I$ should consist of indicators literals for each (group of) constraint(s) as well as already known true literals.

The key part is finding the best next step. \cref{alg:oneStep} shows the gist of the algorithm used in \cite{ecai/BogaertsGCG20}.
It takes as input the formula \formulac, a cost function $f$ quantifying quality of explanations, an interpretation $I$ containing all already derived literals in the sequence so far, and the interpretation-to-explain $\Iend$. 
To compute an explanation, this procedure iterates over the literals that are still to explain, computes for each of them an associated MUS and subsequenty selects the best of the found such MUSs.  
The reason this works is because there is a one-to-one correspondence between MUSs of $\formulac \land I \land \neg l$ and so-called \emph{non-redundant explanation} of $l$ in terms of $\formulac$ and $I$~\cite{ecai/BogaertsGCG20}. 

Experiments have shown that such a MUS-based approach can easily take hours, and hence that algorithmic improvements are needed to make it more practical. 
We see three main points of improvement, all of which will be tackled by our generic OCUS algorithms presented in the next section. 
% \tias{in the exps, MUS-based does not, also because we don't do the constraint trick. What to do?} \bart{Maybe the experiment section is a good place to discuss this in detail. That we there clearly state: in Bogaerts et al: the MUS based thing from algorithm \onestep is found not sufficient. To get better MUSs they implement this trick that walks a different balance between ``doing it for too many subsets and exploding there'' and ``getting terrible explanations''. What we show is that with our constrained stuff, you can get the best of both worlds: we get optimal explanatiosn (better than the subset stuff) plus we are as efficient as the no-subset approach). 
%  But... in the main text I would not go into detail because it would distract from our main message. }
\begin{inparaenum}
 \item First of all, since the algorithm is based on \call{MUS} calls, there is no guarantee that the explanation found is indeed the best (with respect to the given cost function) possible. 
 Most likely, the reason for choosing \call{MUS}es is that currently, \textit{there are no algorithms for unsatisfiable subset \textbf{optimization}}. 
 \item Second, this algorithm uses \call{MUS} calls for every literal to explain separately. The goal of all of these calls is to find a single unsatisfiable subset of $\formulac \land I \land \overline{(\Iend\setminus I)}$ that contains exactly one literal from $\overline{(\Iend\setminus I)}$. This begs the questions whether it is possible \textit{to compute a single (optimal) unsatisfiable subset \textbf{subject to constraints}}, where in our case, the constraint is on the number of literals from $\overline{(\Iend\setminus I)}$ to include. 
 \item Finally, the algorithm that computes an entire explanation sequence makes use of repeated calls to \onestep and hence will solve many very similar problems. This raises the issue of \textit{\textbf{incrementality}: can we re-use data structures to achieve speed-ups in later calls}? 
\end{inparaenum}

The first two points lead to the following definition. 


\begin{definition}
    If $\formula$ is a formula, $f:2^{\formula} \to \nat$ a cost function and  $p$ a predicate $p: 2^{\formula}\to \{\ltrue,\lfalse\}$, then we call %a set $U\subseteq \formulag$ a \emph{$p$-constrained $f$-OUS} of \formulag ($(p,f)$-OUS) \tias{what with the OCUS name here?} \bart{I propsoe to say. We call 
    $X \subseteq \formula$ an OCUS of \formula (with respect to $f$ and $p$) if \begin{compactitem}                                      
      \item $X$ is unsatisfiable,
      \item $p(X)$ is true
      \item all other unsatisfiable $X'\subseteq \formula$ with $p(X')=\ltrue$ satisfy $f(X')\geq f(X)$.
    \end{compactitem}
\end{definition}

Rephrased in this terms, the task of the procedure \onestep is to compute an OCUS of the formula $\formula := \formulac\land I\land \overline{\Iend\setminus I}$ with $p$ the predicate that holds when $\formulag$ that contain exactly one literal of $\overline{\Iend}$, see Algorithm~\eqref{alg:oneStepOCUS}. 
In the rest of this paper, we study (incremental) algorithms for computing an OCUS. 






\section{Optimal Constrained Unsatisfiable Subsets} \label{sec:ocus}
The first two considerations from the previous section lead to the following definition. 

\begin{definition}
   Let $\formula$ be a formula, $f:2^{\formula} \to \nat$ a cost function and  $p$ a predicate $p: 2^{\formula}\to \{\ltrue,\lfalse\}$. We call %a set $U\subseteq \formulag$ a \emph{$p$-constrained $f$-OUS} of \formulag ($(p,f)$-OUS) \tias{what with the OCUS name here?} \bart{I propsoe to say. We call 
    $\m{S} \subseteq \formula$ an OCUS of \formula (with respect to $f$ and $p$) if \begin{compactitem}                                      
      \item $\m{S}$ is unsatisfiable,
      \item $p(\m{S})$ is true
      \item all other unsatisfiable $\m{S}'\subseteq \formula$ with $p(\m{S}')=\ltrue$ satisfy $f(\m{S}')\geq f(\m{S})$.
    \end{compactitem}
\end{definition}

Rephrased in these terms, the task of the procedure \onestep is to compute an OCUS of the formula $\formula := \formulac\land I\land \overline{\Iend\setminus I}$ with $p$ the predicate that holds for subsets  that contain exactly one literal of $\overline{\Iend}$, see \cref{alg:oneStepOCUS}. 
%In the rest of this paper, we study (incremental) algorithms for computing an OCUS. 

%\emilio{why is the text formatting weird ? Due to the algorithm names ? }
In order to compute an OCUS of a given formula, we propose to build on the hitting set duality of \cref{prop:MCS-MUS-hittingset}. 
For this, we will assume to have access to a solver \cohs that can compute hitting sets of a given collection of sets that are \emph{optimal} (w.r.t.\ a given cost function $f$) among all hitting sets \emph{satisfying a condition $p$}. 
The choice of the underlying hitting set solver will thus determine which types of cost functions and constraints are possible. 
In our implementation we use a cost function $f$ as well as a condition $p$ that can easily be encoded as linear constraints, thus allowing the use of highly optimized mixed integer programming (MIP) solvers. The \cohs formulation is as follows:
%\bart{should be minimize or -f}\emilio{adapted}
\begin{align*}
\small
  minimize_S \quad & f(S) \\ 
  s.t. \quad & p(S) \\
       & sum(H) \geq 1, \quad &\forall H \in \setstohit \\
       & s \in \{0,1\}, \quad &\forall s \in S
\end{align*}
%\bart{This is not what the reviewers asked for! They asked for MIP models of our ``arbitrary objective functions''. A mip encoding of a generic hitting set problem with only some ``at least one'' constraints is not going to help us, I think. }
% well, its not arbitrary but linear, and it is a weighted sum; will have to do
%\bart{indeed, it is not arbitrary! But that's exactly the point: if we say ``we only support linear objective functions'', then that reviewers concern ``how will you encode this as MIP?'' is resolved. We simply did not say that in the prior work. In any case, the model also doesn't hurt and breaks text a bit... } 
where $S$ is a set of MIP decision variables, one for every clause in $\formula$. In our case, $p$ is expressed as $\sum_{s \in \overline{\Iend\setminus I}} s = 1$. 
%On top of that, the $p$ can be used to enforce that some constraints in \formula are hard constraints and should always be included in the hitting set. %make a distinction between hard constraints (those that \emph{must} be included in the OCUS), which can be useful in case constraints are reified using assumption literals, or to \emph{group} constraints that should be enabled/disabled simultaneously. 
%\emilio{not trivial at all!!! If I understand correctly C = Hard + indicator variables where p imposes constraints on not using the Hard, but only the indicator variables}
%\emilio{Added an example:}
%For example, when explaining logic grid puzzles, the cnf-translated hard constraints ($\mathcal{C}_{hard}$) are reified using assumption literals i.e weighted soft constraints ($\mathcal{C}_{soft}$). In that case, $\mathcal{C} = \mathcal{C}_{hard} \ \cup \ \mathcal{C}_{soft}$.
% When aiming to explain satisfaction problems in terms of the subset of constraints and literals needed to derive a new literal, the initial interpretation $I$ should consist of indicators literals for each (group of) constraint(s) as well as already known true literals. }
%\bart{Previous sentence instead of earlier ``indicator'' sentence?}
%
%In our application, 
$f$ is a weighted sum over the variables in $S$. For example, (unit) clauses representing previously derived facts can be given small weights and regular constraints can be given large weights, such that explanations are penalized for including many constraints when previously derived facts can be used instead. %relevant facts are directly available.



\newcommand\onestepo{\ensuremath{\call{explain-One-Step-ocus}}\xspace}
\begin{algorithm}[t]
  \DontPrintSemicolon
  
  \caption{$\onestepo(\formulac,f,I,\Iend)$}
  \label{alg:oneStepOCUS}
  $p \leftarrow$ exactly one of $\overline{\Iend\setminus I}$\;
  \Return{$\comus(\formulac\land I\land \overline{\Iend\setminus I}, f, p)$} 
\end{algorithm}
\begin{algorithm}[t]
  \DontPrintSemicolon
%  $\setstohit  \gets \emptyset$ \; %\label{omus-line1} 
  \While{true}{
    $\m{S} \gets \cohs(\setstohit,f,p) $  \;%\tcp*{\small Find \textb    $\setstohit  \gets \setstohit  \cup \{  \formula \setminus \F''\}$ \;
% f{optimal} solution}
    % \tcp{\small set with all unique clauses from hitting set}
%     (sat?, $\kappa$) $\gets$ \texttt{SatSolver}($hs$)\;
    % \tcp{If SAT, $\kappa$ contains the satisfying truth assignment}
    % \tcp{IF UNSAT, $hs$ is the OUS }
    \If{ $\lnot \sat(\m{S})$}{
      \Return{$\m{S}$} \;
    }
    $\m{S} \gets  \grow(\m{S},\F) $ \label{line:grow}\;
    $\setstohit  \gets \setstohit  \cup \{  \formula \setminus \m{S}\}$ \;
  }
  \caption{$\comus(\formula,f,p)$ }
  \label{alg:comus}
\end{algorithm}


%\tias{I would not show the above one as it is rather vague \bart{I would disagree with the vagueness. It makes abstraction of several things (what is $p$? what is $f$? How is Grow and CondOptHS implemented? But in my opinion that is good, since it shows close relation to the basic OUS algo as well as illustrating what is really going on and modularity.}, but immediately rewrite it as Alg2 the singleStepExplain:}
%\bart{I would avoid that :-) That way we mix up ``how to compute constrained OUSs?'' with ``how to compute explanations using constrained OUSs?''. These are two different concerns. We should show that we also tackle the first .  That way, our new ``singlestepexplain2'' will also look a lot simpler than singleStepExplain1 (if we use one oracle call to cOUS}

%\begin{algorithm}[ht]
%  \caption{$\call{ExplainCSPcOUS}({\cal C},f)$}
%  \label{alg:explainCSPcOUS}
%$E \gets \langle \rangle$\;
%$I_{end} \gets optimalPropagate({\cal C})$\;
%$\formulag \gets {\cal C} \cup I_{end} \cup \overline{\Iend}$\;
%$\setstohit \gets \{\formulag \setminus \{{\cal C} \cup I_{end}\}\}$\;
%// preseeding\\
%\For{$l \in I_{end}:$}{
%  $\setstohit \gets \setstohit \cup \{\formulag \setminus \grow(-l,\formulag)\} $\;
%}
%$I \gets \emptyset$\;
%$p \gets \{$exactly one of $\overline{\Iend}$ in the hitting set$\}$\; %, none of $I_{end}$ can be in the hitting set$\}$\;
%
%\While{$I \neq I_{end}$}{
%	update $p$ such that none of $\{I_{end} \setminus I\}$ and none of $\bar{I}$ can be in the hitting set\;
%    $\m{S} \gets \comus(\formulag,f,p,\setstohit)$\;
%	$I_{\mathit{best}} \gets I\cap \m{S}$\;
%    ${\cal C}_{\mathit{best}}\gets{\cal C}\cap \m{S}$\;
%	$N_{\mathit{best}} \gets \{I_{end} \setminus I\} \cap optimalPropagate(\m{S}) $\;
%	add $\{I_{\mathit{best}} \wedge {\cal C}_{\mathit{best}} \implies N_{\mathit{best}}\}$ to $E$\;
%	$I \gets I \cup N_{\mathit{best}}$\; 
%  }
%\Return{E}\;
%\end{algorithm}
Our generic algorithm for computing OCUSs is depicted in \cref{alg:comus}. It combines the hitting set-based approach for MUSs of \cite{ignatiev2015smallest} with the use of a MIP solver for (weighted) hitting sets as proposed for maximum satisfiability \cite{DBLP:conf/sat/DaviesB13}. The key novelty is the ability to add structural constraints to the hitting set solver, without impacting the duality principles of \cref{prop:MCS-MUS-hittingset} as we will show.

Ignoring \cref{line:grow} for a moment, 
the algorithm alternates calls to a hitting set solver with calls to a \sat oracle on a subset $\m{S}$ of $\formula$. 
In case the \sat oracle returns true, i.e., the subset $\m{S}$ is satisfiable, the complement of $\m{S}$ is a correction subset and is added to \setstohit. In this way, the hitting set size always grows and a hitting set $\m{S}$ will always be a subset of $\formula$.

Instead of directly adding the complement of $\m{S}$ to \setstohit as done by \citet{ignatiev2015smallest}, our algorithm includes a call to \grow, which extends $\m{S}$ into a larger subset of $\formula$ that is still satisfiable (if possible).
%\tias{here base grow impls?} \bart{Your next sentence is enough for me here. Alternatively we can in one stentence say that we consider a greedy and a maxsat-based implementation? }
We discuss the different possible implementations of \grow later and evaluate their performance in \cref{sec:experiments}.

\emilio{The result of algorithm \ref{alg:comus} represents a subset of the clauses of F if applied on a toy example with imaginary weights. Consider the following, cnf $\formula$ formula over variables $x_1, x_2, x_3$ with model $\{x_1, \lnot x_2, x_3 \}$, $p\triangleq$ \textit{exactlyoneof($c_6$, $c_7$)} and $f = \sum w_i$ the sum over clause weights $w_1 = 20, w_2=20, w_3=20, w_4=20, w_5=1, w_6=1, w_7=1$:
	\[ c_1 = x_1 \text{\hspace*{20pt}} c_2 = \lnot x_1 \vee \lnot x_2 \vee x_3\] \[ c_3 = \lnot x_1 \vee \lnot x_2 \vee x_3 \text{\hspace*{20pt}} c_4 = \lnot x_3 \vee \lnot x_2 \]
	 If we already know $I= \{x_1\}$, then remaining literals to explain are defined as $\overline{\Iend\setminus I} = c_6 \wedge c_7$:
\[c_5 = x_1 \text{\hspace*{20pt}} c_6 = x_2 \text{\hspace*{20pt}} c_7 = \lnot x_3 \]
Let $\setstohit =\{ \{ c_6, c_7\} \}$ be collection of hitting sets bootstrapped with the at least one constraint.
}

\begin{table}[!h]
	
	\begin{adjustbox}{max width=\columnwidth}
		\emilio{
	\begin{tabular}{|c|c|c|r|}
		\hline
		\rule{0pt}{2ex}$\m{S}$ & $\sat(\m{S})$ & Grow($\m{S}$, $\formula$) & $\setstohit  \gets \setstohit  \cup \{  \formula \setminus \m{S}\}$\\ 
\hline
		\hline
		\rule{0pt}{2ex} $\{ c_6 \}$ & True & $\{ c_6,  c_1, c_5, c_2, c_3\}$    & $\{ \{ c_6, c_7\}, \{c_7, c_4\}\}$\rule{0pt}{2ex}   \\
		\hline		
		 \rule{0pt}{2ex}\multirow{2}*{ $\{ c_6, c_4 \}$}  & \multirow{2}*{True} & \multirow{2}*{$\{ c_6,  c_4, c_2, c_3\}$}   & $\{ \{ c_6, c_7\}, \{c_7, c_4\}$  \\  
		   &  &    & $\{c_7, c_1, c_5\}\}$\rule{0pt}{2ex} \\\hline
				 \rule{0pt}{2ex} \multirow{2}*{$\{ c_6, c_4, c_5 \}$}  & \multirow{2}*{True} &  \multirow{2}*{$\{ c_6,c_1, c_4, c_5, c_7 \}$}  & $\{ \{ c_6, c_7\}, \{c_7, c_4\}$  \\  
				    &  &    & $\{c_7, c_1, c_5\}, \{c_2, c_3\}\}$\rule{0pt}{2ex} \\\hline
		\rule{0pt}{2ex} $\{ c_6, c_4,c_5, c_2 \}$ & False & \multicolumn{2}{|c|}{$\{ c_6, c_4,c_5, c_2 \}$ is an OCUS of $\formula$.} \\
		\hline
	\end{tabular}
}
\end{adjustbox}
\end{table}


Soundness and completeness of the proposal follow from the fact that all sets added to \setstohit are correction subsets, and \cref{thm:soundcomplete}, which states that what is returned is indeed a solution and that a solution will be found if it exists. 
 
\begin{theorem}\label{thm:soundcomplete}
  Let $\m{H}$ be the the set of correction sets of \formula. 
  If $\m{S}$ is a hitting set of \m{H} that is $f$-optimal among the hitting sets of \m{H} satisfying a predicate $p$, and  $\m{S}$ is unsatisfiable, then $\m{S}$ is an OCUS of \formula. 
  
  If  $\m{H}$ has no hitting sets satisfying $p$, then $\formula$ has no OCUSs.
\end{theorem}
\begin{proof}
For the first claim, it is clear that $\m{S}$ is unsatisfiable and satisfies $p$. Hence all we need to show is $f$-optimality of $\m{S}$.
  If there were some other unsatisfiable subset $\m{S}'$ that satisfies $p$ with $f(\m{S}')\leq f(\m{S})$, we know that $\m{S}'$ would hit every minimal correction set of \m{F}, and hence also every set in \m{H} (since every correction set is the superset of a minimal correction set).
  Since $\m{S}$ is $f$-optimal among hitting sets of $\m{H}$ satisfying $p$ and $\m{S}'$ also hits $\m{H}$ and satisfies $p$, it must thus be that $f(\m{S})=f(\m{S}')$. 
%  

The second claim immediately follows from \cref{prop:MCS-MUS-hittingset} and the fact that an OCUS is an unsatisfiable subset of $\formula$. 
\end{proof}
% 
% 

Perhaps surprisingly, correctness of the proposed algorithm does \emph{not} depend on monotonicity properties of $f$ nor $p$. In principle, any (computable) cost function and condition on the unsatisfiable subsets can be used. In practice however, one is bound by limitations of the chosen hitting set solver. 

\emilio{Extending theorem \ref{thm:soundcomplete}, if $f$ is defined as a stricly increasing monotone function and p also satisfying monotonicity property then \comus is guaranteed to be a \emph{minimal} unsatisfiable subset.}

\begin{proposition}
	\emilio{
	The complexity of extracting the \mm{\call{O(C)US}} is of the second level of polynomial hierarchy FP$^{\sum^{P}_2}$. 
}
\end{proposition}
\begin{proof}
	\emilio{
	The extraction of a cardinality-minimal MUS (SMUS) for an unsatisfiable cnf $\formula$ is of the second level of polynomial hierarchy FP$^{\sum^{P}_2}$ \cite{ignatiev2015smallest}. In the case of $f$ being a weighted sum, the \omus extraction is reducible to an SMUS-extraction on a relaxed formula $\formula^{R}$ in polynomial time. 

		\noindent\textit{Reduction.} Associate for each clause $c_i \in \formula$ of weight $w_i=n \in \mathbb{N}^+$, the relaxation variables $r_{i,j}$ where $j \in 1..n-1$:
		\[ \formula^{R}  \triangleq \bigcup_{c_i \in \formula}(c_i \leftrightarrow (r_{i, 1} \wedge ..  r_{i, n-1})) \]

%	$F^{R} \triangleq \cup \$
%	, by adding $w_i - 1$ relaxation variables $(r_{i,1} \wedge .. \wedge r_{i,n-1}) \leftrightarrow c_i$, the problem of OUS is reduced to the problem of SMUS.
	}
	\paragraph{Constrainedness.} \todo{Add proof for Constrainedness}
\end{proof}
% Now, since the search for optimal hitting sets is --- in implicit hitting set algorithms --- usually done with a MIP solver, it suffices to express the predicate $p$ as constraints on the MIP. Since the variables of the MIP encoding represent inclusion of certain constraints in the unsatisfiable subset, this is simple for  the 3 constraints that we need to obtain meaningful explanations. %needed to have meaningful in practice only predicates $p$ that can easily be encoded in MIP are useful. In such cases, we can directly use the MIP solver to implement \cohs as well. 

% \paragraph{Application to Explanations}
% %To apply this idea to the context of explanations, we note that at each step, the current interpretation, will be fixed. 
% %At each step, we are looking for an OUS that contains \emph{exactly one} negation of a derivable literal. 
% %Such an exactly-one constraint is easily expressible in MIP.
% %Furthermore, also the ``subtheory constraint'', as introduced for incremental MUS solving can be expressed in MIP. Namely, in \cref{sec:incremental}, we assumed that each OUS call would be done given a subtheory of the original theory. However, constraints of the form ``the OUS should be a subset of the given set \formula'' are easily expressible in MIP as well. 
% %As such, the idea of constrained OUS computation is actually more general than the formalization of incremental OUS. 
% % 
% Given such a constrained OUS algorithm, the procedure to find the single best explanation step now simplifies to \cref{alg:singleStepExplain3}.
% 
% \begin{algorithm}[t]
%   \caption{$\call{bestStep--c-OUS}({\cal C},f,I,I_{end})$}
%   \label{alg:singleStepExplain3}
% $\formulag \gets {\cal C} \cup I_{end} \cup \overline{\Iend}$\;
% set $p$ such that exactly one of $\overline{\Iend}$ in the hitting set \textit{and} none of $\{I_{end} \setminus I\}$ \textit{and} none of $\bar{I}$ can be in the hitting set\;
% \Return{$\comus(\formulag,f,p)$}\;
% \end{algorithm}

% \tias{hard/soft temporarily hidden}
% \ignore{



\section{Efficient OCUS Computation for Explanations}\label{sec:ocusEx}
In this section, we discuss optimizations to the basic algorithm to compute $(p,f)$-OUSs. 
While our optimizations are inspired by the explanation generation problem, and tailored to solving this as efficiently as possible, the ideas presented here can also be used when other forms of domain knowledge are present.  


\paragraph{Incremental cOUS computation}
\todo{copy paste some old text about reuse of MSSs}


\paragraph{Using Initialization to Add Domain Knowledge}
\todo{TO BE DETERMIEND IF WE WANT TO SAY SOMETHING AOBUT PRESEEDING.}\bart{ I still dont believe it makes a difference. TO be tested first!} 



\paragraph{Domain-Specific Implementations of \grow}
\todo{``old maxsat'' comes here}


\paragraph{Using Constraints to Encode Domain Knowledge}

{\color{OliveGreen} OLD TEXT TO BE REWRITTEN
The constraints on OUSs can not only be used to restrict the set of solutions, but also to improve the solver performance by encoding domain knowledge.
Indeed, if we know that all ``good'' OUSs will satisfy certain constraints, or if we know that it suffices to search for OUSs satisfying certain constraints (because each OUS can easily be extended to one such OUS),  we can also encode that knowledge in $p$, thereby restricting the possible options of the hitting set solver, aiming to improve overall performance of the algorithm. 

In the explanation application, we encountered this phenomenon as follows. 
The clues to be used in explanations were high-level (first-order) constraints. They were translated into clauses, using among other, a Tseitin transformation.
Hence, in the end the transformation of a single high-level clue consists of several clauses, of which some are definitions of newly introduced variables. 
Now, the associated cost function was only concerned with the issue ``\emph{was a certain clue used or not?}'', which translates at the lower level to ``\emph{does the OUS contain at least one clause from the translation of the clue?}''.
Using such a cost function means that to compute the cost of an OUS, it does not matter if a single, or if all clauses corresponding to a given clue are used. As such, we might as well include all of them, which can be encoded in $p$ as well. 

An alternative view on the same property is that we can \emph{reify} the high level constraint by considering an indicator variable defining satisfaction of the entire constraint. 
We can then add the property to $p$ that all reified constraints are \emph{hard constraints}, in the sense that they have to be included in each OUS (and thus each hitting set). With that, only the truth/falsity of the single indicator variable is considered to be a clause of $\formulac$ that can be enabled/disabled by the hitting set algorithm. 
% This variable then represent whether or not the high level constraint is active.
It is easy to see that there is a one to one correspondence between the OUSs produced by the two approaches. In our implementation, we opted for the latter because of its simplicity. 
%\tias{is this really to $p$? higher up we argued that we push $p$ into the MIP, but all hard clauses are kept outside of the MIP... I guess saying that har dlcauses are 'always included' is somehow doing that? it also means they are 'constant' in the MIP objective and hence can be removed from it, that is perhaps a more pragmatic view on it...}
%\emilio{phrases are loooong.}


}


\section{Experiments}\label{sec:experiments}
% !TeX root = ./main.tex

We now experimentally validate the performance of the different versions of our algorithm.
Our benchmarks were run on a compute cluster, where each explanation sequence generation was assigned a single core on a 10-core INTEL Xeon Gold 61482 (Skylake) processor, a timelimit of 120 minutes and a memory-limit of 4GB. 
All code available was implemented in Python on top of PySAT.\footnote{\url{https://pysathq.github.io}} The MIP solver used is Gurobi 9.0 and when a (Max)SAT solver is used it is RC2 as bundled with PySAT. In the MUS-based approach we used PySAT's deletion-based MUS extractor MUSX~\cite{marques2010minimal}.

All of our experiments were run on a direct translation to PySAT of the 10 puzzles of \citet{ecai/BogaertsGCG20}\footnote{In one of the puzzles, an error in the automatic translation of the natural language constraints was found and fixed.}. %Because of this error, it was missing in the experimental results of the previous work.} 
In all puzzles, we used a cost of 60 for puzzle-agnostic constraints; 100 for puzzle-specific constraints; and cost 1 for facts.
The selected clauses in an explanation for such a puzzle consist of one or more constraints together with some previously derived facts. 
% When generating an explanation sequence for such puzzles, the clauses of the unsatisfiable subset represent using 1 or more constraints combined with previously known or derived facts.
%
Our experiments are designed to answer the following research questions: 
\begin{compactdesc}
 \item[Q1] What is the effect of requiring optimality of the generated MUSs on the \textbf{quality} of the generated explanations and \emilio{the generated explanation sequence} ?
 \item[Q2] Which \textbf{domain-specific \grow methods} perform best?
 \item[Q3] What is the effect of the use of \textbf{constrainedness} on the time required to compute an explanation sequence?
 \item[Q4] Does \textbf{re-use} of computed satisfiable subsets improve efficiency?
\end{compactdesc}

% \bart{too much detail... -> Requries explaingin what ``transtiivity'' etc mean in this setting) }\emilio{ok modified}

% \begin{figure}[ht]
%   \centering
%   \includegraphics[width=\columnwidth]{figures/rq1.pdf}
%   \caption{Q1 - Optimal vs Greedy explanation generation quality}
%   \label{fig:rq1}
% \end{figure}

\begin{figure}[ht]
  \centering
  \includegraphics[width=0.6\columnwidth]{figures_post_paper/heatmap_costs_mus_cous.pdf}
  \caption{Q1 - \emilio{Explanation quality comparison of optimal vs. minimal explanations in the generated puzzle explanation sequences.}}
  \label{fig:rq1_heatmap}
\end{figure}


\paragraph{Explanation quality}\label{paragraph:explanationquality}
To evaluate the effect of optimality on the quality of the generated explanations, we reimplemented a MUS-based explanation generator based on \cref{alg:oneStep}. 
Before presenting the results, we want to stress that this is \emph{not} a fair comparison with the implementation of \citet{ecai/BogaertsGCG20}, since they --- in order to avoid the quality problems we will illustrate below --- implemented an extra inner loop with \emph{even more} calls to \call{MUS} for a selected set of subsets of \formulac of increasing size (this is their Algorithm~3). 
While this yields better explanations, it comes at the expense of computation time, thereby leading to several hours to generate the explanation of a single puzzle. 
We will see in our later experiments is that we can both \emph{outperform}, in terms of computational cost, this simple \call{MUS} based implementation while also guaranteeing cost-optimality, as such achieving a Pareto improvement.

To answer \textbf{Q1}, we ran the \call{MUS}-based algorithm as described in \cref{alg:oneStep} and compared at every step the cost of the produced explanation with the cost of the optimal explanation. 
% 
% Similar to \citet{ecai/BogaertsGCG20}, a weight assigned to each type of constraint used. The weighted sum of the constraints present in the 
% (\textit{minimum} / \textit{optimal}) 
% minimal/optimal
% unsatisfiable subset become a proxy to how difficult the generated explanation really is.
% commaring the explanation quality of mus and ous
% To answer \textbf{Q1}, we start by generating a sequence of explanations using the \comus from an initial assignment $I_0$.. Then, using this sequence as a starting point, we look for a MUS-based explanation given the partial assignments $I_i$ at step i.
These costs are plotted on a heatmap in Figure \ref{fig:rq1_heatmap}, where the darkness represents the number of occurrences of the combination at hand. 
% Note that, the distribution of the dotts show the quality of the OUS explanation is either similar, when along the same-cost diagonal or better in the upper part of the graph. The dotts' sizes demonstrates 
We see that the difference in quality is striking in many cases, with the MUS-based solution often missing very cheap explanations (as seen by the large dark vertical column around cost 60), thereby confirming the need for a cost-based \omus/\comus approach.
%\begin{table}[ht]
%		\centering
%	\begin{adjustbox}{max width=\columnwidth}
%\begin{tabular}{|c|c|c|c|c|}
%       \rule{0pt}{2ex} \textbf{MUS}& \textbf{OUS} &  \textbf{OUS+SS.Caching} &\textbf{OCUS} &  \textbf{OCUS+Incr.HS}\\
%        	\midrule
%	   \rule{0pt}{2ex}$90\pm23$ & $117\pm26$ & $113\pm15$   &  $119\pm22$ & $121\pm24$    \\
%	   	\bottomrule
%\end{tabular}
%	\end{adjustbox}
%\caption{Average number of explanation steps per config}
%\label{tab:config-number-explanation-steps}
%	\end{table}
% \tias{Please add avg (and maximum?) MUS size to show that 'smaller is not better' more explicitly}
\paragraph{Notation} In the following, \emph{+SS. caching} corresponds to caching satisfiable subsets in order to re-use subsets between \omus calls, while \emph{+Incr.HS} corresponds to keeping the hitting set solver warm throughout the different calls to \onestepo.

\begin{table}[!h]
	\centering
			\begin{adjustbox}{max width=\columnwidth}
	\begin{tabular}{rrrrrr}
%		 \cline{2-6}
	& \multirow{2}{*}{\textbf{MUS}} & \multirow{2}{*}{\textbf{OUS}} &\textbf{OUS}  & \multirow{2}{*}{\textbf{OCUS}} &\textbf{OCUS}\\
		\rule{0pt}{2ex}&  & & \textbf{+SS.Caching} &  & \textbf{+Incr.HS}  \\
		\midrule
\multicolumn{1}{r|}{\textbf{\textit{avg. $\#$ expl.}}} &  \textbf{90} &  117 &  113 &  119 &  121 \\
%\multicolumn{1}{r|}{\rule{0pt}{3ex}\textbf{\textit{ $\overbar{\text{\# expl.}}$ }}} &  $90\pm23$ &  $117\pm26$ &  $113\pm15$ &  $119\pm22$ &  $121\pm24$ \\
%\multicolumn{1}{|r|}{\rule{0pt}{2ex}\textbf{\textit{std. \# expl.}}} &  $23$ &   &  15 &  $22$ &  $24$ \\
%		\hline
\multicolumn{1}{r|}{\textbf{\rule{0pt}{2ex}\textit{max. expl. size}}} &  \textbf{14}&  13 & 13 &  13 &  13 \\
\multicolumn{1}{r|}{\rule{0pt}{2ex}\textbf{\textit{avg. expl. size}}} &   \textbf{5.02} &   4.07 &   4.08 &  4.05 &   $4.07$ \\
%\multicolumn{1}{|r|}{\rule{0pt}{2ex}\textbf{\textit{std-size}}} &   $2.08$ &   $1.47$ &   $1.47$ &   $1.42$ &   $1.41$ \\
%\hline
\multicolumn{1}{r|}{\rule{0pt}{2ex}\textbf{\textit{avg. $\#$ lits deriv.}}}&   \textbf{1.72} &   1.32 &   1.33 &   1.30 &   1.28 \\
%\multicolumn{1}{|r|}{\rule{0pt}{2ex}\textbf{\textit{std lits deriv.}}} &   1.24 &   0.95 &   0.97 &   0.94 &   $0.91$ \\
%\hline
\bottomrule
	\end{tabular}
		\end{adjustbox}
	\caption{\emilio{Statistics on the composition of the explanation sequences}}
	\label{tab:config-number-explanation-steps-avg-max}
\end{table}

%\begin{table}[ht]
%	\centering
%%		\begin{adjustbox}{max width=\columnwidth}
%\begin{tabular}{lrrrrr}
%	\cline{2-6}
%	&     \multicolumn{5}{c}{\textit{config}} \\
%	\cline{2-6}
%	\rule{0pt}{2ex} &     OCUS+I &       OCUS &      OUS+I &        OUS &        MUS \\
%	\midrule
%	max-size &  13 &  13 &  13 &  13 &  14 \\
%	avg-size &   4.07 &   4.05 &   4.08 &   4.07 &   5.02 \\
%	std-size &   1.41 &   1.41 &   1.47 &   1.47 &   2.08 \\
%	\bottomrule
%\end{tabular}
%%	\end{adjustbox}
%\caption{Average number of explanation steps per config}
%\label{tab:config-number-explanation-steps-avg-max}
%\end{table}

%\begin{table}[ht]
%	\centering
%			\begin{adjustbox}{max width=\columnwidth}
%	\begin{tabular}{cr|r|r|r|r|r|}
%		\cline{3-7}
%		&&     \multicolumn{5}{c|}{\textit{config}} \\
%		\cline{3-7}
%		\rule{0pt}{2ex}&  &     OCUS+I &       OCUS &      OUS+I &        OUS &        MUS \\
%		\hline
%	 \multicolumn{1}{|c|}{\multirow{3}{*}{\rotatebox[origin=c]{90}{\parbox[c]{1cm}{\centering MUS-size}}}}	& max &  13 &  13 &  13 &  13 &  14 \\
%	\multicolumn{1}{|c|}{}	& avg &   4.07 &   4.05 &   4.08 &   4.07 &   5.02 \\
%	\multicolumn{1}{|c|}{}	& std &   1.41 &   1.41 &   1.47 &   1.47 &   2.08 \\
%		\hline
%	\end{tabular}
%		\end{adjustbox}
%	\caption{Average number of explanation steps per config}
%	\label{tab:config-number-explanation-steps-avg-max}
%\end{table}



\emilio{\paragraph{Explanation Sequence} To identify the effect on the quality of the generated explanation sequences, we report statistics on the generated sequences in Table \ref{tab:config-number-explanation-steps-avg-max} for each configuration. The notation used in the table, is defined as follows: `avg. $\#$ expl.' is the average number of explanations in the explanation sequence; `max. expl. size' and `avg. expl. size' are respectively the maximum and average size of an explanation; avg. $\#$ lits derived is the average number of literals derived in an explanation. 
	
	The average number of explanations is not equal for all configurations, the reason for this is that %one of the two OUS implementations timed-out before completing the most difficult puzzle (hence explaining their lower number) and
	the quality of the MUS-based explanations is worse. Worse explanations (that make more assumptions) often explain more literals at once and hence the total number of explanation steps is lower. The last 2 lines of table \ref{tab:config-number-explanation-steps-avg-max} support the validity of this statement. }

\paragraph{Domain-specific \grow} The core computation\emilio{al} complexity for generating explanations not only lies in the calls to the optimal hitting set solver, but more importantly in computing \emph{high quality} satisfiable subsets. This requires a trade-off between \emph{efficiency} of the \grow strategy and \emph{quality} of the produced satisfiable subset.

\begin{figure}[ht]
  \centering
  \includegraphics[width=\columnwidth]{figures_post_paper/cumul_grow_avg_time_lits_derived.pdf}
  \caption{Q2 - Explanation specific \grow strategies.}
  \label{fig:grow_strategies}
\end{figure}
% \tias{make 'Timeout' top entry of legend, sort legend according to ordering in graph}


\deleted{Thus, to answer \textbf{Q2}, we depict in Figure \ref{fig:grow_strategies} the cumulative time of different \grow strategies to generate the explanation sequence of all puzzles combined.}\emilio{Thus, to answer \textbf{Q2}, we depict in Figure \ref{fig:grow_strategies} the average cumulative time of different \grow strategies to derive n-literals for all puzzles combined.} The configurations are the following: \emph{Full} refers to using the full unsatisfiable formula $\mathcal{F}$ given with every call of \comus while \emph{Actual} refers to using only the constraints that hold in the actual interpretation (Section~\ref{sec:ocusEx}); \emph{Greedy} to growing using a sat-based refinement heuristic; the weighing scheme is either with uniform weights (\emph{unif}), with the original weights (\emph{pos}) or with the inverse of the original costs (\emph{inv}) when using the \maxsat solver (\emph{Max}).

As conjectured in \cref{sec:ocusEx}, the \maxsat-\grow on the actual interpretation \emilio{$\m{I}$} indeed performs best. When repeating this experiment on other variants of the implementation (e.g., disabling constrainedness) the pattern was observed.
% Notice in Figure~\ref{fig:grow_strategies} that only the \maxsat solver on actual variables and with uniform weights is able to complete the generation of a sequence for all puzzles. Furthermore, giving the \maxsat constraints on $\mathcal{F}$ drastically increases the execution time and, even within the best weighing scheme (\emph{unif}), it takes less than 10 steps\tias{how do you conclude this? non-obvious...} on average for every puzzle to time out.
The greedy strategy provides some improvement on the cumulative runtime, with only marginal gains when the actual interpretation is used.

\deleted{
We also observed (see \cref{tab:percentage_exec}) that in each of our configurations, the majority of the time is spent on computing hitting sets. As such, it makes sense to optimize the \grow procedure to provide results that are as good as possible and limit the choices of the hitting set solver as much as possible. 
Slightly surprising, performing a \maxsat-\grow with the full interpretation takes \emph{less} time than on the actual interpretation. The gains of using the actual interpretation seems to come most from the fact that it generates better  sets to hit. 
}
% \tias{TODO: explain table or cut}
%\begin{table}[t]
%  \centering
%  \resizebox{0.82\textwidth}{!}{\begin{minipage}{\textwidth}
%  		    \begin{tabular}{r|c|c|c|c|c}
%      %\hline
%      config &    $\#$HS &   $\%$HS &  $\%$SAT &  $\%$\grow & Time [s] \\
%      \hline
%      Greedy-SAT-Actual &   8679 &  89.0 &   0.0 &    9.0 &      66782 \\
%      Greedy-SAT-Full &   8787 &  88.0 &   0.0 &    10.0 &      66297 \\
%      Max-Actual-Unif &   6831 &  81.0 &   0.0 &   19.0 &       5449 \\
%      Max-Full-Unif &  17280 &  85.0 &   0.0 &   15.0 &      72000 \\
%      %\hline
%    \end{tabular}
%  \end{minipage}}
%  \caption{Number of hitting sets and total time (in seconds) spent in core parts of \comus algorithm across all puzzles.}
%  \label{tab:percentage_exec}
%\end{table}

\begin{figure}[ht]
  \centering
  \includegraphics[width=\columnwidth]{figures_post_paper/cumul_incr_avg_time_lits_derived.pdf}
  \caption{Q3 - Cumulative runtime evolution enhancements on incrementality and constrainedness.}
  \label{fig:incrementality_constraindness}
\end{figure}
% \tias{make 'Timeout' top entry of legend}

\paragraph{Constrainedness and incrementality}
To answer \textbf{Q3} and \textbf{Q4}, we compare the effect of constrainedness in the search for explanations (C), and incrementality, \emilio{i.e., reusing satisfiable subsets between \omus calls (+SS caching) and keeping the hitting set solver warm for \comus (+Incr.HS).}
% 
Figure \ref{fig:incrementality_constraindness} shows the results of the different variants, including the MUS-based variant discussed above. 
\emilio{In this figure, the configurations are compared in a similar fashion to figure \ref{fig:grow_strategies}.}
\deleted{In this figure, the total number of explanation steps is not equal for all configurations, the reason for this is that the two OUS implementations timed-out before completing two of the puzzles (hence explaining their lower number) and the quality of the MUS-based explanations is worse, as observed in \textbf{Q1}. Worse explanations (that make more assumptions) often explain more literals at once and hence the total number of explanation steps is lower.}
%We choose to answers \textbf{Q3} and \textbf{Q4} using the cactus plot of Figure \ref{fig:incrementality_constraindness} with the same logarithmic scale for consistency. Furthermore, both incrementality and constraindness go hand in hand.
All O(C)US variants use Max-Actual-Unif as \grow strategy, after we experimentally verified that this result of \textbf{Q2} also applied to the other OUS variants. %apply as well to all \mm{\call{O(C)US(+I)}} configurations illustrated on Figure \ref{fig:incrementality_constraindness}.

\paragraph{Constrainedness}
When comparing all five configurations, we see that the non-constrained optimal implementations cannot keep up with the plain MUS based implementation, as is to be expected since a harder problem is tackled. 
However, when adding constrainedness, by switching to Algorithm \ref{alg:oneStepOCUS}, we effectively reduce the number of calls to the \omus solver.
The constrainedness enhancement as shown in Figure \ref{fig:incrementality_constraindness} proves the importance of our approach.
The explanation generation times for a whole sequence are on close to and sometimes even outperform  the MUS approach, while still solving the harder problem of unsatisfiable subset optimization. 

% Note that, for the same given puzzles, the number of explanations for MUS is lower COMUS, which supports the claim that the unsat subsets are of high cost as presented in \hyperref[paragraph:explanationquality]{explanation quality}.


\paragraph{Incrementality}
% The naive, but optimal approach of pure 'OUS' requires notably more time than the non-optimal MUS-based algorithm. This observation demonstrates that the problem complexity increases when using an optimality criterion. We observe with 'OUS+I' that reusing information throughout the calls to the OUS solver burdens the \onestep algorithm with extra processing in order to keep track of all generated satisfiable subsets.
To our surprise, we noticed that \emilio{both in the constrained and non-constrained OUS setting, incrementality had no noticeable effect.
	Even more so, naively restarting \textit{OUS} and \textit{OCUS} at every explanation step, without tracking the satisfiable subsets, provides improvements on the more difficult puzzles.
	One potential explanation is when a branch-and-bound-based MIP solver such as Gurobi is kept warm, the solver will start in a search space close to the previous solution. However there is no immediate guarantee that the new best explanation is closely related to the previous one.}

% 
\deleted{When combining constrainedness with incrementality, on the other hand, we see that this incrementality yields important improvements. One potential explanation for this behaviour is the fact that, as discussed above, in the OCUS+I implementation, we only initialize a MIP and reuse it  across the different \onestep calls of algorithm \ref{alg:oneStepOCUS}.
By keeping the MIP solver warm, we take advantage of its efficiency to handle the many satisfiable subsets which are represented under the form of constraints on the variables of the problem specification.
Figure \ref{fig:incrementality_constraindness} demonstrates that contrary to \emph{OUS+I}, incrementality in \textbf{\comus+I} significantly decreases the overall computation time. }

\ignore{
\begin{table}[t]
	\centering
	\resizebox{0.82\textwidth}{!}{\begin{minipage}{\textwidth}
			\begin{tabular}{r|c|c|c|c|c}
				%\hline
				config &    $\#$HS &   $\%$HS &  $\%$SAT &  $\%$\grow & Time [s] \\
				\hline
  Greedy-Sat-Actual &  80519 &  99.6 &   0.2 &    0.1 &    67753 \\
Greedy-Sat-Full &  58791 &  99.7 &   0.1 &    0.1 &    69990 \\
Max-Actual-Unif &  53728 &  59.1 &   0.8 &   37.3 &     5199 \\
Max-Full-Unif &  32604 &  57.3 &   0.0 &   42.3 &    69900 \\
				%\hline
			\end{tabular}
	\end{minipage}}
	\caption{Number of hitting sets and total time (in seconds) spent in core parts of \emilio{not incremental} \comus algorithm across all puzzles.}
	\label{tab:percentage_exec}
\end{table}
}
\begin{table}[!h]
	\centering
	\begin{adjustbox}{max width=\columnwidth}
\begin{tabular}{c|c|c|c|c|c|c|c|c}
	Time [s] &\textbf{0-1} & \textbf{1-2} & \textbf{2-5} & \textbf{5-10} & \textbf{10-20} & \textbf{20-50} & \textbf{50-100} & \textbf{100+}\\
	\hline
	\rule{0pt}{2ex} $\# expl$ &565 & 287 & 193 & 61 & 46 & 19 & 10 & 10\\

\end{tabular}
\end{adjustbox}
	\caption{Frequency of explanation-generation-time in the explanation sequence.}
	\label{tab:explanation-time}
\end{table}
%\begin{figure}[ht]
%	\centering
%	\includegraphics[width=\columnwidth]{figures_post_paper/explanation_times.pdf}
%	\caption{Distribution of the explanation generation time in the generated sequence.}
%	\label{fig:explanation_time}
%\end{figure}

%\tias{Try Fig 5 with log-y axis?}
\paragraph{Interactivity} \emilio{To analyze how the explanations perform in an interactive context, we represent the frequency of the explanation-generation times in table \ref{tab:explanation-time}. We see that a major part of the explanations take less than 2 to 5 seconds to be generated. Even though, the difficult explanation steps can take more time ranging from 5 to 100+ seconds to be generated, it is only a small fraction of all generated explanations.}



%\todo{Add Discussion on the time to find optimal explanations}
% \resizebox{3cm}{!}{
%   \begin{minipage}
%     \begin{table}
%       \centering
%       \begin{tabular}{|r|c|c|c|c|c|}
%         \hline
%         config &    $\#$HS &   $\%$HS &  $\%$SAT &  $\%$Grow &  OCUS [s] \\
%         \hline
%         Greedy-Actual &   8679 &  89.0 &   9.0 &    0.0 &      66782 \\
%         Greedy-Full &   8787 &  88.0 &   10.0 &    0.0 &      66297 \\
%         Max-Actual-Unif &   6831 &  81.0 &   0.0 &   19.0 &       5449 \\
%         Max-Full-Unif &  17280 &  85.0 &   0.0 &   15.0 &      72000 \\
%         \hline
%       \end{tabular}
%       \caption{Time spent in core parts of \comus algorithm}
%     \end{table}
%   \end{minipage}
%   }
  

% \begin{figure}[ht]
%   \centering
%   \includegraphics[width=\columnwidth]{figures/rq4.pdf}
%   \caption{Q4 - Grow strategies until timeout or full explanation sequence generation}
%   \label{fig:rq4}
% \end{figure}


% \begin{table*}[]
%     \centering
%     \caption{Execution time generic grow version}
%     \begin{tabular}{|r||c|c|c|c|c|}
%     \hline
%       p &             sat &          subset &      maxsat\_pos &     maxsat\_inv &    maxsat\_unif \\
%     \hline
%      6 &  15.25 - [18]  &  17.46 - [33]  &  48.41 - [9]  &  14.24 - [9]  &  14.71 - [16]  \\
%      8 &  38.95 - [6]  &  21.31 - [10]  &  1558.78 - [6]  &  14.26 - [6]  &  26.47 - [6]  \\
%      p &  14.09 - [9]  &  19.96 - [10]  &  646.94 - [9]  &  6.66 - [1]  &  9.84 - [4]  \\
%      2 &  18.6 - [9]  &  10.12 - [15]  &  171.94 - [9]  &  14.62 - [7]  &  16.54 - [7]  \\
%      7 &  10.24 - [7]  &  17.52 - [8]  &  1542.48 - [1]  &  13.03 - [1]  &  12.31 - [4]  \\
%      1 &  26.25 - [13]  &  20.52 - [13]  &  333.4 - [13]  &  13.28 - [12]  &  16.49 - [13]  \\
%      4 &  40.52 - [101]  &  22.86 - [108]  &  360.07 - [16]  &  14.18 - [11]  &  16.59 - [20]  \\
%      10 &  13.75 - [8]  &  26.72 - [8]  &  487.54 - [2]  &  17.88 - [4]  &  13.89 - [4]  \\
%      3 &  22.58 - [21]  &  19.57 - [10]  &  319.77 - [10]  &  13.49 - [9]  &  14.92 - [9]  \\
%     \hline
%     \end{tabular}
% \end{table*}





\ignore{\color{OliveGreen} old results to be removed
\begin{table}[ht]
  \centering
  \begin{tabular}{r||c|c|c|c|c}
    % \begin{tabular}{|r||c|c|c|c|c|c|}
      % \hline
      \textbf{p} & \textbf{MUS} & \textbf{OUS}  & \textbf{OUS+I} & \textbf{\comus} & \textbf{\comus+I} \\
      \hline
      1 &       569 &         4114 &     4727 &           803 &         \textbf{299} \\
      2 &       438 &         3834 &     3972 &           607 &         \textbf{238} \\
      3 &       477 &         4220 &     4938 &           932 &         \textbf{607} \\
      4 &       624 &         3508 &     4820 &           388 &          \textbf{97} \\
      5 &      3382 &         Timeout &     Timeout &          3556 &        \textbf{1537} \\
      6 &       568 &         3849 &     3854 &           498 &         \textbf{155} \\
      7 &       372 &         4411 &     4380 &           685 &         \textbf{414} \\
      8 &       474 &         4679 &     5552 &           669 &         \textbf{448} \\
      9 &       766 &         Timeout &     Timeout &          2383 &        \textbf{1135} \\
      p &       224 &         2601 &     2528 &           651 &         \textbf{537} \\
      % \hline
    \end{tabular}
    \caption{Computation time (s) compared between executions}
    \label{table:computationTime}
  \end{table}
}

\ignore{
We now experimentally validate the the performance of the different versions of our algorithms for explaining satisfiable constraint satisfaction problems.

We consider the following benchmarks: CNF instances from the SATLIB problems Benchmark \cite{hoos2000satlib} and a CNF encoding of the logic grid puzzle ``Origin'' of \cite{ecai/BogaertsGCG20}. All code was implemented in Python on top of %CPpy~\footnote{} and
PySAT.\footnote{\url{https://pysathq.github.io}} The MIP solver used is Gurobi 9.0 and when a (Max)SAT solver is used it is RC2 as bundled with PySAT. Experiments were run on a Intel(R) Xeon(R) CPU E3-1225 with 4 cores and 32 Gb memory, running linux 4.15.0.

Based on the theoretical findings of the previous sections, we aim to answer the following research questions:
\begin{compactdesc}
\item[RQ1] what is the effect of postponing optimal hitting set computation, of incremental OUS solving and of pre-seeding \satsets when solving multiple variants of the same problem?
\item[RQ2] how do the different variants of \omus perform when explaining an elaborate constraint satisfaction problem?
\ignore{
\item[RQ3] how do the sequences found when using (constrained) \omus search compare to those found using a heuristic MUS approach?
}
\end{compactdesc}


\paragraph{RQ1}
To answer the first research question, we use 10 CNF instances from the SATLIB Benchmark and randomly choose 10 literals that are entailed by the CNF. For each variant of the algorithm, we compute the OUS of the same 10 literals in the same order within a total time limit of 10 minutes. 
We compare the following enhancements options to the basic \omus algorithm: postponing optimization (+P), incrementality by reusing satisfiable subsets between \omus calls (+I), and pre-seeding $\satsets$ as described in Section~\ref{sec:incremental} (+W). Options can be combined, for example {\omus}+IPW characterises running the \omus algorithm postponing the optimization phase, with incrementality between the successive calls, and warm starting (pre-seeding) with satisfiable subsets of the original CNF formula.
%The executions are set to timeout after 10 minutes, a limit fixed based on the results of experiment 2.

% \rule{\textwidth}{10pt}

% \begin{table*}[t!]
%     \centering
%     \begin{tabular}{|c|c|c|c|c|c|c|c|c|}
%         \hline
%         % p    & nv& nc&           \omus &      {\omus}+Incr &      {\omus}+Post &  {\omus}+Incr+Warm &   {\omus}+Incr+Post & {\omus}+Incr+Post+Warm \\
%         p    & nv& nc&           \omus &      {\omus}+I &      {\omus}+P &  {\omus}+IW &   {\omus}+IP & {\omus}+IPW \\
%         \hline
%         aim-50-1\_6-yes1-4 & 50& 80&   0.88 s  &   0.38 s  &   0.37 s  &   0.81 s  &    0.65 s  &      \textbf{0.33} s  \\
%         par8-2 & 350 & 1157 &   122.42 s  &  94.07 s  &  96.84 s  &  120.55 s  &  126.15 s  &     \textbf{87.31} s  \\
%         zebra\_v155\_c1135 & 155& 1135&   130.38 s  &  87.97 s  &  84.75 s  &  104.7 s  &  124.48 s  &     \textbf{80.92 s}  \\
%         \hline
%         \end{tabular}
%         \caption{Execution time of the \omus variants for deriving 10 literals evaluated on CNF instances.}
%         \label{table:experiment1}
% \end{table*}

% \begin{table*}[t!]
%     \centering
%     \begin{tabular}{|c|c|c|c|c|c|c|c|c|}
%         \hline
%         % p    & nv& nc&           \omus &      {\omus}+Incr &      {\omus}+Post &  {\omus}+Incr+Warm &   {\omus}+Incr+Post & {\omus}+Incr+Post+Warm \\
%         p    & nv& nc&           \omus &      {\omus}+I &      {\omus}+P &  {\omus}+IW &   {\omus}+IP & {\omus}+IPW \\
%         \hline
%         % 1 & 50& 80&   0.88 s  &   0.38 s  &   0.37 s  &   0.81 s  &    0.65 s  &      \textbf{0.33} s  \\
%         % 2 & 350 & 1157 &   122.42 s  &  94.07 s  &  96.84 s  &  120.55 s  &  126.15 s  &     \textbf{87.31} s  \\
%         % 3 & 155& 1135&   130.38 s  &  87.97 s  &  84.75 s  &  104.7 s  &  124.48 s  &     \textbf{80.92 s}  \\
%         par8-5  & 350  & 1171&      --- &     --- &     --- &     --- &      --- &        --- \\
%         par16-1 & 317 & 1264           &      --- &  --- &  --- &     --- &      --- &     --- \\
%         par16-2& 349 & 1392         &      --- &     --- &     --- &     --- &      --- &        --- \\
%         par16-3 & 334 & 1332        &      --- &     --- &     --- &     --- &      --- &        --- \\
%         par16-4-c & 324 & 1292        &      --- &     --- &     --- &     --- &      --- &        --- \\
%         par16-4  & 1015 & 3324        &      --- &     --- &     --- &     --- &      --- &        --- \\
%         hanoi4  & 718 & 4934       &      1 &     1 &     1 &     1 &      1 &        1 \\
%         \hline
%         \end{tabular}
%         \caption{Number of decision variables explained by the variants of the \omus for timed-out CNF instances.}
%         \label{table:experiment1}
% \end{table*}

% \begin{table*}[t!]
%     \centering
%     \begin{tabular}{c|cccc|cccccc}
%         % \hline
%         p &  time [s] &  \#steps &   $\overline{cost}$ & max(cost) &    1 bij &  1trans &  1 clue & 1 clue+i & 1 mult-i & mult-c. \\
%         \hline
%         1 &  1287.27 &     115 &     25.87  &    25.87  &  31.83\% &  50.57\% &  1.09\% &    16.52 \% &     0\% &    0.0\% \\
%         % \hline
%         \end{tabular}
%         \caption{Puzzle Properties, execution statistics and explanation sequence composition for the origin puzzle.}
%         \label{table:experiment3}
% \end{table*}

% ------------------------------- EMILIO LOCK -----------------------------------------
The results can be seen in Table \ref{table:experiment1} and can be summarized as follows: p, nv and nc represent the instance name, the number of variables and the number of clauses respectively. 
Only for instances aim-50-1\_6-yes1-4, par8-2.cnf and zebra\_v155\_c1135.cnf, is the algorithm able to complete the search for OUSs on the 10 decision variables within the required time constraint of 10 minutes.
For these instances, the overall winner is \emph{{\omus}+IPW}. 
All variants time out on the larger instances (par8-5, par16-1, par16-2, par16-3, par16-4-c, par16-4, hanoi4) before finding the OUSs for all 10 decision variables. For instance par8-5, all variants are able to find 6 out of the 10 variables. On all instances that timed-out, \emph{{\omus}+IPW} remains the fastest. Similar results are observed for the remaining instances for all variants par16-* and hanoi4 with the \omus found for only 1 variable.


A further analysis of the overall execution times highlights that much time is spent in the \grow procedure, for which we start from the partial assignment found by the SAT check and use the RC2 MaxSAT solver to complete it. 
We reran the same experiments with a greedy \grow algorithm instead and observed that \omus is not even able to finish for zebra\_v155\_c1135 and all runtimes increase considerably.
Furthermore, we see that in this case postponing the MIP call effectively redistributes 50\% of the computational load to growing \satsets and the remaining 50 \% are evenly distributed between (i) the SAT solver, (ii) the MIP solver, and (iii) the the greedy and incremental hitting set heuristics. Hence, while the portion of time spent growing satisfiable subsets is reduced, much more iterations are needed to find the optimal OUSs. %runtime of grow is decreased, the number of 
% A further analysis of the overall execution times highlights that the main bottleneck of the algorithm is the time spent WE NEED TEXT HERE AFTER TEH RUNS ARE READY\todo{emilio}

From this experiment we conclude that in the short time limit provided, the best configuration for computing multiple related OUS's is \emph{{\omus}+IPW}, taking advantage of the repeated calls to the OUS algorithm, thus reusing the computed \satsets.
% ------------------------------- EMILIO LOCK -----------------------------------------

% \begin{table*}[h!]
%     \begin{tabular}{|c|c|c|c|c|c|c|c|c|}
%         \hline
%         % p    & nv& nc&           \omus &      {\omus}+Incr &      {\omus}+Post &  {\omus}+Incr+Warm &   {\omus}+Incr+Post & {\omus}+Incr+Post+Warm \\
%         p    & nv& nc&           \omus &      {\omus}+I &      {\omus}+P &  {\omus}+IW &   {\omus}+IP & {\omus}+IPW \\
%         \hline
%         1 & 50& 80&   0.88 s  &   0.38 s  &   0.27 s  &   0.81 s  &    0.65 s  &      0.33 s  \\
%         2 & 350 & 1157 &   22.42 s  &  14.07 s  &  76.84 s  &  20.55 s  &  126.15 s  &     87.31 s  \\
%         3 & 155& 1135&   130.38 s  &  87.97 s  &  64.75 s  &  104.7 s  &  124.48 s  &     80.92 s  \\
%         4..10 & x $\cdot$ $10^2$ & x $\cdot$ $10^3$           &      --- &     --- &   --- &  --- &   --- &     --- \\
%         % 5 & 317 & 1264           &      --- &  --- &  --- &     --- &      --- &     --- \\
%         % 6 & 324 & 1292        &      --- &     --- &     --- &     --- &      --- &        --- \\
%         % 7 & 334 & 1332        &      --- &     --- &     --- &     --- &      --- &        --- \\
%         % 8& 349 & 1392         &      --- &     --- &     --- &     --- &      --- &        --- \\
%         % 9  & 1015 & 3324        &      --- &     --- &     --- &     --- &      --- &        --- \\
%         % 10  & 718 & 4934       &      --- &     --- &     --- &     --- &      --- &        --- \\
%         \hline
%         \end{tabular}
%         \caption{Comparison of \omus variants evaluated on CNF instances.}
%         \label{table:experiment1}
% \end{table*}

% \begin{table*}
%     \begin{tabular}{|c|c|c|c|c|c|c|}
%         \hline
%         p                  &           \omus &      {\omus}+Incr &      {\omus}+Post &  {\omus}+Incr+Warm &   {\omus}+Incr+Post & {\omus}+Incr+Post+Warm \\
%         \hline
%         1 &    0.88 s | 10 &   0.38 s | 10 &   0.27 s | 10 &   0.81 s | 10 &    0.65 s | 10 &      0.33 s | 10 \\
%         2            &   22.42 s | 10 &  14.07 s | 10 &  76.84 s | 10 &  20.55 s | 10 &  126.15 s | 10 &     87.31 s | 10 \\
%         3  &  130.38 s | 10 &  87.97 s | 10 &  64.75 s | 10 &  154.7 s | 10 &  124.48 s | 10 &     80.92 s | 10 \\
%         4            &      600 s | 1 &  600 s | 2 &  600 s | 1 &     600 s | 1 &      600 s | 1 &     600 s | 1 \\
%         5         &      600 s | 1 &     600 s | 1 &     600 s | 1 &     600 s | 1 &      600 s | 1 &        600 s | 1 \\
%         6         &      600 s | 1 &     600 s | 1 &     600 s | 1 &     600 s | 1 &      600 s | 1 &        600 s | 1 \\
%         7         &      600 s | 1 &     600 s | 1 &     600 s | 1 &     600 s | 1 &      600 s | 1 &        600 s | 1 \\
%         8          &      600 s | 6 &     600 s | 6 &     600 s | 2 &     600 s | 6 &      600 s | 2 &        600 s | 2 \\
%         9         &      600 s | 1 &     600 s | 1 &     600 s | 1 &     600 s | 1 &      600 s | 1 &        600 s | 1 \\
%         10            &      600 s | 6 &     600 s | 6 &   600 s | 6 &  600 s | 6 &   600 s | 6 &     600 s | 6 \\
%         \hline
%         \end{tabular}
%         \caption{Comparison of \omus variants evaluated on CNF instances.}
%         \label{table:experiment1}
% \end{table*}


\paragraph{RQ2}
The second research question is: how do the different variants perform when explaining an elaborate constraint satisfaction problem? For this, we tested the complete generation of an explanation sequence. 
In this comparison, we expect that constrained versions of our algorithm perform best as they will allow performing an entire step of the explanations in a single call.
For this reason, we only include different variations on the constrained configuration and the single best variant of non-constrained algorithms found in the previous experiment. 

% \begin{figure}[t]
%     \centering
%     \includegraphics[width=\columnwidth]{figures/omusConstrCumulative.png}
%     \caption{Experiment 2}
%     \label{fig:exp2}
% \end{figure}

We generate the explanation sequence as far as possible within a time limit of one hour. 
The results for the  ``origin'' puzzle is shown in Figure~\ref{fig:exp2}.
It shows the number of literals explained on the X-axis, and the cumulative time taken on the Y-axis. 
Only three configurations find the full explanation sequence (note that there can be multiple optimal sequences with a different length, which explains the difference in length between the configurations).
We  see that the best non-constrained implementation is unable to explain all of the literals within the time limit; especially around step 95 there is a big jump in runtime. The vanilla constrained-OUS approach is not able to finish in time either, with big jumps in time on specific (large and costly) explanation steps.

When combining constrained-OUS with either pre-seeding, post-poned optimisation or both, then our approach is able to fully explain the solution. Best results are obtained with constrained-OUS with just pre-seeding at the beginning. The post-poned optimisation in this case may spent a lot of time generating MCSs that are not or little relevant to the constrained OUSs we are seeking. 

\paragraph{Concluding notes}
While a direct comparison of the runtime needed to find an explanation sequence of our tool versus the one of \citet{ecai/BogaertsGCG20} %built on the IDP system \cite{WarrenBook/DeCatBBD14} 
would shed more light on the performance impact, we can not do a fair comparison as the solvers and hardware used are different.

However, the authors reported that explaining a single puzzle easily takes one to two hours due to the many MUS calls. In contrast, Figure~\ref{fig:exp2} shows that three of our constrained-OMUS approaches fully explain a puzzle one of their larger puzzles in 20 to 30 minutes.
Furthermore, our algorithms guarantee that each explanation step is \emph{optimal} with respect to $f$. As such we know that the generated sequences are at least as good for the cost function provided.

% do not include such an extensive experiment here since it would not be a fair comparison of the underlying algorithms. Indeed, different solving technology is used for finding MUSs, and satisfying assignments. 
%We can, however, give an indication of the speed by mentioning that in some preliminary experiments (e.g., on the origin puzzle), using constrained \omus calls, we can find the entire sequence in roughly 20 minutes, whereas the IDP-based tool takes over two hours.


% Finally, for \textbf{RQ3} \emilio{Bart: concluding phrases on expl. generation}
%  we compare the sequence found by our proposed method with the sequence reported on in~\cite{ecai/BogaertsGCG20} for the origin puzzle (puzzle 1). 
% The explanation sequence for the puzzle is generated using \omus Constr with pre-seeding and according to the same cost function as Bogaerts et al.~\cite{ecai/BogaertsGCG20}. We report statistics relating to the explanation generation in table~\ref{table:experiment3}.
% Evidently, one of the most important observations is the speed-up provided by \omus Constr. 
% As a matter of fact, the sequence is generated in a bit more than 21 minutes compared to a few (2-3) hours in~\cite{ecai/BogaertsGCG20}, meaning that \omus Constr is 8-10x faster, while also finding the optimal explanations in each step.
% Table \ref{table:experiment3} also reports that the explanation sequence has become easier to understand: the average cost is slightly lower and so is $max(cost)$, the cost of the most difficult explanation in the puzzle. 


% \begin{table*}
%     \begin{tabular}{ccc|ccc|cccccc}
%         % \hline
%         types &  $|dom|$ &  $|grid|$ &  time [s] &  \#steps &    cost &    1 bij &  1trans &  1 clue & 1 clue+i & 1 mult-i & mult-c. \\
%         \hline
%         4 &      5 &     150 &  1287.27 &     115 &        25.87 &  27.83\% &  49.57\% &  6.09\% &    11.3\% &     5.22\% &    0.0\% \\
%         % \hline
%         \end{tabular}
%         \caption{Puzzle Properties, execution statistics and explanation sequence composition for the origin puzzle.}
%         \label{table:experiment3}
% \end{table*}

%
%\begin{figure*}[ht]
%    \centering
%    \includegraphics[width=\columnwidth]{figures/omusNonConstrCumulative.png}
%    \caption{}
%    \label{}
%\end{figure*}

% \begin{figure}[ht]
%     \centering
%     \includegraphics[width=\columnwidth]{figures/explanation_cost.png}
%     \caption{}
%     \label{}
% \end{figure}

}


\section{Related Work}\label{sec:related}
% !TeX root = ./main.tex

\begin{itemize}
    \item XAIP
    \item Smallest MUS (QMaxSAT, Ignatiev)
    \item MaxHS-nonOPT
\end{itemize}

\section{Conclusion, Challenges and Future work}\label{sec:conclusion}
% !TeX root = ./main.tex

Summarised, while \cite{ecai/BogaertsGCG20} has to explore multiple candidate explanations requiring many calls to MUS extractions methods, the step-wise explanations generation directly benefits from the \comus approach of finding the optimal explanation candidate at any point in the sequence.

There are many trade-offs to be made in the algorithm, including time spent computing optimal or non-optimal hitting sets, how to reuse information and how to grow the satisfiable subsets. While we studied the key algorithmic dimensions of information reuse, a deeper study of alternative grow approaches is needed, including the trade-off between finding smaller satisfiable subsets versus having to compute more hitting sets.

The concept of OUS, incremental OUS and constrained OUS are not limited to explanations of satisfaction problems and we are keen to explore other applications too.

From the explanation point of view, a next challenge is how to compute explanations for optimisation problems, that is, where decisions are made based on search and not just propagation. We believe a constrained OUS algorithm can also play a key part in that. Finally, an open challenge is that of defining appropariate cost functions for generating 'simple' explanations, and how to evaluate what a 'good' explanation sequence is. While we currently find a sequence where each step is optimal with respect to the cost function, we have yet to consider whether it is possible to optimize a cost function over the entire sequence as well.

%\begin{itemize}
%    \item Conclusion on resutls
%    \item Challenges next : extension to XOPT ? 
%    \item Applicability on other problems ?
%    \item Characterizing explanation difficulty
%\end{itemize}

{
\footnotesize
%     
% The file named.bst is a bibliography style file for BibTeX 0.99c
\bibliographystyle{named}
\bibliography{krrlib,ref} 
}
     


\end{document}


