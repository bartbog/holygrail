\chapter{Een volledige specificatie}
\label{ch:specificatie}

Hoofdstuk~\ref{ch:framework} introduceerde een framework voor semantische analyse. Hoofdstukken~\ref{ch:lexicon}~en~\ref{ch:grammatica} beschreven hoe we dit framework konden gebruiken voor het vertalen van logigrammen naar logica. Hoofdstuk~\ref{ch:types} voegde types toe aan dit framework. Op basis van deze vier hoofdstukken kunnen we nu de zinnen van een logigram omzetten naar zinnen in eerste-orde-logica met types.

In dit hoofdstuk beschrijven we de ontbrekende delen die nodig zijn om een volledige specificatie op te stellen van een logigram. Voor deze specificatie ontbreekt nog een formeel vocabularium en een aantal axioma's die inherent zijn aan logigrammen. Hierbij zullen we ook gebruik maken van type-informatie.

\section{Een formeel vocabularium}
\label{sec:vocabularium}
\paragraph{} We stellen een formeel vocabularium op dat ook getypeerd is. De types worden vertaald naar een subset van de gehele getallen of naar constructed types. Constructed types zijn types die bestaat uit een set van constanten met twee extra axioma's ingebouwd, namelijk het Domain Closure Axiom en het Unique Names Axiom. Onder het Domain Closure Axiom verstaan we dat de set van constanten de enige mogelijke elementen zijn van dat type. Het Unique Names Axiom drukt dan weer uit dat alle constanten verschillend zijn van elkaar.

Als er een substantief is van een bepaald type, dan geeft die een naam aan het type. De resterende types worden gewoon genummerd (\texttt{type1}, \texttt{type2}, ...).

\paragraph{} De meest voorkomende types zijn types die een niet-numeriek domein voorstellen. In dat geval zijn er eigennamen van dat type. De eigennamen vormen de domeinelementen en kunnen vertaald worden naar de constanten van de constructed type. Het is echter ook mogelijk dat er 1 domeinelement ontbreekt. Niet alle domeinelementen hoeven namelijk voor te komen in de zinnen van een logigram. De bekendste logigram is genoemd naar zo'n ontbrekend element. \textit{De Zebrapuzzel} \cite{zebra} is een logigram waar de lezer gevraagd wordt welke persoon een zebra heeft. De zebra komt voor de rest niet voor in het logigram.

De gebruiker moet daarom opgeven hoeveel elementen er in elk domein zitten. Het ontbrekende element wordt dan toegevoegd als een extra constante. Omwille van het Domain Closure Axiom mag deze constante zeker niet ontbreken. Bij de oplossing van het logigram kan er dan bijvoorbeeld \texttt{the\_other\_person} komen te staan. Men weet dan over welke persoon het gaat. Er kan immers maar één element ontbreken. Als er twee elementen zouden ontbreken, dan zou men het onderscheid niet kunnen maken tussen de twee elementen. Er is echter altijd een unieke oplossing.

\paragraph{} Daarnaast is het mogelijk dat een niet-numerieke type niet overeenkomt met een domein van het logigram. Er zijn dan geen eigennamen van dat type. Het is een tussenliggend type om twee andere domeinen te verbinden. Bijvoorbeeld ``tour'' in de zin ``John follows the tour with 54 people''. Er zijn twee echte domeinen in deze zin: de personen (met hun naam) en de groottes van een groep.

Voor zo'n type, introduceren we een constructed type met evenveel constanten als er domeinelementen zijn. Deze constanten spelen geen rol buiten voor het verbinden van de andere types. We zullen dus ook symmetrie-brekende axioma's moeten opleggen aan zo'n types om tot een unieke oplossing te komen.

\paragraph{} Ten slotte kan een type een numeriek domein van het logigram voorstellen. In tegenstelling tot voor een niet-numeriek type kunnen we hier de nodige domeinelementen niet afleiden uit de tekst. We hebben bovendien de exacte subset nodig om tot een unieke oplossing te komen. Het volstaat niet om alle getallen buiten één te kennen. Bovendien kan het zelfs zijn dat er meer dan één getal ontbreekt. Voor numerieke types vragen we dus de domeinelementen aan de gebruiker.

Het is mogelijk dat zo'n numeriek type nog een afgeleid type heeft. Dit gaat dan om een \textit{TypeDiff} uit sectie~\ref{sec:aanpassingen}. Het domein van het afgeleide type kan automatisch berekend worden uit het domein van het type dat een domein voorstelt.

\paragraph{} De predicaten uit het formeel vocabularium worden allemaal geïnduceerd door voorzetsels en transitieve werkwoorden. De types van het predicaat worden bepaald door het type van het woord dat het induceerde. ``lives in'' is bijvoorbeeld een werkwoord dat een mens als onderwerp neemt en een land als lijdend voorwerp. \texttt{lives\_in} zal daarom een predicaat zijn met een eerste argument van type \texttt{human} en tweede argument van type \texttt{country}.

Via types is het bovendien mogelijk om de ontbrekende predicaten uit sectie~\ref{sec:npMissingRelation} (Een naamwoordgroep met onbekende relatie) te achterhalen. We weten het type van $x$ en $y$ en dus het domein. Bovendien is er binnen logigrammen altijd maar één functie tussen twee domeinen, namelijk de bijectie die we zoeken. We zoeken dus een werkwoord of voorzetsel met het juiste type-paar. Neem bijvoorbeeld ``the 2008 graduate...''. Aangezien ``2008'' een jaartal is, ``graduate'' een persoon en ``graduated in'' een werkwoord tussen een persoon en een jaartal kan de naamwoordgroep herschreven worden als ``the graduate who graduated in 2008, ...''.  Indien er (minstens) één bestaat kunnen we het predicaat dat overeenkomt met dat woord kiezen. Dit verhoogt de leesbaarheid van de formele vertaling. Het kan echter zijn dat er zo geen woord bestaat. In dat geval introduceren we een nieuw predicaat met de juiste types.

We gebruiken voor alle predicaten de predicaat-syntax (en dus niet de functionele syntax). De theorie bevat de nodige axioma's om deze predicaten te beperken tot bijecties.

% \section{Het juiste predicaat} */
% \paragraph{} Een tweede inferentie bestaat eruit om ontbrekende predicaten uit sectie~\ref{sec:npMissingRelation} (Een naamwoordgroep met onbekende relatie) te achterhalen. Dit zal gebeuren voor we vertalen naar het formele vocabularium. */

% We weten het type van $x$ en $y$ en dus het domein. Bovendien is er binnen logigrammen altijd maar één functie tussen twee domeinen, namelijk de bijectie die we zoeken. We zoeken dus het werkwoord of voorzetsel met het juiste type-paar. Indien er één bestaat kunnen we het predicaat dat overeenkomt met dat woord kiezen. Dit verhoogt de leesbaarheid van de formele vertaling. Het kan echter zijn dat er zo geen woord bestaat. In dat geval introduceren we een nieuw predicaat met de juiste types. */

\section{De ontbrekende axioma's}
Naast de zinnen van een logigrammen zijn er ook altijd een aantal beperkingen die eigen zijn aan een logigram en die niet expliciet vermeld worden. Daarom bevat een correcte theorie voor logigrammen nog extra axioma's. Er zijn 5 soorten axioma's. Een deel hiervan is gebaseerd op de types van de predicaten.

\begin{itemize}
  \item Elk predicaat is een bijectie. Voor elk predicaat is er dus een extra zin in de vorm van $\forall x \cdot \exists y \cdot pred(x, y) \land \forall y \cdot \exists x \cdot pred(x, y)$ die dit uitdrukken. Bijvoorbeeld voor $lives\_in$: $$\forall x \cdot \exists y \cdot lives\_in(x, y) \land \forall y \cdot \exists x \cdot lives\_in(x, y).$$
  \item Er zijn 3 soorten axioma's die uitdrukken dat er equivalentieklassen bestaan van verschillende domeinelementen die samenhoren. Elke klasse bevat één domeinelement van elk domein. Twee elementen zijn equivalent als ze gelinkt zijn via een predicaat of als ze aan elkaar gelijk zijn (op die manier is aan reflexiviteit voldaan).
    \begin{itemize}
      \item \textbf{Synonymie} Er is exact één bijectie tussen twee domeinen. Twee predicaten met dezelfde types zijn dus altijd synoniemen. $\forall x \forall y \cdot pred_1(x, y) \Leftrightarrow pred_2(x, y)$. M.a.w. er is maar één equivalentierelatie. Bijvoorbeeld voor $from(person, country)$ and $lives\_in(person, country)$: $$\forall x \forall y \cdot lives\_in(x, y) \Leftrightarrow from(x, y)$$
      \item \textbf{Symmetrie} Predicaten met een omgekeerde signatuur stellen elkaars inverse voor $\forall x \forall y \cdot pred_1(x, y) \Leftrightarrow pred_2(y, x)$. Bijvoorbeeld $gave(student, presentation)$ en $given\_by(presentation, student)$: $$\forall x \forall y \cdot gave(x, y) \Leftrightarrow given\_by(y, x)$$
      \item \textbf{Transitiviteit} Ten slotte zijn er nog axioma's om de verschillende bijecties te linken. Deze axioma's zorgen voor de transitiviteit van de equivalentierelatie. Bijvoorbeeld de predicaten $pred_1(t_x, t_y)$, $pred_2(t_x, t_z)$ en $pred_3(t_z, t_y)$ introduceren het axioma $\forall x \forall y \cdot pred_1(x, y) \Leftrightarrow \exists z \cdot pred_2(x, z) \land pred_3(z, y)$. Bijvoorbeeld voor $spoke\_for(student, time)$, $gave(student, presentation)$ en $lasted\_for(presentation, time)$: $$\forall x \forall y \cdot spoke\_for(x, y) \Leftrightarrow \exists z \cdot gave(x, z) \land lasted\_for(z, y).$$
    \end{itemize}
  \item Niet-numerieke types die geen domein voorstellen introduceren een aantal nieuwe constanten (bijvoorbeeld het type van ``tour'' in ``John follows the tour with 54 people''). Deze constanten zijn inwisselbaar. Om een unieke oplossing te bekomen introduceren we een symmetriebrekend axioma dat de constanten van zo'n type linkt aan de domeinelementen van een domein naar keuze. Bijvoorbeeld $$with(TourA, 54) \land with(TourB, 64) \land with(TourC, 74)$$
\end{itemize}

\section{Conclusie}
Een volledige specificatie van een logigram bestaat niet alleen uit de vertaling van de zinnen naar logica. Er is ook nood aan een formeel vocabularium en een aantal extra axioma's die niet expliciet vermeld worden. Beiden kunnen echter opgesteld worden aan de hand van informatie i.v.m. de types van woorden.

Zowel de constructie van het vocabularium als de extra axioma's zijn specifiek voor logigrammen. Zo is het domein van numerieke types bij logigrammen altijd beperkt tot een subset van de gehele getallen. Voor andere specificaties kan zo'n type vertaald worden naar de gehele getallen. Het is ook triviaal om in te zien dat de axioma's specifiek zijn voor logigrammen.
