\section{Evaluation}
A grammar was constructed based on the first 10 logic grid puzzles from Puzzle Baron's Logic Puzzles Volume 3 \cite{logigrammen} and evaluated on the next 10 puzzles. The question arises how many adaptions are necessary to represent these unseen puzzles in the constructed grammar. Another question is whether or not it is possible to deduce the types of a logic grid puzzle or not. Finally, we want to know if all puzzles are represented correctly such that the solution can automatically be derived with the IDP system \cite{IDP}.

It turns out that once modified, the types of all unseen puzzles can be derived automatically and the clues can be translated correctly into logic.

Table~\ref{tbl:resultaten} gives an overview of the different adaptations. In total 65 adaptations were necessary, 30 of which were purely grammatical. They used grammatical structures that didn't appear in the training set. The next 5 adaptations were adaptations similar to those from the training set. There are 19 adaptations due to badly typed words. E.g. a verb ``to order'' that was used to order both food and drinks. The assumption of one type per word wasn't satisfied in that case. There was one adaptation to a proper noun because the same domain element appeared in two different word forms. We assumed there is only one word form per domain element. Finally, there were 10 cases where an extra word was added or removed.

We refer to the full master thesis for a detailed overview of all the adaptations.

\begin{table}[h]
  \centering
  \begin{tabular}{lc}
    \hline
    \textbf{Problem} & \textbf{Count} \\ 
    \hline
    ``the one'' & 15 \\
    Wrong use of copular verb & 6 \\
    Badly structured subordinate clause & 6 \\
    Passive sentence & 1 \\
    Possessive pronoun & 1 \\
    Badly structured noun phrase & 1 \\
    \hline
    Rational numbers & 3 \\
    Superlative & 1 \\
    ``Of the two ...'' & 1 \\
    \hline
    Badly typed verb & 14 \\
    Badly typed noun phrase & 5 \\
    \hline
    Double word form domain element & 1 \\
    \hline
    Redundant word & 7 \\
    Missing word (ellipse) & 3 \\
    \hline
  \end{tabular}
  \caption{An overview of the different adaptations}
  \label{tbl:resultaten}
\end{table}
