% !TeX root = ./main.tex

In the last few years, driven by the increasingly many successes of Artificial Intelligence (AI), there is a growing need for \textbf{eXplainable Artificial Intelligence (XAI)}~\cite{miller2019explanation}.
In the research community, this need manifests itself through the emergence of (interdisciplinary) workshops and conferences on this topic~\cite{xai-ijcai,FAT} and American and European incentives to stimulate research in the area~\cite{gunning2017explainable,hamonrobustness,fetproact}. 
%Also on the \textbf{legislation} side there is increased attention for explainability \cite{regulation2016regulation}. % there is some common misconception about the 'right for an explanation' which is not a right

While the main focus of XAI research has been on explaining black-box machine learning systems \cite{lundberg2017unified,guidotti2018survey,ignatiev2019abduction}, also model-based systems, which are typically considered more transparent, are in need of explanation mechanisms. 
Indeed, by advances in solving methods in research fields such as constraint programming \cite{fai/Rossi06} and SAT \cite{faia/2009-185}, as well as by hardware improvement, such systems now easily consider millions of alternatives in short amounts of time. 
Because of this complexity, the question arises how to generate human-interpretable explanations of the conclusions they make. 
Explanations for model-based systems have been considered mostly for explain \textit{unsatisfiable} problem instances~\cite{junker2001quickxplain}, and have recently seen a rejuvenation in various subdomains of constraint reasoning \cite{fox2017explainable,vcyras2019argumentation,chakraborti2017plan,ecai/BogaertsGCG20}.

Our current work is motivated by a concrete algorithmic need that arose in this context. 
Specifically, the work of \citet{ecai/BogaertsGCG20} shows the need for algorithms that can find optimal MUSs with respect to a given cost function, where the cost function approximates human-understandability of the corresponding explanation step.

The closest related work can be found in the literature on generating or enumerating MUSs \cite{conf/sat/LynceM04}.
Different techniques are employed to find MUSs, including  manipulating resolution proofs produced by SAT solvers \cite{goldberg,DBLP:journals/fmsd/GershmanKS08,DBLP:conf/sat/DershowitzHN06}, incremental solving to enable/disable clauses and branch-and-bound search \cite{DBLP:conf/dac/OhMASM04}, or by BDD-manipulation methods \cite{huang}.
Other methods work by means of translation into a so-called Quantified \maxsat \cite{DBLP:journals/constraints/IgnatievJM16}, a field that combines the expressivity of Quantified Boolean Formulas (QBF) \mycite{QBF} with optimization as known from \maxsat \mycite{DBLP:series/faia/LiM09}, or by exploiting the so-called hitting set duality \cite{ignatiev2015smallest}. 
An \textit{abstract} framework for describing \hitsetbased algorithms, including optimization was developed by \citet{DBLP:conf/kr/SaikkoWJ16}. While our approach can be seen to fit the framework, the terminology is focussed on MaxSAT rather than MUS and would complicate our exposition.
To the best of our knowledge, only few have considered \emph{optimizing} MUSs: the only criterion considered yet is cardinality-minimality \cite{conf/sat/LynceM04,ignatiev2015smallest}. 


\ignore{
Our paper builds on the algorithm of \citet{ignatiev2015smallest}, which fits in a general class of so-called \emph{implicit hitting set algorithms}.
While these algorithms find their root in early work of \citet{ai/Reiter87}, they only really boosted in popularity when applied in the context of \maxsat solving \cite{DBLP:conf/cp/DaviesB11,DBLP:conf/sat/DaviesB13,davies}, where \hitsetbased solvers are often among the best solvers in the competitions. 
%\bart{SOME MORE APPLICATIONS OF IMPLICIT HITTING SET ALGORITHMS?}
}




\tias{for conclusion}
The goal of explaining satisfaction problems fits in wider goal of human-machine decision making, which is related to interactive constraint solving~\cite{putnam2019exploring}. The use of MUSs in this context has been explored by \citet{van2017kb}. For interactivity, finding a single O(C)US fast would be the focus, where incrementality can play a big role.

Notably, MUSs and \hitsetbased algorithms are also investigated in the context of explaining machine learning decisions~\cite{ignatiev2019abduction}. An open direction for future work is to see how optimal MUS search can be of benefit there, and whether constrained properties can further boost the possibilities.


