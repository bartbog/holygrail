Before detailing how we construct a complete specification, we give a very brief introduction to the \idp system. More information can be found in \mycite{idp}. 
\subsection{Preliminaries: the \idp system} 
The \idp system \mycite{idp} is a knowledge base system \mycite{KBS} for a rich extension of first-order logic. 
In practice, when problem-solving with \idp, one typically makes use of the following components: vocabularies, theories, structures, inference methods, and procedures.
A vocabulary is (as in standard first-order logic) a set of symbols. In \idp these symbols are furthermore \emph{typed}. 
A (complete) structure over a vocabulary is an assignment of values (of the right type) to symbols (for instance, a set of tuples to a predicate symbol), this is typically called a variable assignment or candidate solution in CP. A \emph{partial} structure may in addition contain partial information, such as $(1,1)$ is an element of the interpretation of $P$ and $(2,1)$ is not, without fully specifying the interpretation of $P$; in a CP context, it can be seen as the current \textit{domain} of each symbol. 
Partial structures are ordered according to \emph{precision}. Intuitively, $S_1$ is more precise than $S_2$ (notation $S_1\geqp S_2$) if $S_1$ contains at least all information $S_2$ contains. 
A \emph{theory} in \idp is an expression in an extension of first-order logic. It represents a piece of information (for instance, a clue) but does not represent a problem; it can be seen as a (sub)set of constraints in CP. 
In order to solve a problem using this knowledge, we make use of \emph{inference methods} (solvers), which are generic algorithms that tackle a task given some of the aforementioned components. Some commonly used inference methods are:
\begin{itemize}
 \item \textbf{modelexpand(T,S)} This method takes as input a theory and a partial structure $S$, it returns one or more (depending on options) structures that are more precise than $S$ and that satisfy $T$.
%  \item \textbf{entails(T1,T2)} This procedure takes as input two theories and checks whether the first one logically entails the second. 
 \item \textbf{optimalpropagate(T,S)} This inference method takes as input a theory $T$ and a partial structure $S$, it returns the most precise partial structure $S'$ that approximates all models of $T$ that expand $S$ (i.e., such that for each model $M$ of $T$ with $M\geqp S$, it holds that $S'\leqp M$). Thus, $S'$ contains all consequences derivable from $T$ starting from the structure $S$.
 \item \textbf{unsatstructure(T,S,V)} This inference method takes as input a theory $T$ and a structure $S$ and optionally a vocabulary $V$, whereby $T$ has no models more precise than  $S$, that is, the combination of $T$ and $S$ has no solution. It returns structure $S'$ less precise than $S$, but equal to $S$ outside $V$ such that $T$ still has no satisfying solution more precise than $S'$; furthermore the returned structured is minimally precise among such structures. Intuitively, this inference methods finds the reason for the inconsistency: it explains why there are no models of $T$ expanding $S$ by identifying a minimal set of facts in $S$ that cause the lack of satisfying solutions. 
 Internally, this is implemented using unsatisfiable core extraction \cite{conf/sat/LynceM04}. 
\end{itemize}
Finally, \idp provides built-in support for the lua scripting language, where all above components are first-class citizens. Lua procedures are typically used to glue together different calls to different inference methods to solve an actual problem.

\subsection{Step D. From logic to a complete IDP specification} 
In order to build a complete specification of the puzzle from the Discourse Representation Theory(DRT) returned by the typed Blackburn and Bos framework, we compute the interpretation of the different types. 
As mentioned before, the list of entities occurring in the puzzle needs to be given to build the lexicon. If additionally they are also partitioned in types (this information can e.g., be taken from a grid-representation), nothing else needs to be done here. 
If the partitioning of the entities in types is \emph{not} given,
we  use \textbf{type inference} to compute an equivalence relation on the set of proper nouns occurring in the clues (two proper nouns are equivalent if they occur in the same position of a verb/preposition; for instance if ``the Englishman smokes cigarettes'' and ``the person who owns a dog does not smoke cigars'' we derive that cigars and cigarettes are nouns of the same type). It might happen that this does not yield enough information to completely determine the types for two reasons. 
First of all, not all proper nouns might occur in the clues (for instance the zebra in Einsteins famous zebra puzzle). 
However, since the solution of a logic grid puzzle is always unique, there can at most be one such missing entity per type (otherwise by symmetry there would be multiple solutions) and hence, we can then simply add an anonymous element. 
Secondly, there might be a large variation in the verbs used to denote the same relation. In that case, without using knowledge on the partitioning of entities, we cannot decide which entities belong to the same type. Our system then queries the user to ask which verb are -- for the purpose of the puzzle -- synonyms. 

Once the types are completed, we construct the IDP vocabulary containing: all the types and a \textit{relation} for each transitive verb or preposition. For instance if the clues contain a sentence ``John lives in the red house'', then the vocabulary will contain a binary relation $livesIn(\cdot,\cdot)$ with the first argument of type $person$ and the second argument of type $house$. %  \tias{needs example, like livesIn() below}
Additionally, we ensure that there is at least one relation between each two types, even if this relation does not occur in the clues. This is not important for solving the puzzle, but it plays an important role in explaining (more on that follows in the next section). 
% \tias{NOT SURE WHAT TO DO?} \bart{ REMARK: Essentially, by including these relations, we change the \emph{language of learning}, which is known to sometimes allow for much more efficient inference TIAS:CITATION? Here it turns out that not just for speeding up the solver, but also for clarifying the solution process to a human, the language of learning is of crucial importance. IT WOULD BE NICE TO HAVE A SENTENCE LIKE THIS HERE, BUT I NEED A REFERENCE FOR EXPONENTIAL PROOF SIZE REDUCTOIN BY EXTENDING THE LANGUAGE OF LEARNING... }
The interpretation of the types is encoded in \idp by means of a \emph{constructed type}.  A constructed type consist of a set of constants with two extra axioms implied: Domain Closure Axiom (DCA) and Unique Names Axiom (UNA). DCA states that the set of constants are the only possible elements of the domain. UNA states that all constants are different from each other.

% A type that corresponds to a numerical domain is translated to a subset of the natural numbers. In that case, the system asks the user the exact subset as it often cannot be automatically inferred from the clues.

After the vocabulary, we construct IDP theories: 
\begin{itemize}
% \item 
 \item we translate each clue into the \idp language, and
 \item we add implicit constraints present in logic grid puzzles.
\end{itemize}
%The translation of DRT into first-order logic is well-known. We slightly extended it to take types into account and translate into typed first-order logic. 
% \todo{overlap with previous part}
% \begin{itemize}
%  \item using the type information as a sanity check: if a verb (which is translated into a relation) is used twice, then the two occurrences must have the same typing
%  \item type inference: if the grid (the interpretation of the logical types underlying the puzzle) is not given as input, we derive it based on occurrences in the puzzle, e.g., using the fact that if two different entities are used as the subject of the same verb, then they must have the same type,
% %  \item 
% \end{itemize}

%Next, there are some implicit constraints stemming from the fact that this is a logic grid puzzle. 
The implicit constraints are stemming from the following: First of all, our translation might generate multiple relations between two types. For instance if there are clues ``The tea drinker is from France'' and ``The person who owns a dog lives in England'', then the translation will create two relations $\mathit{from}$ and $\mathit{livesIn}$ between persons and countries. This happens regularly since logigram designers tend to vary their vocabulary to keep the puzzles interesting. However, we know that there is only one relation between two types, hence we add a theory containing \emph{synonymy} axioms; for this case concretely: 
\[\forall x \forall y : livesIn(x, y) \Leftrightarrow from(x, y).\]
Similarly, if two relations have an inverse signature, they represent the inverse functions (for instance $\mathit{isOwnedBy}$ and $\mathit{likes}$) in the clues ``The Englishman likes cats'' and ``The dog is owned by the Belgian''). In this case we add constraints of the form
\[\forall x \forall y : likes(x, y) \Leftrightarrow isOwnedBy(y,x).\]
Next, we add axioms that state that each relation between two types is actually a \emph{bijection}, e.g. 
\[\forall x : \#\{y\mid from(x, y)\} = 1.\]\[\forall y : \#\{ x \mid from(x, y)\}=1.\]
Finally, we add \emph{transitivity} axioms that link the different relations. For instance if the dog is kept in the red house and the Englishman lives in the red house, then the Englishman keeps a dog. This kind of axioms is expressed as:
\[
 \forall x \forall y \forall z: keptIn(x,y) \land livesIn(z,y) \limplies keeps(z,x).
\]


%\todo{More on the Usage of types?}


\paragraph{Solving the puzzle}
The conjunction of all the logical theories created in the previous paragraph completely characterizes the constraints underlying a logic grid puzzle.
In order to solve the puzzle, we use \idp's built-in \emph{model expansion} inference, which searches for a solution in a given finite domain. Under the hood, \idp uses \minisatid \mycite{minisatid}, a solver using SAT \mycite{SAT} and CP \mycite{CP} technology, in particular lazy clause generation \mycite{lcg} and conflict-driven clause learning \mycite{cdcl}. In our experience, the solving part is often quite trivial for logic grid puzzles, since they are usually crafted to be solvable using the grid and single clues. 
\conferencepaper{\bart{We agreed to drop this claim here now, right, and possibly insert it again in the conference paper...

However, it deserves to be mentioned that theoretically, arbitrary logic grid puzzles are NP hard.

\todo{do we wish to say something about this: if there does not have to be a solution and it does not have to be unique, then we can get NP hardness. However, if there has to be a unique solution proving hardness is harder}

REDUCTION OF 3SAT TO LOGIC GRID PUZZLE. 
Create a type with elements ``true'' and ``false''. 
Given a 3CNF over \voc. For each atom in \voc create a type with two elements: $p$ and $\lnot p$.
For each clause $p \lor \lnot q \lor r$, add a clue 
``At least one of the following hold: p is associated with true, $q$ is associated with false or $r$ is associated with 
}


}
