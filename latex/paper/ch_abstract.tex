This paper researches the value of the semantical framework of Blackburn and Bos for knowledge representation. Specifically, the framework is used to translate logigrams into logic. %The grammar is manually derived from existing logigrams.

Additionally, we introduce types into the framework. Based on these types a knowledge base system could type check sentences in natural language. Therefor, it could reject sentences that are grammatical but without meaning. In this paper however, type inference is used instead. The framework can infer the domains of a logigram from its sentences in natural language.

% Specifically, we add types to this framework. Based on types we can reject sentences that are grammatical but without meaning.
% We explore the value of types in the framework by translating 

% Deze thesis stelt een aangepassing voor aan het semantische framework van Blackburn
% en Bos [6, 7]. Specifiek passen we dit framework aan voor het oplossen van logigrammen
% door ze te vertalen naar logica. Deze aanpassingen bestaan enerzijds uit het opstellen van
% een lexicon en een grammatica voor logigrammen. Anderzijds voegen we types toe aan
% het framework. Dankzij types kunnen grammaticaal correcte zinnen zonder betekenis
% toch uitgesloten worden. Bovendien kan men op basis van types achtergrondinformatie
% afleiden. Binnen logigrammen kunnen de verschillende domeinen geleerd worden m.b.v.
% types.
