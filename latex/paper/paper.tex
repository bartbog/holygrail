%%*************************************************************************
%% Legal Notice:
%% This code is offered as-is without any warranty either expressed or
%% implied; without even the implied warranty of MERCHANTABILITY or
%% FITNESS FOR A PARTICULAR PURPOSE! 
%% User assumes all risk.
%% In no event shall the IEEE or any contributor to this code be liable for
%% any damages or losses, including, but not limited to, incidental,
%% consequential, or any other damages, resulting from the use or misuse
%% of any information contained here.
%%
%% All comments are the opinions of their respective authors and are not
%% necessarily endorsed by the IEEE.
%%
%% This work is distributed under the LaTeX Project Public License (LPPL)
%% ( http://www.latex-project.org/ ) version 1.3, and may be freely used,
%% distributed and modified. A copy of the LPPL, version 1.3, is included
%% in the base LaTeX documentation of all distributions of LaTeX released
%% 2003/12/01 or later.
%% Retain all contribution notices and credits.
%% ** Modified files should be clearly indicated as such, including  **
%% ** renaming them and changing author support contact information. **
%%*************************************************************************


% *** Authors should verify (and, if needed, correct) their LaTeX system  ***
% *** with the testflow diagnostic prior to trusting their LaTeX platform ***
% *** with production work. The IEEE's font choices and paper sizes can   ***
% *** trigger bugs that do not appear when using other class files.       ***                          ***
% The testflow support page is at:
% http://www.michaelshell.org/tex/testflow/

\documentclass[journal]{IEEEtran}
% Some very useful LaTeX packages include:
% (uncomment the ones you want to load)

% *** CITATION PACKAGES ***
%
%\usepackage{cite}
% cite.sty was written by Donald Arseneau
% V1.6 and later of IEEEtran pre-defines the format of the cite.sty package
% \cite{} output to follow that of the IEEE. Loading the cite package will
% result in citation numbers being automatically sorted and properly
% "compressed/ranged". e.g., [1], [9], [2], [7], [5], [6] without using
% cite.sty will become [1], [2], [5]--[7], [9] using cite.sty. cite.sty's
% \cite will automatically add leading space, if needed. Use cite.sty's
% noadjust option (cite.sty V3.8 and later) if you want to turn this off
% such as if a citation ever needs to be enclosed in parenthesis.
% cite.sty is already installed on most LaTeX systems. Be sure and use
% version 5.0 (2009-03-20) and later if using hyperref.sty.
% The latest version can be obtained at:
% http://www.ctan.org/pkg/cite
% The documentation is contained in the cite.sty file itself.

% *** GRAPHICS RELATED PACKAGES ***
%
\ifCLASSINFOpdf
  % \usepackage[pdftex]{graphicx}
  % declare the path(s) where your graphic files are
  % \graphicspath{{../pdf/}{../jpeg/}}
  % and their extensions so you won't have to specify these with
  % every instance of \includegraphics
  % \DeclareGraphicsExtensions{.pdf,.jpeg,.png}
\else
  % or other class option (dvipsone, dvipdf, if not using dvips). graphicx
  % will default to the driver specified in the system graphics.cfg if no
  % driver is specified.
  % \usepackage[dvips]{graphicx}
  % declare the path(s) where your graphic files are
  % \graphicspath{{../eps/}}
  % and their extensions so you won't have to specify these with
  % every instance of \includegraphics
  % \DeclareGraphicsExtensions{.eps}
\fi
% graphicx was written by David Carlisle and Sebastian Rahtz. It is
% required if you want graphics, photos, etc. graphicx.sty is already
% installed on most LaTeX systems. The latest version and documentation
% can be obtained at: 
% http://www.ctan.org/pkg/graphicx
% Another good source of documentation is "Using Imported Graphics in
% LaTeX2e" by Keith Reckdahl which can be found at:
% http://www.ctan.org/pkg/epslatex
%
% latex, and pdflatex in dvi mode, support graphics in encapsulated
% postscript (.eps) format. pdflatex in pdf mode supports graphics
% in .pdf, .jpeg, .png and .mps (metapost) formats. Users should ensure
% that all non-photo figures use a vector format (.eps, .pdf, .mps) and
% not a bitmapped formats (.jpeg, .png). The IEEE frowns on bitmapped formats
% which can result in "jaggedy"/blurry rendering of lines and letters as
% well as large increases in file sizes.
%
% You can find documentation about the pdfTeX application at:
% http://www.tug.org/applications/pdftex

% *** MATH PACKAGES ***
%
%\usepackage{amsmath}
% A popular package from the American Mathematical Society that provides
% many useful and powerful commands for dealing with mathematics.
%
% Note that the amsmath package sets \interdisplaylinepenalty to 10000
% thus preventing page breaks from occurring within multiline equations. Use:
%\interdisplaylinepenalty=2500
% after loading amsmath to restore such page breaks as IEEEtran.cls normally
% does. amsmath.sty is already installed on most LaTeX systems. The latest
% version and documentation can be obtained at:
% http://www.ctan.org/pkg/amsmath

% *** SPECIALIZED LIST PACKAGES ***
%
%\usepackage{algorithmic}
% algorithmic.sty was written by Peter Williams and Rogerio Brito.
% This package provides an algorithmic environment fo describing algorithms.
% You can use the algorithmic environment in-text or within a figure
% environment to provide for a floating algorithm. Do NOT use the algorithm
% floating environment provided by algorithm.sty (by the same authors) or
% algorithm2e.sty (by Christophe Fiorio) as the IEEE does not use dedicated
% algorithm float types and packages that provide these will not provide
% correct IEEE style captions. The latest version and documentation of
% algorithmic.sty can be obtained at:
% http://www.ctan.org/pkg/algorithms
% Also of interest may be the (relatively newer and more customizable)
% algorithmicx.sty package by Szasz Janos:
% http://www.ctan.org/pkg/algorithmicx

% *** ALIGNMENT PACKAGES ***
%
%\usepackage{array}
% Frank Mittelbach's and David Carlisle's array.sty patches and improves
% the standard LaTeX2e array and tabular environments to provide better
% appearance and additional user controls. As the default LaTeX2e table
% generation code is lacking to the point of almost being broken with
% respect to the quality of the end results, all users are strongly
% advised to use an enhanced (at the very least that provided by array.sty)
% set of table tools. array.sty is already installed on most systems. The
% latest version and documentation can be obtained at:
% http://www.ctan.org/pkg/array

% IEEEtran contains the IEEEeqnarray family of commands that can be used to
% generate multiline equations as well as matrices, tables, etc., of high
% quality.

% *** SUBFIGURE PACKAGES ***
%\ifCLASSOPTIONcompsoc
%  \usepackage[caption=false,font=normalsize,labelfont=sf,textfont=sf]{subfig}
%\else
%  \usepackage[caption=false,font=footnotesize]{subfig}
%\fi
% subfig.sty, written by Steven Douglas Cochran, is the modern replacement
% for subfigure.sty, the latter of which is no longer maintained and is
% incompatible with some LaTeX packages including fixltx2e. However,
% subfig.sty requires and automatically loads Axel Sommerfeldt's caption.sty
% which will override IEEEtran.cls' handling of captions and this will result
% in non-IEEE style figure/table captions. To prevent this problem, be sure
% and invoke subfig.sty's "caption=false" package option (available since
% subfig.sty version 1.3, 2005/06/28) as this is will preserve IEEEtran.cls
% handling of captions.
% Note that the Computer Society format requires a larger sans serif font
% than the serif footnote size font used in traditional IEEE formatting
% and thus the need to invoke different subfig.sty package options depending
% on whether compsoc mode has been enabled.
%
% The latest version and documentation of subfig.sty can be obtained at:
% http://www.ctan.org/pkg/subfig

% *** FLOAT PACKAGES ***
%
%\usepackage{fixltx2e}
% fixltx2e, the successor to the earlier fix2col.sty, was written by
% Frank Mittelbach and David Carlisle. This package corrects a few problems
% in the LaTeX2e kernel, the most notable of which is that in current
% LaTeX2e releases, the ordering of single and double column floats is not
% guaranteed to be preserved. Thus, an unpatched LaTeX2e can allow a
% single column figure to be placed prior to an earlier double column
% figure.
% Be aware that LaTeX2e kernels dated 2015 and later have fixltx2e.sty's
% corrections already built into the system in which case a warning will
% be issued if an attempt is made to load fixltx2e.sty as it is no longer
% needed.
% The latest version and documentation can be found at:
% http://www.ctan.org/pkg/fixltx2e

%\usepackage{stfloats}
% stfloats.sty was written by Sigitas Tolusis. This package gives LaTeX2e
% the ability to do double column floats at the bottom of the page as well
% as the top. (e.g., "\begin{figure*}[!b]" is not normally possible in
% LaTeX2e). It also provides a command:
%\fnbelowfloat
% to enable the placement of footnotes below bottom floats (the standard
% LaTeX2e kernel puts them above bottom floats). This is an invasive package
% which rewrites many portions of the LaTeX2e float routines. It may not work
% with other packages that modify the LaTeX2e float routines. The latest
% version and documentation can be obtained at:
% http://www.ctan.org/pkg/stfloats
% Do not use the stfloats baselinefloat ability as the IEEE does not allow
% \baselineskip to stretch. Authors submitting work to the IEEE should note
% that the IEEE rarely uses double column equations and that authors should try
% to avoid such use. Do not be tempted to use the cuted.sty or midfloat.sty
% packages (also by Sigitas Tolusis) as the IEEE does not format its papers in
% such ways.
% Do not attempt to use stfloats with fixltx2e as they are incompatible.
% Instead, use Morten Hogholm'a dblfloatfix which combines the features
% of both fixltx2e and stfloats:
%
% \usepackage{dblfloatfix}
% The latest version can be found at:
% http://www.ctan.org/pkg/dblfloatfix

%\ifCLASSOPTIONcaptionsoff
%  \usepackage[nomarkers]{endfloat}
% \let\MYoriglatexcaption\caption
% \renewcommand{\caption}[2][\relax]{\MYoriglatexcaption[#2]{#2}}
%\fi
% endfloat.sty was written by James Darrell McCauley, Jeff Goldberg and 
% Axel Sommerfeldt. This package may be useful when used in conjunction with 
% IEEEtran.cls'  captionsoff option. Some IEEE journals/societies require that
% submissions have lists of figures/tables at the end of the paper and that
% figures/tables without any captions are placed on a page by themselves at
% the end of the document. If needed, the draftcls IEEEtran class option or
% \CLASSINPUTbaselinestretch interface can be used to increase the line
% spacing as well. Be sure and use the nomarkers option of endfloat to
% prevent endfloat from "marking" where the figures would have been placed
% in the text. The two hack lines of code above are a slight modification of
% that suggested by in the endfloat docs (section 8.4.1) to ensure that
% the full captions always appear in the list of figures/tables - even if
% the user used the short optional argument of \caption[]{}.
% IEEE papers do not typically make use of \caption[]'s optional argument,
% so this should not be an issue. A similar trick can be used to disable
% captions of packages such as subfig.sty that lack options to turn off
% the subcaptions:
% For subfig.sty:
% \let\MYorigsubfloat\subfloat
% \renewcommand{\subfloat}[2][\relax]{\MYorigsubfloat[]{#2}}
% However, the above trick will not work if both optional arguments of
% the \subfloat command are used. Furthermore, there needs to be a
% description of each subfigure *somewhere* and endfloat does not add
% subfigure captions to its list of figures. Thus, the best approach is to
% avoid the use of subfigure captions (many IEEE journals avoid them anyway)
% and instead reference/explain all the subfigures within the main caption.
% The latest version of endfloat.sty and its documentation can obtained at:
% http://www.ctan.org/pkg/endfloat
%
% The IEEEtran \ifCLASSOPTIONcaptionsoff conditional can also be used
% later in the document, say, to conditionally put the References on a 
% page by themselves.

% *** PDF, URL AND HYPERLINK PACKAGES ***
%
%\usepackage{url}
% url.sty was written by Donald Arseneau. It provides better support for
% handling and breaking URLs. url.sty is already installed on most LaTeX
% systems. The latest version and documentation can be obtained at:
% http://www.ctan.org/pkg/url
% Basically, \url{my_url_here}.

% *** Do not adjust lengths that control margins, column widths, etc. ***
% *** Do not use packages that alter fonts (such as pslatex).         ***
% There should be no need to do such things with IEEEtran.cls V1.6 and later.
% (Unless specifically asked to do so by the journal or conference you plan
% to submit to, of course. )

% correct bad hyphenation here
\hyphenation{op-tical net-works semi-conduc-tor}

\usepackage{../thesis/prologconfig}

\begin{document}
%
% paper title
% Titles are generally capitalized except for words such as a, an, and, as,
% at, but, by, for, in, nor, of, on, or, the, to and up, which are usually
% not capitalized unless they are the first or last word of the title.
% Linebreaks \\ can be used within to get better formatting as desired.
% Do not put math or special symbols in the title.
\title{Automatic Translation of Logigrams into Logic}
%
%
% author names and IEEE memberships
% note positions of commas and nonbreaking spaces ( ~ ) LaTeX will not break
% a structure at a ~ so this keeps an author's name from being broken across
% two lines.
% use \thanks{} to gain access to the first footnote area
% a separate \thanks must be used for each paragraph as LaTeX2e's \thanks
% was not built to handle multiple paragraphs
%

\author{Jens~Claes}
%\thanks{M. Shell was with the Department of Electrical and Computer Engineering, Georgia Institute of Technology, Atlanta, GA, 30332 USA e-mail: (see http://www.michaelshell.org/contact.html).}% <-this % stops a space
%\thanks{J. Doe and J. Doe are with Anonymous University.}% <-this % stops a space
%\thanks{Manuscript received April 19, 2005; revised August 26, 2015.}}

% note the % following the last \IEEEmembership and also \thanks - 
% these prevent an unwanted space from occurring between the last author name
% and the end of the author line. i.e., if you had this:
% 
% \author{....lastname \thanks{...} \thanks{...} }
%                     ^------------^------------^----Do not want these spaces!
%
% a space would be appended to the last name and could cause every name on that
% line to be shifted left slightly. This is one of those "LaTeX things". For
% instance, "\textbf{A} \textbf{B}" will typeset as "A B" not "AB". To get
% "AB" then you have to do: "\textbf{A}\textbf{B}"
% \thanks is no different in this regard, so shield the last } of each \thanks
% that ends a line with a % and do not let a space in before the next \thanks.
% Spaces after \IEEEmembership other than the last one are OK (and needed) as
% you are supposed to have spaces between the names. For what it is worth,
% this is a minor point as most people would not even notice if the said evil
% space somehow managed to creep in.


% The paper headers
%\markboth{Journal of \LaTeX\ Class Files,~Vol.~14, No.~8, August~2015}%
%{Shell \MakeLowercase{\textit{et al.}}: Bare Demo of IEEEtran.cls for IEEE Journals}
% The only time the second header will appear is for the odd numbered pages
% after the title page when using the twoside option.
% 
% *** Note that you probably will NOT want to include the author's ***
% *** name in the headers of peer review papers.                   ***
% You can use \ifCLASSOPTIONpeerreview for conditional compilation here if
% you desire.




% If you want to put a publisher's ID mark on the page you can do it like
% this:
%\IEEEpubid{0000--0000/00\$00.00~\copyright~2015 IEEE}
% Remember, if you use this you must call \IEEEpubidadjcol in the second
% column for its text to clear the IEEEpubid mark.



% use for special paper notices
%\IEEEspecialpapernotice{(Invited Paper)}




% make the title area
\maketitle

% As a general rule, do not put math, special symbols or citations
% in the abstract or keywords.
\begin{abstract}
  This paper presents an application of the semantical framework of Blackburn and Bos to logigrams. We construct a grammar and lexicon adapted for translating logigrams into logic.

Additionally, we introduce types into the framework. Based on these types we could reject sentences that are grammatical but without meaning. In this paper type inference is used to infer the domains of a logigram from the sentences in natural language.

% Specifically, we add types to this framework. Based on types we can reject sentences that are grammatical but without meaning.
% We explore the value of types in the framework by translating 

% Deze thesis stelt een aangepassing voor aan het semantische framework van Blackburn
% en Bos [6, 7]. Specifiek passen we dit framework aan voor het oplossen van logigrammen
% door ze te vertalen naar logica. Deze aanpassingen bestaan enerzijds uit het opstellen van
% een lexicon en een grammatica voor logigrammen. Anderzijds voegen we types toe aan
% het framework. Dankzij types kunnen grammaticaal correcte zinnen zonder betekenis
% toch uitgesloten worden. Bovendien kan men op basis van types achtergrondinformatie
% afleiden. Binnen logigrammen kunnen de verschillende domeinen geleerd worden m.b.v.
% types.

\end{abstract}

% Note that keywords are not normally used for peerreview papers.
% \begin{IEEEkeywords}
% IEEE, IEEEtran, journal, \LaTeX, paper, template.
% \end{IEEEkeywords}






% For peer review papers, you can put extra information on the cover
% page as needed:
% \ifCLASSOPTIONpeerreview
% \begin{center} \bfseries EDICS Category: 3-BBND \end{center}
% \fi
%
% For peerreview papers, this IEEEtran command inserts a page break and
% creates the second title. It will be ignored for other modes.
\IEEEpeerreviewmaketitle

\section{Introduction}

\IEEEPARstart{K}{nowledge} base systems have been around for a while. They take a knowledge base as input. This knowledge base describes the world, i.e. it is a specification of how something should work. Based on this knowledge base, different kinds of inference can be applied. De Cat et al. \cite{IDP} give the example of an university course-management system. The knowledge base contains rules like ``In every auditorium at a any time, there should be at most one course.''. The different kinds of inferences that can be applied, include (the examples are from De Cat et al. \cite{IDP}): \textit{propagation} (e.g. automatically selecting required prerequisites when constructing an individual study program), \textit{model expansion} (e.g. getting a full individual study program from a partial one) and \textit{querying} (e.g. getting the schedule for a specific student).

These knowledge base systems have triggered a lot of research into formal languages and their expressivity. These languages are often hard to read, write and learn.

One method to ease the writing of formal theories is to develop languages that are somewhere in between natural language and formal languages. Such languages are called Controlled Natural Languages (CNL). A CNL is a subset of a natural language that, for example, allows to write specifications in a more consistent language. Kuhn~\cite{Kuhn2014} made an overview of 100 CNL's. Some of these languages have formal semantics and can be translated automatically into a formal logic. They can be used as a knowledge representation language within a knowledge base system. They are often easy to read. Unfortunately, the translations of such CNL's into logic are not well documented which makes it hard to expand the language.

This paper therefore constructs a new CNL based on the language used in a small domain, namely logic grid puzzles. It is then tested on unseen puzzles. The first goal of this paper is to show that the framework introduced by Blackburn and Bos \cite{Blackburn2005, Blackburn2006} can be used to automatically translate such a CNL into a formal logic. The second goal is to show that adding types to this framework is useful. It makes it possible to translate to a typed logic and allows more inference. E.g. The system can automatically infer the domains of a logic grid puzzle.

% allows more inference. The system can automatically infer domains of a logic grid puzzle.

% In this paper, we take the next step for bridging the gap between natural language and logic. We use the well-documented framework of Blackburn and Bos and extend it with types. With this extension, we constructed the first typed CNL with a formal semantics in a typed logic. This new CNL is based on the language used in a small domain, namely logic grid puzzles. It is then tested on unseen puzzles. The first goal of this paper is to show that the framework introduced by Blackburn and Bos \cite{Blackburn2005, Blackburn2006} can be used to automatically translate such a CNL into a formal logic. We choose the IDP language \cite{IDP} as our formal logic. The second goal is to show that adding types to this framework is useful. It makes it possible to translate to a typed logic and allows more inference. E.g. The system can automatically infer the domains of a logic grid puzzle.

Therefore, we take the next step for bridging the gap between natural language and logic. We extend the well documented framework of Blackburn and Bos with types. With this extension, we constructed the first typed CNL with a formal semantics in a typed logic.

The result of this research is a fully automated tool that can understand a logic grid puzzle and reason about it (e.g. the system can solve the puzzle automatically).


\section{Logic Grid Puzzles}
Logic grid puzzles are logical puzzles. The most famous such puzzle is probably the Zebra Puzzle \cite{zebra}, sometimes also known as Einstein's Puzzle. These puzzles consist of a number of sentences or clues in natural language. In the clues, a number of domains appear (e.g. nationality, color, animal, ...), each with a number of domain elements (e.g. Norwegian, Canadian, blue, red, cow, horse, ...). Between each domain there is a bijection. The goal of such a puzzle is to find the value of these bijections, i.e. to find which domain elements belong together. E.g. The Norwegian lives in a blue house and keeps the horse. Every puzzle has an unique solution.
 
Logic grid puzzles can be considered as small specifications. Moreover, they can be expressed in fairly simple logical statements. Finally, it is easy to find numerous examples of these types of puzzles. It is these three properties that make logic grid puzzles the ideal domain to test our assumptions. Namely that the framework of Blackburn and Bos allows to construct a CNL with formal semantics that can be used for knowledge representation (within a small domain).

\section{Motivation}
Since knowledge representation languages can be hard to read, write and learn, an expert in formal languages is required to write and maintain the specification. A formal language that is more accessible for non-experts, like a CNL with formal semantics, can solve this issue without compromising on the strength of knowledge base systems. It is still possible to apply different kind of inferences on a specification that is more accessible for non-experts.

For example, when the clues of a logic grid puzzle are expressed in a CNL (with formal semantics), it is possible to automatically construct a complete specification such that a knowledge base system can reason about the puzzle. Such a system can
% With a CNL for logic grid puzzles, it is possible to apply different kind of inferences to these puzzles. It is for example possible to
\begin{itemize}
  \item Solve the puzzle automatically.
  \item Given a (partial) solution by the user, indicate which clues have already been incorporated in the solution, which clues still hold some new information and which clues have been violated.
  \item Given a partial solution by the user, automatically derive a subset of clues that can be used to expand the solution. This can be part of a ``hint''-system for the user.
  \item Given a set of clues, indicate which solutions are still possible. This can help the author while writing new puzzles. Based on the solutions that are still possible, the author can construct a new clue to eliminate some of the possibilities until only one solution remains.
\end{itemize}

\section{Related work}

Two of the most advanced controlled natural languages that can be used for knowledge representation are Attempto Controlled English (ACE) and Processable English (PENG). ACE \cite{Fuchs2008} is a general purpose CNL. The language contains a large built-in vocabularium. This has as advantage that the user doesn't have to provide the vocabularium. However, for specifications in small domains, we usually do want to provide the vocabularium to make sure the specification only talks about the modelled domain. PENG \cite{Schwitter2002} does ask the user to provide its own content words. Moreover, there is a tool named ECOLE which gives suggestions while writing PENG sentences, namely which linguistic constructs can follow. This makes it easier to write correct PENG sentences. These languages mainly show that we can translate English into logic, but the literature on both these CNL's lacks a description of this translation process. Therefore, extending these CNL's is hard.

Another important CNL is RuleCNL \cite{Njonko2014}. It is a CNL that translates (a subset of) English into business rules. It's notable because it is the only CNL we could find that uses types. Interestingly, they don't really describe their type system nor the inferences they support. However figure~6 from \cite{Njonko2014} indicates that they do support type checking of business rules in RuleCNL.

Finally, Baral et al. \cite{Baral2008, Costantini2010, Baral2012, Baral2012a} researched translating natural language into ASP programs. They do this based on $\lambda$-calculus and a Combinatorial Categorial Grammar (CCG). Each word gets one $\lambda$-ASP-expression per CCG category it can represent. The CCG then tells how to combine these expressions via $\lambda$-application. In their last paper \cite{Baral2012a}, Baral et al. explain how to learn both these $\lambda$-ASP-expressions and a probabilistic CCG grammar from a set of logigrams. They used a supervised machine learning method for this goal. Based on these methods they can solve unseen logigrams without adapting these logigrams.

However, the goal of this paper is not to solve (unseen) logigrams without adapting them. The goal is to create a CNL with formal semantics that can be applied to logigrams and can be used as a knowledge representation language for these logigrams. This paper illustrates that for a small domain, it is possible to have such a language. For unseen logigrams, slight adaptions are necessary to make sure the clues are grammatically correct according to the constructed CNL. As mentioned earlier, once we have a representation of the clues in a formal logic, inference becomes possible. Solving the logigram is only one of the examples.

\chapter{Een framework voor semantische analyse}
\label{ch:framework}
In dit hoofdstuk bespreken we het framework van Blackburn en Bos \cite{Blackburn2005, Blackburn2006} voor semantische analyse. Het framework bestaat uit een lexicon (het vocabularium), de grammatica, de semantiek van de grammaticale regels en de semantiek van de woorden in het lexicon. Het hele framework is gebaseerd op lambda-calculus en Frege's compositionaliteitsprincipe dat stelt dat de betekenis van een woordgroep enkel afhangt van de betekenissen van de woorden waaruit ze bestaat.

\paragraph{} \textit{Dit hele hoofdstuk is een samenvatting van de relevante hoofdstukken van de boeken van Blackburn en Bos \cite{Blackburn2005, Blackburn2006}. De formules komen grotendeels overeen met die uit hun boeken en uit de code die erbij hoort. De aanpassingen die zijn gebeurd, dienen vooral om de leesbaarheid te verhogen. De nadruk op de signaturen van de verschillende categorieën is nieuw, al komt de getypeerde lambda-calculus met de twee types wel aan bod in appendix B van hun eerste boek \cite{Blackburn2005}.}

\section{Lexicon}
Het lexicon bestaat uit een opsomming van alle woorden met een aantal (taalkundige) features (zoals de categorie van het woord). Tabel \ref{tbl:lexicon} geeft een voorbeeld van een lexicon. Het lexicon is meer dan een woordenboek. Het bevat alle woordvormen, niet enkel de basisvorm. Zo komt ``love'' drie keer voor in het lexicon. ``love'' zelf komt één keer voor als infinitief en één keer als meervoud van de tegenwoordige tijd. ``loves'' is dan weer het enkelvoud van de tegenwoordige tijd.

De meeste woordvormen hebben ook een feature \texttt{Symbool} die zal gebruikt worden bij de semantiek van de woorden. Deze feature maakt het onderscheid tussen de verschillende woorden van dezelfde categorie. Het symbool komt overeen met de naam van de constante of van het predicaat die het woord introduceert in het formeel vocabularium (zie sectie~\ref{sec:vocabularium}).

\begin{table}[!]
  \centering
  \begin{tabular}{@{}llll@{}}
    \toprule
    \textbf{Woordvorm} & \textbf{Categorie} & \textbf{Symbool} & \textbf{Andere features} \\ \midrule
    man                & zelfstandig naamwoord     & man     & num=sg            \\
    men                & zelfstandig naamwoord     & man     & num=pl            \\
    woman              & zelfstandig naamwoord     & woman   & num=sg            \\
    women              & zelfstandig naamwoord     & woman   & num=pl            \\
    John               & eigennaam                 & John    &                   \\
    sleep              & onovergankelijk werkwoord & sleeps  & inf=inf           \\
    sleeps             & onovergankelijk werkwoord & sleeps  & inf=fin, num=sg   \\
    sleep              & onovergankelijk werkwoord & sleeps  & inf=fin, num=pl   \\
    love               & overgankelijk werkwoord   & loves   & inf=inf           \\
    loves              & overgankelijk werkwoord   & loves   & inf=fin, num=sg   \\
    love               & overgankelijk werkwoord   & loves   & inf=fin, num=pl   \\
    a                  & determinator              & /       & type=existential  \\
    every              & determinator              & /       & type=universal    \\
    \bottomrule
  \end{tabular}
  \caption{Een voorbeeld van een lexicon}
  \label{tbl:lexicon}
\end{table}

\section{Grammatica}
De grammatica bepaalt welke woorden samen woordgroepen vormen, welke woordgroepen samen andere woordgroepen vormen en welke woordgroepen een zin vormen. Op die manier ontstaat er een boom van woorden.

\paragraph{Een simpele grammatica} Grammatica~\ref{dcg:simple-gramm} bevat een simpele grammatica. Een zin bestaat in deze grammatica uit een \texttt{np} gevolgd door een \texttt{vp}, beiden met hetzelfde getal. Een naamwoordgroep (noun phrase of \texttt{np}) is een woordgroep die naar één of meerdere entiteiten verwijst. Zo'n groep wordt ook wel een nominale constituent genoemd. Een verb phrase (\texttt{vp}) of verbale constituent drukt meestal een actie uit.

\begin{dcg}{Een simpele grammatica}{dcg:simple-gramm}
s() --> np([num:Num]), vp([num:Num]).
s() --> [if], s(), s().
np([num:sg]) --> pn().
np([num:Num]) --> det([num:Num]), n([num:Num]).
vp([num:Num]) --> iv([num:Num]).
vp([num:Num]) --> tv([num:Num]), np([num:_]).
\end{dcg} 

Een zin kan ook bestaan uit het functiewoord ``if'', gevolgd door twee zinnen (bijvoorbeeld ``If a man breathes, he lives''). Functiewoorden zijn deel van de grammatica en komen niet voor in het lexicon. Ze helpen om de structuur van de zin te herkennen. De betekenis van deze woorden komt via de semantiek van de grammatica naar boven.

Een naamwoordgroep (\texttt{np}) kan bestaan uit een eigen naam (proper name of \texttt{pn}) of uit een determinator (\texttt{det}, bijvoorbeeld een lidwoord) en een zelfstandig naamwoord (noun of \texttt{n}) die overeenkomen in getal. Een eigennaam is in deze grammatica altijd in het enkelvoud.

Een verbale constituent (\texttt{vp}) bestaat uit een onovergankelijk werkwoord (intransitive verb of \texttt{iv}) of uit een vergankelijk werkwoord (transitive verb of \texttt{tv}) gevolgd door een nieuwe naamwoordgroep (als lijdend voorwerp). Het werkwoord moet in getal overeenkomen met het getal van de verbale constituent. Daardoor zal het getal van het werkwoord en het onderwerp altijd overeenkomen.

Grammatica~\ref{dcg:simple-gramm} bevat een aantal categorieën (\texttt{pn}, \texttt{det}, \texttt{n}, \texttt{iv} en \texttt{tv}) die overeenkomen met de lexicale categorieën. Voor deze categorieën wordt er in het lexicon gezocht naar een woord met de juiste features.

Bovenstaande grammatica is nog heel beperkt. De moeilijkheid ligt erin om de grammatica simpel te houden maar toch zoveel mogelijk gewenste zinnen toe te laten. Om logigrammen automatisch te kunnen vertalen moet er dus een grammatica opgesteld worden die de zinnen van deze logigrammen omvat.

\paragraph{Een boom} Op basis van deze grammatica kunnen we ook een parse tree opbouwen voor elke geldige zin. Zo wordt ``Every woman loves john omgezet in de boom

\Tree[.s [.np [.det every ] [.n woman ]] [.vp [.tv loves ] [.np [.pn john ]]]]

Op elke knoop in deze boom zullen we later Frege's compositionaliteitsprincipe toepassen: de betekenis van een woordgroep is gelijk aan een combinatie van de betekenissen van de woord(groep)en waaruit ze bestaat.

\paragraph{Conclusie} De grammatica bepaalt welke combinaties van woorden zinnen vormen. Ze bepaalt dus welke zinnen in de taal liggen en welke er buiten vallen. Bovendien geeft de grammatica ons een boom. Deze boom zullen we gebruiken om de betekenis van de woorden naar boven toe te laten propageren volgens Frege's compositionaliteitsprincipe tot de betekenis van de zin.

\section{Semantiek}
Het lexicon bepaalt welke woorden gebruikt mogen worden, de grammatica hoe deze woorden een zin kunnen vormen. De vraag die nog rest is welke betekenis de zin heeft. Daarvoor doen we een beroep op de lambda-calculus. Eerst bespreken we hoe we de betekenis van een woordgroep kunnen afleiden uit de betekenis van de woordgroepen waaruit ze bestaan. Daarna bespreken we de betekenis van de woorden uit het lexicon.

\subsection{Semantiek van de grammaticale regels}
\paragraph{Een getypeerde lambda-calculus}
Voor de betekenis van de taal zullen we gebruiken maken van een getypeerde lambda-calculus. Er zijn twee basistypes in onze lambda-calculus, namelijk $e$ voor entiteiten en $t$ voor waarheidswaarden (\textit{waar} en \textit{onwaar}). Ten slotte is er de type-constructor voor een functie-type die we noteren met $\rightarrow$. Zo is $\lambda x \cdot man(x)$ een lambda-expressie van type $e \rightarrow t$. Elk woord zal een lambda-formule als betekenis krijgen. Deze formules zullen gecombineerd worden tot één formule voor de zin die geen lambda's meer bevat. Die formule vormt de betekenis van de zin. De betekenis van een zin is dus van type $t$.

\paragraph{Frege's compositionaliteitsprincipe} Voor de betekenis van de grammatica berusten we op Frege's compositionaliteitsprincipe: De betekenis van een woordgroep bestaat uit een combinatie van de betekenissen van de woorden of woordgroepen waaruit ze bestaat. Deze combinatie kan afhangen van de manier waarop de woorden gecombineerd worden. Op deze manier propageren we de semantiek van de woorden (die de bladeren in de boom vormen) naar boven toe om zo de betekenis van de zin te verkrijgen. We hernemen bovenstaande grammaticale regels nu en voegen de semantiek toe. Later zullen we aantonen dat deze simpele combinatieregels tot een zinnig resultaat kunnen leiden.

\begin{table}[h]
  \begin{tabular}{@{}ll}
    \hline
    \textbf{Grammaticale regel} & \textbf{Semantiek} \\
    \hline
    \texttt{s ---> np([num:Num]), vp([num:Num]).}              & $\sem{s} = \app{\sem{vp}}{\sem{np}}$ \\
    \texttt{s ---> [if], s$_1$, s$_2$.}                        & $\sem{s} = \drs{}{\sem{s_1} \Rightarrow \sem{s_2}}$ \\
    \texttt{np([num:sg]) ---> pn.}                             & $\sem{np} = \sem{pn}$ \\
    \texttt{np([num:Num]) ---> det([num:Num]), n([num:Num]).}  & $\sem{np} = \app{\sem{det}}{\sem{n}}$ \\
    \texttt{vp([num:Num]) ---> iv([num:Num]).}                 & $\sem{vp} = \sem{iv}$ \\
    \texttt{vp([num:Num]) ---> tv([num:Num]), np([num:\_]).}   & $\sem{vp} = \app{\sem{tv}}{\sem{np}}$\\
    \hline
  \end{tabular}
  \centering
  \caption{De semantiek van grammatica~\ref{dcg:simple-gramm}}
  \label{tbl:grammar-sem}
\end{table}

Tabel~\ref{tbl:grammar-sem} geeft een overzicht van de semantiek van grammatica~\ref{dcg:simple-gramm}. Een idee achter het framework is om de betekenis van de zinnen zoveel mogelijk in de betekenis van de woorden te stoppen. De betekenis van een grammaticale regel wordt, indien mogelijk, beperkt tot lambda-applicaties. Voor woordgroepen die maar uit één woord bestaan is de betekenis van de woordgroep gelijk aan die van het woord zelf. Voor de meeste andere woordgroepen is de betekenis enkel een lambda-applicatie van de betekenissen van de woordgroepen waaruit ze bestaan. Meestal heeft een woordgroep één woord dat essentieel is aan de groep. De woordgroep is hier dan ook naar vernoemd\footnote{Voor de semantiek van een naamwoordgroep wordt het lidwoord toch als het belangrijkste woord aanzien door Blackburn en Bos}. Dit woord is de functor van de lambda-applicatie, het andere woord het argument. Enkel voor de speciale zinsstructuur van de conditionele zin is er ook een speciale constructie nodig in de semantiek. Indien we het functiewoord ``if'' naar het lexicon zouden verhuizen, dan zouden twee lambda-applicaties volstaan\footnote{$\sem{if} = \lambdaf{S1}{\lambdaf{S2}{\drs{}{S1 \Rightarrow S2}}}$ en $\sem{s} = \app{\app{\sem{if}}{\sem{s_1}}}{\sem{s_2}}$}.

\subsection{Semantiek van het lexicon} We weten nu hoe we de semantiek van de woorden kunnen combineren tot de semantiek van de woordgroepen en bij uitbreiding tot die van een zin. Er ontbreekt enkel nog de semantiek van de woorden zelf.

Voor we de betekenis van woorden kunnen opstellen moeten we eerst de signatuur achterhalen. Cruciaal voor het framework is dat elke grammaticale categorie exact 1 signatuur heeft. Zodanig dat we altijd woordgroepen van dezelfde categorie kunnen uitwisselen. We gebruiken de functie $\tau$ om de signatuur aan te duiden.

We beginnen met de signatuur van een zelfstandig naamwoord (\texttt{n}) te achterhalen. Daarna bekijken we achtereenvolgens de nominale constituent (\texttt{np}), de eigennaam (\texttt{pn}) en het lidwoord (\texttt{det}). Ten slotte bekijken we de hiërarchie van de verbale constituenten: de verbale constituent zelf (\texttt{vp}), onovergankelijk werkwoord (\texttt{iv}) en overgankelijk werkwoord (\texttt{tv}).

\paragraph{De signatuur van een zelfstandig naamwoord.} Een zelfstandig naamwoord, zoals bijvoorbeeld $\mathit{man}$, stelt een verzameling entiteiten voor. In het voorbeeld van $\mathit{man}$ stelt dit de verzameling van alle mannen voor. In logica wordt zo'n verzameling typisch voorgesteld door een unair predicaat. In deze setting gebeurt net hetzelfde: we stellen een zelfstandig naamwoord voor als een functie van entiteiten naar waarheidswaarden, of in formule $e \to t$. Dit is de functionele variant van een unair predicaat. De functie van een substantief bepaalt of een bepaalde entiteit omschreven kan worden met dat substantief of niet.

\paragraph{De signatuur voor een nominale constituent (np)} Een nominale constituent of naamwoordgroep is een woordgroep die kan dienen als onderwerp of lijdend voorwerp. De betekenis ervan is een verwijzing naar een entiteit (of een groep van entiteiten). In een zin zeggen we altijd iets over deze entiteiten, bijvoorbeeld ``a man sleeps''. Een zin is waar als de entiteit(en) van de naamwoordgroep voldoen aan een bepaalde eigenschap, bijvoorbeeld als er een man is die de eigenschap heeft dat hij slaapt. Dat is wat signatuur $$\tau(np) = (e \to t) \to t$$ uitdrukt. Het eerste argument van type $e \to t$ is een eigenschap waaraan entiteiten kunnen voldoen. Bijvoorbeeld voor ``sleeps'' is dit (in eerste-orde-logica) $\lambdaf{x}{sleeps(x)}$.

Een naamwoordgroep voldoet aan een eigenschap als er \textit{genoeg} entiteiten zijn die omschreven worden door de naamwoordgroep en die voldoen aan de eigenschap. Een voorbeeldimplementatie voor ``a man'' (in eerste-orde-logica) is $$\lambdaf{E}{\exists x \cdot man(x) \land E(x)}$$ Er moet namelijk minstens één man zijn die aan de eigenschap $E$ voldoet. In het geval van ``every man'' moeten alle mannen voldoen aan de eigenschap $E$ $$\lambdaf{E}{\forall x \cdot man(x) \Rightarrow E(x)}$$

\paragraph{Eigennaam en lidwoord} De signatuur van een eigennaam is gelijk aan die van een naamwoordgroep. Dat volgt uit de semantiek van de grammatica. $$\tau(pn) = \tau(np) = (e \rightarrow t) \rightarrow t$$ Uit de semantiek van de grammatica volgt ook dat de signatuur van een lidwoord gelijk is aan die van een zelfstandig naamwoord naar een naamwoordgroep. $$ \tau(det) = \tau(n) \rightarrow \tau(np) = (e \to t) \rightarrow (e \rightarrow t) \rightarrow t$$ Bijvoorbeeld voor ``every'' wordt dit (met $R$ een restrictie vanuit het substantief) $$\lambdaf{R}{\lambdaf{E}{\forall x \cdot R(x) \Rightarrow E(x)}}$$

\paragraph{Verbale constituent} De signatuur voor een verbale constituent (\texttt{vp}) volgt opnieuw uit de grammatica en de andere signaturen $$\tau(vp) = \tau(np) \rightarrow \tau(s) = ((e \rightarrow t) \rightarrow t) \rightarrow t$$ Een zin is waar als het onderwerp voldoet aan de eigenschap die het werkwoord uitdrukt.

De signatuur van een onovergankelijk werkwoord (\texttt{iv}) is gelijk aan die van een verbale constituent (volgens de grammatica). $$\tau(iv) = \tau(vp) = ((e \rightarrow t) \rightarrow t) \rightarrow t$$ De signatuur van een overgankelijk werkwoord (\texttt{tv}) wordt dan $$\tau(tv) = \tau(np) \rightarrow \tau(vp) = ((e \rightarrow t) \rightarrow t) \rightarrow ((e \rightarrow t) \rightarrow t) \rightarrow t$$

\paragraph{} Tabel~\ref{tbl:signaturen} vat alle signaturen nog eens samen. Op basis van deze signaturen kunnen we de betekenis van het lexicon opstellen. Blackburn en Bos gebruiken hiervoor \textit{semantische macro's}. Dat wil zeggen dat ze voor elke lexicale categorie een functie hebben die een woordvorm uit het lexicon afbeeldt op een lambda-expressie. Hiervoor worden enkel de lexicale features van de woordvorm in kwestie gebruikt.

\begin{table}[h]
  \begin{tabular}{@{}lll}
    \hline
    \textbf{Grammaticale categorie}             & \textbf{Signatuur} \\
    \hline 
    Zin (\texttt{s})                          & $t$ \\
    Naamwoordgroep (\texttt{np})              & $(e \rightarrow t) \rightarrow t$ \\
    Eigennaam (\texttt{pn})                   & $(e \rightarrow t) \rightarrow t$ \\
    Zelfstandig naamwoord (\texttt{n})        & $(e \rightarrow t)$ \\
    Lidwoord (\texttt{det})                   & $(e \rightarrow t) \rightarrow (e \rightarrow t) \rightarrow t$ \\
    Verbale constituent (\texttt{vp})         & $((e \rightarrow t) \rightarrow t) \rightarrow t$ \\
    Onovergankelijk werkwoord (\texttt{iv})   & $((e \rightarrow t) \rightarrow t) \rightarrow t$ \\
    Vergankelijke werkwoord (\texttt{tv})     & $((e \rightarrow t) \rightarrow t) \rightarrow ((e \rightarrow t) \rightarrow t) \rightarrow t$ \\
    \hline
  \end{tabular}
  \centering
  \caption{De signaturen van de grammaticale categorieën uit grammatica~\ref{dcg:simple-gramm}}
  \label{tbl:signaturen}
\end{table}

Grammatica~\ref{dcg:simple-gramm} telt 5 lexicale categorieën: \texttt{pn}, \texttt{n}, \texttt{det}, \texttt{iv} en \texttt{tv}.
\begin{itemize}
  \item Een eigennaam (\texttt{pn}) introduceert een constante met als naam het symbool dat bij die eigennaam hoort. Een eigennaam voldoet aan de eigenschap $E$ als de entiteit waarnaar verwezen wordt eraan voldoet. Meer formeel wilt dit zeggen dat de functiewaarde van die constante voor de functie $E$ gelijk is aan de waarheidswaarde \textit{waar} of $\sem{pn} = \lambdaf{E}{\app{E}{\textit{Symbool}}}$ of voor ``John'' $\sem{John} = \lambdaf{E}{\app{E}{John}}$.
  \item Een zelfstandig naamwoord (\texttt{n}) test of een referentie kan omschreven worden met dat naamwoord of niet. $\sem{n} = \lambdaf{x}{\drs{}{\textit{Symbool}(x)}}$. Bijvoorbeeld voor ``man'' $\sem{man} = \lambdaf{x}{\drs{}{man(x)}}$
  \item Een lidwoord of meer algemeen een determinator (\texttt{det}) introduceert een nieuwe referentie die een bepaalde scope heeft. Hier zijn er meerdere mogelijke vertalingen, één voor elk type van determinator. Een determinator heeft 2 argumenten: de \textit{restriction} en de \textit{nuclear scope}. De \textit{restriction} $R$ wordt opgevuld door het zelfstandig naamwoord (met eventuele bijzinnen). De \textit{nuclear scope} $S$ wordt opgevuld door de verbale constituent. Het is de eigenschap waaraan de naamwoordgroep moet voldoen.
    \begin{itemize}
      \item Voor een universele determinator (zoals ``every'') moet de variabele met een universele kwantor gebonden zijn. In eerste-orde-logica krijgen we dan $\sem{det_{universeel}} = \lambdaf{R}{\lambdaf{S}{\forall x \cdot \left( \app{R}{x} \Rightarrow \app{S}{x} \right)}}$. In DRS wordt dit $$\sem{det_{universeel}} = \lambdaf{R}{\lambdaf{S}{\drs{}{\drsImpl{\drsMerge{\drs{x}{}}{\app{R}{x}}}{\app{S}{x}}}}}$$
      \item De existentiële determinator (zoals ``a'') introduceert een variabele die gebonden is door een existentiële kwantor. De $\oplus$-operator vormt de logische conjunctie tussen twee DRS-structuren. $$\sem{det_{existentieel}} = \lambdaf{R}{\lambdaf{S}{\drsTriMerge{\drs{x}{}}{\app{R}{x}}{\app{S}{x}}}}$$
      \item De negatieve determinator (zoals ``no''): Deze determinator drukt uit dat er geen entiteit bestaat die tegelijk aan de restrictie $R$ en de eigenschap $S$ voldoet. Deze introduceert dus een negatie van een existentiële kwantor. $$\sem{det_{negatief}} = \lambdaf{R}{\lambdaf{S}{\drsNot{\ \drs{x}{} \oplus \app{R}{x} \oplus \app{S}{x}}}}$$
    \end{itemize}
  \item Een onovergankelijk werkwoord (\texttt{iv}) neemt de naamwoordgroep die het onderwerp vormt als argument. Het moet testen of dit onderwerp $O$ voldoet aan de eigenschap van het werkwoord $\lambdaf{x}{\drs{}{\textit{Symbool}(x)}}$, namelijk of het onderwerp de \textit{actie} van het werkwoord uitvoert. $$\sem{iv} = \lambdaf{O}{\app{O}{\lambdaf{x}{\drs{}{\textit{Symbool}(x)}}}}$$
  \item Een overgankelijk werkwoord (\texttt{tv}) is gelijkaardig maar krijgt twee naamwoordgroepen als argument. Het eerste argument is het lijdend voorwerp $L$, het tweede het onderwerp $O$. Er zijn meerdere vertaling mogelijk. Neem bijvoorbeeld de zin ``Every man loves a woman''. Is er één vrouw waarvan alle mannen houden of kan dit een verschillende vrouw zijn voor elke man? Dit wordt een \textit{Quantifier Scope Ambiguïteit} genoemd omdat de ambiguïteit ligt in de volgorde van de kwantoren. Zo worden de twee lezingen respectievelijk $\exists w \cdot woman(w) \land (\forall m \cdot man(m) \Rightarrow loves(m, w))$ en $\forall m \cdot man(m) \Rightarrow (\exists w \cdot woman(w) \land loves(m, w))$. Blackburn en Bos lossen deze ambiguïteiten op door de kwantoren in de vertaling in dezelfde volgorde te laten als in de natuurlijke taal\footnote{Ze leggen daarnaast ook uit hoe men de andere lezingen kan verkrijgen. We verwijzen nar hoofdstuk 3 uit hun eerste boek \cite{Blackburn2005} voor de details.}. Voor het overgankelijke werkwoord wordt dit $$\sem{tv} = \lambdaf{L}{\lambdaf{O}{\app{O}{\lambdaf{x_o}{\app{L}{\lambdaf{x_l}{\drs{}{\textit{Symbool}(x_o, x_l)}}}}}}}$$ Een entiteit $x_o$ omschreven door het onderwerp voldoet aan de verbale constituent als voor die $x_o$ het lijdend voorwerp voldoet aan de eigenschap $\lambdaf{x_l}{\drs{}{\textit{Symbool}(x_o, x_l)}}$.

We passen dit toe op ``Every man loves a woman''. Een man $x_o$ voldoet aan de verbale constituent (``loves a woman'') als er voor die man $x_o$ een vrouw $x_l$ is zodat $\drs{}{\textit{loves}(x_o, x_l)}$ waar is. Het onderwerp (als geheel) voldoet aan de verbale constituent als elke man $x_o$ voldoet aan bovenstaande eigenschap. Als er dus voor elke man $x_o$ een vrouw $x_l$ bestaat waarvan hij houdt. Voor elke man kan er dit een andere vrouw $x_l$ zijn.
\end{itemize}

Merk op dat al deze lambda-expressies voldoen aan de signaturen van tabel~\ref{tbl:signaturen}

\section{Een voorbeeld}
In appendix~\ref{app:vb-framework} illustreren we het framework en bovenstaande formules aan de hand van de zin ``If every man sleeps, a woman loves John''. De vertaling wordt 

  \begin{align*}
    \sem{s} &= \drs{}{\drsImpl{\drs{}{\drsImpl{\drs{x}{man(x)}}{\drs{}{sleeps(x)}}}}{\drs{y}{woman(y) \\ loves(y, john)}}} \\
            &= \bigg( \forall x \cdot man(x) \Rightarrow sleeps(x) \bigg) \Rightarrow \bigg( \exists y \cdot woman(y) \land loves(y, john) \bigg)
  \end{align*}

Dit is de vertaling zoals we die zouden verwachten.

\section{Evaluatie}
Het framework dat Blackburn en Bos voorstellen, is uitermate geschikt voor semantische analyse van natuurlijke taal. De verschillende onderdelen staat vrij los van elkaar. Zo kan men de doeltaal vrij kijzen. Blackburn en Bos vertalen in hun eerste boek \cite{Blackburn2005} naar eerste-orde-logica. In hun tweede boek \cite{Blackburn2006} vertalen ze naar \textit{Discourse Representation Structures}. Ook de vorm van de grammatica is vrij te kiezen. Zowel Blackburn en Bos als deze thesis gebruiken DCG's om de grammatica te specifiëren. Baral et al. \cite{Baral2008} gebruiken een gelijkaardig framework maar met behulp van een Combinatorische Categorische Grammatica.

\paragraph{} Dankzij het framework van Blackburn en Bos, wordt het probleem van semantische analyse herleid tot het opstellen van een lexicon dat de woorden van het logigram bevat; het opstellen van een grammatica die de zinsstructuren van een logigram omvat; het opstellen van de semantiek van deze grammaticale regels; en ten slotte het opstellen van semantische macro's voor alle lexicale categorieën die we introduceren.

\chapter{Een lexicon voor logigrammen}
In dit hoofdstuk bespreken we de gebruikte lexicale categorieën. We beperken ons tot een set die het makkelijk maakt voor de vertaling van logigrammen naar DRS-structuren. We bespreken zowel de categorieën zelf alsook hun vertaling naar deze DRS-structuren. Er wordt een onderscheid gemaakt tussen open en gesloten lexicale categorieën. De open categorieën zijn open voor uitbreiding. De woorden uit die categorie zijn verschillend per logigram. De gesloten categorieën bevatten woorden die gemeenschappelijk zijn voor alle logigrammen. Tabel~\ref{tbl:lexiconCategories} geeft een overzicht van de gebruikte lexicale categorieën.

\begin{table}[t]
  \centering
  \begin{tabular}{llll}
    \toprule
    \textbf{Categorie} & \textbf{Afkorting} & \textbf{Open?} & \textbf{Voorbeeld}  \\ \midrule
    Determinator       & det                & gesloten & a, an, the \\
    Hoofdtelwoord      & number             & gesloten & three, 5      \\
    Eigennaam          & pn                 & open     & John, ``the black darts'' \\
    Substantief        & noun               & open     & man, year, \\
    Voorzetsel         & prep               & gesloten & in, to \\
    Betrekkelijk voornaamwoord & relpro     & gesloten & who, which, that \\
    Transitief werkwoord & tv               & open     & loves, ``had a final score of'' \\
    Hulpwerkwoord      & av                 & gesloten & does, ``doesn't'' \\
    Koppelwerkwoord    & cop                & gesloten & is, ``is not'' \\
    Comparatief        & comp               & open     & below, ``older than'' \\
    Onbepaalde woorden & some               & open     & somewhat, sometime \\
    Voegwoord          & coord              & gesloten & and, or, ``neither ... nor ...'' \\
    \bottomrule
  \end{tabular}
  \caption{Een overzicht van de lexicale categorieën}
  \label{tbl:lexiconCategories}
\end{table}

\section{Determinator}
Een determinator kan zowel een lidwoord als een kwantor zijn. In het geval van logigrammen volstaan de lidwoorden ``a'', ``an'' en ``the''. Deze drie determinatoren krijgen alle drie de vertaling van de existentiële determinator uit het vorige hoofdstuk $$\sem{det} = \sem{det_{existentieel}} = \lambdaf{R}{\lambdaf{S}{\left( \drs{x}{} \oplus \app{R}{x} \oplus \app{S}{x} \right)}}$$ Er is dus geen nood aan een universele of negatieve determinator. Bij logigrammen zijn we namelijk op zoek naar de waarde van bijecties. Er is dus altijd exact één iemand die een bepaalde drank drinkt of een bepaald huisdier heeft. Er is nooit sprake van ``Every man who drinks vodka, ...'' maar altijd van ``The man who drinks vodka, ...''. Men kan ook nooit gebruik maken van de negatieve determinator (bv. ``No man drinks vodka'') aangezien er altijd één iemand moet zijn die vodka drinkt.

\section{Hoofdtelwoord}
Hoofdtelwoorden kunnen gebruikt worden als determinatoren die een aantal uitdrukken. In deze thesis krijgen hoofdtelwoorden echter een eigen lexicale categorie omdat op sommige plaatsen enkel een hoofdtelwoord past en geen andere determinatoren (bijvoorbeeld ``five'' in de zin ``The five different people are ...''). De hoofdtelwoorden mogen in cijfers voorkomen maar ook in woorden \footnote{In de praktijk zitten enkel de eerste 15 getallen in woorden in het lexicon}.

De signatuur van een hoofdtelwoord is gelijk aan die van een determinator. Er zijn twee mogelijke lezingen voor een hoofdtelwoord: de collectieve en de distributieve lezing. We verduidelijken aan de hand van de voorbeeldzin ``Twee mannen gaan naar zee''. In de collectieve lezing vormen de ``twee mannen'' één geheel. Ze gaan dus samen naar zee. In de distributieve lezing zijn er twee mannen die elk naar zee gaan. In logigrammen komt enkel de collectieve lezing aan bod. Meestal gaat het immers om een numerieke eigenschap van iets of iemand. Bijvoorbeeld ``John is 10 years old'' of ``John is 3 years younger than Mary''.

$$\sem{number} = \lambdaf{R}{\lambdaf{S}{\drsMerge{\app{R}{Number}}{\app{S}{Number}}}}$$

\section{Eigennaam}
Een eigennaam is een open lexicale categorie. Dat wil zeggen dat de eigennamen verschillend zijn per logigram. De semantiek is identiek aan die in het vorige hoofdstuk. $$\sem{pn} = \lambdaf{P}{\app{P}{\textit{Symbool}}}$$ We staan vanaf nu echter wel toe dat woordgroepen die taalkundig geen eigennaam zijn, toch gebruikt kunnen worden als een eigennaam. Zo kan ``the black darts'' (wat normaal een determinator + adjectief + substantief is) aanzien worden als een eigennaam. Dit maakt het vertalen van de zinnen makkelijker maar tegelijkertijd wordt het opstellen van het lexicon voor een logigram moeilijker. Het lexicon is niet meer enkel afhankelijk van taalkundige informatie. Alle mogelijke waarden van een (niet-numeriek) concept moeten namelijk als een eigennaam aangegeven worden in het lexicon dat hoort bij het logigram. Bovendien moet dit gebeuren in de vorm zoals het voorkomt in de zinnen van het logigram. Zo is ``black'' een waarde van het concept kleur, toch moet ``the black darts'' ingegeven worden in het lexicon. Deze eigennamen zullen later vertaald worden naar een constante uit een constructed type (zie ook hoofdstuk \ref{ch:types}). Daarmee is het lexicon dus een mengeling van een taalkundig en een formeel vocabularium.

\paragraph{}Een logigram kan 3 soorten eigennamen hebben: een eigennaam in het enkelvoud (bv. ``John''), een eigennaam in het meervoud (bv. ``The Turkey Rolls'') en een \textit{numerieke eigennaam}. Bij de eerste twee wordt het symbool afgeleid van de woordvorm. Bij de laatste gebeurt dit door de gebruiker. Numerieke eigennamen worden namelijk gebruikt om woorden om te zetten in getallen. Zo kan ``March'' omgezet worden in 3. Op die manier wordt het zinvol om te spreken over ``1 maand na maart''. Voor een \textit{numerieke eigennaam} is het symbool gelijk aan die numerieke waarde. Deze wordt apart meegegeven in het probleem-specifiek lexicon. Deze vertaling van woorden naar getallen moet door de gebruiker gebeuren omdat hier achtergrond kennis voor nodig is. Het is opnieuw een voorbeeld van hoe onze vertaling van een logigram deels in het lexicon zit.
%De numerieke eigennamen zullen geen aanleiding geven tot constanten in constructed types. Het is wel een andere voorbeeld van hoe de vertaling van de logigram in het lexicon kruipt.

\section{Substantief}
Ook substantieven zijn een open categorie. Hun semantiek nemen we voorlopig over van het vorige hoofdstuk. $$\sem{n} = \lambdaf{x}{\drs{}{\textit{Symbool}(x)}}$$ In hoofdstuk~\ref{ch:types} (over types) zullen we DRS uitbreiden met types en zal het predicaat op x verdwijnen en vervangen worden door een echte type-constraint in DRS. Een substantief in het logigram-specifiek lexicon bevat een enkelvoudsvorm en een meervoudsvorm. Het symbool is gelijk aan de enkelvoudsvorm. Op die manier is het symbool voor het enkelvoud en het meervoud gelijk.

\section{Voorzetsel}
De voorzetsels (in het Engels \texttt{prepositions} of \texttt{prep}) vormen een gesloten woordklasse die bestaat uit woorden zoals ``from'', ``in'', ``with''. Ze worden op twee manieren gebruikt in de zinnen van een logigram. Enerzijds bij een werkwoord. Dan staat het vlak voor het lijdend voorwerp. Bijvoorbeeld de ``with'' in ``to finish with 500 points''. Anderzijds als belangrijkste woord in een voorzetselconstituent (ook wel \texttt{prepositional phrase} of \texttt{pp} genoemd). Bijvoorbeeld de ``from'' in ``the man from France''. Het voorzetsel dat bij een werkwoord hoort, zien we als deel van het werkwoord. In dat geval heeft het voorzetsel geen vertaling.

In het geval van een voorzetselconstituent is er wel een vertaling. We kunnen zo'n voorzetselconstituent zien als een extra beperking op een substantief. Of een transformatie van een substantief naar een nieuw substantief ($\tau(pp) = \tau(n) \rightarrow \tau(n)$). Een voorzetsel is het belangrijkste woord in zo'n voorzetselconstituent. Men kan het dus zien als een functie van een naamwoordgroep (\texttt{noun phrase} of \texttt{np}) naar een voorzetselconstituent. Of in formulevorm $\tau(prep) = \tau(np) \rightarrow \tau(pp) = \tau(np) \rightarrow \tau(n) \rightarrow \tau(n) = [(e \rightarrow t) \rightarrow t] \rightarrow (e \rightarrow t) \rightarrow (e \rightarrow t)$. De betekenis ziet er uit als
$$\sem{prep} = \lambdaf{N}{\lambdaf{S}{\left( \lambdaf{y}{\drsMerge{\app{S}{y}}{\app{N}{\lambdaf{x}{\drs{}{\textit{Symbool}(y, x)}}}}}} \right)}$$

Een voorzetsel neemt een naamwoordgroep $N$ en een substantief $S$ als argument en geeft een beperking op een entiteit $y$ terug. De beperking op $y$ bestaat enerzijds uit de beperking van het substantief ($\app{S}{y}$). Anderzijds voegt het voorzetsel zelf ook nog een beperking toe. Hiervoor haalt het de entiteit $x$ uit de box van de naamwoordgroep $N$ en linkt het de entiteiten $x$ en $y$ via een binair predicaat met dezelfde naam als het voorzetsel in kwestie. Het symbool van een voorzetsel is namelijk gelijk aan het voorzetsel zelf.

\section{Betrekkelijk voornaamwoord}
Een betrekkelijk voornaamwoord (\texttt{relative pronoun} of \texttt{relpro}) is een woord aan het begin van een betrekkelijke bijzin (ook wel \texttt{relative clause} of \texttt{rc}). Voorbeelden zijn ``that'', ``which'' en ``who''. Net als een voorzetselconstituent staat zo'n betrekkelijke bijzin bij een substantief en legt ze een extra beperking op.

$$\sem{relpro} = \lambdaf{V}{\lambdaf{S}{\left( \lambdaf{x}{\drsMerge{\app{S}{x}}{\app{V}{\lambdaf{B}{\app{B}{x}}}}} \right)}}$$
Een betrekkelijk voornaamwoord neemt een verbale constituent $V$ en een substantief $S$ als argument en geeft een beperking op een entiteit $x$ terug. De beperkingen op $x$ bestaan enerzijds uit de beperking van het substantief $S$ (namelijk $\app{S}{x}$) en anderzijds uit de beperkingen van de verbale constituent. De verbale constituent heeft $x$ als onderwerp dus $V$ krijgt als argument een lege box rond $x$ mee ($\lambdaf{B}{\app{B}{x}}$)

\section{Transitief werkwoord}
Een transitief werkwoord is een werkwoord met een lijdend voorwerp. Een logigram heeft enkel transitieve werkwoorden. Er zijn geen intransitieve (zonder lijdend voorwerp) of ditransitieve werkwoorden (met een meewerkend voorwerp). De vertaling van een transitief werkwoord is gelijk aan die van het vorige hoofdstuk $$\sem{tv} = \lambdaf{N1}{\lambdaf{N2}{\app{N2}{\lambdaf{x2}{\app{N1}{\lambdaf{x1}{\drs{}{\textit{Symbool}(x2, x1)}}}}}}}$$ Het lijdend voorwerp ($N1$) en het onderwerp ($N2$) zijn de argumenten van het werkwoord. Eerst wordt de box van het onderwerp uitgepakt. Daarbinnen wordt dan weer de box van het lijdende voorwerp uitgepakt. De kwantor van het onderwerp komt dus altijd voor die van het lijdend voorwerp (met deze vertaling). Ten slotte worden de entiteiten van het onderwerp ($x2$) en het lijdend voorwerp ($x1$) gebonden via een predicaat met het symbool van het werkwoord.

\paragraph{}Een transitief werkwoord is een open lexicale categorie. Dat wil zeggen dat de woorden verschillend zijn per logigram. In het logigram-specifiek lexicon staat de infinitief, de werkwoordsvorm in de derde persoon enkelvoud alsook een voltooid of onvoltooid deelwoord. De enkelvoudsvorm kan zowel in de verleden tijd als de tegenwoordige tijd zijn, afhankelijk van hoe het werkwoord gebruikt wordt in de logigram. Ten slotte wordt ook nog het voorzetsel gegeven dat voor het lijdend voorwerp wordt gezet (indien van toepassing) en eventueel een achtervoegsel aan het einde van de zin (bv. ``to print'' in ``The design took 8 minutes to print''). Het symbool van het werkwoord (en dus ook de naam van het predicaat) wordt afgeleid uit de enkelvoudsvorm, het voorzetsel en het achtervoegsel.

\paragraph{} Net zoals met de eigennamen wordt er vrij los omgesprongen met de werkwoorden. Zo is ``to be recognized as endangered in'' een werkwoord in één van de logigrammen.

\section{Hulpwerkwoord}
De woordklasse van hulpwerkwoorden is een gesloten woordklasse. Binnen de logigrammen zijn het de werkwoordsvormen van ``to do'' en ``to be'' die deel uitmaken van deze klasse. Alsook het hulpwerkwoord voor de toekomst ``will''. Naast de woordvorm bevat het lexicon ook informatie over de polariteit van die woordvorm: positief of negatief. ``does'' is positief, ``doesn't'' is negatief. De beide polariteiten hebben elk een andere betekenis.

Een hulpwerkwoord is een woord dat een verbale constituent (\texttt{verb phrase} of \texttt{vp}) omvormt tot een nieuwe verbale constituent. Voor een hulpwerkwoord met positieve polariteit is dit de identieke transformatie \footnote{Merk op dat de tijd niet uitmaakt voor een logigram en ``will clean'' dus niet anders vertaald moet worden dan ``cleans''}. $$\sem{av_{pos}} = \lambdaf{V}{V}$$ Voor een hulpwerkwoord met een negatieve polariteit bestaat de transformatie uit een negatie van de verbale constituent. Merk op dat er dus geen negatie is van het onderwerp. Hiermee wordt de vertaling van ``Everyone doesn't work'' naar logica $\forall x \cdot \lnot work(x)$ i.p.v. $\lnot \forall x \cdot work(x)$

$$\sem{av_{neg}} = \lambdaf{V}{\lambdaf{N}{\app{N}{\lambdaf{x}{\drsNot{\app{V}{\lambdaf{P}{\app{P}{x}}}}}}}}$$

Het hulpwerkwoord krijgt een verbale constituent ($V$) en een onderwerp ($N$) als argument. Het haalt de entiteit van het onderwerp uit de box en geeft dan een negatie van de verbale constituent terug. Die verbale constituent krijgt een lege doos met enkel de entiteit in ($\lambdaf{P}{\app{P}{x}}$) mee als ``onderwerp''.

\section{Koppelwerkwoord}
\label{sec:lex-koppelwerkwoord}
De categorie van koppelwerkwoorden is een gesloten lexicale categorie. Ze bestaat uit verschillende vormen van het werkwoord ``to be'' (bv. is, isn't, is not, was, are, were, ...). Er is een enkelvoud- en meervoudsvorm. Bovendien is er sprake van een positieve of negatieve polariteit. Deze hebben elk een licht andere semantiek. Ten slotte kan een koppelwerkwoord ook op drie verschillende manieren gebruikt worden in een zin van een logigram:

\begin{itemize}
  \item Samen met een \texttt{nominale constituent} (\texttt{noun phrase} of \texttt{np}): Bijvoorbeeld ``John is a man''. Dit type van gebruik heeft dezelfde signatuur als een overgankelijk werkwoord. De semantiek zegt dat de twee referenties die in de box van het onderwerp en het lijdend voorwerp zitten, gelijk zijn.
  $$\sem{cop_{np, pos}} = \lambdaf{N1}{\lambdaf{N2}{\app{N2}{\lambdaf{x2}{\app{N1}{\lambdaf{x1}{\drs{}{x2 = x1}}}}}}}$$
  In negatie wordt dit een ongelijkheid. De negatie bevindt zich echter al voor het ``uitpakken'' van de box van het lijdend voorwerp. Zodanig dat er een correcte negatie van de kwantoren is. In de zin ``Mary is not a man'' is het belangrijk dat er geen enkele man is die gelijk is aan Mary. De negatie moet dus voor de existentiële kwantor komen en dus voor het uitpakken van de box van het lijdend voorwerp.
  $$\sem{cop_{np, neg}} = \lambdaf{N1}{\lambdaf{N2}{\app{N2}{\lambdaf{x2}{\drsNot{\app{N1}{\lambdaf{x1}{\drs{}{x2 = x1}}}}}}}}$$
  \item Samen met een \texttt{adjectiefconstituent} (\texttt{adjective phrase} of \texttt{ap}): Bijvoorbeeld ``John is 30 years old''. De vertaling die wij gebruiken ligt echter vrij ver van de taalkundige structuur. Zo wordt het koppelwerkwoord + adjectief gezien als een soort van transitief werkwoord. De betekenis is gelijkaardig aan die van het koppelwerkwoord met een nominale constituent. De adjectieven die gebruikt kunnen worden, moeten meegegeven worden via het lexicon van een logigram. Het symbool komt overeen met de woordvorm van het adjectief.
  $$\sem{cop_{ap, pos}} = \sem{tv} = \lambdaf{N1}{\lambdaf{N2}{\app{N2}{\lambdaf{x2}{\app{N1}{\lambdaf{x1}{\drs{}{\textit{Symbool}(x2, x1)}}}}}}}$$
  $$\sem{cop_{ap, neg}} = \lambdaf{N1}{\lambdaf{N2}{\app{N2}{\lambdaf{x2}{\drsNot{\app{N1}{\lambdaf{x1}{\drs{}{\textit{Symbool}(x2, x1)}}}}}}}}$$
  \item Samen met een \texttt{voorzetselconstituent} (\texttt{prepositional phrase} of \texttt{pp}): Bijvoorbeeld ``John is from France''. Dit kan aanzien worden als een ellips van ``a person'': ``John is a person from France''. De signatuur is $ \tau(cop_{pp}) = \tau(pp) \rightarrow \tau(vp) = \tau(pp) \rightarrow \tau(np) \rightarrow t = [(e \rightarrow t) \rightarrow e \rightarrow t] \rightarrow [(e \rightarrow t) \rightarrow t] \rightarrow t$.

  $$\sem{cop_{pp, pos}} = \lambdaf{Q}{\lambdaf{N}{\app{N}{\lambdaf{x}{\app{\app{Q}{\lambdaf{y}{\drs{}{}}}}{x}}}}}$$
  De voorzetselconstituent $Q$ heeft als eerste argument een lambda-functie die een beperking oplegt vanuit het substantief waarbij het hoort. In dit geval is er echter geen substantief en dus geen beperking. Dit vertaalt zich in een lege DRS-structuur. Het tweede argument, de entiteit die beperkt wordt door de voorzetselconstituent, is de entiteit uit het onderwerp. Die moeten we dus eerst uitpakken uit de naamwoordgroep $N$.

  De betekenis van de negatieve vorm is $$\sem{cop_{pp, neg}} = \lambdaf{Q}{\lambdaf{N}{\app{N}{\lambdaf{x}{\drsNot{\app{\app{Q}{\lambdaf{y}{\drs{}{}}}}{x}}}}}}$$
\end{itemize}

\section{Comparatief en onbepaalde woorden}
Binnen een logigram is er altijd minstens één concept van numerieke aard. Vaak is er dan ook een zin die uitdrukt dat iemand een hogere of lagere numerieke waarde heeft dan iemand anders. Bijvoorbeeld ``John scored 15 points less than Mary'' of ``Mary finished sometime after Tom''. De woordgroepen ``less than'' en ``after'' zijn deel van de lexicale categorie \texttt{comparatief}. ``sometime'' is dan weer een voorbeeld van een onbepaald woord. Wat op het eerste zicht niet opvalt aan deze zinnen is dat ze een belangrijke ellips bevatten. Tom is namelijk geen tijdstip maar een persoon. De tweede zin is voluit ``Mary finished sometime after Tom finished''. Deze ellips zal opgelost worden in de grammatica.

%Een comparatief is een functie die twee naamwoordgroepen als argument neemt en een nieuwe naamwoordgroep maakt die als lijdend voorwerp kan dienen. $\tau(comp) = \tau(np) \rightarrow \tau(np) \rightarrow \tau(np) = [(e \rightarrow t) \rightarrow t] \rightarrow [(e \rightarrow t) \rightarrow t] \rightarrow [(e \rightarrow t) \rightarrow t]$. 
\paragraph{} Er zijn opnieuw 2 betekenissen afhankelijk van het woord: één voor ``hoger'' (bv. ``after'') en één voor ``lager'' (bv. ``less than'')
$$\sem{comp_{lager}} = \lambdaf{N1}{\lambdaf{N2}{\left( \lambdaf{P}{\app{N1}{\lambdaf{x1}{\app{N2}{\lambdaf{x2}{\drsMerge{\drs{x}{x = x2 - x1}}{\app{P}{x}}}}}}} \right)}}$$
Een comparatief neemt twee naamwoordgroepen ($N1$ en $N2$) als argument. Het resultaat is een nieuwe naamwoordgroep die kan dienen als lijdend voorwerp. Een naamwoordgroep is van type $(e \rightarrow t) \rightarrow t$ en dus krijgen we nog een argument $P$ van type $e \rightarrow t$ mee. We kunnen $\lambda P$ ook zien als het aanmaken van een box waar we later een nieuwe entiteit kunnen insteken. Daarna kunnen we de naamwoordgroepen uitpakken. We doen dit in volgorde van het voorkomen in de zin (dus eerst $N1$ en dan $N2$). Vervolgens maken we een nieuwe entiteit $x$ aan. Deze is gelijk aan het verschil $x2-x1$ (bijvoorbeeld de score van Mary - 15). Ten slotte stoppen we de nieuwe entiteit $x$ in de box van onze resulterende naamwoordgroep.

De andere betekenis van de comparatief, die voor ``hoger'', is gelijkaardig aan de eerste met dat verschil dat $x = x2+x1$ i.p.v. $x = x2-x1$.

De comparatief is een open lexicale categorie. Elke logigram heeft eigen comparatieven. In het logigram-specifiek lexicon bevindt zich naast de woordvorm ook telkens het type (``hoger'' of ``lager'').

\paragraph{} De onbepaalde woorden hebben dezelfde signatuur als een naamwoordgroep $\tau(some) = \tau(np) = (e \rightarrow t) \rightarrow t$. We kunnen het zien als een box rond een existentiële kwantor voor een natuurlijk getal \footnote{Het getal moet strikt positief zijn want anders kan $x=x2-x1$ ook hoger dan $x2$ uitkomen bij een comparatief met als betekenis ``lager dan''}. Meer concreet hebben we $$\sem{some} = \lambdaf{P}{\drsMerge{\drs{x}{x > 0}}{\app{P}{x}}}$$ %We maken een box aan ($\lambda P$). Vervolgens maken we een entiteit $x$ aan dat een positief getal ($x > 0$) voorstelt. (Het getal moet strikt positief zijn want anders kan $x=x2-x1$ ook hoger dan $x2$ uitkomen bij een comparatief met als betekenis ``lager dan''). Ten slotte stoppen we de entiteit in de box ($\app{P}{x}$).

\section{Voegwoord}
Een voegwoord is een woord dat twee woordgroepen van dezelfde categorie verbindt. Voorbeelden zijn ``and'', ``or'' and ``nor''. Een voegwoorden kan twee zinnen verbinden maar ook twee nominale constituenten. Binnen logigrammen zijn de voegwoorden nevenschikkend. Dat wil zeggen dat beide woordgroepen even belangrijk zijn \footnote{In tegenstelling tot een onderschikkend voegwoord. Zo'n voegwoord verbindt bijvoorbeeld een hoofdzin met een bijzin. Daarbij is de bijzin minder belangrijk dan de hoofdzin.}. Er komen drie soorten voegwoorden voor, elk met een eigen vertaling: een conjunctief voegwoord (``A and B''), een disjunctief voegwoord (``A or B'') en een negatief voegwoord (``A nor B''). Een voegwoord kan uit twee delen bestaan (bv. ``either ... or'' en ``neither ... nor''). Daarom bestaan er twee lexicale categorieën: \texttt{coord} voor het voegwoord zelf (bv. ``or'') en \texttt{coordPrefix} voor de eventuele prefix (bv. ``either'').

De vertaling van het voegwoord is onafhankelijk van de categorie van de woordgroepen die worden verbonden. Belangrijk is wel dat de signatuur van die woordgroepen een functie met als resultaat iets van type $t$ is ($\tau(woordgroep) = \alpha \rightarrow t$). Een signatuur van het voegwoord is dan $\tau(coord) = (\alpha \rightarrow t) \rightarrow (\alpha \rightarrow t) \rightarrow (\alpha \rightarrow t) = (\alpha \rightarrow t) \rightarrow (\alpha \rightarrow t) \rightarrow \alpha \rightarrow t $. De betekenis van een voegwoord bestaat er telkens uit door het derde argument (van type $\alpha$) door te geven aan de eerste twee argumenten en de resultaten op de juiste manier terug te combineren

$$\sem{coord_{conjunctief}} = \lambdaf{A}{\lambdaf{B}{\left( \lambdaf{C}{\drsMerge{\app{A}{C}}}{\app{B}{C}} \right)}}$$
$$\sem{coord_{disjunctief}} = \lambdaf{A}{\lambdaf{B}{\left( \lambdaf{C}{\drs{}{\app{A}{C} \lor \app{B}{C}}} \right)}}$$
$$\sem{coord_{negatief}} = \lambdaf{A}{\lambdaf{B}{\left( \lambdaf{C}{\drsMerge{\drsNot{\app{A}{C}}}}{\drsNot{\app{B}{C}}} \right)}}$$

Deze vertaling is echter niet de enigste mogelijke vertaling. In het geval van een disjunctie kan men ook naar een exclusieve of vertalen (zeker voor de constructie ``either ... or ...'' houdt dit steek). Omwille van de bijecties bij logigrammen maakt dit echter geen verschil. Er is immers altijd maar exact één iemand die eraan voldoet dus er is in praktijk geen verschil tussen een inclusieve en exclusieve disjunctie.

\paragraph{} Echter ook voor de conjunctie is een andere vertaling mogelijk, meer bepaald voor de conjunctie van twee naamwoordgroepen. Zo heeft de zin ``John and Mary went to the sea'' twee mogelijke vertalingen. De bovenstaande vertaling wordt de distributieve lezing genoemd en komt overeen met dat beiden (apart) naar de zee zijn gegaan. In de andere vertaling, de collectieve lezing, zijn John en Mary samen naar de zee gegaan. Binnen logigrammen komt de collectieve lezing bijna niet aan bod. De enigste plaats waar dat wel gebeurt, is in een soort van \textit{alldifferent}-constraint. Bijvoorbeeld ``John, Bob and Charles are three different politicians''. Bij de vertaling van de grammatica lossen we dit probleem op.

\chapter{Een grammatica voor logigrammen}
\label{ch:grammatica}

In dit hoofdstuk overlopen we de gebruikte grammaticale categorieën en de grammaticale regels die erbij horen, samen met hun semantiek. We beginnen bij de categorieën die bestaan uit lexicale categorieën en werken naar boven toe richting de categorie van een zin. We gebruiken de DCG-notatie uit hoofdstuk~\ref{sec:DCG}. De feature \texttt{sem} staat voor de semantiek van een woordgroep.

\paragraph{}De grammatica is opgesteld vertrekkend van code van Blackburn en Bos \cite{Blackburn2006} en op basis van de eerste 10 logigrammen uit Puzzle Baron's Logic Puzzles Volume 3 \cite{logigrammen}. De grammatica regels van Blackburn en Bos die niet van toepassing zijn voor logigrammen zijn verwijderd.

\section{Lexicale categorieën}
\subsection{Een mapping naar het lexicon}
\label{sec:lexgram}
De meeste van de grammaticale categorieën die de link vormen met het lexicon volgend de structuur uit grammatica~\ref{dcg:lexcat}. Hierbij staat \textit{cat} voor de categorie in kwestie. De grammaticale regel bestaat uit het opzoeken van een woord in het lexicon (m.b.v. de functie \texttt{lexEntry}). Hierbij kijken we niet alleen naar de lexicale categorie maar ook naar andere features zoals getal; controleren dat het woord uit het lexicon het volgende woord is in de zin (lijn 3); en tenslotte het opzoeken van de betekenis van het woord (m.b.v. de functie \texttt{semLex}). Dit resulteert in de formules van hoofdstuk~\ref{ch:lexicon}.

\begin{dcg}{Een grammaticale regel voor een lexicale categorie \textit{cat}}{dcg:lexcat}
cat([feature1:Feature1, type:Type, sem:Sem]) -->
  { lexEntry(cat, [feature1:Feature1, type:Type, syntax:Word]) },
  Word,
  { semLex(cat, [feature1:Feature1, type:Type, sem:Sem]) }.
\end{dcg}

De categorieën die deze structuur volgen zijn determinatoren (\texttt{det}), hoofdtelwoorden (\texttt{number}), substantieven (\texttt{noun}), voorzetsels (\texttt{prep}), hulpwerkwoorden (\texttt{av}), koppelwerkwoorden (\texttt{cop}) en comparatieven (\texttt{comp}).

\paragraph{}Bijvoorbeeld voor een substantief wordt dit grammatica~\ref{dcg:noun}. Het getal van de grammaticale categorie komt overeen met het getal uit het lexicon. Het symbool dat wordt gebruikt in de betekenis van een woord komt ook uit het lexicon.
\begin{dcg}{De grammaticale regel voor een substantief}{dcg:noun}
noun([num:Num, sem:Sem]) -->
  { lexEntry(noun, [symbol:Sym, num:Num, syntax:Word]) },
  Word,
  { semLex(noun, [symbol:Sym, sem:Sem]) }.
\end{dcg}

\subsection{Eigennaam}
Grammatica~\ref{dcg:pn} geeft de grammaticale regel voor een eigennaam weer. Dit lijkt zeer sterk op de algemene mapping van de grammatica naar het lexicon buiten de optionele ``the''. Deze is nodig omdat in sommige puzzels een bepaalde term zowel met als zonder ``the'' voorkomt. Indien we deze twee verschillende termen allebei apart in het lexicon zouden ingeven, zouden deze worden vertaald naar verschillende symbolen. Dit zouden we graag vermijden \footnote{Een alternatief was om de twee vormen met hetzelfde symbool in het lexicon op te nemen. Dat is equivalent aan onderstaande grammatica}.

\begin{dcg}{De grammaticale regel voor een eigennaam}{dcg:pn}
pn([num:Num, sem:Sem]) -->
  { lexEntry(pn, [symbol:Sym, syntax:Word, num:Num]) },
  optional([the]),
  Word,
  { semLex(pn, [symbol:Sym, sem:Sem]) }.
optional(X) -->
  X.
optional(X) -->
  [].
\end{dcg}

\subsection{Transitief werkwoord}
\label{sec:gramm-tv}
Er zijn twee grammaticale regels voor een transitief werkwoord (grammatica~\ref{dcg:tv}). Enerzijds is er de standaard mapping van grammatica naar lexicon. De voorzetsels en achtervoegsels uit het lexicon worden als feature meegegeven aan de grammaticale woordgroep. Het getal (\texttt{num}) en de vorm (\texttt{inf}) van de grammaticale woordgroep moet ook overeenkomen met die uit het lexicon.

Anderzijds wordt de combinatie koppelwerkwoord + adjectief ook als een transitief werkwoord gezien. Hierbij is het adjectief het achtervoegsel. De semantiek van deze combinatie is die van het koppelwerkwoord ($\sem{cop_{ap}}$) zoals we die hebben afgeleid in hoofdstuk~\ref{sec:lex-koppelwerkwoord}.
\begin{dcg}{De grammaticale regels voor een transitief werkwoord}{dcg:tv}
tv([inf:Inf, num:Num, positions:Pre-Post, sem:Sem]) -->
  { lexEntry(tv, [symbol:Sym, syntax:Word-Pre-Post, inf:Inf, num:Num]) },
  Word,
  { semLex(tv, [symbol:Sym, sem:Sem]) }.

tv([inf:Inf, num:Num, positions:[]-Post, sem:Sem]) -->
  cop([type:ap, inf:Inf, num:Num, sem:Sem, symbol:Sym]),
  { lexEntry(copAdj, [symbol:Sym, adj:Post]) }.
\end{dcg}

\subsection{Onbepaalde woorden en betrekkelijke voornaamwoorden}
Het is mogelijk om een onbepaald woord te verzwijgen, bijvoorbeeld ``John finished [sometime] before Mia''. Daarom is er een extra regel die zegt dat het woord niet per se hoeft voor te komen in de zin (lijn 5 en 6). Hetzelfde geldt voor betrekkelijke voornaamwoorden.

\begin{dcg}{De grammaticale regels voor onbepaalde woorden en betrekkelijke voornaamwoorden}{dcg:someAndRelpro}
some([sem:Sem]) -->
  { lexEntry(some, [syntax:Word]) },
  Word,
  { semLex(some, [sem:Sem]) }.
some([sem:Sem]) -->
  { semLex(some, [sem:Sem]) }.

relpro([sem:Sem]) -->
  { lexEntry(relpro, [syntax:Word]) },
  Word,
  { semLex(relpro, [sem:Sem]) }.
relpro([sem:Sem]) -->
  { semLex(relpro, [sem:Sem]) }.
\end{dcg}

\subsection{Voegwoord}
Voegwoorden kunnen uit twee delen bestaan: het voegwoord zelf (\texttt{coord}) en een optionele prefix (\texttt{coordPrefix}). De prefix heeft geen vertaling bovenop het voegwoord. Dit zien we ook terug in de eerste grammaticale regel van grammatica~\ref{dcg:coord}. De tweede grammaticale regel stelt ook dat het optioneel is.

De grammaticale regel voor het voegwoord zelf volgt dezelfde structuur als die van sectie~\ref{sec:lexgram}. Een voegwoord kan soms echter weggelaten worden. De grammaticale categorie \texttt{noCoord} stelt een ellips van zo'n voegwoord voor. Er is gekozen voor een aparte categorie omdat een ellips niet overal mag voorkomen.
\begin{dcg}{De grammaticale regels i.v.m. voegwoorden}{dcg:coord}
coordPrefix([type:Type]) -->
  { lexEntry(coordPrefix, [syntax:Word, type:Type]) },
  Word.
coordPrefix([type:_]) -->
  [].

coord([type:Type, sem:Sem]) -->
  { lexEntry(coord, [syntax:Word, type:Type]) },
  Word,
  { semLex(coord, [type:Type, sem:Sem]) }.

noCoord([type:Type, sem:Sem]) -->
  { semLex(coord, [type:Type, sem:Sem]) }.
\end{dcg} 

\section{(Getransformeerde) Substantieven}
Grammatica~\ref{dcg:n} geeft een overzicht van de grammatica regels i.v.m. substantieven en transformaties op die substantieven. De eerste 2 regels zeggen dat een getransformeerd substantief (hiervoor gebruiken we de categorie \texttt{n}) kan bestaan uit een gewoon substantief (\texttt{noun}) al dan niet gevolgd door een transformatie op dat substantief (\texttt{nmod} naar \texttt{noun modifier}). In het geval van een transformatie is de betekenis van het getransformeerde substantief gelijk aan de applicatie van de transformatie op dat van het echte substantief of $\sem{n} = \app{\sem{nmod}}{\sem{noun}}$.

Lijnen 9 t.e.m. 14 van grammatica~\ref{dcg:n} definiëren de mogelijke transformaties: ofwel gaat het om een voorzetselconstituent (\texttt{pp}), ofwel om een betrekkelijke bijzin (\texttt{rc}). In het laatste geval moet het getal van het substantief en het werkwoord uit de bijzin overeenkomen.

\begin{dcg}{De grammaticale regels i.v.m. substantieven}{dcg:n}
n([num:Num, sem:Sem]) -->
  noun([num:Num, sem:Noun]),
  { Sem = Noun }.
n([num:Num, sem:Sem]) -->
  noun([num:Num, sem:Noun]),
  nmod([num:Num, sem:NMod]),
  { Sem = app(NMod, Noun) }.

nmod([num:_, sem:Sem]) -->
  pp([sem:PP]),
  { Sem = PP }.
nmod([num:Num, sem:Sem]) -->
  rc([num:Num, sem:RC]),
  { Sem = RC }.

pp([sem:Sem]) -->
  prep([sem:Prep]),
  np([coord:_, num:_, gap:[], sem:NP]),
  { Sem = app(Prep, NP) }.

rc([num:Num, sem:Sem]) -->
  relpro([sem:RelPro]),
  vp([coord:no, inf:fin, num:Num, gap:[], sem:VP]),
  { Sem = app(RelPro, VP) }.
\end{dcg}

\paragraph{} Lijnen 16 t.e.m. 19 beschrijven de voorzetselconstituent (\texttt{pp}) zoals ``from France''. Deze constituent bestaat namelijk uit een voorzetsel (\texttt{prep}) gevolgd door een nominale constituent (\texttt{np}). De feature \texttt{gap} van de nominale constituent heeft te maken met ellipsen die kunnen voorkomen in die constituent. De lege lijst wilt zeggen dat er geen ellipsen mogelijk zijn.

De betekenis van een voorzetselconstituent bestaat uit een simpele lambda-applicatie ($\sem{pp} = \app{\sem{prep}}{\sem{np}}$).

\paragraph{} Lijnen 21 t.e.m. 24 beschrijven de betrekkelijke bijzin (\texttt{rc}). Zo'n betrekkelijke bijzin bestaat uit een betrekkelijk voornaamwoord (\texttt{relpro}) gevolgd door een verbale constituent. Een voorbeeld van zo'n betrekkelijke bijzin is ``that loves Mary''. De verbale constituent moet simpel zijn (\texttt{coord:no}). Dat wil zeggen dat er geen voegwoorden mogen in voorkomen. Bovendien moet het werkwoord vervoegd zijn (\texttt{inf:fin}) en dus niet in de infinitief voorkomen. Ten slotte moet het getal overeenkomen met dat van het substantief.

De betekenis bestaat opnieuw uit één lambda-applicatie, namelijk $\sem{rc} = \app{\sem{relpro}}{\sem{vp}}$.

\section{Nominale constituent}
Een nominale constituent of naamwoordgroep (\texttt{np}) is een woordgroep waarin het naamwoord het belangrijkste woord is en die een entiteit aanduidt. Deze woordgroepen kunnen dienen als onderwerp of als lijdend voorwerp van een werkwoord.

De grammaticale categorie heeft 4 mogelijke features: De feature \texttt{coord} duidt aan of het gaat om een simpele naamgroep (\texttt{coord:no}) of een complexere naamgroep. Zo heeft een naamwoordgroep met een vergelijking de waarde \texttt{comp} voor deze feature. De feature \texttt{num} duidt het getal van de woordgroep aan (enkelvoud \texttt{sg} of meervoud \texttt{pl}). De feature \texttt{gap} geeft aan welke woorden gebruikt zijn in een ellips binnen deze woordgroep. De waarde \texttt{[]} voor deze feature wilt dan weer zeggen dat er geen ellips zit in deze naamwoordgroep. Zoals altijd zit de semantiek in de feature \texttt{sem}.

\subsection{Een eenvoudige naamwoordgroep}
Grammatica~\ref{dcg:np1} geeft de eenvoudige naamwoordgroepen weer. De eerste grammaticale regel stelt dat een naamwoordgroep kan bestaan uit een eigennaam (\texttt{pn}) met hetzelfde getal. In dit geval is de semantiek gelijk aan die van de naamwoordgroep.

Daarnaast kan een naamwoordgroep ook bestaan uit een lidwoord (of meer algemeen, een determinator \texttt{det}) en een zelfstandig naamwoord. We gebruiken hier de categorie \texttt{n} i.p.v. \texttt{noun} zodat ook een zelfstandig naamwoord met een bijzin mogelijk is. De semantiek bestaat uit een simpele lambda-applicatie $\sem{np} = \app{\sem{det}}{\sem{n}}$.

Zoals we ook in sectie~\ref{sec:lex-number} besproken hebben, kan een hoofdtelwoord ook aanzien worden als een determinator. We krijgen dus nog een gelijkaardige regel als de vorige maar nu met een hoofdtelwoord i.p.v. een determinator.

Ten slotte kan een getal ook op zichzelf staan zonder zelfstandig naamwoord, bijvoorbeeld als jaartal. De grammaticale regel van lijn 15 t.e.m. 17 dekt dit geval. De semantiek is vergelijkbaar met de vorige grammaticale regel maar er is geen extra restrictie op het getal. Vandaar de lege DRS-structuur in $\sem{np} = \app{\sem{number}}{\lambdaf{x}{\drs{}{}}}$.

\begin{dcg}{De grammaticale regels voor een simpele naamwoordgroep}{dcg:np1}
np([coord:no, num:Num, gap:[], sem:Sem]) -->
  pn([num:Num, sem:PN]),
  { Sem = PN }.

np([coord:no, num:Num, gap:[], sem:Sem]) -->
  det([num:Num, sem:Det]),
  n([num:Num, sem:N]),
  { Sem = app(Det, N) }.

np([coord:no, num:Num, gap:[], sem:Sem]) -->
  number([sem:Number]),
  n([num:Num, sem:N]),
  { Sem = app(Number, N) }.

np([coord:no, num:_, gap:[], sem:Sem]) -->
  number([sem:Number]),
  { Sem = app(Number, lam(_, drs([], []))) }.
\end{dcg}

\subsection{Een naamwoordgroep met onbekende relatie}
Grammatica~\ref{dcg:np2} beschrijft twee grammaticale regels met dezelfde semantiek en gelijkaardige structuur. Het gaat om naamwoordgroepen respectievelijk zoals ``the Arkansas native'' en ``John's dog''. Ze bestaan allebei uit een simpele naamwoordgroep (\texttt{np} met \texttt{coord:no}) en een substantief~(\texttt{n}). Er is telkens een link tussen de entiteit van het substantief en dat van de simpele naamwoordgroep maar er is geen kennis over welke link dit juist is. Voor nu laten we het dus nog open. In hoofdstuk~\ref{ch:types} zullen we met behulp van types afleiden welk predicaat nodig is. De entiteit van de substantief is het belangrijkste en is ook de entiteit van de naamwoordgroep als geheel. Als resultaat krijgen we $$\sem{np} = \lambdaf{P}{\appH{\app{\sem{det}}{\sem{n}}}{\lambdaf{x}{\app{\sem{np}}{\lambdaf{y}{\drsMerge{\drs{}{\textit{unknown}(x, y)}}{\app{P}{x}}}}}}}$$

Het is een naamwoordgroep en dus een box ($\lambda P$). Daarbinnen pakken we de woordgroep die bestaat uit het lidwoord en het substantief uit naar de entiteit $x$. Vervolgens pakken we de naamdwoordgroep uit naar $y$. Er is een relatie tussen $x$ en $y$ maar deze is voorlopig nog onbekend. Ten slotte stoppen we $x$ in de box van het gehele naamwoordgroep want $x$ is de entiteit waar de naamwoordgroep als geheel naar verwijst.

\paragraph{}Bij de tweede grammaticale regel is er geen expliciet lidwoord aanwezig maar de betekenis hiervan is wel nog steeds nodig om een scope te geven aan het zelfstandig naamwoord. Er wordt dus gebruik gemaakt van dezelfde betekenis alsof er wel een lidwoord had gestaan (lijn 10).

Beiden naamwoordgroepen krijgen de waarde \texttt{np} voor de feature \texttt{coord} naar de extra naamwoordgroep die zich in deze naamwoordgroep bevindt.

\begin{dcg}{De grammaticale regels voor een naamwoordgroep met een onbekende relatie}{dcg:np2}
np([coord:np, num:Num, gap:[], sem:Sem]) -->
  det([num:Num, sem:Det]),
  np([coord:no, num:_, gap:[], sem:NP]),
  n([num:Num, sem:N]),
  { Sem = lam(P, app(app(Det, N), lam(X, app(NP, lam(Y, merge(
    drs([], [rel(_, X, Y)]),
    app(P, X))))))) }.

np([coord:np, num:Num, gap:[], sem:Sem]) -->
  { semLex(det, [num:sg, sem:Det]) },
  np([coord:no, num:_, gap:[], sem:NP]),
  [s],
  n([coord:_, num:Num, sem:N]),
  { Sem = lam(P, app(app(Det, N), lam(X, app(NP, lam(Y, merge(
    drs([], [rel(_, X, Y)]),
    app(P, X))))))) }.
\end{dcg}

\subsection{Een vergelijkende naamwoordgroep}
\label{sec:gramNpComp}
Grammatica~\ref{dcg:np3} geeft de grammaticale regels voor een vergelijkende naamwoordgroep, zoals ``3 years after John''. Zo'n naamwoordgroep bestaat uit een hoeveelheid (\texttt{quantity}), een comparatief (\texttt{comp}) en een andere naamwoordgroep (\texttt{np}). De betekenis bestaat uit twee simpele lambda-applicaties $\sem{np} = \appH{\app{\sem{comp}}{\sem{np_1}}}{\sem{np_2}}$

Zo'n hoeveelheid kan onbepaald zijn (lijn 8-10) zoals ``sometime'' of een simpele naamwoordgroep (lijn 10-12) zoals ``2 years''.

\paragraph{} Deze naamwoordgroep heeft de waarde \texttt{comp} voor de feature \texttt{coord}. Die feature is nodig om niet in een oneindige lus te geraken. Een vergelijkende naamwoordgroep kan namelijk starten met een simpele naamwoordgroep (lijn 2 en 10-12). Zonder de feature \texttt{coord} zou een vergelijkende naamwoordgroep ook kunnen starten met een vergelijkende naamwoordgroep. Bij het zoeken of de volgende woorden een vergelijkende naamwoordgroep kan zijn, moeten we dus eerst kijken of ze een vergelijkende naamwoordgroep zijn. Bij het gebruik van een andere parser zou dit niet per se nodig zijn.

\paragraph{} Binnen logigrammen zit er ook altijd een ellips in zo'n vergelijkende naamwoordgroep, namelijk die van het hoofdwerkwoord binnen de naamwoordgroep waarmee vergeleken wordt. In de zin ``John finished after Mary'' ontbreekt het woord ``finished'' op het einde van de zin: ``John finished after Mary [finished]''. Lijn 1 en 5 zeggen dat de hele naamwoordgroep een ellips heeft a.s.a. de naamwoordgroep waarmee vergeleken wordt een ellips heeft.

Lijn 14 t.e.m. 20 geven dan weer de betekenis van zo'n naamwoordgroep met een ellips. In formulevorm wordt dit $$\sem{np} = \lambdaf{P}{\drsTriMerge{\drs{z}{}}{\appH{\app{\sem{tv}}{\lambdaf{P}{\app{P2}{z}}}}{\sem{np_1}}}{\app{P}{z}}}$$
De betekenis is een nieuwe entiteit $z$ in een box. Bovendien is er de conditie dat (een lege box rond) $z$ het lijdend voorwerp vormt van het (verzwegen) werkwoord en de onderliggende naamwoordgroep het onderwerp.

\begin{dcg}{De grammaticale regels i.v.m. een vergelijkende naamwoordgroep}{dcg:np3}
np([coord:comp, num:Num, gap:G, sem:Sem]) -->
  quantity([num:Num, sem:NP1]),
  comp([sem:Comp]),
  np([coord:_, num:_, gap:G, sem:NP2]),
  { Sem = app(app(Comp, NP1), NP2) }.

quantitity([num:_, sem:S]) -->
  some([sem:S]),
  { Sem = S }.
quantity([num:Num, sem:Sem]) -->
  np([coord:no, num:Num, gap:[], sem:NP]),
  { Sem = NP}.

np([coord:no, num:Num, gap:[useTVGap, tv:TV | G], sem:Sem]) -->
  np([coord:_, num:Num, gap:G, sem:NP]),
  { Sem = lam(P, merge(
      merge(
        drs([Z], []),
        app(app(TV, lam(N, app(N, Z))), NP)),
      app(P, Z))) }.
\end{dcg}

\subsection{Een naamwoordgroep met een voegwoord}
Ten slotte is er nog een naamwoordgroep met een voegwoord in of ook wel gecoördineerde naamwoodgroep genoemd. Grammatica~\ref{dcg:np4} geeft de twee grammaticale regels die dit soort naamwoordgroepen beschrijven. De eerste regel zegt dat een gecoördineerde naamwoordgroep kan bestaat uit een eventuele prefix van een voegwoord, een naamwoordgroep (een simpele of ééntje met een onbekende relatie), een voegwoord en opnieuw een naamwoordgroep (een simpele, ééntje met een onbekende relatie of een naamwoordgroep met een voegwoord van het zelfde type als het voegwoord van deze naamwoordgroep).

De tweede grammaticale regel is gelijkaardig maar behandelt het geval dat het voegwoord is verzwegen, zoals in ``John, Pete and Mary''. De naamwoordgroep ``John, Pete'' bevat geen voegwoord. De grammaticale regel is vrij gelijkaardig. De enige andere verandering is dat de tweede naamwoordgroep sowieso opnieuw gecoördineerd moet zijn. Het type van voegwoord in die tweede naamwoordgroep is ook het type van het voegwoord dat is verzwegen.

\paragraph{} Het getal van een gecoördineerde naamwoordgroep is afhankelijk van het type van het voegwoord (lijn 2, 12 en 20-22). Een conjunctief voegwoord (``and'') resulteert in een naamwoordgroep in het meervoud. De andere twee types resulteren in een naamwoordgroep in het enkelvoud.

De betekenis bestaat zoals bij zo vele grammaticale regels uit twee lambda-applicaties. $$\sem{np} = \app{\app{\sem{coord}}{\sem{np_1}}}{\sem{np_2}}$$

\begin{dcg}{De grammaticale regels voor een naamwoordgroep met een voegwoord}{dcg:np4}
np([coord:CoordType, num:CoordNum, gap:G, sem:Sem]) -->
  { npCoordNum(CoordType, CoordNum) },
  coordPrefix([type:CoordType]),
  { Coord1 = no ; Coord1 = np},
  np([coord:Coord1, num:sg, gap:G, sem:NP1]),
  coord([type:CoordType, sem:Coord]),
  { Coord2 = CoordType ; Coord2 = no ; Coord2 = np },
  np([coord:Coord2, num:_, gap:G, sem:NP2]),
  { Sem = app(app(Coord, NP1), NP2) }.

np([coord:CoordType, num:CoordNum, gap:G, sem:NP, vType:Type]) -->
  { npCoordNum(CoordType, CoordNum) },
  coordPrefix([type:CoordType]),
  { Coord1 = no ; Coord1 = np},
  np([coord:Coord1, num:sg, gap:G, sem:NP1, vType:Type]),
  noCoord([type:CoordType, sem:Coord]),
  np([coord:CoordType, num:_, gap:G, sem:NP2, vType:Type]),
  { Sem = app(app(Coord, NP1), NP2) }.

npCoordNum(conj, pl).
npCoordNum(disj, sg).
npCoordNum(neg, sg).
\end{dcg}

\section{Verbale constituent}
\subsection{Een simpele verbale constituent}
Grammatica~\ref{dcg:vp1} geeft de grammaticale regels voor een aantal simpele verbale constituent.

De eenvoudigste verbale constituent bestaat uit een transitief werkwoord, gevolgd door haar prefix en een naamwoordgroep (als lijdend voorwerp) en ten slotte het achtervoegsel van het werkwoord. Het getal van het werkwoord is ook deze van de verbale constituent. De betekenis is een lambda-applicatie $\sem{vp} = \app{\sem{tv}}{{\sem{np}}}$.

De tweede grammaticale regel drukt iets gelijkaardigs uit maar nu in het geval het lijdend voorwerp een vergelijking is (\texttt{coord:comp}). In dat geval zijn het voor- en achtervoegsel optioneel. Neem bijvoorbeeld de zinnen ``John won in 2008'' en ``Mary won 2 years after John''. Het voorvoegsel ``in'' verdwijnt in het geval van een vergelijkende naamwoordgroep. Bovendien is er een ellips van het werkwoord in de naamwoordgroep (zie ook sectie~\ref{sec:gramNpComp}). Dit reflecteert zich in de feature \texttt{gap} van de naamwoordgroep. 

Merk op dat we in de eerste grammaticale regel niet toestaan dat er een ellips is van het werkwoord binnen het lijdende voorwerp. Dit zou kunnen lijden tot extra betekenissen die niet nuttig zijn binnen een logigram.

\paragraph{} In bijzinnen kan het ook soms voorkomen dat er een naamwoordgroep voor het werkwoord komt te staan, bijvoorbeeld ``The horse that Mary won, ...''. In dat geval is de naamwoordgroep van de verbale constituent het onderwerp en de naamwoordgroep waar de bijzin bijstaat het lijdend voorwerp. De derde grammaticale regel (lijn 14-17) drukt dit uit. De semantiek is $$\sem{vp} = \lambdaf{N1}{\app{\app{\sem{tv}}{N1}}{\sem{np}}}$$

Een verbale constituent kan ook een hulpwerkwoord bevatten. In dat geval komt het getal en de vorm (infinitief, deelwoord, ...) van de constituent overeen met dat van het hulpwerkwoord. Het hoofdwerkwoord moet in zo'n constituent voorkomen als infinitief (\texttt{inf}) of als deelwoord (\texttt{part}). De semantiek is nogmaals een lambda-appliatie $$\sem{vp} = \app{\sem{av}}{\sem{vp_1}}$$

\begin{dcg}{De grammaticale regels voor een simpele verbale constituent}{dcg:vp1}
vp([coord:no, inf:I, num:Num, gap:G, sem:Sem]) -->
  tv([inf:I, num:Num, positions:Pre-Post, sem:TV]),
  Pre,
  np([coord:_, num:_, gap:G, sem:NP]),
  Post,
  { Sem = app(TV, NP) }.
vp([coord:no, inf:I, num:Num, gap:G, sem:Sem]) -->
  tv([inf:I, num:Num, positions:Pre-Post, sem:TV]),
  optional(Pre),
  np([coord:comp, num:_, gap:[tv:TV | G], sem:NP]),
  optional(Post),
  { Sem = app(TV, NP) }.

vp([coord:no, inf:I, num:Num, gap:[], sem:Sem]) -->
  np([coord:_, num:_, gap:[], sem:NP]),
  tv([inf:I, num:Num, positions:_-[], sem:TV]),
  { Sem = lam(N1, app(app(TV, N1), NP))}.

vp([coord:no, inf:Inf, num:Num, gap:[], sem:Sem]) -->
  av([inf:Inf, num:Num, sem:AV]),
  { Inf2 = inf ; Inf2 = part },
  vp([coord:_, inf:Inf2, num:_, gap:[], sem:VP]),
  { Sem = app(AV, VP) }.
\end{dcg}

\subsection{Een verbale constituent met koppelwerkwoord}
\label{sec:gram-koppelwerkwoord}
Sectie~\ref{sec:lex-koppelwerkwoord} maakte een onderscheid tussen drie betekenissen van een koppelwerkwoord afhankelijk van de categorie van het \textit{argument} van dat werkwoord. Grammatica~\ref{dcg:vp2} beschrijft de grammatica voor een koppelwerkwoord met een nominale of voorzetselconstituent. Sectie~\ref{sec:gramm-tv} beschreef reeds het geval waarbij een adjectiefzin het argument is.

De eerste twee grammaticale regels van grammatica~\ref{dcg:vp2} zeggen dat een verbale constituent kan bestaan uit een koppelwerkwoord met een nominale constituent of met een voorzetselconstituent. Het getal en de vorm van de verbale constituent komen overeen met die van het werkwoord. De betekenis is voor beide regels een simpele lambda-applicatie.

\paragraph{} De derde grammaticale regel drukt een \textit{alldifferent}-constraint uit. Deze is van toepassing voor zinnen als ``John and Mary are two different persons'' of ``John, the person with the horse and the man from France are all different people''. Deze verbale constituent bestaat uit een koppelwerkwoord; een getal of het woordje ``all''; het woord ``different''; en tenslotte en substantief. De betekenis is $$\sem{vp} = \lambdaf{N}{\app{N}{\lambdaf{x}{\drs{}{alldifferent(x)}}}}$$ Dit drukt uit dat de entiteit van het onderwerp deel is van een alldifferent constraint. Zoals we in sectie~\ref{sec:lex-coord} reeds aanhaalden is er geen collectieve lezing voor een naamwoordgroep. Een alldifferent constraint drukt echter uit dat er binnen het collectief van het onderwerp geen twee dezelfde entiteiten bevinden. We passen echter de vertaling van DRS-structuren naar eerste-orde-logica lichtjes aan om dit te omzeilen.

\[
  \left(\ \drs{x_1, \ldots, x_n}{
      \gamma_1 \\
      \ldots \\
      \gamma_m \\
      alldifferent(y_1) \\
     \ldots \\
      alldifferent(y_k)
    }\ \right)^{fo} = \left(\ \drs{x_1, \ldots, x_n}{
      \gamma_1 \\
      \ldots \\
      \gamma_m \\
      y_1 \neq y_2 \\
      \ldots \\
      y_1 \neq y_k \\
      alldifferent(y_2) \\
      \ldots \\
      alldifferent(y_k)
    }\ \right)^{fo} 
\]

M.a.w. alle alldifferent constraints die zich binnen één DRS-structuur bevinden, behoren tot één groep van entiteiten die allemaal verschillend zijn. Men kan dit ook schalen naar meerdere groepen door de groep te reïficeren binnen de alldifferent constraint (i.e. $alldifferent(g_i, x_j)$ i.p.v. $alldifferent(x_j)$).

Er wordt geen rekening gehouden met de betekenis van het getal of het koppelwerkwoord. ``John and Mary are three different persons.'' is dus een geldige zin, alhoewel de ``three'' niet overeenkomt met het aantal entiteiten in het onderwerp.

\begin{dcg}{De grammaticale regels voor een verbale constituent met een koppelwerkwoord}{dcg:vp2}
vp([coord:no, inf:Inf, num:Num, gap:[], sem:Sem]) -->
  cop([type:np, inf:Inf, num:Num, sem:Cop]),
  np([coord:_, num:_, gap:[], sem:NP]),
  { Sem = app(Cop, NP) }.

vp([coord:no, inf:Inf, num:Num, gap:[], sem:Sem]) -->
  cop([type:pp, inf:Inf, num:Num, sem:Cop]),
  pp([sem:PP]),
  { Sem = app(Cop, PP) }.

vp([coord:no, inf:Inf, num:Num, gap:[], sem:Sem]) -->
  cop([type:np, inf:Inf, num:Num, sem:_]),
  numberOrAll(),
  [different],
  n([coord:_, num:Num, sem:_]),
  { Sem = lam(N, app(N, lam(X, drs([], [alldifferent(X)])))) }.

numberOrAll() -->
  number([sem:_]).
numberOrAll() -->
  [all].
\end{dcg}

\subsection{Een verbale constituent met voegwoord}
Een gecoördineerde verbale constituent is gelijkaardig aan die van een nominale constituent. Er is echter één klein verschil in de betekenis ervan. Een simpele lambda-applicatie volstaat hier niet altijd. Indien het onderwerp een kwantor bevat, dan valt de verbale constituent namelijk onder de scope van deze kwantor. We willen dus ook dat de vertaling van het voegwoord binnen de scope van de kwantor valt. Als we $\sem{vp} = \app{\app{\sem{coord}}{\sem{vp_1}}}{\sem{vp_2}}$ als vertaling nemen, dan valt de vertaling van het voegwoord buiten de scope van het onderwerp. De juiste betekenis is

$$\sem{vp} = \lambdaf{N}{\app{N}{\lambdaf{x}{\appH{\app{\app{\sem{coord}}{\sem{vp_1}}}{\sem{vp_2}}}{\lambdaf{P}{\app{P}{x}}}}}}$$ We halen de entiteit $x$ eerst uit het onderwerp en binnen de scope van de naamwoordgroep geven we x (in een lege box) terug aan de gecoördineerde verbale constituent.

Voor logigrammen resulteren beide betekenissen echter tot een correct resultaat. Het onderwerp zal namelijk altijd volledig gespecifieerd zijn. Er is maar één entiteit die aan de omschrijving voldoet. Daardoor maakt het niet uit dat er twee kantoren zijn die gecombineerd worden met het voegwoord in kwestie i.p.v. dat het voegwoord zich bevindt in de scope van de kwantor. Dit voorbeeld toont echter nog eens aan dat het opstellen van de formules niet altijd triviaal is.

\begin{dcg}{De grammaticale regels voor een verbale constituent met een voegwoord}{dcg:vp3}
vp([coord:yes, inf:Inf, num:Num, gap:[], sem:Sem]) -->
  vp([coord:no, inf:Inf, num:Num, gap:[], sem:VP1]),
  coord([type:_, sem:Coord]),
  vp([coord:_, inf:Inf, num:Num, gap:[], sem:VP2]),
  { CoordinatedVP = app(app(Coord, VP1), VP2) },
  { Sem = lam(N, app(N, lam(X, app(CoordinatedVP, lam(P, app(P, X)))))) }.
\end{dcg}

\section{Zin}
Er zijn drie soorten zinnen binnen logigrammen. Een normale zin bestaat uit een nominale en een verbale constituent die overeenkomen in getal. Dat is de eerste grammaticale regel van grammatica~\ref{dcg:s}. De betekenis is een simpele lambda-applicatie $\sem{s} = \app{\sem{vp}}{\sem{np}}$.

\paragraph{} Een tweede soort zin is een andere vorm voor een alldifferent constraint en wordt gebruikt om alle domeinelement van een bepaald domein op te sommen. Een voorbeeldzin is ``The four players are John, Mary, the person who plays with cards and the man from France''. Zo'n zin bestaat uit het woordje ``the'', een getal (het aantal domeinelementen), een substantief, een koppelwerkwoord en ten slotte een gecoördineerde naamwoordgroep. De betekenis is gelijk aan die uit sectie~\ref{sec:gram-koppelwerkwoord}

\paragraph{} Een laatste soort zin is van de vorm ``Of John and Mary, one is from France and the other plays with cards''. De zin bestaat dus uit twee nominale en twee verbale constituenten. De twee entiteiten van de nominale constituenten zijn verschillend en bovendien is elk het onderwerp van één van de twee verbale constituenten. 

$$\sem{s} = \app{\sem{np_1}}{\lambdaf{x1}{\app{\sem{np_2}}{\lambdaf{x2}{\\ \drsMerge{\drs{}{x1 \neq x2}}{\drsOr{\drsMerge{\app{\sem{vp_1}}{\lambdaf{P}{\app{P}{x1}}}}{\app{\sem{vp_2}}{\lambdaf{P}{\app{P}{x2}}}}}{\drsMerge{\app{\sem{vp_1}}{\lambdaf{P}{\app{P}{x2}}}}{\app{\sem{vp_2}}{\lambdaf{P}{\app{P}{x1}}}}}}}}}}$$

We halen de entiteiten $x1$ en $x2$ uit hun box. We stellen dat ze verschillend zijn. Bovendien is (een lege box rond) entiteit $x1$ het onderwerp van de verbale constituent 1 en $x2$ het onderwerp van verbale constituent 2 of omgekeerd.

\begin{dcg}{De grammaticale regels voor een zin}{dcg:s}
s([coord:no, sem:Sem]) -->
  np([coord:_, num:Num, gap:[], sem:NP]),
  vp([coord:_, inf:fin, num:Num, gap:[], sem:VP]),
  { Sem = app(VP, NP) }.

s([coord:no, sem:Sem]) -->
  [the],
  number([sem:_, vType:_]),
  n([coord:_, num:pl, sem:_]),
  cop([type:np, inf:fin, num:pl, sem:_]),
  np([coord:conj, num:_, gap:[], sem:NP]),
  { Sem = app(NP, lam(X, drs([], [alldifferent(X)]))) }.

s([coord:no, sem:Sem]) -->
  [of],
  np([coord:_, num:sg, gap:[], sem:NP1]),
  [and],
  np([coord:_, num:sg, gap:[], sem:NP2]),
  [one],
  vp([coord:no, inf:fin, num:sg, gap:[], sem:VP1]),
  [and, the, other],
  vp([coord:no, inf:fin, num:sg, gap:[], sem:VP2]),
  { Sem = app(NP1, lam(X1, app(NP2, lam(X2, 
      merge(
        drs([], [not(drs([], [eq(X1, X2)]))]),
        drs([], [or(
          merge(
            app(VP1, lam(N, app(N, X1))), 
            app(VP2, lam(N, app(N, X2)))), 
          merge(
            app(VP1, lam(N, app(N, X2))),
            app(VP2, lam(N, app(N, X1)))))])))))) }.
\end{dcg}

\section{Conclusie}
Een grammatica voor logigrammen is redelijk beperkt. De betekenis van een grammaticale regel komt vaak neer op een simpele lambda-applicatie maar soms zijn ook complexere formules nodig om de scope van de kwantoren correct te krijgen.

Sommige grammaticale regels zijn specifiek voor logigrammen. Andere zijn dan weer algemeen bruikbaar. Het is soms wel moeilijk om maar 1 vertaling toe te staan. Zo laten we enkel voor een vergelijkende naamwoordgroep een ellips van het werkwoord toe. Dit is namelijk de enige plaats waar dit gebeurt in logigrammen. Als we deze grammatica willen uitbreiden naar andere specificaties moeten we dus meerdere betekenissen toelaten of bijvoorbeeld het gebruik van ellipsen verbieden.

\section{A typed language}
In natural language, it is possible to construct sentences that are grammatically correct but without meaning. E.g. ``The grass is drinking the house''. The grass is not something that can drink and a house is not something that can be drunk. We say the sentence is wrongly typed. Based on grammar alone, we can never know exclude these setences. Therefore, we add types to the framework.

Some words and phrases have a feature \textit{number} to indicate whether it is singular or plural. In a sentence, the \textit{number} of the noun phrase and the verb phrase must unify, i.e. the subject and the verb must agree in number.

Similarly to this, we add the feature \textit{type} to most words and phrases to indicate its type. In this paper we use a very basic type system. There are a number of basic types and a \textit{pair-type} that consists of two basic types. E.g. a noun has a basic type, a transitive verb a pair of types. One for its subject and one for its object. Also phrases get types. A noun phrase gets a basic type as well. The same goes for a verb phrase. In the grammar we can then express that the type of the noun phrase and the verb phrase should unify when they make a sentence. This way, badly typed sentences get excluded.

As indicated earlier, most words (like noun, verb, ...) should also get a type in the lexicon. In this paper we explored a form of type inference. We assume that every word has exactly one type but that it is unknown. We then try to find the types of all words. We cannot allow a word to have more than one type because it is the only information we have to unify types. If we know the type of all words, we know to which domain every domain element belongs, i.e. we know that ``France'' and ``Italy'' are from the same domain (without necessarily knowing that they are countries). The ultimate goal for the type inference system is to have the types match up with the domains, i.e. to know which domain elements form a certain domain.

However, based on types alone, the system cannot always deduce enough. This is the case when a lot of synonyms are used. There is no indication of synonyms in the lexicon so the principle of one type per word doesn't help us in that case. Therefore we also ask the user how many domains are in the logigram. In case the system cannot unify enough types, it will ask the user some linguistic questions that could be rewritten as ``Is ... a meaningful sentence?''.

In the case a certain domain always occurs as part of noun phrases with an unknown relation (e.g. ``the 2008 graduate''), we cannot construct such a question. In that case, the system does ask the user if two proper nouns are of the same type. In all other cases, the type inference system can deduce the correct types based on only linguistic information.

Having a typed natural language means we can translated to a typed formal language. In this paper we translate to the IDP language \cite{IDP}. We can construct the formal vocabulary based on linguistic information. When the types correspond to a domain for logigrams, we translate to to a constructed type or to a subset of the natural numbers. The former in case of a non-numerical domain. The different proper nouns that belong to this type form the constants of the constructed type. However, it is possible that one constant is missing (i.e. that it didn't occur in the clues of a logigram). Therefore, we provide the system with the number of domain elements that are in each domain. In case one is missing, the system adds an extra constant. It is not possible that two elements are missing as a logigram has a unique solution and these two elements would be interchangeable. The system translates to a subset of the natural numbers in case of numerical domain. In that case, the system asks the user the exact subset.

There can also be types that don't correspond with domain elements. There is a derived numerical type which contains the set of numbers that are a difference of two numbers from a numerical domain. This type is only necessary because the inferences in IDP would otherwise not be finite.

It is also possible to have an intermediate type that links two domains, e.g. \textit{tour} in ``John follows the tour with 54 people''. These types are translated to constructed types with as many constants as there are domain elements in a domain. Symmetry-breaking axioms are added to the theory to link every constant with one domain element.

Every transitive verb and preposition introduces a predicate. The noun phrases with an unknown predicate can now be resolved. Based on type information, we know which predicates are suitable, namely the predicates that link the corresponding types. Because there is exactly one bijection between every two domains, it doesn't matter which predicate the system picks.

With a formal vocabulary and a translation of all clues into logic, the system can almost solve the logigram (and apply other inferences). The theory only needs to be expanded with a number of logigram-specific axioms. We need type information to construct some of these axioms. 

The first type of axioms is the symmetry-breaking axioms mentioned earlier ($pred(C1, John) \land pred(C2, Mary) \land pred(C3, Charles)$). Another type is the bijection axioms to state that every predicate is a bijection ($\forall x \cdot \exists y \cdot pred(x, y) \land \forall y \cdot \exists x \cdot pred(x, y)$).

Finally there are three types of axioms to express that there is an equivalence relation between domain elements. Two elements are equivalent if they are linked through a predicate or if they are equal. In second order logic $x \sim y \Leftrightarrow \exists P \cdot P(x, y) \lor x = y$. The reflexivity is satisfied by definition.

\begin{itemize}
  \item \textbf{Synonymy} There is exactly one bijection between every two domains. Therefore, two predicates with the same types are always synonyms. E.g. $\forall x \forall y \cdot pred_1(x, y) \Leftrightarrow pred_2(x, y).$
  \item \textbf{Symmetry} Predicates with a reversed signature are each other inverse. E.g. $\forall x \forall y \cdot pred_1(x, y) \Leftrightarrow pred_2(y, x).$
  \item \textbf{Transitivity} Finally there are axioms linking the different predicates. They express the transitivity of the equivalence relation. E.g. $\forall x \forall y \cdot pred_1(x, y) \Leftrightarrow \exists z \cdot pred_2(x, z) \land pred_3(z, y).$
\end{itemize}


\section{Evaluation}
A grammar was constructed based on the 10 first logic grid puzzles from Puzzle Baron's Logic Puzzles Volume 3 \cite{logigrammen} and evaluated on the next 10 puzzles. The question arises how many adaptions are necessary to represent these unseen puzzles in the constructed grammar. Another question is whether or not it is possible to deduce the types of a logic grid puzzle or not. Finally, we want to know if all puzzles are represented correctly such that the solution can automatically be derived with the IDP system \cite{IDP}.

It turns out that once modified, the types of all unseen puzzles can be derived automatically and the clues can be translated correctly into logic.

Table~\ref{tbl:resultaten} gives an overview of the different adaptations. In total 65 adaptations were necessary, 30 of which were purely grammatical. They used grammatical structures that didn't appear in the training set. The next 5 adaptations were adaptations that also happened during training. There are 19 adaptations due to badly typed words. E.g. a verb ``to order'' that was used to order both food and drinks. The assumption of one type per word wasn't satisfied in that case. There was one adaptation to a proper noun because the same domain element appeared in two different word forms. We assumed there is only one word form per domain element. Finally, there were 10 cases where an extra word was added or removed.

We refer to the full master thesis for a detailed overview of all the adaptations.

\begin{table}[h]
  \centering
  \begin{tabular}{lc}
    \hline
    \textbf{Problem} & \textbf{Count} \\ 
    \hline
    ``the one'' & 15 \\
    Wrong use of copular verb & 6 \\
    Badly structured subordinate clause & 6 \\
    Passive sentence & 1 \\
    Possessive pronoun & 1 \\
    Badly structured noun phrase & 1 \\
    \hline
    Floating point numbers & 3 \\
    Superlative & 1 \\
    ``Of the two ...'' & 1 \\
    \hline
    Badly typed verb & 14 \\
    Badly typed noun phrase & 5 \\
    \hline
    Double word form domain element & 1 \\
    \hline
    Redundant word & 7 \\
    Missing word (ellipse) & 3 \\
    \hline
  \end{tabular}
  \caption{An overview of the different adaptations}
  \label{tbl:resultaten}
\end{table}

\section{Conclusion}
The framework of Blackburn and Bos can be used to translate a CNL into logic. This way, we assign semantics to such a CNL. Therefore, we can use such a CNL as a knowledge representation language within a knowledge base system. This allows multiple types of inference on the (constructed) natural language. For a logic grid puzzle, this can mean automatically solving the puzzle but also helping the user by giving hints or helping the author by giving the possible solutions.

Moreover, we can expand the framework with types. This way we can translate the natural language to a typed logic. We have proven that type inference is possible as well. More research into a typed CNL is definitely necessary. For example, into a more complex type system. Or a type system which allows multiple types per word. The type system can then derive the correct instance of the verb. E.g. it can differentiate between ordering drinks and ordering food.


% \section{Introduction}

\IEEEPARstart{K}{nowledge} base systems have been around for a while. They take a knowledge base as input. This knowledge base describes the world, i.e. it is a specification of how something should work. Based on this knowledge base, different kinds of inference can be applied. De Cat et al. \cite{IDP} give the example of an university course-management system. The knowledge base contains rules like ``In every auditorium at a any time, there should be at most one course.''. The different kinds of inferences that can be applied, include (the examples are from De Cat et al. \cite{IDP}): \textit{propagation} (e.g. automatically selecting required prerequisites when constructing an individual study program), \textit{model expansion} (e.g. getting a full individual study program from a partial one) and \textit{querying} (e.g. getting the schedule for a specific student).

These knowledge base systems have triggered a lot of research into formal languages and their expressivity. These languages are often hard to read, write and learn.

One method to ease the writing of formal theories is to develop languages that are somewhere in between natural language and formal languages. Such languages are called Controlled Natural Languages (CNL). A CNL is a subset of a natural language that, for example, allows to write specifications in a more consistent language. Kuhn~\cite{Kuhn2014} made an overview of 100 CNL's. Some of these languages have formal semantics and can be translated automatically into a formal logic. They can be used as a knowledge representation language within a knowledge base system. They are often easy to read. Unfortunately, the translations of such CNL's into logic are not well documented which makes it hard to expand the language.

This paper therefore constructs a new CNL based on the language used in a small domain, namely logic grid puzzles. It is then tested on unseen puzzles. The first goal of this paper is to show that the framework introduced by Blackburn and Bos \cite{Blackburn2005, Blackburn2006} can be used to automatically translate such a CNL into a formal logic. The second goal is to show that adding types to this framework is useful. It makes it possible to translate to a typed logic and allows more inference. E.g. The system can automatically infer the domains of a logic grid puzzle.

% allows more inference. The system can automatically infer domains of a logic grid puzzle.

% In this paper, we take the next step for bridging the gap between natural language and logic. We use the well-documented framework of Blackburn and Bos and extend it with types. With this extension, we constructed the first typed CNL with a formal semantics in a typed logic. This new CNL is based on the language used in a small domain, namely logic grid puzzles. It is then tested on unseen puzzles. The first goal of this paper is to show that the framework introduced by Blackburn and Bos \cite{Blackburn2005, Blackburn2006} can be used to automatically translate such a CNL into a formal logic. We choose the IDP language \cite{IDP} as our formal logic. The second goal is to show that adding types to this framework is useful. It makes it possible to translate to a typed logic and allows more inference. E.g. The system can automatically infer the domains of a logic grid puzzle.

Therefore, we take the next step for bridging the gap between natural language and logic. We extend the well documented framework of Blackburn and Bos with types. With this extension, we constructed the first typed CNL with a formal semantics in a typed logic.

The result of this research is a fully automated tool that can understand a logic grid puzzle and reason about it (e.g. the system can solve the puzzle automatically).


% \section{Conclusion}
The framework of Blackburn and Bos can be used to translate a CNL into logic. This way, we assign semantics to such a CNL. Therefore, we can use such a CNL as a knowledge representation language within a knowledge base system. This allows multiple types of inference on the (constructed) natural language. For a logic grid puzzle, this can mean automatically solving the puzzle but also helping the user by giving hints or helping the author by giving the possible solutions.

Moreover, we can expand the framework with types. This way we can translate the natural language to a typed logic. We have proven that type inference is possible as well. More research into a typed CNL is definitely necessary. For example, into a more complex type system. Or a type system which allows multiple types per word. The type system can then derive the correct instance of the verb. E.g. it can differentiate between ordering drinks and ordering food.



% if have a single appendix:
%\appendix[Proof of the Zonklar Equations]
% or
%\appendix  % for no appendix heading
% do not use \section anymore after \appendix, only \section*
% is possibly needed

% use appendices with more than one appendix
% then use \section to start each appendix
% you must declare a \section before using any
% \subsection or using \label (\appendices by itself
% starts a section numbered zero.)
%
% \appendices
% \section{Proof of the First Zonklar Equation}
Appendix one text goes here.

% you can choose not to have a title for an appendix
% if you want by leaving the argument blank
\section{}
Appendix two text goes here. \cite{IDP}



% % use section* for acknowledgment
% \section*{Acknowledgment}
% The authors would like to thank...


% Can use something like this to put references on a page
% by themselves when using endfloat and the captionsoff option.
\ifCLASSOPTIONcaptionsoff
  \newpage
\fi



% trigger a \newpage just before the given reference
% number - used to balance the columns on the last page
% adjust value as needed - may need to be readjusted if
% the document is modified later
%\IEEEtriggeratref{8}
% The "triggered" command can be changed if desired:
%\IEEEtriggercmd{\enlargethispage{-5in}}

% references section

% can use a bibliography generated by BibTeX as a .bbl file
% BibTeX documentation can be easily obtained at:
% http://mirror.ctan.org/biblio/bibtex/contrib/doc/
% The IEEEtran BibTeX style support page is at:
% http://www.michaelshell.org/tex/ieeetran/bibtex/
%\bibliographystyle{IEEEtran}
% argument is your BibTeX string definitions and bibliography database(s)
%\bibliography{IEEEabrv,../bib/paper}
%
% <OR> manually copy in the resultant .bbl file
% set second argument of \begin to the number of references
% (used to reserve space for the reference number labels box)
% \begin{thebibliography}{1}

\bibliographystyle{IEEEtran}
\bibliography{../thesis/referenties}
% \bibitem{IEEEhowto:kopka}
% H.~Kopka and P.~W. Daly, \emph{A Guide to \LaTeX}, 3rd~ed.\hskip 1em plus
%   0.5em minus 0.4em\relax Harlow, England: Addison-Wesley, 1999.
% \end{thebibliography}

% biography section
% 
% If you have an EPS/PDF photo (graphicx package needed) extra braces are
% needed around the contents of the optional argument to biography to prevent
% the LaTeX parser from getting confused when it sees the complicated
% \includegraphics command within an optional argument. (You could create
% your own custom macro containing the \includegraphics command to make things
% simpler here.)
%\begin{IEEEbiography}[{\includegraphics[width=1in,height=1.25in,clip,keepaspectratio]{mshell}}]{Michael Shell}
% or if you just want to reserve a space for a photo:

% \begin{IEEEbiography}{Michael Shell}
% Biography text here.
% \end{IEEEbiography}

% % if you will not have a photo at all:
% \begin{IEEEbiographynophoto}{John Doe}
% Biography text here.
% \end{IEEEbiographynophoto}

% % insert where needed to balance the two columns on the last page with
% % biographies
% %\newpage

% \begin{IEEEbiographynophoto}{Jane Doe}
% Biography text here.
% \end{IEEEbiographynophoto}

% You can push biographies down or up by placing
% a \vfill before or after them. The appropriate
% use of \vfill depends on what kind of text is
% on the last page and whether or not the columns
% are being equalized.

%\vfill

% Can be used to pull up biographies so that the bottom of the last one
% is flush with the other column.
%\enlargethispage{-5in}



% that's all folks

\end{document}


