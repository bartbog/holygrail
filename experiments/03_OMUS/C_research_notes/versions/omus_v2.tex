\documentclass{article}
\usepackage{fullpage}
\usepackage{graphicx}
\usepackage{hyperref}
\usepackage{xcolor}
\usepackage{amsmath,amssymb}
\usepackage[ruled,linesnumbered]{algorithm2e}
\usepackage{float}
\usepackage{amsthm}
\usepackage{thmtools} 
\usepackage[parfill]{parskip}

\definecolor{vuborange}{rgb}{1.0,0.40,0.0}
\newcommand\m[1]{\mathcal{#1}}

\newtheorem{thm}{Theorem}
\newtheorem{definition}[thm]{Definition}
\newtheorem{prop}{Property}
\newtheorem{property}[prop]{Property}
\newtheorem{lem}{Lemma}
\newtheorem{lemma}[lem]{Lemma}
\newtheorem{propo}{Proposition}
\newtheorem{proposition}[propo]{Proposition}
% \newtheorem{prof}{Proof}
% \newtheorem{proof}[prof]{Proof}
\newtheorem{ex}{Example}
\newtheorem{example}[ex]{Example}

\bibliographystyle{apalike}
\newcommand\comment[1]{\marginpar{\tiny #1}}
\renewcommand\comment[1]{#1}
% \newcommand{\tias}[1]{{\comment{\color{blue}\textsc{TG:}#1}}}
\newcommand{\emilio}[1]{{\comment{\color{red}#1}}}
\newcommand{\todo}[1]{{\comment{\color{red}\textsc{TODO:}#1}}}

\begin{document}

\begin{definition}
  A subset $\m{U} \subseteq \m{F}$ is \textbf{minimal unsatisfiable subset} (MUS) if $\m{U}$ \emph{unsatisfiable} and $\forall \m{U}' \subset \m{U}$, $\m{U}'$ is \emph{satisfiable}. An MUS of $\m{F}$ with an optimal value w.r.t an \emph{objective function $f$} is called an \textbf{optimal MUS} (OMUS).
\end{definition}

\begin{definition}
  A subset $\m{C}$ of $\m{F}$ is an \textbf{minimal correction subset} (MCS) if $\m{F}  \setminus \m{C}$ is \emph{satisfiable} and $\forall \m{C}' \subseteq \m{C} \wedge \m{C}' \neq \emptyset$, $(\m{F} \setminus \m{C}) \cup \m{C}'$ is \emph{unsatisfiable}.
\end{definition}

\begin{definition}
  A \emph{satisfiable subset} $\m{S} \subseteq \m{F}$ is a \textbf{Maximal Satisfiable Subset} (MSS) if  $\forall \ \m{S}' \subseteq \m{F}'$ s.t $ \m{S} \subseteq \m{S}'$, $\m{S}'$ is \emph{unsatisfiable}.
\end{definition}

An MSS can also be defined as the complement of an MCS (and vice versa). If $\m{C}$ is a MCS then $ \m{S} = \m{F} \setminus \m{C}$ is a MSS. On the other hand, MUSes and MCSes are related by the concept of minimal hitting set.


\begin{definition}
  A \emph{minimal correction subset} $\m{C}$ of $\m{F}$ is an \textbf{Optimal Correction Subset} of $\m{F}$, OCS($\m{F}$) if $\forall \ \m{C}' \in MCSes(\m{F}'): f(\m{C}) \leq f(\m{C}')$.
\end{definition}


\begin{definition}
  A \emph{maximal satisfiable subset} $\m{S} \subseteq \m{F}$ is an \textbf{Optimal Satisfiable Subset} (OSS) if $\forall \ \m{S}' \in MSSes(\m{F}'): f(\m{S}) \leq f(\m{S}')$.
\end{definition}

\begin{proof}
  Let $\m{C}$ be the collection of MCSes($\m{F}$), $\m{S}$ be the collection of MSSes($\m{F}$), we know that 
  \begin{itemize}
    \item $|\m{C}| = |\m{S}|$ and,
    \item $\forall \ i \in \{1..|\m{C}|\}: \m{S}_i = \m{F} \setminus \m{C}_i$ 
  \end{itemize}


  Let $\m{S}*$ be the OSS, 

  Let $\m{C}*$ be the OCS, 
\end{proof}

% $\m{C}$
% $\m{S}$

\begin{lemma}
  Given a CNF formula $\m{F}$, let OMUSes($\m{F}$) and OCSes($\m{F}$) be the set of all OMUSes and OCSes of $\m{F}$ respectively. Then the following holds:
  \begin{enumerate}
    \item A subset $\m{U}$ of $\m{F}$ is an OMUS iff $\m{U}$ is an optimal hitting set of OCSes($\m{F}$)
    \item A subset $\m{C}$ of $\m{F}$ is an OCS iff $\m{U}$ is an optimal hitting set of OMUSes($\m{F}$)
  \end{enumerate}
\end{lemma}


\begin{definition}
  Let $\Gamma$ be a collection of sets and MHS($\Gamma$) the set of all minimal hitting sets on $\Gamma$ and let $f$ be an objective function with input a set of constraints. Then a hitting set $ h \in \Gamma$ is said to be an \textbf{optimal} hitting set if $\forall$ $h' \in OHS(\Gamma)$ we have that %$|h| \leq |h'|$ and 
  $f(h) \leq f(h')$ \cite{davies2011solving}.
\end{definition}

\begin{property}
  The \textbf{optimal} hitting set of a collection of sets $\Gamma$ is denoted by $OHS(\Gamma)$.
\end{property}

% The algorithm is based on the following observation:

\begin{proposition}
  A set $\m{U} \subseteq \m{F}$ is an OMUS of $\m{F}$ if and only if $\m{U}$ is an optimal hitting set of MCSes($\m{F}$)
\end{proposition}

\begin{lemma}\label{lemma:K}
  Let $\m{K}  \subseteq$ MCSes($\m{F}$). Then a subset $\m{U}$ of $\m{F}$ is an OMUS if $\m{U}$ is a optimal hitting set on $\m{K}$ and $\m{U}$ is unsatisfiable
\end{lemma}

\textbf{Proof Lemma \ref{lemma:K}}


\begin{itemize}
  \item \emilio{Is an OCS obligatory an MCS ?}
  \item \emilio{Does an OCS have to be optimal and minimal ? }
  \item \emilio{How is MSS afected ?}
\end{itemize}

\newpage
\begin{algorithm}
  %  \begin{algorithmic}
  \SetKwInOut{Input}{input}
  \SetKwInOut{OptInput}{optional input}
  \SetKwInOut{Output}{output}
  \SetKwComment{command}{/*}{*/}

  \Input{$\m{F}$ \textit{a CNF formula } }
  \Input{$f$ \textit{a cost function} }
  \OptInput{$\m{H}_0$ \textit{initial set of disjoint Minimum Correction Sets}}
  \Begin{
    $\m{H} \gets \m{H}_0$ \;
    \While{true}{
      \tcp{\small cost minimal hitting set w.r.t cost function f}
      $h \gets$ \texttt{OHS}($\m{H}, f$)\;

      \tcp{\small set with all clauses from hitting set}
      $\m{F}' \gets \{ c_i | e_i \in h\}$ \;

      \If{\textbf{not} \texttt{SAT($\m{F}'$)} }{
        \Return{$\m{OMUS} \gets \m{F}' $} \;}
      %\Else{
      \tcp{\small written as \texttt{grow($\m{F}'$)} which is Minimum Correction Set of $\m{F}'$}
      \tcp{\small find the biggest one with the biggest cost}
      %$\m{C} \gets \m{F} \setminus grow(\m{F}', \m{F}, f)$ \;  
      $\m{C} \gets \m{F} \setminus grow(\m{F}', \m{F}, f)$ \;
      %}

      $\m{H} \gets \m{H} \cup \{ \m{C}\}$ \;

    }
  }
  % \end{algorithmic}
  \caption{OMUS($\m{F},f_o, \m{H}_0$) \cite{ignatiev2015smallest}}
  % \caption{OMUS($\m{F},\m{H}_0 = \emptyset ,f_o$)}
  \label{alg:omus}
\end{algorithm}




\

\textbf{Proof 1.} 
\begin{itemize}
  \item use the cost of the constraints
  \item if cost $\leq$ ... and cost $\geq$ ... then it means cost ....
  \item Use the min cost of OCS and OMUS
\end{itemize}


\textbf{Proof 1.1 $\Rightarrow$} \textit{A subset $\m{U}$ of $\m{F}$ is an OMUS $\Rightarrow$ $\m{U}$ is an optimal hitting set of OCSes($\m{F}$)} 

\textbf{Proof 1.2 $\Leftarrow$} \textit{A subset $\m{U}$ of $\m{F}$ is an OMUS $\Leftarrow$ $\m{U}$ is an optimal hitting set of OCSes($\m{F}$)} 

\

\textbf{Proof 2.}

\textbf{Proof 2.1 $\Rightarrow$} \textit{A subset $\m{C}$ of $\m{F}$ is an OCS $\Rightarrow$ $\m{U}$ is an optimal hitting set of OMUSes($\m{F}$)} 

\textbf{Proof 2.2 $\Leftarrow$} \textit{A subset $\m{C}$ of $\m{F}$ is an OCS $\Leftarrow$ $\m{U}$ is an optimal hitting set of OMUSes($\m{F}$)} 

\newpage



% \end{algorithm}
\newpage
For the set of clauses $\{c_1, ... c_{|C|}\}$ in the collection of sets $\m{H}$. For Example:
\[ \{c_3 \},\ \{c_2, c_4\},\ \{c_1, c_4\},\ \{c_1, c_5, c_7\} \]
The optimal hitting set can be formulated as an integer linear program.
\begin{equation} \label{eq:ILP:objective}
  min_{x} \sum_{ i \in \{1..|C|\}} c_i \cdot x_i
\end{equation}
\begin{equation} \label{eq:ILP:hittingset}
  \sum_{i \in \{1..|C|\}} x_i \cdot w_{ij} \geq 1, \  \forall j \in \{1..|hs|\}
\end{equation}
\begin{equation} \label{eq:ILP:bool:xi}
  x_i = \{0,1\}
\end{equation}
\begin{equation} \label{eq:ILP:bool:wij}
  w_{ij} = \{0, 1\}
\end{equation}
\begin{itemize}
  \item $c_i$ is the cost associated with clause i in
  \item $x_i$ is a boolean decision variable if constraint/clause $c_i$ is chosen or not.
  \item Equation \ref{eq:ILP:hittingset} is a boolean decision variable if constraint/clause i is in hitting set j.
\end{itemize}

\newpage

\bibliography{omusrefs}

\end{document}
%%%%%%%%%%%%%%%%%%%%%%%%%%%%%%%%%%%%%%%%%%%%%%%%%%%%%%%%%%%%%%%%%%%%%%


% \begin{algorithm}
%   %  \begin{algorithmic}
%   \SetKwInOut{Input}{input}\SetKwInOut{Output}{output}
%   \SetKwComment{command}{/*}{*/}

%   \Input{A set of constraint sets $\m{H}$}
%   \Begin{

%   }
%   % \end{algorithmic}
%   \caption{OMHS($\m{H}, f_o$)}
%   \label{alg:ominhs}


% \begin{enumerate}
%   \item Engineering Parallelize OMUS with parallel calls to OMHS
%   \item Ideas from \cite{davies2011solving} and from \cite{de2014future}:
%   \begin{enumerate}
%     \item Formalize : Dissociate SAT from from minimum cost hitting set using Design model
%     \item OMHS can be written as a Integer Linear Program:
%   \end{enumerate}
%   \item Ideas from \cite{ignatiev2015smallest}:
%   \begin{enumerate}
%     \item Reducing the number of SAT Calls
%   \end{enumerate}
% \end{enumerate}