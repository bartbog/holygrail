\chapter{Types}
\label{ch:types}

\paragraph{} In dit hoofdstuk voegen we types toe aan het framework van Blackburn en Bos. Aan de hand hiervan kunnen we vertalen naar een getypeerde logica en kunnen we automatisch de domeinen van een logigram afleiden uit de zinnen in natuurlijke taal.

We beschrijven eerst het achterliggende idee en hoe we dit kunnen toepassen op logigrammen. Dan bekijken we welke aanpassingen het lexicon en de grammatica moeten ondergaan. Ten slotte leggen we uit hoe we de domeinen van een logigram automatisch kunnen afleiden. 

\section{Het achterliggende idee}
In natuurlijke taal kan men vele zinnen vormen zonder echte betekenis die wel grammaticaal correct zijn. Bijvoorbeeld de zin ``Het gras drinkt het zingende huis'' houdt weinig steek. Idealiter zouden we zo'n zinnen willen bestempelen als foutief. Gras is namelijk niet iets dat kan drinken en een huis kan ook niet zingen of gedronken worden. We zeggen dat ze slecht getypeerd zijn.

\paragraph{} In de bestaande literatuur over gecontroleerde natuurlijke talen (CNL's) wordt hier weinig of geen aandacht aan besteed\footnote{In hoofdstuk~\ref{ch:related} was RuleCNL de enige taal die ook gebruik maakt van types.}. Deze thesis introduceert types in het framework van Blackburn en Bos. 

Concreet voegen we types toe als extra feature aan de meeste grammaticale en lexicale categorieën. Net zoals de feature getal ervoor zorgde dat het onderwerp en het werkwoord van een zin overeenkomen in getal, zorgt de feature \texttt{vType}\footnote{We kiezen voor \texttt{vType} omdat sommige grammaticale categorieën reeds een feature \texttt{type} hebben. De \texttt{v} staat voor vocabularium.} dat het onderwerp en de verbale constituent overeenkomen in type.

Bijvoorbeeld voor ``every man orders rice'': 

\Tree[.s [.np:$person$ every [.n:$person$ man ]] [.vp:$person$ [.tv:$\langle person,food\rangle$ orders ] [.np:$food$ rice ]]]

Een naamwoordgroep krijgt een type dat overeenkomt met het type van de entiteit waarnaar het verwijst. Het type van een verbale constituent is gelijk aan dat van het onderwerp. Het type van een transitief werkwoord bestaat dan weer uit een type-paar. Het eerste voor het type van het onderwerp (en dus de verbale constituent) en het andere voor het type van het lijdende voorwerp. Ook een voorzetsel heeft een paar als waarde voor \texttt{vType}. Het ene komt overeen met dat van het substantief dat getransformeerd wordt en het andere komt overeen met dat van de naamwoordgroep uit de voorzetselconstituent.

\paragraph{} Als men het lexicon annoteert met types, kan het systeem zinnen in natuurlijke taal type-checken. In deze thesis houden we het lexicon echter zo simpel mogelijk. De gebruiker moet daarom geen types opgeven. Het systeem zal aan type-inferentie doen en deze types proberen afleiden. Concreet willen we dus te weten komen dat ``France'' en ``Italy'' van hetzelfde type zijn, zonder per se te weten wat het type juist is. We hoeven dus niet te weten dat het om landen gaat. Wat ons vooral interesseert is het groeperen van de verschillende eigennamen volgens hun domein. 

In sommige toepassingen van het KBS-paradigma, zoals requirements engineering, wordt er eerst een domein opgesteld. Dit domein zou dan ook in een lexicon met types gegoten kunnen worden. Daarna kan men een specificatie schrijven die ook juist getypeerd is. Zo kan men fouten in de specificatie of het domein opsporen. Men zou bijvoorbeeld niet kunnen spreken over het bestellen van drank als men in het domeinmodel enkel eten kan bestellen. Verder onderzoek over de haalbaarheid en nuttigheid hiervan is nodig.

\section{Types voor logigrammen}
\paragraph{} Voor type-inferentie in logigrammen maken we één aanname: elk woord uit het lexicon heeft binnen één logigram exact één type. Later zal blijken dat hier niet altijd aan voldaan is (zie hoofdstuk~\ref{ch:evaluatie}). Het type-systeem zelf is vrij simpel: het bestaat uit een aantal basistypes en type-paren. Een type-paar moet altijd twee basistypes als argument hebben. Er is geen type-hiërarchie. Men kan dus niet uitdrukken dat mensen en dieren allebei levende wezens zijn. Type-checking is volledig gebaseerd op unificatie. Een zin is geldig als het type van het onderwerp en van de verbale constituent unificeren. We onderscheiden twee soorten basistypes: een basistype kan ofwel numeriek zijn ofwel niet. 


\paragraph{} Met een getypeerde natuurlijke taal kunnen we ook vertalen naar een getypeerde formele taal. Daarvoor breiden we DRS-structuren uit met types: elke entiteit is van een bepaald type. Vanaf nu vertalen we ook altijd naar eerste-orde-logica met types. De vertaling van een getypeerde DRS naar getypeerde eerste-orde logica is analoog met de vertaling van DRS naar eerste-orde-logica. De types worden vertaald zoals men verwacht:

\[
  \left(\ \drs{x_1[t_1], \ldots, x_n[t_n]}{
      \gamma_1 \\
      \ldots \\
      \gamma_m
    }\ \right)^{fo} = \exists x_{1}[t_1] \ldots \exists x_n [t_n]\left((\gamma_{1})^{fo} \land \ldots \land (\gamma_m)^{fo}\right)
\]

\[\left(\ \drs{x_1[t_1], ..., x_n[t_n]}{\gamma_1 \\ ... \\ \gamma_m} \Rightarrow B\right)^{fo} =  \forall x_1[t_1]\ldots \forall x_n[t_n]\Bigg(\Big((\gamma_1)^{fo} \land ... \land (\gamma_m)^{fo}\Big) \Rightarrow \Big(B\Big)^{fo} \Bigg)\]

\section{Aanpassingen}
\label{sec:aanpassingen}
\paragraph{} In het lexicon geven we elk woord één type dat nog onbekend is. De vertaling van een zelfstandig naamwoord is nu veel simpeler: $\sem{noun} = \lambdaf{P}{\drs{}{}}$. Dit komt overeen met $\sem{noun} = \lambdaf{P}{true}$ in eerste-orde-logica. Er is geen restrictie meer vanuit het zelfstandig naamwoord, deze restrictie zit volledig in het type van de entiteit die we opleggen bij het quantificeren. Een determinator heeft namelijk als vertaling $$\sem{det} = \lambdaf{R}{\lambdaf{S}{\left( \drs{x[\textit{Type}]}{} \oplus \app{R}{x} \oplus \app{S}{x} \right)}}$$ Hierbij komt \textit{Type} overeen met de waarde van de feature \texttt{vType}. Via unificatie krijgt de determinator het type van het zelfstandige naamwoord mee.

\paragraph{} Ook bij een comparatief en een onbepaald woord is er een introductie van een nieuwe variabele. Bij een comparatief gebeurt dit gelijkaardig aan de determinator. Het type komt namelijk uit een lexicale feature \texttt{vType} die via de grammatica geünificeerd wordt met andere types.

Een onbepaald woord introduceert een positief geheel getal (zie sectie~\ref{sec:lex-some}). De $x$ heeft dus $int$ als type. 

$$\sem{some} = \lambdaf{P}{\drsMerge{\drs{x[\textit{int}]}{x > 0}}{\app{P}{x}}}$$

In de praktijk blijkt het echter nodig om het domein te beperken tot een subset van de gehele getallen. Anders eindigt de inferentie met onderliggende tool IDP \cite{IDP} niet. Deze subset wordt voorgesteld door \textit{TypeDiff}, een numeriek type waarvan het domein is afgeleid van het domein van \textit{Type} (de waarde van de feature \texttt{vType}). Stel bijvoorbeeld dat \textit{Type} het domein \texttt{\{2007, 2008, 2009, 2010, 2011\}} heeft, dan heeft \textit{TypeDiff} het domein \texttt{\{1, -1, 2, -2, 3, -3, 4, -4\}}. Een getal uit \textit{TypeDiff} stelt namelijk een verschil voor van twee getallen uit \textit{Type}.

\paragraph{} De uitbreiding van de grammatica is vrij voor de hand liggend. Grammatica~\ref{dcg:types} geeft een aantal voorbeelden. De eerste grammaticale regel zegt dat het type van het onderwerp moet unificeren met dat van de verbale constituent. Het type van een verbale constituent komt dan weer overeen met het eerste type van het type-paar van het transitieve werkwoord. Het tweede type van dat paar komt overeen met het lijdend voorwerp.

\begin{dcg}{Een aantal grammaticale regels met type}{dcg:types}
s([coord:no, sem:Sem]) -->
  np([coord:_, num:Num, gap:[], sem:NP, vType:Type]),
  vp([coord:_, inf:fin, num:Num, gap:[], sem:VP, vType:Type]),
  { Sem = app(VP, NP) }.

vp([coord:no, inf:I, num:Num, gap:G, sem:Sem, vType:TypeSubj]) -->
  tv([inf:I, num:Num, positions:Pre-Post, sem:TV, vType:TypeSubj-TypeObj]),
  Pre,
  np([coord:_, num:_, gap:G, sem:NP, vType:TypeObj]),
  Post,
  { Sem = app(TV, NP) }.
\end{dcg}

\section{Type Inferentie}
% If many
% synonyms are used, the system enters an interactive
% mode and asks the user some linguistic questions
% that can be rewritten as “Is ... a meaningful sen-
% tence?” until the types are unified enough.
% This process fails if (and only if) a certain domain
% always occurs as part of noun phrases with an
% unknown relation (e.g. “the 2008 graduate”). In
% such a scenario, the system cannot fallback on a
% linguistic question and therefore asks the user if two
% proper nouns are of the same type. This is the exact
% question the system was supposed to solve.
% The described system can thus correctly derive
% which domain elements belong together based on
% type constraints and the answers to some linguistic
% questions. Only when they do not provide enough
% information, is the user asked to help identify the
% elements that belong together.
\paragraph{} Om de types automatisch af te leiden veronderstellen we dat elk woord uit het lexicon één type heeft (maar dit type wordt niet meegegeven door de gebruiker). Vervolgens zoeken we de types van alle woorden. Tijdens het vertalen worden de types geünificeerd volgens het principe van één type per woord. Neem bijvoorbeeld de zinnen ``John lives in France'' en ``The man with the dog lives in Italy''. Op basis hiervan weten we dat ``France'' en ``Italy'' hetzelfde type hebben, beiden zijn namelijk het lijdende voorwerp van ``lives in''. Ook weten we dat ``John'' een ``man'' is. Ten slotte weten we dat ``lives in'' een relatie is tussen een man en iets van het type van ``France'' en ``Italy'', terwijl ``with'' een relatie is tussen een man en iets van het type van ``dog''.

Als een logigram veel synoniemen bevat, kan het systeem deze types niet automatisch af leiden. Stel bijvoorbeeld dat ``Mary is from Spain'' deel uitmaakt van het vorige voorbeeld. Dan kan het systeem niet afleiden dat ``Spain'' tot hetzelfde domein behoort als ``France'' en ``Italy''. Het systeem weet namelijk niet dat ``from'' en ``lives in'' dezelfde relatie uitdrukken. De gebruiker kan dit ook niet ingeven in het lexicon. In dat geval wordt één domein nog steeds voorgesteld door de combinatie van een aantal types. Om de onderliggende axioma's van een logigram correct te kunnen uitdrukken, moeten deze types overeenkomen met de domeinen.

% De reden dat de groepering niet compleet hoeft te zijn, heeft te maken met synoniemen. Het programma is namelijk niet op de hoogte van welke woorden synoniemen zijn. De gebruiker kan deze (in ons syteem) namelijk niet opgeven in het lexicon. In het worst-case scenario komt elk woord maar één keer voor. In dat geval leert het principe ons niets en zijn alle eigennamen volledig gescheiden van elkaar. 

\paragraph{} Daarom stelt het systeem een aantal ja-nee-vragen aan de gebruiker. Hiervoor gebruikt het zoveel mogelijk taalkundige informatie. Er zijn 4 soorten vragen: ``Zijn A en B dezelfde relatie (synoniemen)?'', ``Zijn A en B een omgekeerde relatie?'', ``Is A een mogelijk lijdend voorwerp van B?'' en ``Zijn A en B van hetzelfde type?''. De eerste drie soorten vragen zijn puur taalkundig en zouden herschreven kunnen worden ``Is ... een zin met een betekenis?'', bijvoorbeeld ``Is 'John lives in the dog' een zin met een betekenis?''.

De laatste soort vraag is eigenlijk het doel van het hele type-systeem, namelijk achterhalen of twee domeinelementen van hetzelfde type zijn. Indien een bepaald domein in het logigram altijd wordt voorgesteld door een naamwoordgroep met een onbekende relatie (zie sectie~\ref{sec:npMissingRelation}) dan heeft men geen informatie over het type van deze domeinelementen en geen mogelijke relatie waar deze domeinelementen deel van kunnen maken. Er is dus geen mogelijke taalkundige vraag in dat geval.

\section{Conclusie}
\todo[inline]{Write one}
