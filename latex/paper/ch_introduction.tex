\section{Introduction}

\IEEEPARstart{K}{nowledge} base systems have been around for a while. A knowledge base is the input for such a system. Based on this knowledge base, different kinds of inference could be applied. De Cat et al. \cite{IDP} give the example of an university course-management system. The knowledge base describes the world, i.e. it is a specification of how something should work. E.g. in the course-management domain, the knowledge base could contain a rule like ``In every auditorium at a any time, there should be at most one course.''. Based on this knowledge base, different kinds of inferences can be applied. De Cat et al. list some inferences including \textit{propagation} (e.g. automatically selecting required prerequisites when constructing an individual study program), \textit{model expansion} (e.g. getting a full individual study program from a partial one) and \textit{querying} (e.g. getting the schedule for a specific student).

These systems have triggered a lot of research into formal languages and their expressivity. These languages are often hard to read, write and learn.

One method to ease the writing of formal theories is to develop languages that are somewhere in between natural language and formal languages. Such languages are called Controlled Natural Languages (CNL). A CNL is a subset of a natural language that, for example, allows to write specifications in a more consistent language. Kuhn~\cite{Kuhn2014} made an overview of 100 CNL's. Some of these languages have formal semantics and can be translated automatically into a formal logic. They could be used as a knowledge representation language within a knowledge base system. They are often easy to read. Unfortunately, the translations of such CNL's into logic are not well documented which makes it hard to expand the language.

This paper therefore constructs a new CNL based on the language used in a small domain, namely logic grid puzzles. It is then tested on unseen puzzles. The first goal of this paper is to show that the framework introduced by Blackburn and Bos \cite{Blackburn2005, Blackburn2006} can be used to automatically translate such a CNL into a formal logic. We choose the IDP language \cite{IDP} as our formal logic. The second goal is to show that adding types to this framework is useful. It makes it possible to translate to a typed logic and allows more inference. E.g. The system can automatically infer the domains of a logic grid puzzle.

% allows more inference. The system can automatically infer domains of a logic grid puzzle.

% In this paper, we take the next step for bridging the gap between natural language and logic. We use the well-documented framework of Blackburn and Bos and extend it with types. With this extension, we constructed the first typed CNL with a formal semantics in a typed logic. This new CNL is based on the language used in a small domain, namely logic grid puzzles. It is then tested on unseen puzzles. The first goal of this paper is to show that the framework introduced by Blackburn and Bos \cite{Blackburn2005, Blackburn2006} can be used to automatically translate such a CNL into a formal logic. We choose the IDP language \cite{IDP} as our formal logic. The second goal is to show that adding types to this framework is useful. It makes it possible to translate to a typed logic and allows more inference. E.g. The system can automatically infer the domains of a logic grid puzzle.

In this paper, we take the next step for bridging the gap between natural language and logic. We use the well-documented framework of Blackburn and Bos and extend it with types. With this extension, we constructed the first typed CNL with a formal semantics in a typed logic.

The result of this research is a fully automated tool that can understand a logic grid puzzle and reason about it (e.g. the system can solve the puzzle automatically).

