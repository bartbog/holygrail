\documentclass[handout]{beamer}
%\documentclass{beamer}
\usepackage[ruled,linesnumbered]{algorithm2e}
\usepackage{natbib}
%\usepackage{enumitem}
\definecolor{vuborange}{rgb}{1.0,0.40,0.0}
\usetheme{Boadilla}
\usepackage{amsmath,amssymb}
\usepackage{booktabs}
\title{Efficiently Explaining CSPs with Unsatisfiable Subset Optimization}
\institute[shortinst]
{\inst{1} Vrije Universiteit Brussel, Belgium \\ % Your institution for the title page
	\inst{2} KULeuven, Belgium \\ % Your institution for the title page
	\href{mailto:emilio.gamba@vub.be}{\underline{emilio.gamba@vub.be}}, \href{mailto:bart.bogaerts@vub.be}{bart.bogaerts@vub.be}, \href{mailto:tias.guns@kuleuven.be}{tias.guns@kuleuven.be} % Your email address
}
\date{IJCAI 2021}

\author{\underline{Emilio Gamba}\inst{1} \and  Bart Bogaerts\inst{1} \and   Tias Guns\inst{1,2}}

\newcommand\m[1]{\ensuremath{\mathcal{#1}}\xspace}
\newcommand\allconstraints{\m{T_P}}
\newcommand\formula{\ensuremath{\m{F} }\xspace}
\newcommand\formulac{\ensuremath{\m{C} }\xspace}

\makeatletter
\setbeamertemplate{footline}
{
	\leavevmode%
	\hbox{%
		\begin{beamercolorbox}[wd=.2\paperwidth,ht=2.25ex,dp=1ex,center]{author in head/foot}%
			\usebeamerfont{author in head/foot}Emilio Gamba (VUB)
		\end{beamercolorbox}%
		\begin{beamercolorbox}[wd=.6\paperwidth,ht=2.25ex,dp=1ex,center]{title in head/foot}%
			\usebeamerfont{title in head/foot}Efficiently Explaining CSPs with Unsatisfiable Subset Optimization
		\end{beamercolorbox}%
		\begin{beamercolorbox}[wd=.2\paperwidth,ht=2.25ex,dp=1ex,right]{date in head/foot}%
			\usebeamerfont{date in head/foot}August 2021\hspace*{1em}
			\insertframenumber{} / \inserttotalframenumber\hspace*{2ex} 
	\end{beamercolorbox}}%
	\vskip0pt%
}
\makeatother

\begin{document}
	
	\begin{frame}
		\maketitle
	\end{frame}
	
	\begin{frame}
		\frametitle{Outline}
		
		\begin{enumerate}
			\item Background
			\item Motivation
			\item Contributions
			\begin{enumerate}
				\item {\color{vuborange}Optimality}
				\item {\color{vuborange}Constrainedness}
				\item {\color{vuborange}Incrementality}
			\end{enumerate}
			\item Applications
			\item Results
			\item Conclusion
			\item Ideas for future work
		\end{enumerate}
	\end{frame}
%	
	\begin{frame}{Background}
	\framesubtitle{Examples of Constraint Satisfaction Problems}
		\begin{minipage}[t]{0.6\textwidth}
			\begin{figure}[h]
				\includegraphics[width=\textwidth]{logic_puzzle.png}
				\caption{Logic grid puzzle: Pasta Puzzle}
			\end{figure}
		\end{minipage}
	\hfill
			\begin{minipage}[t]{0.29\textwidth}
			\begin{figure}[h]
				\includegraphics[width=\textwidth]{sudoku.jpg}
				\caption{Sudoku}
			\end{figure}
		\end{minipage}
	\end{frame}

	\begin{frame}{Background}
	\framesubtitle{Constraint Satisfaction Problem}
	In \cite{bogaerts2020step}, we proposed a method for step-wise explaining solutions of Constraint Satisfaction Problems (CSPs).\pause
	\begin{description}
		\item[$\m{C}$] constraints over variables $V$ with domains $D$
		\item[$\m{I}$] a partial interpretation (partial solution) .a.k.a \emph{facts}
		\item[$f$] a cost-function quantifying the difficulty of an explanation step\pause
		\item[$\m{I}_{end}$] interpretation-to-explain (solution) of the CSP \pause  
		\item[$\m{I}_{end} \setminus \m{I}$] remaining facts to explain
	\end{description} \pause

%	\begin{definition}
%		An \emph{\color{vuborange}explanation} is defined as a \textit{sequence} of simple inference steps.
%	\end{definition}\pause
	\end{frame}

	\begin{frame}{Background}
	\framesubtitle{Explaining how to solve Constraint Satisfaction Problems}
%		\begin{definition}
%			An \emph{\color{vuborange}explanation} is defined as a \textit{sequence} of simple inference steps. 
%		\end{definition}\pause
		Given a formula $\m{C}$, a partial interpretation $\m{I}$, a new fact $l \in {\m{I}_{end} \setminus \m{I}}$ 
		\begin{definition}
			A \emph{\color{vuborange}non-redundant explanation} corresponds to computing a Minimal Unsatisfiable Subset (\emph{MUS}) of $\m{C} \wedge \m{I} \wedge \lnot l$.
		\end{definition}\pause
	
	\begin{block}{Explanation sequence generation problem}
		Find a \emph{non-redundant} explanation sequence that explains all derivations need to solve the CSP.
	\end{block}\pause
	\end{frame}

	\begin{frame}{Background}
	\framesubtitle{Explaining how to solve Constraint Satisfaction Problems}
	At every step of the explanation sequence,
	\begin{enumerate}
		\item Loop over all facts to find the `best' fact to explain (i.e smallest cost w.r.t $f$)
		\item For every fact, incrementally combine part of the constraints to find smaller MUSes and keep the `best' one for that fact.
	\end{enumerate}\pause

	\begin{alertblock}{Note}
		The cost-function is not encoded in the explanation computation, but in the explanation finding heuristic. There is no guarantee that the non-redundant explanation is provably cost-optimal!
	\end{alertblock}
			
	\end{frame}
%	\begin{definition}
%		\emph{\color{vuborange}Simplicity} of an explanation is measured by the number and types of constraints and facts used. 
%	\end{definition}

%		\begin{definition}
%	Given one or a subset of the constraints ( $S_i \subseteq T_P$  ) and facts we know ($E_i\subseteq \mathcal{I}_i$), an \textbf{explanation} is an implication of the form $E_i \wedge S_i  \implies N_i $, where $N_i$ is the new information from $S_i \cup E_i \models N_i$. \cite{bogaerts2020step}
%\end{definition}
%	


%	\begin{frame}{Motivation}
%		\framesubtitle{Step-wise Explanations of CSPs}
%		\cite{bogaerts2020step} proposed a method for step-wise explaining solutions of constraint satisfaction problems.\pause
%	
%		
%%		\begin{definition}
%%			\emph{\color{vuborange}Simplicity} of an explanation is measured by the number and types of constraints and facts used. 
%%		\end{definition}
%	\end{frame}

%	\begin{frame}{Motivation}
%	\framesubtitle{Optimal explanations}
%	
%	Let \emph{f} be a cost function quantifying explanation difficulty.
%	\begin{definition}
%		An \emph{\color{vuborange} optimal} explanation is \emph{f}. 
%	\end{definition}\pause
%	
%\end{frame}


	\begin{frame}{Motivation}
		\framesubtitle{Challenges and open questions}

		\begin{description}[font=\color{vuborange}\itshape]
			\item[Optimality] Explanations not provably optimal, \emph{heuristically} found. \pause
			\item[Efficiency] Explanation generation takes a lot of time. \pause
			\item[Incrementality] Can we reuse information from an explanation call to another? \pause
			\item[Constrainedness] Can we avoid looping over the literals when searching for the next best explanation ?
		\end{description}
	\end{frame}

	\begin{frame}{Optimality}
	\framesubtitle{Explanation difficulty and cost function}
	\begin{definition}
		An \emph{\color{vuborange}optimal} non-redundant explanation corresponds to computing a \emph{weighted} Minimal Unsatisfiable Subset (\emph{MUS}) of $\m{C} \wedge \m{I} \wedge \lnot l$.
	\end{definition}
	
	\end{frame}

	\begin{frame}{Constrainedness}
	\framesubtitle{Explanation difficulty and cost function}
	
	
	\end{frame}

	\begin{frame}{Incrementality}
	\framesubtitle{Can we reuse information ?}
	
	
	\end{frame}

	\begin{frame}{Efficiency}
	
	
	\end{frame}


%	\begin{frame}{Motivation}
%		\framesubtitle{Logic Grid Puzzles}
%		\begin{figure}
%			\includegraphics[width=0.9\textwidth]{logicpuzzles.png}
%		\end{figure}
%		\begin{center}
%			\url{https://bartbog.github.io/zebra/pasta/}
%		\end{center}
%	\end{frame}
%	
%	\begin{frame}{Motivation}
%		\framesubtitle{CSPs: a little formal Background}
%		
%		\begin{itemize}
%			%               \setlength{\leftmargin}{0pt}
%			\item[$\m{C}$] \emph{constraints} we can use to reason (alldifferent, George did not take pasta, …)
%			\item[$\mathcal{I}_0$] an \emph{Initial Interpretation}
%			\item[$\mathcal{I}_{end}$] Everything we can derive from $ \m{C} \wedge \mathcal{I}_0$
%		\end{itemize}
%		\begin{figure}
%			\includegraphics[width=0.7\textwidth]{sequence_explanation.png}
%		\end{figure}
%	\end{frame}
%	
%	
%	\begin{frame}{Motivation}
%		\framesubtitle{Gentle reminder: Explanations for CSPs}
%		\begin{definition}
%			Given one or a subset of the constraints ( $S_i \subseteq T_P$  ) and facts we know ($E_i\subseteq \mathcal{I}_i$), an \textbf{explanation} is an implication of the form $E_i \wedge S_i  \implies N_i $, where $N_i$ is the new information from $S_i \cup E_i \models N_i$. \cite{bogaerts2020step}
%		\end{definition}
%		
%	\end{frame}
%	
%	\begin{frame}{Motivation}
%		\begin{figure}
%			\includegraphics[width=0.8\textwidth]{logic_puzzles_bij.png}
%		\end{figure}\pause
%		\begin{itemize}
%			\item[$E_i$] farfalle goes with arrabiata sauce \pause
%			\item[$S_i$] an entity of type pasta can only be linked to one entity of type sauce\pause
%			\item[$N_i$] taglioni, rotini and capellini cannot be linked to arriabiata
%		\end{itemize}
%		
%	\end{frame}
%	
%	\newcommand\onestep{\ensuremath{\call{explain-One-Step}}\xspace}
%	
%		\begin{frame}{Motivation}
%		\framesubtitle{Gentle reminder: Explanations for CSPs}
%		\begin{definition}
%			Given one or a subset of the constraints ( $S_i \subseteq T_P$  ) and facts we know ($E_i\subseteq \mathcal{I}_i$), an \textbf{explanation} is an implication of the form $E_i \wedge S_i  \implies N_i $, where $N_i$ is the new information from $S_i \cup E_i \models N_i$. \cite{bogaerts2020step}
%		\end{definition} 
%		
%		An \textbf{explanation sequence} is of the form $$<(\mathcal{I}_0,(\emptyset,\emptyset,\emptyset)),\text{\hspace{3pt}}(\mathcal{I}_1,(E_1, S_1, N_1)),\text{\hspace{3pt}}...,\text{\hspace{3pt}}(\mathcal{I}_{end},(E_{n}, S_n, N_n ))>$$
%		
%		\begin{figure}
%			\includegraphics[width=0.7\textwidth]{sequence_explanation2.png}
%		\end{figure}
%		
%	\end{frame}
%	
%	\begin{frame}{Motivation}
%		\framesubtitle{Generating an explanation sequence for CSPs}
%\cite{bogaerts2020step}: At every step in the explanation sequence:
%		\begin{figure}
%			\includegraphics[width=0.45\textwidth]{algo_mus2.png}
%		\end{figure}
%		\begin{itemize}
%			\item Finding non-redundant explanations: MUS($\formulac \land \mathcal{I} \land \neg l$) extraction (IDP system)
%			
%		\end{itemize}
%	\end{frame}
%
%	\begin{frame}{Motivation}
%	\framesubtitle{Generating an explanation sequence for CSPs}
%\cite{bogaerts2020step}: At every step in the explanation sequence:
%	\begin{figure}
%		\includegraphics[width=0.45\textwidth]{algo_mus2_b.png}
%	\end{figure}
%	\begin{itemize}
%		\item Finding non-redundant explanations: MUS($\formulac \land \mathcal{I} \land \neg l $) extraction (IDP system)
%		
%		\begin{itemize}
%			\item In practice, not all constraints at once!
%			\item Consider power sets of constraints sorted by increasing cost
%		\end{itemize}
%	\end{itemize}
%\end{frame}
%
%	
%	\begin{frame}{Motivation}
%		\framesubtitle{Open Questions}
%		
%		\begin{description}[font=\color{vuborange}\itshape]
%			\item[\hspace{0.9cm}Optimality] Explanations not optimal, heuristically found \pause
%			\item[\hspace{1.05cm}Efficiency] Explanation generation takes a lot of time \pause
%			\item[\hspace{0.3cm}Incrementality] Can we reuse information from an explanation call to another? \pause
%			\item[Constrainedness] Can we avoid looping over the literals when searching for the next best explanation ?
%		\end{description}
%	\end{frame}
%	
%	\begin{frame}{Optimality}
%		\pause
%		\begin{minipage}{0.49\textwidth}
%						\begin{figure}
%				\includegraphics[width=\textwidth]{ihs.png}
%			\end{figure}
%			\pause
%		\end{minipage}
%		\begin{minipage}{0.5\textwidth}
%		\begin{itemize}
%			\item Inspired by \cite{ignatiev2015smallest}
%			\item MIP solver instead of SAT-based
%			\item[+] Optimal Hitting set (\emph{\textbf{Optimality}})
%		\end{itemize}
%		\end{minipage}
%		\vfill
%		\begin{figure}
%			\includegraphics[width=\textwidth]{mus_to_ous.png}
%		\end{figure}
%		
%	\end{frame}
%
%\begin{frame}{Optimality}
%	\framesubtitle{Verifying explanation quality}
%	\pause
%			\begin{figure}
%		\includegraphics[width=0.6\textwidth]{heatmap_costs_mus_cous.pdf}
%	\end{figure}
%\end{frame}
%	
%	
%	
%	
%		\begin{frame}{Incrementality}
%			\framesubtitle{Improving efficiency of OUS}
%		
%		\begin{minipage}{0.59\textwidth}
%			\begin{enumerate}
%				\item {\color{vuborange} Incremental \emph{OUSs} }
%				\begin{itemize}
%					\item Keep Satisfiable Subsets between OUS-calls
%				\end{itemize}	
%			\end{enumerate}
%		\end{minipage}
%		\begin{minipage}{0.4\textwidth}
%			\begin{figure}
%				\includegraphics[width=\textwidth]{ihs.png}
%			\end{figure}
%		\end{minipage}
%		\vfill
%		\begin{figure}
%			\includegraphics[width=\textwidth]{mus_to_ous_i.png}
%		\end{figure}
%	\end{frame}
%	
%	
%	
%	\begin{frame}{Incrementality}
%		
%		\begin{minipage}{0.59\textwidth}
%			\begin{enumerate}
%				\item {\color{vuborange} Incremental \emph{OUSs} }
%				\begin{itemize}
%					\item Keep Satisfiable Subsets between OUS-calls
%					\item Naïve OUS-restart $\rightarrow$ Keep 1 MIP solver warm per literal
%					\begin{itemize}
%						\item[$\implies$] No need to keep track of Satisfiable Subsets
%					\end{itemize}
%				\end{itemize}	
%			\end{enumerate}
%		\end{minipage}
%		\begin{minipage}{0.4\textwidth}
%			\begin{figure}
%				\includegraphics[width=\textwidth]{ihs.png}
%			\end{figure}
%		\end{minipage}
%	\vfill
%		\begin{figure}
%		\includegraphics[width=\textwidth]{mus_to_ous_i_litincr.png}
%	\end{figure}
%	\end{frame}
%
%	\begin{frame}{Incrementality}
%	
%	\begin{minipage}{0.59\textwidth}
%		\begin{enumerate}
%			\item {\color{vuborange} Incremental \emph{OUSs}}
%			\begin{itemize}
%				\item Keep Satisfiable Subsets between OUS-calls
%				\item Naïve OUS-restart $\rightarrow$ Keep 1 MIP solver warm per literal
%				\begin{itemize}
%					\item[$\implies$] No need to keep track of Satisfiable Subsets
%				\end{itemize}
%			\end{itemize}	
%		\end{enumerate}
%	\end{minipage}
%	\begin{minipage}{0.39\textwidth}
%		\begin{figure}
%			\includegraphics[width=\textwidth]{ihs_cost.png}
%		\end{figure}
%	\end{minipage}
%	\begin{enumerate}
%		\item {\color{vuborange} + Optimizations {\color{blue} (Bounded)}}
%		\begin{itemize}
%			\item Keep track of cost for computed OUSs 
%			\begin{itemize}
%				\item[$\implies$] Use previously found bound on OUS to interrupt if cost(HS) too high 
%			\end{itemize}
%			\item Sort the literals based on cost
%		\end{itemize}
%	\end{enumerate}
%	
%	\begin{figure}
%		\includegraphics[width=\textwidth]{mus_to_ous_i_bounded.png}
%	\end{figure}
%\end{frame}
%
%\begin{frame}{Optimality \& Constrainedness}
%	\begin{figure}
%		\includegraphics[width=0.45\textwidth]{algo_mus2.png}
%	\end{figure}\pause
%	{\Huge$$\downarrow$$}
%	\begin{figure}
%		\includegraphics[width=0.5\textwidth]{exactly_one.png}
%	\end{figure}
%\end{frame}
%
%\begin{frame}{Optimality \& Constrainedness}
%	\begin{minipage}{0.49\textwidth}
%		\begin{figure}
%			\includegraphics[width=\textwidth]{ihs_constrained.png}
%		\end{figure}
%	\end{minipage}
%	\begin{minipage}{0.5\textwidth}
%		% TODO: ADD algorithm  one step !
%		\begin{itemize}
%			\item Inspired by \cite{ignatiev2015smallest}
%			\item MIP solver instead of SAT-based
%			\item[+] Optimal Hitting set (\emph{\textbf{Optimality}})
%			\item[+] {\color{red} 1 literal explained (\emph{\textbf{Constrained}})}
%		\end{itemize}
%	\end{minipage}
%	\vfill
%	\begin{figure}[h]
%		\includegraphics[width=\textwidth]{mus_to_ocus.png}
%	\end{figure}
%	
%\end{frame}
%
%\begin{frame}{Optimality \& Constrainedness + Incrementality}
%		\begin{enumerate}
%			\item \color{vuborange} Incremental \emph{OUSs} + optimizations {\color{blue} (Bounded)}\pause
%			\item {\color{vuborange} Incremental \emph{OCUS}}
%			\begin{itemize}
%				\item Naïve OCUS-restart $\rightarrow$ Keep only 1 MIP solver warm
%			\end{itemize}
%		\end{enumerate}
%		\begin{figure}
%			\includegraphics[width=\textwidth]{mus_to_ocus_i.png}
%		\end{figure}
%	\end{frame}


\begin{frame}{Results}
	\framesubtitle{Explanation configurations compared}
	\begin{figure}
		\includegraphics[width=0.9\textwidth]{figures/cumul_incr_avg_time_lits_derived_new_annotated.pdf}
	\end{figure}
\end{frame}

\begin{frame}{Conclusion}
	In this paper, we introduce (cost-)\textbf{\underline{O}ptimal} \underline{U}nsatisfiable \underline{S}ubsets (OUS) using the \emph{implicit hitting set duality}.
	\begin{description}[font=\color{vuborange}\itshape]
		\item[Optimality] A cost-function quantifies explanation difficulty. \pause
		\item[Efficiency] Add bounded and sorted OUS calls to improve time to find the next best explanation.\pause
		\item[Constrainedness] Analyze the structural constraint on the form of the explanations.\pause
		\item[Incrementality] Reuse-information between successive explanation calls to improve the efficiency of explanation sequence generation.\pause
	\end{description}
	\vfill
\end{frame}

	\begin{frame}
		
		\begin{center}
			{\huge Questions?}
		\end{center}
%		\begin{itemize}
%			\item \href{mailto:emilio.gamba@vub.be}{\underline{emilio.gamba@vub.be}}
%%			\item \href{https://arxiv.org/abs/2105.11763}{\emph{Link to paper}}
%%			\item \href{}{\emph{2 minute talk recording}}
%%			\item \href{}{\emph{15 minute talk recording}}
%		\end{itemize}

		
		
	\end{frame}

	
	\begin{frame}[allowframebreaks]
		\frametitle{References}
		\bibliographystyle{named}
		\bibliography{refs}
	\end{frame}
	

	
\end{document}