\textbf{Logic grid puzzles.}
While our proposed method is applicable to constraint satisfaction problems in general, we use \textit{logic grid puzzles} as example domain, as it requires no expert knowledge to understand.

A logic grid puzzle (also known as Zebra puzzle or Einstein puzzle) consists of natural language sentences (from hereon referred to as ``clues'') over a set of \emph{entities} occurring in those sentences. 
For instance, our running example in Figure~\ref{fig:zebrascreen} contains as second clue ``The person who chose arrabiata sauce is either Angie or Elisa'' and (among others) entities ``arrabiata sauce'', ``Angie'' and ``Elisa''. 

The set of entities is sometimes left implicit if it can be derived from the clues, but often it is given in the form of a grid. Furthermore, in such a puzzle the set of entities is partitioned into equally-sized groups (corresponding to \emph{types}); in our example, ``person'' and ``sauce'' are two such types. 
% 
The goal of the puzzle is to find relations between each two types such that
\begin{compactitem}
	\item each clue is respected, 
	\item each entity of one type is matched with exactly one entity of the second type, e.g., each person chose exactly one sauce and each sauce is linked to one person (this type of constraint will be referred to as \emph{bijectivity}), and 
	\item the relations are logically linked, e.g., if Angie chose arrabiata sauce and arrabiata sauce was paired with farfalle, then Angie must also have eaten farfalle (from now on called \emph{transitivity}). 
\end{compactitem}
In Section \ref{sec:holistic} we explain how we obtain a vocabulary and first-order theory in a mostly automated way from the clues. The result is a vocabulary with types corresponding to the groups of entities in the clues, and the names and types of the binary relations to find (e.g \textit{chose(person, sauce)}, \textit{paired(sauce, pasta)}, \textit{eaten(person, pasta)});
%We furthermore assume that the interpretation of the types is fixed (and hence all interpretations agree on this). 
as well as constraints (first-order sentences) corresponding to the clues, and the bijectivity and transitivity constraints. Let $T_P$ be a theory containing all of these constraints for a given puzzle $P$.

Our running example is a puzzle about people having dinner in a restaurant and ordering different types of pasta. It is the hardest logic grid puzzle we encountered (as a reference, at a recent AI conference, when presenting our tool \cite{DBLP:conf/bnaic/ClaesBCGG19}, only four out of 80 researchers who tried managed to solve it).    
The entire puzzle can be seen in Figure \ref{fig:zebrascreen}; the full final explanation generated for it can be found at \url{http://bartbog.github.io/zebra/pasta}.
% Dout of  and its final produced explanation still contains some challenging steps. 

\myparagraph{Typed first-order logic.}
Our constraint solving method is based on \emph{typed first-order logic}. %, with links to \emph{typed second-order logic}.
Part of the input is a logical vocabulary consisting of a set of type symbols, (typed) constant symbols, and (typed) relation symbols with associated type signature (i.e., each relation symbol is typed $T_1\times \dots \times T_n$ with $T_i$ types).\footnote{We here omit function symbols since they are not used in this paper.} For example, type \textit{person} with constant symbol \textit{Angie} of type \textit{person} and a relation \textit{chose(.,.)} with signature \textit{person $\times$ sauce}.


A \emph{first-order theory} is a set of sentences (well-formed variable-free first-order formulas in which each quantified variable has an associated type), also referred to as constraints.
Since we work in a fixed and finite domain, the vocabulary, the interpretation of the types (the domains) and the constants are fixed.
This justifies the following definition. 
\begin{definition}
 A \emph{(partial) interpretation} is a finite set of literals, i.e., expressions of the form $P(\ddd)$ or $\lnot P(\ddd)$ where $P$ is a relation symbol typed $T_1\times\dots \times T_n$ and $\ddd$ is a tuple of domain elements where each $d_i$ is of type $T_i$. 

 A partial interpretation is \emph{consistent} if it does not contain both an atom and its negation, it is called a \emph{full} interpretation if it either contains $P(\ddd)$ or $\lnot P(\ddd)$ for each well-typed atom $P(\ddd)$. 
\end{definition}
For instance in the partial interpretation $I_{Ang-Ar}=\{chose(Angie,arrabiata),$ $\lnot chose(Elisa,arrabiata)\}$ it is known that $Angia$ had arrabiata sauce and Elisa did not. 
% \tias{This needs a running example example}
% 
% Given a logical vocabulary $V$, a \emph{\textbf{partial} interpretation} $I$ assigns to each type symbol $T$ (e.g. \textit{person}) a finite set $I(T)$ and to each 
% relation symbol $P$ with type signature $T_1\times \dots \times T_n$ a function 
% \[I(P): I(T_1)\times \dots \times I(T_n)\to \{\ltrue,\lunkn,\lfalse\},\] 
% where $\ltrue$ stands for true, $\lunkn$ for unknown, and $\lfalse$ for false. In case all functions $I(P)$ map all tuples into $\{\ltrue,\lfalse\}$ (i.e., no more tuples are unknown), we call $I$ a \emph{full interpretation} (this is sometimes also called a \emph{total} or \emph{two-valued} interpretation, or simply \emph{an interpretation}). 
A partial interpretation $I_1$ is \emph{more precise} than partial interpretation $I_2$ (notation $I_1\geqp I_2$) if $I_1\supseteq I_2$.
The partial interpretation $\{chose(Angie,arrabiata),$ $\lnot chose(Elisa,arrabiata),$ $ \lnot chose(damon,arrabiata)\}$ is more precise than $I_{Ang-Ar}$. 

Since variable-free literals are also sentences, we will freely use a partial interpretation as (a part of) a theory in solver calls or in statements of the form $I\land T \models J$, meaning that everything in $J$ is a consequence of $I$ and $T$, or stated differently, that $J$ is less precise than any model $M$ of $T$ satisfying $M\geqp I$. 

In the context of first-order logic, the task of finite-domain constraint solving is better known as \emph{model expansion} \cite{MitchellTHM06}: given a logical theory $T$ (corresponding to the constraint specification) and a partial interpretation $I$ with a finite domain (corresponding to the initial domain of the variables), find a model $M$ more precise than $I$ (a more restricted domain, e.g. a partial solution that satisfies $T$).
% Viewing (partial) interpretations as sets of literals allows us to also use them interchangeably as a theory; we will make use of this in our algorithms.

% 
% In a second-order theory also quantifiers over relations have an associated type signature. 
% \bart{Clear or do I make it more formal}
% 
% \bart{There might be some related work in interactive theorem proving. If the system proposes a ``simple'' proof of a theorem... Very similar to what we do here.}
