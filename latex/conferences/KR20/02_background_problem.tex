A \emph{(partial) interpretation} is defined as a finite set of literals, i.e., expressions of the form $P(\ddd)$ or $\lnot P(\ddd)$ where $P$ is a relation symbol typed $T_1\times\dots \times T_n$ and $\ddd$ is a tuple of domain elements where each $d_i$ is of type $T_i$.
A partial interpretation is \emph{consistent} if it does not contain both an atom and its negation, it is called a \emph{full} interpretation if it either contains $P(\ddd)$ or $\lnot P(\ddd)$ for each well-typed atom $P(\ddd)$. In the context of first-order logic, the task of finite-domain constraint solving is better known as \emph{model expansion} \cite{MitchellTHM06}: given a logical theory $T$ (corresponding to the constraint specification) and a partial interpretation $I$ with a finite domain (corresponding to the initial domain of the variables), find a model $M$ more precise than $I$ (a partial solution that satisfies $T$).

\subsection{Explanation of reasoning steps}
We assume that a theory $\m{T}$ and an initial partial interpretation $m{I}_0$ are given and fixed. We define the \textbf{maximal consequence} of a theory $\allconstraints$ and partial interpretation $I$ (denoted $max(I,T)$) as the precision-maximal partial interpretation $J$ such that  $I \wedge \allconstraints \entails J$. 
The 