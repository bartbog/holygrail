% !TeX root = ./main.tex

Summarised, while \citet{ecai/BogaertsGCG20} had to generate many MUSs for finding a single next explanation step, the step-wise explanation generation directly benefits from the \comus approach of finding the optimal explanation candidate at any point in the sequence.

There are many trade-offs to be made in the algorithm, including time spent computing optimal or non-optimal hitting sets, how to reuse information and how to grow the satisfiable subsets. While we studied the key algorithmic dimensions of information reuse, a deeper study of alternative approaches to balance these trade-offs is needed, for instance comparing different methods to grow a satisfiable subset, and balancing the trade-off between finding hitting sets of high quality versus having to compute fewer hitting sets.

Additionally, the problem of computing a $(p,f)$-OUS could be encoded in QMaxSAT \cite{DBLP:journals/constraints/IgnatievJM16}. An in-depth comparison between our algorithms and QMaxSAT solvers could reveal additional points of improvement.

The concept of OUS, incremental OUS and constrained OUS are not limited to explanations of satisfaction problems and we are keen to explore other applications too.

From the explanation point of view, a next challenge is how to compute explanations for optimisation problems, that is, where decisions are made based on search and not just propagation. We believe a constrained OUS algorithm can also play a key part in that. Finally, an open challenge is that of defining appropriate cost functions for generating explanations, including how to evaluate what a ``good'' explanation sequence is. While we currently find a sequence where each step is optimal with respect to the cost function, we have yet to consider whether it is possible to optimize a cost function over the entire sequence as well.

%\begin{itemize}
%    \item Conclusion on resutls
%    \item Challenges next : extension to XOPT ? 
%    \item Applicability on other problems ?
%    \item Characterizing explanation difficulty
%\end{itemize}
