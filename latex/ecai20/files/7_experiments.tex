The explanation-generating search has been tested against unseen puzzles on a computer with the following specifications:

\begin{table}
	\centering
	\resizebox{0.8\columnwidth}{!}{%
		\begin{tabular}{|l|l|l|l|}
			\hline 
			CPU & CPU Freq. & cores & RAM Size \\ 
			\hline 
			Intel(R) Xeon(R) CPU E3-1225 & 3.30 GHz & 4 & 32 Gb \\ 
			\hline 
		\end{tabular} 
	}
\caption{Computer system specifications}
\label{table:system_specifications}
\end{table}

We analyze the results in 3 directions :

\paragraph{Q1. How do different puzzles compare with respect to the generated explanations ?} For future reference, we denote \textbf{c} = $c_1, c2, ...c_n$ the costs, where $c_i$ is the cost, based on the cost function $f(I, C)$, at reasoning step $i$ and n the number of reasoning steps to solve a problem. Table \ref{table:sequence_leve} shows how difficult the puzzles are by means of the total cost and how the expensive a new conclusion is based on the current state of the grid.

\begin{table}
	\centering
	\resizebox{\columnwidth}{!}{%
\begin{tabular}{|c||c|c|c|c|c||c|c|c|c|c|c|} 
\hline 
\textbf{Puzzle} & \textbf{n} & $\sum_{i = 1}^{n} c_i$  & max(\textbf{c}) & $\overline{\text{\textbf{c}}}$ & \textbf{5 Highest costs} & \textbf{\% bij.} & \textbf{\% trans.} & \textbf{\% comb.} & \textbf{\% clue} & \textbf{\% multi. clues} \\ 
\hline 
p5 & 113 & 626 & 25 & 5.59 & [25, 21, 21, 21, 21] & 31.0 & 50.0 & 0 & 19.0 & 0 \\ 
\hline 
p16 & 122 & 680 & 23 & 5.62 & [23, 21, 21, 21, 21] & 21.0 & 60.0 & 0 & 19.0 & 0 \\ 
\hline 
p93 & 119 & 659 & 21 & 5.58 & [21, 21, 21, 21, 20] & 34.0 & 47.0 & 0 & 19.0 & 0 \\ 
\hline 
\textbf{p12} & \textbf{115} & \textbf{591} & \textbf{23} & \textbf{5.18} & \textbf{[23, 21, 21, 21, 20]} & \textbf{29.0} & \textbf{54.0} & \textbf{0} & \textbf{17.0} & \textbf{0} \\ 
\hline 
\textbf{p18} & \textbf{116} & \textbf{615} & \textbf{22} & \textbf{5.35} & \textbf{[22, 22, 22, 21, 21]} & \textbf{28.0} & \textbf{55.0 }& \textbf{0 }& \textbf{17.0} & \textbf{0 }\\ 
\hline 
p20 & 116 & 552 & 22 & 4.8 & [22, 21, 21, 21, 20] & 27.0 & 58.0 & 0 & 15.0 & 0 \\ 
\hline 
\textbf{p25} & \textbf{111} & \textbf{595} & \textbf{24} & \textbf{5.41} & \textbf{[24, 23, 22, 21, 21]} & \textbf{37.0} & \textbf{45.0} & \textbf{0} &\textbf{ 17.0} & \textbf{0} \\ 
\hline 
p19 & 123 & 630 & 22 & 5.16 & [22, 21, 21, 21, 21] & 25.0 & 59.0 & 0 & 16.0 & 0 \\ 
\hline
\end{tabular} 
	}
\caption{Puzzle explanation cost based on the cost function $f(I, C)$ and statistics on puzzle constraints}
\label{table:sequence_leve}
\end{table}

While the first part of the table \ref{table:sequence_leve} only analyzes the difficulty of a puzzle, the second part focuses on the explanations. 
Most of the explanations are either generated using bijectivity (bij.) or Transitivity (trans.). The rest are found using the 1 or more clues, or by combining (comb.) transitivity with bijectivity. This is due to the way puzzles are formulated in natural language for logic grid puzzles.

\paragraph{Q2. How does the level of difficulty progress throughout the explanation?} Out of the problems in table \ref{table:sequence_leve}, 3 puzzle instances were selected: \textit{p12}, \textit{p18} and \textit{p25}. These puzzles present a similar number of explanation steps and total costs.

\begin{figure}
	\centering
	\includegraphics[width=0.7\linewidth]{figures/plot_cost_steps.png}
		\caption{Explanation cost for puzzle instances p12, p18, p25}
			\label{fig:plotcoststeps}
\end{figure}

\begin{table}
	\centering
	\resizebox{\columnwidth}{!}{%
		\begin{tabular}{|l||c|c|c|c|c|c|c|} 
			\hline 
			\textbf{puzzle} & \textbf{tot exec time} & avg clue time  & avg bij. & avg trans. & avg time/cost of clue & avg time/cost of bij. & avg time/cost of bij. \\ 
			\hline 
			p5.output.json & 386.0 & 129.29 & 226.27 & 204.23 & 6.13 & 145.18 & 102.11 \\ 
			\hline 
			p16.output.json & 576.0 & 192.97 & 370.55 & 426.41 & 9.22 & 213.63 & 213.21 \\ 
			\hline 
			p93.output.json & 946.0 & 213.13 & 604.13 & 425.58 & 10.44 & 340.27 & 212.79 \\ 
			\hline 
			p12.output.json & 1103.0 & 271.13 & 744.93 & 799.79 & 12.87 & 432.21 & 399.89 \\ 
			\hline 
			p18.output.json & 1924.0 & 323.82 & 856.52 & 881.44 & 15.14 & 421.34 & 440.72 \\ 
			\hline 
			p20.output.json & 411.0 & 117.57 & 235.34 & 235.29 & 5.69 & 147.65 & 117.64 \\ 
			\hline 
			p25.output.json & 886.0 & 291.12 & 659.88 & 623.77 & 13.42 & 404.61 & 311.89 \\ 
			\hline 
			p19.output.json & 1207.0 & 265.82 & 538.64 & 602.01 & 12.77 & 293.38 & 301.0 \\ 
			\hline 
		\end{tabular} 
	}
	\caption{Puzzle explanation cost based on the cost function $f(I, C)$ and statistics on puzzle constraints}
	\label{table:sequence_leve}
\end{table}


%\begin{figure}
%	\centering
%	\begin{minipage}{0.45\linewidth}
%	\centering
%	\includegraphics[width=\textwidth]{figures/plot_exec_time.png}
%	\caption{Execution time for puzzle instances p12, p18, p25}
%	\label{fig:plotcoststeps2}
%	\end{minipage}\hfill
%	\begin{minipage}{0.45\linewidth}
%	\centering
%	\includegraphics[width=\textwidth]{figures/plot_cost_steps.png}
%	\caption{Explanation cost for puzzle instances p12, p18, p25}
%	\label{fig:plotcoststeps}
%	\end{minipage}
%\end{figure}
%\begin{figure}
%	\centering
%	\begin{subfigure}{.5\textwidth}

%	\end{subfigure}%
%	\begin{subfigure}{.5\textwidth}
%
%	\end{subfigure}
%	\caption{A figure with two subfigures}
%	\label{fig:test}
%\end{figure}

\paragraph{Q3. How is the performance and the level of difficulty affected by removing parts of the algorithm ? } bla bla bla

% TODO table : Leaving some parts of the algo out, to see step-wise improvements of the components?  Some experiment with different levels of abstraction (e.g. all constraints at once, clues separate but all implicit at once, 2 groups of implicits, full split of implicits) with 'set of measures', so probably a table

\paragraph{Bonus. How well does it compare to human solving process ?} 

% TODO Compare with the tutorial puzzle of logicgridpuzzles.com? Ideally, a small human evaluation (e.g. ask people to solve a 3-with-3 puzzle and note the order of derivations and clues used, compare this 'ranking' to our ranking, discuss some differences.

