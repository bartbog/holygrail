\documentclass{article}
%\usepackage{fullpage}
\usepackage{geometry}
 \geometry{
 a4paper,
 total={170mm,257mm},
 left=10mm,
 top=10mm,
 }
\usepackage{graphicx}
\usepackage{hyperref}
\usepackage{xcolor}
\usepackage{amsmath,amssymb}
\usepackage[ruled,linesnumbered]{algorithm2e}
\usepackage{float}
\usepackage{amsthm}
\usepackage{xspace}
\usepackage{thmtools} 
\usepackage[parfill]{parskip}
\usepackage[normalem]{ulem}

\definecolor{vuborange}{rgb}{1.0,0.40,0.0}
\newcommand\m[1]{\mathcal{#1}}

\newtheorem{thm}{Theorem}
\newtheorem{definition}[thm]{Definition}
\newtheorem{prop}{Property}
\newtheorem{property}[prop]{Property}
\newtheorem{lem}{Lemma}
\newtheorem{lemma}[lem]{Lemma}
\newtheorem{propo}{Proposition}
\newtheorem{proposition}[propo]{Proposition}
\newtheorem{ex}{Example}
\newtheorem{example}[ex]{Example}

\bibliographystyle{apalike}
\newcommand\comment[1]{\marginpar{\tiny #1}}
\renewcommand\comment[1]{#1}
\newcommand{\tias}[1]{{\comment{\color{blue}\textsc{TG:}#1}}}
\newcommand{\emilio}[1]{{\comment{\color{red}#1}}}
\newcommand{\bart}[1]{{\comment{\color{green}#1}}}
\newcommand{\todo}[1]{{\comment{\color{blue}#1}}}

\newcommand\setstohit{\ensuremath{\m{H} }\xspace}
\newcommand\F{\ensuremath{\m{F} }\xspace}
% \newcommand\ohs{\ensuremath{\m{OHS} }\xspace}
\newcommand\setstohitall{\ensuremath{\m{H}_\mathit{all} }\xspace}
\newcommand\Iend{\ensuremath{I_\mathit{end} }\xspace}
\newcommand\formula{\ensuremath{\m{F} }\xspace}
\newcommand\formulac{\ensuremath{\m{C} }\xspace}
\newcommand\formulag{\ensuremath{\m{G} }\xspace}
\newcommand\mm[1]{\ensuremath{#1}\xspace}
\newcommand\nat{\mm{\mathbb{N}}}
\newcommand\ltrue{\mm{\textbf{t}}}
\newcommand\lfalse{\mm{\textbf{f}}}
\newcommand\uservars{\ensuremath{\m{U} }\xspace}

\newcommand\call[1]{\mm{\textsc{#1}}}
\newcommand\geths{\mm{\call{GetHittingSet}}}
\newcommand\ohs{\mm{\call{OptHittingSet}}}
\newcommand\ghs{\mm{\call{GreedyHittingSet}}}
\newcommand\ihs{\mm{\call{IncrementalHittingSet}}}
\newcommand\cohs{\mm{\call{CondOptHittingSet}}}
\newcommand\chs{\mm{\call{CondHittingSet}}}
\newcommand\sat{\texttt{{SAT}}}
\newcommand\grow{\mm{\call{Grow}}}
\newcommand\omus{\mm{\call{OUS}}}
\newcommand\comus{\mm{\call{c-OUS}}}
\newcommand\omusinc{\mm{\call{OUS-Inc}}}
\newcommand\store{\mm{\call{Store}}}
\newcommand\optprop{\mm{\call{OptimalPropagate}}}
\newcommand\initsat{\mm{\call{InitSat}}}
\newcommand\hitsetbased{hitting set--based\xspace} %en-dash!
\newcommand\satsets{\mm{\mathbf{SSs}}}
\newcommand\fall{\mm{\formula_{\mathit{all}}}}
\newcommand\algemilio[1]{\emilio{#1}\;}
\begin{document}

\SetKwInOut{Input}{Input}
\SetKwInOut{OptInput}{Optional}
\SetKwInOut{Output}{Output}
\SetKwInOut{State}{State}
\SetKwInOut{Ext.}{Ext}
\SetKwComment{command}{/*}{*/}

\section{Prioirity TaskList}
\subsection{short-term goals + High priority}
\begin{itemize}
  \item \todo{Re-read OCUS paper review-style}
  \item \todo{Document what has been done + what not + choices}
  \item \todo{Clean code !}
  \item \todo{Re-read SMUS paper and see whath we missed and what choices we made in the OCUS (pocus-pas) paper.} 
  \item \todo{Re-read postponing optimisation paper and see whath we missed and what choices we made in the OCUS (pocus-pas) paper.}
  \item \todo{Keeping track of the subsets in the same way as we do for OCUS incr}
  \item \todo{Ability to zoom-in on explanations generated i.e \textit{using nested explanations ?}}
  \item \todo{Generate the explanation sequence with OCUS with the same cost function as in ECAI}
\end{itemize}
\subsection{Long-term goals (low priority)}
\begin{itemize}
  \item \todo{Trasnlate the weighted cosntriants from OUS into an SMUS cnf specification}
  \item \todo{setup experiment for quantifying difference in computation of this approach}
  \item \todo{Look for a better cost-function defined on the constraints.}
  \item \todo{Look for meta-constraints added to the specification}
  \item \todo{Define how the nested explanations are constructed: which constriants are activable in the nested explanation reasoning and which can't or are prohibited.}
  \item \todo{Present the user with alternative explanations (of the same cost or similar costs)}
\end{itemize}


\section{IJCAI}
\subsection{SMUS}
From early experiments we see that SMUS outperforms OUS on CNF instances when they both solve the same problem, i.e. OUS with unit weights on cosntraints.

\begin{enumerate}
  \item OUS als SMUS
  \begin{description}
    \item[Experiments] \todo{Test on a large scale how both systems perform on CNF isntances}
    \item[Theory] \todo{Re-read SMUS paper and see whath we missed and what choices we made in the OCUS (pocus-pas) paper.} 
    % \item \todo{}
  \end{description}
  \item SMUS als OUS
  \begin{description}
    \item[Implementation] \todo{Trasnlate the weighted cosntriants from OUS into an SMUS cnf specification}
    \begin{itemize}
      \item Ex: Constriants weighted 10, is implemneted as 10 literals wich together have to be satisfied in order to use the constraint.
    \end{itemize}
    \item[Experiment] \todo{setup experiment for quantifying difference in computation of this approach}
    % \item \todo{}
  \end{description}
\end{enumerate}

\subsection{Postponing optimisation}
Experiments from $\m{OCUS}$ show that postponing optimisation doesn't improve the computation time for building the whole explanation sequencE.
\begin{description}
  \item[Theory] \todo{Re-read postponing optimisation paper and see whath we missed and what choices we made in the OCUS (pocus-pas) paper.}
  \item[Implementation] \todo{Keeping track of the subsets in the same way as we do for OCUS incr}
\end{description}

\subsection{SUDOKU}\label{sudoku}
Sudoku encoding in python notebook in "experiments/03\_OMUS/02\_OUS/explain\_sudoku.ipynb". 
\begin{description}
  \item[Efficiency] Explanation generation is slow!
  \item[Explanation Quality] The cost function defined for explaining generates explanations that can be better. For example, for a given square, the explanation algo, needs to derive a lot of negative literals (i.e. 8) which is cheaper than deriving the positive knowing together with the constraints of the 8 negatives.
  \begin{description}
    \item[Theory] \todo{Look for meta-constraints added to the specification}
    \item[Theory] \todo{Look for a better cost-function defined on the constraints.}
    \item[Future work] \todo{Defining a way to learn cost function strucutures based on the feedback of users?}
    % \item \todo{Idea for a Application paper with explaining sudoku}
    \item[Implementation] \todo{Ability to zoom-in on explanations generated i.e \textit{using nested explanations ?}}
  \end{description}
  % \item \todo{}
\end{description}

\subsection{Visualisation}
Code for generating the visualisation in python is ready. 
\begin{description}
  \item[Implementation] \todo{Generate the explanation sequence with OCUS with the same cost function as in ECAI}
\end{description}

\subsection{Explanations}
Ideas for improving the current OCUS explanations.
\subsubsection{Nested explanations}
When deriving a fact, try to generate explanation with counterfactual reasoning.
\begin{description}
  \item[Implementation] \todo{Generate nested explanations with OCUS}
  \item[Theory] \todo{Define how the nested explanations are constructed. Which constriants are activable in the nested explanation reasoning and which can't or are prohibited.}
  \begin{itemize}
    \item Example: For a given square in the grid. Certain explanations in the puzzles are using 2 bijectives in order to be able to use the information for a third bijectivity. THe problem is that the third bijectivity could have been used in the first place, but \textbf{wasn't} because the explanation algorithm prohibited from using these cosntraints.
  \end{itemize}
\end{description}

\subsubsection{Alternative explanations - User experiments}
See future work of \ref{sudoku}. 
\begin{description}
  \item[Theory] \todo{Present the user with alternative explanations (of the same cost or similar costs)}
\end{description}


\section{FWO}
Content to be discussed at the kick-off meeting.


\end{document}
