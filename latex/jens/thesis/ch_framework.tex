\chapter{Een framework voor semantische analyse}
\label{ch:framework}
In dit hoofdstuk bespreken we het framework van Blackburn en Bos \cite{Blackburn2005, Blackburn2006} voor semantische analyse. Het framework bestaat uit een lexicon (het vocabularium), de grammatica, de semantiek van de grammaticale regels en de semantiek van de woorden in het lexicon. Het hele framework is gebaseerd op lambda-calculus en Frege's compositionaliteitsprincipe dat stelt dat de betekenis van een woordgroep enkel afhangt van de betekenissen van de woorden waaruit ze bestaat.

\paragraph{} \textit{Dit hele hoofdstuk is een samenvatting van de relevante hoofdstukken van de boeken van Blackburn en Bos \cite{Blackburn2005, Blackburn2006}. De formules komen grotendeels overeen met die uit hun boeken en uit de code die erbij hoort. De aanpassingen die zijn gebeurd, dienen vooral om de leesbaarheid te verhogen. De nadruk op de signaturen van de verschillende categorieën is nieuw, al komt de getypeerde lambda-calculus met de twee types wel aan bod in appendix B van hun eerste boek \cite{Blackburn2005}.}

\section{Lexicon}
Het lexicon bestaat uit een opsomming van alle woorden met een aantal (taalkundige) features (zoals de categorie van het woord). Tabel \ref{tbl:lexicon} geeft een voorbeeld van een lexicon. Het lexicon is meer dan een woordenboek. Het bevat alle woordvormen, niet enkel de basisvorm. Zo komt ``love'' drie keer voor in het lexicon. ``love'' zelf komt één keer voor als infinitief en één keer als meervoud van de tegenwoordige tijd. ``loves'' is dan weer het enkelvoud van de tegenwoordige tijd.

De meeste woordvormen hebben ook een feature \texttt{Symbool} die zal gebruikt worden bij de semantiek van de woorden. Deze feature maakt het onderscheid tussen de verschillende woorden van dezelfde categorie. Het symbool komt overeen met de naam van de constante of van het predicaat die het woord introduceert in het formeel vocabularium (zie sectie~\ref{sec:vocabularium}).

\begin{table}[!]
  \centering
  \begin{tabular}{@{}llll@{}}
    \toprule
    \textbf{Woordvorm} & \textbf{Categorie} & \textbf{Symbool} & \textbf{Andere features} \\ \midrule
    man                & zelfstandig naamwoord     & man     & num=sg            \\
    men                & zelfstandig naamwoord     & man     & num=pl            \\
    woman              & zelfstandig naamwoord     & woman   & num=sg            \\
    women              & zelfstandig naamwoord     & woman   & num=pl            \\
    John               & eigennaam                 & John    &                   \\
    sleep              & onovergankelijk werkwoord & sleeps  & inf=inf           \\
    sleeps             & onovergankelijk werkwoord & sleeps  & inf=fin, num=sg   \\
    sleep              & onovergankelijk werkwoord & sleeps  & inf=fin, num=pl   \\
    love               & overgankelijk werkwoord   & loves   & inf=inf           \\
    loves              & overgankelijk werkwoord   & loves   & inf=fin, num=sg   \\
    love               & overgankelijk werkwoord   & loves   & inf=fin, num=pl   \\
    a                  & determinator              & /       & type=existential  \\
    every              & determinator              & /       & type=universal    \\
    \bottomrule
  \end{tabular}
  \caption{Een voorbeeld van een lexicon}
  \label{tbl:lexicon}
\end{table}

\section{Grammatica}
De grammatica bepaalt welke woorden samen woordgroepen vormen, welke woordgroepen samen andere woordgroepen vormen en welke woordgroepen een zin vormen. Op die manier ontstaat er een boom van woorden.

\paragraph{Een simpele grammatica} Grammatica~\ref{dcg:simple-gramm} bevat een simpele grammatica. Een zin bestaat in deze grammatica uit een \texttt{np} gevolgd door een \texttt{vp}, beiden met hetzelfde getal. Een naamwoordgroep (noun phrase of \texttt{np}) is een woordgroep die naar één of meerdere entiteiten verwijst. Zo'n groep wordt ook wel een nominale constituent genoemd. Een verb phrase (\texttt{vp}) of verbale constituent drukt meestal een actie uit.

\begin{dcg}{Een simpele grammatica}{dcg:simple-gramm}
s() --> np([num:Num]), vp([num:Num]).
s() --> [if], s(), s().
np([num:sg]) --> pn().
np([num:Num]) --> det([num:Num]), n([num:Num]).
vp([num:Num]) --> iv([num:Num]).
vp([num:Num]) --> tv([num:Num]), np([num:_]).
\end{dcg} 

Een zin kan ook bestaan uit het functiewoord ``if'', gevolgd door twee zinnen (bijvoorbeeld ``If a man breathes, he lives''). Functiewoorden zijn deel van de grammatica en komen niet voor in het lexicon. Ze helpen om de structuur van de zin te herkennen. De betekenis van deze woorden komt via de semantiek van de grammatica naar boven.

Een naamwoordgroep (\texttt{np}) kan bestaan uit een eigen naam (proper name of \texttt{pn}) of uit een determinator (\texttt{det}, bijvoorbeeld een lidwoord) en een zelfstandig naamwoord (noun of \texttt{n}) die overeenkomen in getal. Een eigennaam is in deze grammatica altijd in het enkelvoud.

Een verbale constituent (\texttt{vp}) bestaat uit een onovergankelijk werkwoord (intransitive verb of \texttt{iv}) of uit een vergankelijk werkwoord (transitive verb of \texttt{tv}) gevolgd door een nieuwe naamwoordgroep (als lijdend voorwerp). Het werkwoord moet in getal overeenkomen met het getal van de verbale constituent. Daardoor zal het getal van het werkwoord en het onderwerp altijd overeenkomen.

Grammatica~\ref{dcg:simple-gramm} bevat een aantal categorieën (\texttt{pn}, \texttt{det}, \texttt{n}, \texttt{iv} en \texttt{tv}) die overeenkomen met de lexicale categorieën. Voor deze categorieën wordt er in het lexicon gezocht naar een woord met de juiste features.

Bovenstaande grammatica is nog heel beperkt. De moeilijkheid ligt erin om de grammatica simpel te houden maar toch zoveel mogelijk gewenste zinnen toe te laten. Om logigrammen automatisch te kunnen vertalen moet er dus een grammatica opgesteld worden die de zinnen van deze logigrammen omvat.

\paragraph{Een boom} Op basis van deze grammatica kunnen we ook een parse tree opbouwen voor elke geldige zin. Zo wordt ``Every woman loves john omgezet in de boom

\Tree[.s [.np [.det every ] [.n woman ]] [.vp [.tv loves ] [.np [.pn john ]]]]

Op elke knoop in deze boom zullen we later Frege's compositionaliteitsprincipe toepassen: de betekenis van een woordgroep is gelijk aan een combinatie van de betekenissen van de woord(groep)en waaruit ze bestaat.

\paragraph{Conclusie} De grammatica bepaalt welke combinaties van woorden zinnen vormen. Ze bepaalt dus welke zinnen in de taal liggen en welke er buiten vallen. Bovendien geeft de grammatica ons een boom. Deze boom zullen we gebruiken om de betekenis van de woorden naar boven toe te laten propageren volgens Frege's compositionaliteitsprincipe tot de betekenis van de zin.

\section{Semantiek}
Het lexicon bepaalt welke woorden gebruikt mogen worden, de grammatica hoe deze woorden een zin kunnen vormen. De vraag die nog rest is welke betekenis de zin heeft. Daarvoor doen we een beroep op de lambda-calculus. Eerst bespreken we hoe we de betekenis van een woordgroep kunnen afleiden uit de betekenis van de woordgroepen waaruit ze bestaan. Daarna bespreken we de betekenis van de woorden uit het lexicon.

\subsection{Semantiek van de grammaticale regels}
\paragraph{Een getypeerde lambda-calculus}
Voor de betekenis van de taal zullen we gebruiken maken van een getypeerde lambda-calculus. Er zijn twee basistypes in onze lambda-calculus, namelijk $e$ voor entiteiten en $t$ voor waarheidswaarden (\textit{waar} en \textit{onwaar}). Ten slotte is er de type-constructor voor een functie-type die we noteren met $\rightarrow$. Zo is $\lambda x \cdot man(x)$ een lambda-expressie van type $e \rightarrow t$. Elk woord zal een lambda-formule als betekenis krijgen. Deze formules zullen gecombineerd worden tot één formule voor de zin die geen lambda's meer bevat. Die formule vormt de betekenis van de zin. De betekenis van een zin is dus van type $t$.

\paragraph{Frege's compositionaliteitsprincipe} Voor de betekenis van de grammatica berusten we op Frege's compositionaliteitsprincipe: De betekenis van een woordgroep bestaat uit een combinatie van de betekenissen van de woorden of woordgroepen waaruit ze bestaat. Deze combinatie kan afhangen van de manier waarop de woorden gecombineerd worden. Op deze manier propageren we de semantiek van de woorden (die de bladeren in de boom vormen) naar boven toe om zo de betekenis van de zin te verkrijgen. We hernemen bovenstaande grammaticale regels nu en voegen de semantiek toe. Later zullen we aantonen dat deze simpele combinatieregels tot een zinnig resultaat kunnen leiden.

\begin{table}[h]
  \begin{tabular}{@{}ll}
    \hline
    \textbf{Grammaticale regel} & \textbf{Semantiek} \\
    \hline
    \texttt{s ---> np([num:Num]), vp([num:Num]).}              & $\sem{s} = \app{\sem{vp}}{\sem{np}}$ \\
    \texttt{s ---> [if], s$_1$, s$_2$.}                        & $\sem{s} = \drs{}{\sem{s_1} \Rightarrow \sem{s_2}}$ \\
    \texttt{np([num:sg]) ---> pn.}                             & $\sem{np} = \sem{pn}$ \\
    \texttt{np([num:Num]) ---> det([num:Num]), n([num:Num]).}  & $\sem{np} = \app{\sem{det}}{\sem{n}}$ \\
    \texttt{vp([num:Num]) ---> iv([num:Num]).}                 & $\sem{vp} = \sem{iv}$ \\
    \texttt{vp([num:Num]) ---> tv([num:Num]), np([num:\_]).}   & $\sem{vp} = \app{\sem{tv}}{\sem{np}}$\\
    \hline
  \end{tabular}
  \centering
  \caption{De semantiek van grammatica~\ref{dcg:simple-gramm}}
  \label{tbl:grammar-sem}
\end{table}

Tabel~\ref{tbl:grammar-sem} geeft een overzicht van de semantiek van grammatica~\ref{dcg:simple-gramm}. Een idee achter het framework is om de betekenis van de zinnen zoveel mogelijk in de betekenis van de woorden te stoppen. De betekenis van een grammaticale regel wordt, indien mogelijk, beperkt tot lambda-applicaties. Voor woordgroepen die maar uit één woord bestaan is de betekenis van de woordgroep gelijk aan die van het woord zelf. Voor de meeste andere woordgroepen is de betekenis enkel een lambda-applicatie van de betekenissen van de woordgroepen waaruit ze bestaan. Meestal heeft een woordgroep één woord dat essentieel is aan de groep. De woordgroep is hier dan ook naar vernoemd\footnote{Voor de semantiek van een naamwoordgroep wordt het lidwoord toch als het belangrijkste woord aanzien door Blackburn en Bos}. Dit woord is de functor van de lambda-applicatie, het andere woord het argument. Enkel voor de speciale zinsstructuur van de conditionele zin is er ook een speciale constructie nodig in de semantiek. Indien we het functiewoord ``if'' naar het lexicon zouden verhuizen, dan zouden twee lambda-applicaties volstaan\footnote{$\sem{if} = \lambdaf{S1}{\lambdaf{S2}{\drs{}{S1 \Rightarrow S2}}}$ en $\sem{s} = \app{\app{\sem{if}}{\sem{s_1}}}{\sem{s_2}}$}.

\subsection{Semantiek van het lexicon} We weten nu hoe we de semantiek van de woorden kunnen combineren tot de semantiek van de woordgroepen en bij uitbreiding tot die van een zin. Er ontbreekt enkel nog de semantiek van de woorden zelf.

Voor we de betekenis van woorden kunnen opstellen moeten we eerst de signatuur achterhalen. Cruciaal voor het framework is dat elke grammaticale categorie exact 1 signatuur heeft. Zodanig dat we altijd woordgroepen van dezelfde categorie kunnen uitwisselen. We gebruiken de functie $\tau$ om de signatuur aan te duiden.

We beginnen met de signatuur van een zelfstandig naamwoord (\texttt{n}) te achterhalen. Daarna bekijken we achtereenvolgens de nominale constituent (\texttt{np}), de eigennaam (\texttt{pn}) en het lidwoord (\texttt{det}). Ten slotte bekijken we de hiërarchie van de verbale constituenten: de verbale constituent zelf (\texttt{vp}), onovergankelijk werkwoord (\texttt{iv}) en overgankelijk werkwoord (\texttt{tv}).

\paragraph{De signatuur van een zelfstandig naamwoord.} Een zelfstandig naamwoord, zoals bijvoorbeeld $\mathit{man}$, stelt een verzameling entiteiten voor. In het voorbeeld van $\mathit{man}$ stelt dit de verzameling van alle mannen voor. In logica wordt zo'n verzameling typisch voorgesteld door een unair predicaat. In deze setting gebeurt net hetzelfde: we stellen een zelfstandig naamwoord voor als een functie van entiteiten naar waarheidswaarden, of in formule $e \to t$. Dit is de functionele variant van een unair predicaat. De functie van een substantief bepaalt of een bepaalde entiteit omschreven kan worden met dat substantief of niet.

\paragraph{De signatuur voor een nominale constituent (np)} Een nominale constituent of naamwoordgroep is een woordgroep die kan dienen als onderwerp of lijdend voorwerp. De betekenis ervan is een verwijzing naar een entiteit (of een groep van entiteiten). In een zin zeggen we altijd iets over deze entiteiten, bijvoorbeeld ``a man sleeps''. Een zin is waar als de entiteit(en) van de naamwoordgroep voldoen aan een bepaalde eigenschap, bijvoorbeeld als er een man is die de eigenschap heeft dat hij slaapt. Dat is wat signatuur $$\tau(np) = (e \to t) \to t$$ uitdrukt. Het eerste argument van type $e \to t$ is een eigenschap waaraan entiteiten kunnen voldoen. Bijvoorbeeld voor ``sleeps'' is dit (in eerste-orde-logica) $\lambdaf{x}{sleeps(x)}$.

Een naamwoordgroep voldoet aan een eigenschap als er \textit{genoeg} entiteiten zijn die omschreven worden door de naamwoordgroep en die voldoen aan de eigenschap. Een voorbeeldimplementatie voor ``a man'' (in eerste-orde-logica) is $$\lambdaf{E}{\exists x \cdot man(x) \land E(x)}$$ Er moet namelijk minstens één man zijn die aan de eigenschap $E$ voldoet. In het geval van ``every man'' moeten alle mannen voldoen aan de eigenschap $E$ $$\lambdaf{E}{\forall x \cdot man(x) \Rightarrow E(x)}$$

\paragraph{Eigennaam en lidwoord} De signatuur van een eigennaam is gelijk aan die van een naamwoordgroep. Dat volgt uit de semantiek van de grammatica. $$\tau(pn) = \tau(np) = (e \rightarrow t) \rightarrow t$$ Uit de semantiek van de grammatica volgt ook dat de signatuur van een lidwoord gelijk is aan die van een zelfstandig naamwoord naar een naamwoordgroep. $$ \tau(det) = \tau(n) \rightarrow \tau(np) = (e \to t) \rightarrow (e \rightarrow t) \rightarrow t$$ Bijvoorbeeld voor ``every'' wordt dit (met $R$ een restrictie vanuit het substantief) $$\lambdaf{R}{\lambdaf{E}{\forall x \cdot R(x) \Rightarrow E(x)}}$$

\paragraph{Verbale constituent} De signatuur voor een verbale constituent (\texttt{vp}) volgt opnieuw uit de grammatica en de andere signaturen $$\tau(vp) = \tau(np) \rightarrow \tau(s) = ((e \rightarrow t) \rightarrow t) \rightarrow t$$ Een zin is waar als het onderwerp voldoet aan de eigenschap die het werkwoord uitdrukt.

De signatuur van een onovergankelijk werkwoord (\texttt{iv}) is gelijk aan die van een verbale constituent (volgens de grammatica). $$\tau(iv) = \tau(vp) = ((e \rightarrow t) \rightarrow t) \rightarrow t$$ De signatuur van een overgankelijk werkwoord (\texttt{tv}) wordt dan $$\tau(tv) = \tau(np) \rightarrow \tau(vp) = ((e \rightarrow t) \rightarrow t) \rightarrow ((e \rightarrow t) \rightarrow t) \rightarrow t$$

\paragraph{} Tabel~\ref{tbl:signaturen} vat alle signaturen nog eens samen. Op basis van deze signaturen kunnen we de betekenis van het lexicon opstellen. Blackburn en Bos gebruiken hiervoor \textit{semantische macro's}. Dat wil zeggen dat ze voor elke lexicale categorie een functie hebben die een woordvorm uit het lexicon afbeeldt op een lambda-expressie. Hiervoor worden enkel de lexicale features van de woordvorm in kwestie gebruikt.

\begin{table}[h]
  \begin{tabular}{@{}lll}
    \hline
    \textbf{Grammaticale categorie}             & \textbf{Signatuur} \\
    \hline 
    Zin (\texttt{s})                          & $t$ \\
    Naamwoordgroep (\texttt{np})              & $(e \rightarrow t) \rightarrow t$ \\
    Eigennaam (\texttt{pn})                   & $(e \rightarrow t) \rightarrow t$ \\
    Zelfstandig naamwoord (\texttt{n})        & $(e \rightarrow t)$ \\
    Lidwoord (\texttt{det})                   & $(e \rightarrow t) \rightarrow (e \rightarrow t) \rightarrow t$ \\
    Verbale constituent (\texttt{vp})         & $((e \rightarrow t) \rightarrow t) \rightarrow t$ \\
    Onovergankelijk werkwoord (\texttt{iv})   & $((e \rightarrow t) \rightarrow t) \rightarrow t$ \\
    Vergankelijke werkwoord (\texttt{tv})     & $((e \rightarrow t) \rightarrow t) \rightarrow ((e \rightarrow t) \rightarrow t) \rightarrow t$ \\
    \hline
  \end{tabular}
  \centering
  \caption{De signaturen van de grammaticale categorieën uit grammatica~\ref{dcg:simple-gramm}}
  \label{tbl:signaturen}
\end{table}

Grammatica~\ref{dcg:simple-gramm} telt 5 lexicale categorieën: \texttt{pn}, \texttt{n}, \texttt{det}, \texttt{iv} en \texttt{tv}.
\begin{itemize}
  \item Een eigennaam (\texttt{pn}) introduceert een constante met als naam het symbool dat bij die eigennaam hoort. Een eigennaam voldoet aan de eigenschap $E$ als de entiteit waarnaar verwezen wordt eraan voldoet. Meer formeel wilt dit zeggen dat de functiewaarde van die constante voor de functie $E$ gelijk is aan de waarheidswaarde \textit{waar} of $\sem{pn} = \lambdaf{E}{\app{E}{\textit{Symbool}}}$ of voor ``John'' $\sem{John} = \lambdaf{E}{\app{E}{John}}$.
  \item Een zelfstandig naamwoord (\texttt{n}) test of een referentie kan omschreven worden met dat naamwoord of niet. $\sem{n} = \lambdaf{x}{\drs{}{\textit{Symbool}(x)}}$. Bijvoorbeeld voor ``man'' $\sem{man} = \lambdaf{x}{\drs{}{man(x)}}$
  \item Een lidwoord of meer algemeen een determinator (\texttt{det}) introduceert een nieuwe referentie die een bepaalde scope heeft. Hier zijn er meerdere mogelijke vertalingen, één voor elk type van determinator. Een determinator heeft 2 argumenten: de \textit{restriction} en de \textit{nuclear scope}. De \textit{restriction} $R$ wordt opgevuld door het zelfstandig naamwoord (met eventuele bijzinnen). De \textit{nuclear scope} $S$ wordt opgevuld door de verbale constituent. Het is de eigenschap waaraan de naamwoordgroep moet voldoen.
    \begin{itemize}
      \item Voor een universele determinator (zoals ``every'') moet de variabele met een universele kwantor gebonden zijn. In eerste-orde-logica krijgen we dan $\sem{det_{universeel}} = \lambdaf{R}{\lambdaf{S}{\forall x \cdot \left( \app{R}{x} \Rightarrow \app{S}{x} \right)}}$. In DRS wordt dit $$\sem{det_{universeel}} = \lambdaf{R}{\lambdaf{S}{\drs{}{\drsImpl{\drsMerge{\drs{x}{}}{\app{R}{x}}}{\app{S}{x}}}}}$$
      \item De existentiële determinator (zoals ``a'') introduceert een variabele die gebonden is door een existentiële kwantor. De $\oplus$-operator vormt de logische conjunctie tussen twee DRS-structuren. $$\sem{det_{existentieel}} = \lambdaf{R}{\lambdaf{S}{\drsTriMerge{\drs{x}{}}{\app{R}{x}}{\app{S}{x}}}}$$
      \item De negatieve determinator (zoals ``no''): Deze determinator drukt uit dat er geen entiteit bestaat die tegelijk aan de restrictie $R$ en de eigenschap $S$ voldoet. Deze introduceert dus een negatie van een existentiële kwantor. $$\sem{det_{negatief}} = \lambdaf{R}{\lambdaf{S}{\drsNot{\ \drs{x}{} \oplus \app{R}{x} \oplus \app{S}{x}}}}$$
    \end{itemize}
  \item Een onovergankelijk werkwoord (\texttt{iv}) neemt de naamwoordgroep die het onderwerp vormt als argument. Het moet testen of dit onderwerp $O$ voldoet aan de eigenschap van het werkwoord $\lambdaf{x}{\drs{}{\textit{Symbool}(x)}}$, namelijk of het onderwerp de \textit{actie} van het werkwoord uitvoert. $$\sem{iv} = \lambdaf{O}{\app{O}{\lambdaf{x}{\drs{}{\textit{Symbool}(x)}}}}$$
  \item Een overgankelijk werkwoord (\texttt{tv}) is gelijkaardig maar krijgt twee naamwoordgroepen als argument. Het eerste argument is het lijdend voorwerp $L$, het tweede het onderwerp $O$. Er zijn meerdere vertaling mogelijk. Neem bijvoorbeeld de zin ``Every man loves a woman''. Is er één vrouw waarvan alle mannen houden of kan dit een verschillende vrouw zijn voor elke man? Dit wordt een \textit{Quantifier Scope Ambiguïteit} genoemd omdat de ambiguïteit ligt in de volgorde van de kwantoren. Zo worden de twee lezingen respectievelijk $\exists w \cdot woman(w) \land (\forall m \cdot man(m) \Rightarrow loves(m, w))$ en $\forall m \cdot man(m) \Rightarrow (\exists w \cdot woman(w) \land loves(m, w))$. Blackburn en Bos lossen deze ambiguïteiten op door de kwantoren in de vertaling in dezelfde volgorde te laten als in de natuurlijke taal\footnote{Ze leggen daarnaast ook uit hoe men de andere lezingen kan verkrijgen. We verwijzen nar hoofdstuk 3 uit hun eerste boek \cite{Blackburn2005} voor de details.}. Voor het overgankelijke werkwoord wordt dit $$\sem{tv} = \lambdaf{L}{\lambdaf{O}{\app{O}{\lambdaf{x_o}{\app{L}{\lambdaf{x_l}{\drs{}{\textit{Symbool}(x_o, x_l)}}}}}}}$$ Een entiteit $x_o$ omschreven door het onderwerp voldoet aan de verbale constituent als voor die $x_o$ het lijdend voorwerp voldoet aan de eigenschap $\lambdaf{x_l}{\drs{}{\textit{Symbool}(x_o, x_l)}}$.

We passen dit toe op ``Every man loves a woman''. Een man $x_o$ voldoet aan de verbale constituent (``loves a woman'') als er voor die man $x_o$ een vrouw $x_l$ is zodat $\drs{}{\textit{loves}(x_o, x_l)}$ waar is. Het onderwerp (als geheel) voldoet aan de verbale constituent als elke man $x_o$ voldoet aan bovenstaande eigenschap. Als er dus voor elke man $x_o$ een vrouw $x_l$ bestaat waarvan hij houdt. Voor elke man kan er dit een andere vrouw $x_l$ zijn.
\end{itemize}

Merk op dat al deze lambda-expressies voldoen aan de signaturen van tabel~\ref{tbl:signaturen}

\section{Een voorbeeld}
In appendix~\ref{app:vb-framework} illustreren we het framework en bovenstaande formules aan de hand van de zin ``If every man sleeps, a woman loves John''. De vertaling wordt 

  \begin{align*}
    \sem{s} &= \drs{}{\drsImpl{\drs{}{\drsImpl{\drs{x}{man(x)}}{\drs{}{sleeps(x)}}}}{\drs{y}{woman(y) \\ loves(y, john)}}} \\
            &= \bigg( \forall x \cdot man(x) \Rightarrow sleeps(x) \bigg) \Rightarrow \bigg( \exists y \cdot woman(y) \land loves(y, john) \bigg)
  \end{align*}

Dit is de vertaling zoals we die zouden verwachten.

\section{Evaluatie}
Het framework dat Blackburn en Bos voorstellen, is uitermate geschikt voor semantische analyse van natuurlijke taal. De verschillende onderdelen staat vrij los van elkaar. Zo kan men de doeltaal vrij kijzen. Blackburn en Bos vertalen in hun eerste boek \cite{Blackburn2005} naar eerste-orde-logica. In hun tweede boek \cite{Blackburn2006} vertalen ze naar \textit{Discourse Representation Structures}. Ook de vorm van de grammatica is vrij te kiezen. Zowel Blackburn en Bos als deze thesis gebruiken DCG's om de grammatica te specifiëren. Baral et al. \cite{Baral2008} gebruiken een gelijkaardig framework maar met behulp van een Combinatorische Categorische Grammatica.

\paragraph{} Dankzij het framework van Blackburn en Bos, wordt het probleem van semantische analyse herleid tot het opstellen van een lexicon dat de woorden van het logigram bevat; het opstellen van een grammatica die de zinsstructuren van een logigram omvat; het opstellen van de semantiek van deze grammaticale regels; en ten slotte het opstellen van semantische macro's voor alle lexicale categorieën die we introduceren.
