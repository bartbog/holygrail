% IJCAI-21 Author's response

% Template file with author's reponse

\documentclass{article}
\pdfpagewidth=8.5in
\pdfpageheight=11in
\usepackage{ijcai21-authors-response}

\usepackage{times}
\usepackage{soul}
\usepackage{url}
\usepackage[hidelinks]{hyperref}
\usepackage[utf8]{inputenc}
\usepackage[small]{caption}
\usepackage{graphicx}
\usepackage{amsmath}
\usepackage{amsthm}
\usepackage{amssymb}
\usepackage{booktabs}
\usepackage{algorithm}
\usepackage{listings}
\usepackage{algorithmic}

\urlstyle{same}
\usepackage{xcolor}
\usepackage{textcomp} % \textquotesingle

\newtheorem{theorem}{Theorem}
\newtheorem{example}{Example}

\newcommand\comment[1]{\marginpar{\tiny #1}}
\renewcommand\comment[1]{#1}
\newcommand{\todo}[1]{{\comment{{\color{red} #1}}}}
\newcommand{\emilio}[1]{{\comment{{\color{blue} #1}}}}
\begin{document}
We thank the reviewers for their time and insightful comments. We first address the major remarks and questions of the reviewers. 

\paragraph{Reviewer \# 90:} ``\textit{A major issue is I wasn't able to run the code provided in the supplementary material and thus verify the experiments. The failing line is} \texttt{from cppy import $cnf\_to\_pysat$}''

The code requires the 'tseitin' branch of the cppy package, which we apparently and unfortunately did not document properly in the dependencies. To reproduce the experiments:
\begin{enumerate}
	\item Clone the 'tseitin' branch of cppy, to be exact commit '676c2904':
\end{enumerate}
{\footnotesize git clone -b tseitin \url{https://github.com/tias/cppy.git}}\\
{\footnotesize cd cppy} \\
{\footnotesize {git checkout 676c29048bbf9f486b89aa4fb8e04ac5a472f9eb}} 
\begin{enumerate}
	\setcounter{enumi}{1}
	\item Extract the .zip archive containing the supplementary code.
	\item Open the \texttt{frietkot.py} file and add the following line on \textbf{line 3} right after \emph{import sys} making sure to replace the path to cppy with your own:
	\begin{center}
	{ sys.path.append(\textquotesingle/home/path/to/cppy/\textquotesingle)}	
	\end{center}
\end{enumerate}
We appreciate the reviewers effort to reproduce the experiments and we regret that the installation instructions were incomplete.


\paragraph{Reviewer \# 85:} ``\textit{It seems to me that the cited [SMUS, Ignatiev et al] paper is also containing a call to grow, what is the difference to the grow method proposed in this paper?}''

The SMUS pseudo-code contains a grow(), but also the following description in the text: ``\emph{one can avoid the computation of MCS $\mathcal{C}$ and use any correction subset $\mathcal{C}'$ s.t. $\mathcal{C} \subseteq \mathcal{C}'$ ... the grow procedure can be skipped}''.

In our case, OCUS requires to account for the costs and we have experimentally validated that using cost-based grow heuristics significantly reduces the time to find an OCUS.

To summarize, OCUS is based on implicit hitting set enumeration, like the SMUS algorithm, with as key differences: 
\begin{itemize}
	\item OCUS uses a cost function guaranteeing incremental computations.
	\item Meta-level constraints are taken into account.
	\item OCUS - uniquely and efficiently - combines the use of meta-level constraints and the incremental computation of explanations. 
\end{itemize}

These are non-trivial extensions at the conceptual level, essentially resulting in problems that have never been considered before.


%Similar to SMUS algorithm, the OCUS algorithm uses a grow. However in practice since any correction subset C', s.t $C \subseteq C'$ can be added, the grow procedure of SMUS is skipped. However, OCUS requires to account for the costs and we have experimentally validated that using better grow heuristics significantly reduces the time to find an OCUS.

\paragraph{Reviewer \# 85:}``\textit{What is the performance goal for this line of research, i..e, can explanation finding be considered a solved problem now or was this only a small step in improving performance (and also quality of the explanation)?}''

With respect to \textit{performance}, our contribution both provides \textit{optimal} (higher quality) and more \textit{efficient} (lower runtime) explanation sequences.

For the more specific use the case of an Intelligent Tutoring System (ITS), the goal is not so much in computing the entire sequence of explanations, but rather in interactively providing one optimal explanation. % which our algorithm can actually do
This would require second-like runtimes which we do not achieve, and for which approximate algorithms should be investigated.

Regarding the \textit{explanation quality}, there are still some questions of what constitutes a good explanation. This requires further investigation into alternative metrics of explanation difficulty with - for example - human-in-the-loop experiments.% which our algorithm can do

\vspace{2em}
\noindent We take note of the remaining comments and will incorporate the pointers towards readability, and suggestions on related and future work. We clarify some of the comments in the following. 

%It is unclear what is computed as a one-step-explanation in case of OUS.
%Perhaps, the result is the set of literals that can be simultaneously (or in parallel) deduced in the current step.
%This should be carefully stated because this is the most essential point that differs from the method of Bogaerts et al.

\paragraph{Reviewer \# 5 and Reviewer \# 89} have highlighted that OUS is a special case of OCUS. In this context, \texttt{one-step-explanation} will replace the many calls to MUS by 1 call to OUS (minimal to optimal) providing an optimal explanation instead of iteratively searching for the best one it can find. 
The information returned by \texttt{one-step-explanation} is in fact a conjunction of clauses that taken together are unsatisfiable. More precisely, given that $C$ and $I$ both represent conjunctions of clauses, the resulting explanation is a subset of (unit-)clauses and the unit-clause with negated literal $\{\lnot l \}$.

\paragraph{Referring to Reviewer \# 90}'s observation on solution uniqueness, OCUS is able to explain puzzles with both unique solution and multiple solutions. In the case of multiple solutions, OCUS will generate an explanation sequence on all new information that can be propagated (e.g. is shared among all solutions) using given CNF formula $\mathcal{C}$ and initial interpretation $\mathcal{I}$.



%
%\paragraph{R5+R90: computational complexity}
%
%Considering the problem of checking if a set C of clauses has a Smallest MUS of weight k is $\sum_{P}^{2}$-hard. [Learning Abstractions for Model Checking, Anubhav Gupta]
%
%\begin{theorem}
%	\textbf{[OUS to SMUS]}
%	Let clause $c_i$ of CNF $\mathcal{F}$ with weight $w_i \in \mathbb{N}$, we introduce auxiliary variables $y_i$  $\in \{1..n-1\}$. The weighted clause $c_i$ is reduced to an unweighted clause $u_i$ using $((y_1 \wedge .. \wedge y_n) \leftrightarrow c_i )$. This reduction is polynomial-time and preserves the normal form.
%\end{theorem}
%\begin{example}
%	Let $c_1 = \{ \lnot Q\}$ with $w_1$ = 3, \[u_1 \leftrightarrow ((y_1 \wedge y_2) \leftrightarrow (\lnot Q))\]  is equivalent to unit-weighted clauses \[(Q \vee y2) \wedge (Q \vee y1) \wedge (\lnot y2 \vee \lnot y1 \vee \lnot Q) \]
%\end{example}
%
%\begin{theorem}
%	The complexity of deciding if there exists an OCUS is ${\Pi}^P_2$-hard. 
%\end{theorem}
%
%\textit{proof.} Let p be a constraint on the variables of C. For every truth-assignment to x, is there a truth assignment to 
%\[ (\forall S) (\exists y) p(SMUS())\]
%
%\paragraph{R85: OCUS vs SMUS} While OCUS is based on the SMUS algorithm: 
%\begin{itemize}
%	\item OCUS uses a cost function guaranteeing incremental computations.
%	\item Meta-level constraints are taken into account.
%	\item OCUS - uniquely and efficiently - combines the use of meta-level constraints and the incremental computation of explanations. 
%\end{itemize}
%These are non-trivial extensions at the conceptual level, essentially resulting in problems that have never been considered before. 


%\paragraph{R85 - explanations} To the best of our knowledge, this kind of explanations are still very recent work and require further investigation into what constitutes a good explanation. Note that, this is also mentioned in future work: "..explanations of satisfaction problems and we are keen to explore other applications too. ".

%
%\[ F = \forall X_0.\exists X1 \phi(X_0, X_1)\]
%
% [\url{https://www.comp.nus.edu.sg/~meel/Papers/ijcai15.pdf}].
%
%https://www.comp.nus.edu.sg/~meel/Papers/ijcai15.pdf
%
%ref: [Learning Abstractions for Model Checking Anubhav Gupta]
%ref: Smallest MUS [ignatiev]
%
%Given that the complexity of extracting a Smallest MUS is \texttt{FP}$^{\sum_{P}^{2}}$ [Ignatiev et al., 2015], 
%
%\todo{Sriram, S., et al. "Symbolic verification of boolean constraints over partially specified functions." ISCAS 2001. The 2001 IEEE International Symposium on Circuits and Systems (Cat. No. 01CH37196). Vol. 5. IEEE, 2001.\\
%	
%	\url{https://ieeexplore.ieee.org/stamp/stamp.jsp?arnumber=921998&casa_token=bPhCC4NxiCAAAAAA:jvIiaLZf762kfi8X7zEjbBWFX1jKqxokjP0Cypcus-C2EeA3-YGtKznNOTygUIowodP49wQfjqdVbA&tag=1}}\\
%
%\todo{Schaefer, Marcus, and Christopher Umans. "Completeness in the polynomial-time hierarchy: A compendium." SIGACT news 33.3 (2002): 32-49.
%	
%	 \url{https://www.hni.uni-paderborn.de/fileadmin/Fachgruppen/Algorithmen/Lehre/Vorlesungsarchiv/WS_2011_12/Komplexitaetstheorie/phcom.pdf}}

%\paragraph{R89} For clarity, we will refer to a clause as a disjunction of literals and I as a set of unit clauses. Regarding the introduction of the OUS-concept, we will introduce and clarify the difference between OUS and OCUS in the introduction. With respect to \texttt{one-step-explanation}, the algorithm replaces the many calls to MUS by 1 call to OUS (minimal to optimal) where the computation of OUS is indeed a special case of OCUS i.e p always returning true.
%
%\paragraph{R90: Solution uniqueness} In case the puzzle doesn't have a unique solution, we can compute explanations for all new information that can be propagated.
%
%\paragraph{R90: OCUS definition}  We will add a discussion on monotonicity and link it with MUS in the paper. 
%Related to the upperbound, based on the restricted assumptions on p and f (a linear function), it would be possible to compute an upper bound.

\end{document}

