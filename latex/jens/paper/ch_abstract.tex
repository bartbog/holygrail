% This paper researches the value of the semantical framework of Blackburn and Bos for knowledge representation. Specifically, the framework is used to translate logic grid puzzles into logic. %The grammar is manually derived from existing logic grid puzzles
This paper extends and evaluates the semantical framework of Blackburn and Bos for knowledge representation. Specifically, the framework is used to translate logic grid puzzles into logic.

% Additionally, we introduce types into the framework. Based on these types a knowledge base system could type check sentences in natural language. Therefore, it could reject sentences that are grammatical but without meaning. In this paper however, only type inference is used. The framework can infer the domains of a logic grid puzzle from its sentences in natural language.

We introduce types into the framework to translate to a typed logic and to allow the system to infer the domains of a logic grid puzzle from its sentences in natural language. This extension also enables type checking of sentences in natural language. Therefore, it can reject sentences that are grammatical but without meaning.

% Specifically, we add types to this framework. Based on types we can reject sentences that are grammatical but without meaning.
% We explore the value of types in the framework by translating 
