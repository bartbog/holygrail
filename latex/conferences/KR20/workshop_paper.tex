\documentclass{llncs}
%
\usepackage{graphicx}
\usepackage{hyperref}
\usepackage{xspace}
\usepackage{amsmath, amssymb}
\usepackage{xcolor}
\usepackage[ruled,linesnumbered]{algorithm2e}
% \usepackage{caption}
% \usepackage{subcaption}
\definecolor{battleshipgrey}{rgb}{0.52, 0.52, 0.51}

\renewcommand\UrlFont{\color{blue}\rmfamily}
\newcommand\m[1]{\ensuremath{#1}\xspace}
\newcommand\xxx{\m{\overline{x}}}
\newcommand\ddd{\m{\overline{d}}}
\newcommand\ltrue{\m{\mathbf{t}}}
\newcommand\lunkn{\m{\mathbf{u}}}
\newcommand\lfalse{\m{\mathbf{f}}}
\newcommand\leqp{\m{\leq_p}}
\newcommand\geqp{\m{\geq_p}}
\newcommand\entails{\m{\models}}
% \newcommand\land{\m{\wedge}}
\newcommand\limplies{\m{\Rightarrow}}
\newcommand\ourtool{\textsc{ZebraTutor}\xspace}
\newcommand\idp{\textsc{IDP}\xspace}
\newcommand\allconstraints{\m{T}}

\makeatletter
\newcommand{\xRrightarrow}[2][]{\ext@arrow 0359\Rrightarrowfill@{#1}{#2}}
\newcommand{\Rrightarrowfill@}{\arrowfill@\equiv\equiv\Rrightarrow}
\newcommand{\xLleftarrow}[2][]{\ext@arrow 3095\Lleftarrowfill@{#1}{#2}}
\newcommand{\Lleftarrowfill@}{\arrowfill@\Lleftarrow\equiv\equiv}
\newcommand{\xLleftRrightarrow}[2][]{\ext@arrow 3399\LleftRrightarrowfill@{#1}{#2}}
\newcommand{\LleftRrightarrowfill@}{\arrowfill@\Lleftarrow\equiv\Rrightarrow}
\makeatother

\begin{document}
%
\title{Step-wise Explaining How to Solve Constraint Satisfaction Problems}
%
\author{Emilio Gamba\orcidID{0000-0003-1720-9428} \and
Bart Bogaerts\orcidID{0000-0003-3460-4251} \and
Tias Guns\orcidID{0000-0002-2156-2155}}

\institute{Vrije Universiteit Brussel, Pleinlaan 2, 1050 Brussel, Belgium
\email{\{firstname.lastname\}@vub.be}
}
%
\maketitle

\begin{abstract}
% The abstract should briefly summarize the contents of the paper in
% 150--250 words.
We investigate the problem of step-wise explaining how to solve constraint satisfaction problems.
More specifically, we study how to explain the inference steps that one can take during propagation.
The main challenge is finding a sequence of \textit{simple} explanations, where each explanation should aim to be cognitively as easy as possible for a human to verify and understand.
This contrasts with the arbitrary combination of facts and constraints that the solver may use  when propagating.
We identify the explanation-production problem of finding the best sequence of explanations for the maximal consequence of a CSP. 
We propose the use of a cost function to quantify how simple an individual explanation of an inference step is.
Our proposed algorithm iteratively constructs the explanation sequence, agnostic of the underlying constraint propagation mechanisms, by using an optimistic estimate of the cost function, to guide the search for the best explanation at each step.
Using reasoning by contradiction, we develop a mechanism to break the most difficult steps up and give the user the ability to \emph{zoom in} on specific parts of the explanation.
% For difficult inference steps, i.e when multiple constraints are involved, we also develop a mechanism that allows to break the most difficult steps up and thus gives the user the ability to \emph{zoom in} on specific parts of the explanation.
\keywords{Explainable Artificial Intelligence \and Constraint Solving \and Explanation \and Automated Reasoning}
\end{abstract}
%
%
\section{Introduction}\label{sec:intro}
% !TeX root = ./workshop_paper.tex
In the last few years, as AI systems employ more advanced reasoning mechanisms and computation power, it becomes increasingly difficult to understand why certain decisions are made.
Explainable (XAI), a subfield of AI, aims to fulfill the need for trustworthy AI systems to understand \emph{how} and \emph{why} the system made a decision, e.g. for verifying correctness of the system, as well as to control for biased or systematically unfair decisions.

Explanations have been investigated in constraint solving before, most notably for explaining over-constrained, and hence unsatisfiable, problems to a user.
The QuickXplain method \cite{junker2001quickxplain} for example, uses a dichotomic approach that recursively partitions the constraints to find a minimal conflict set.
Many other papers consider the same goal and search for explanations of over-constrainedness \cite{leo2017debugging,zeighami2018towards}.

Despite the fact that we do not (specifically) aim to explain over-constrained problems, our algorithms will also internally make use of methods to extract a minimal set of conflicting constraints often called a \emph{\underline{M}inimal \underline{U}nsatisfiable \underline{S}ubset} (MUS) or \emph{Minimal Unsatisfiable Core} \cite{marques2010minimal}.

While explainability of constraint optimisation has received little attention so far, in the related field of \textit{planning}, there is the emerging subfield of \textit{eXplainable AI planning} (XAIP)~\cite{fox2017explainable}, which is concerned with building planning systems that can explain their own behaviour.
This includes answering queries such as ``why did the system (not) make a certain decision?'', ``why is this the best decision?'', etc. In contrast to explainable machine learning research~\cite{guidotti2018survey}, in explainable planning one can make use of the explicit \textit{model-based representation} over which the reasoning happens.
Likewise, we will make use of the constraint specification available to constraint solvers, more specifically typed first-order logic~\cite{atcl/Wittocx13}.

This research fits within the general topic of Explainable Agency~\cite{langley2017explainable}, whereby in order for people to trust autonomous agents, the latter must be able to \textit{explain their decisions} and the \textit{reasoning} that produced their choices.
To provide the constraint solver with Explainable Agency~\cite{langley2017explainable}, we first formalize the problem of step-wise explaining the propagation of a constraint solver through a sequence of small inference steps.
Next, we use an optimistic estimate of a given cost function quantifying human interpretability to guide the search to \textit{simple}, low-cost, explanations thereby making use of minimal unsatisfiable subsets.
We extend this approach using \emph{reasoning by contradiction} to produce additional explanations of still difficult-to-understand inference steps.
Finally, we discuss the challenges and some outlooks to explaining how to solve constraint satisfaction problems.

\paragraph*{Publication history} This workshop paper is an extended abstract of previous papers presented at workshops and conferences \cite{claesuser,DBLP:conf/bnaic/ClaesBCGG19,ecai/BogaertsGCG20} and a journal paper under review \cite{bogaerts2020framework}.
\section{Background and Problem definition}\label{sec:background}
% !TeX root = ./workshop_paper.tex
% The overarching goal of this paper is to generate a sequence of small reasoning steps, each with an interpretable explanation, for that we introduce the necessary background.

A \emph{(partial) interpretation} is defined as a finite set of literals, i.e., expressions of the form $P(\ddd)$ or $\lnot P(\ddd)$ where $P$ is a relation symbol typed $T_1\times\dots \times T_n$ and $\ddd$ is a tuple of domain elements where each $d_i$ is of type $T_i$. 
A partial interpretation is \emph{consistent} if it does not contain both an atom and its negation, it is called a \emph{full} interpretation if it either contains $P(\ddd)$ or $\lnot P(\ddd)$ for each well-typed atom $P(\ddd)$. 

In the context of first-order logic, the task of finite-domain constraint solving is better known as \emph{model expansion} \cite{MitchellTHM06}: given a logical theory $T$ (corresponding to the constraint specification) and a partial interpretation $I$ with a finite domain (corresponding to the initial domain of the variables), find a model $M$ more precise than $I$ (a partial solution that satisfies $T$).

We define the \textbf{maximal consequence} of a theory $\allconstraints$ and partial interpretation $I$ (denoted $max(I,T)$) as the precision-maximal partial interpretation $J$ such that  $I \wedge \allconstraints \entails J$.

Let $I_{i-1}$ and $I_i$ be partial interpretations such that $I_{i-1}\land \allconstraints \models I_i$.
We say that $(E_i,S_i,N_i)$ \emph{explains} the derivation of $I_{i}$ from $I_{i-1}$ if the following hold:
\begin{itemize}
   \item $N_i= I_i \setminus I_{i-1}$ (i.e., $N_i$ consists of all newly defined facts), 
   \item $E_i\subseteq I_i$ (i.e., the explaining facts are a subset of what was previously derived),
   \item $S_i \subseteq \allconstraints$ (i.e., a subset of the constraints used), and 
   \item $S_i \cup E_i \entails N_i$ (i.e., all newly derived information indeed follows from this explanation).
\end{itemize}

Part of our goal of finding easy to interpret explanations is to avoid redundancy.
That is, we want a non-redundant explanation $(E_i,S_i,N_i)$ where none of the facts in $E_i$ or constraints in $S_i$ can be removed while still explaining the derivation of $I_i$ from $I_{i-1}$; in other word: the explanation must be \textit{subset-minimal}. 
While subset-minimality ensures that an explanation is non-redundant, it does not quantify how \textit{interpretable} a explanation is.
For this, we will assume the existence of a cost function $f(E_i,S_i,N_i)$ that quantifies the interpretability of a single explanation.

Formally, for a given theory $\allconstraints$, a cost function $f$ and initial partial interpretation $I_0$, the \textbf{explanation-production problem} consists of finding a non-redundant explanation sequence
\[\langle(I_0,(\emptyset,\emptyset,\emptyset)), (I_1,(E_1,S_1,N_1)), \dots ,(I_n,(E_n,S_n,N_n))\rangle\]
such that a predefined aggregate over the sequence $\left(f(E_i,S_i,N_i)\right)_{i\leq n}$ is minimised.

\subsection{Nested explanation}
Each explanation in the sequence will be non-redundant and hence as small as possible. 
Yet, in our earlier work some explanations were still quite hard to understand, mainly since multiple constraints had to be combined with a number of previously derived facts. 
We propose the use of simple \textit{nested} explanations using reasoning by contraction, hence reusing the techniques from previous section. 

Given a non-trivial explanation $(E,S,N)$, a nested explanation starts from the explaining facts $E$, augmented with the counterfactual assumption of a newly derived fact $n \in N$. 
At each step, it only uses clues from $S$ and each step is easier to understand (has a strictly lower cost) than the parent explanation which has cost $f(E,S,N)$. 
A contradiction is then derived from the counterfactual assumption.
Each of the reasoning steps leading to the contradiction are what constitutes the nested explanation sequence.

\section{Explanation-Producing search}\label{sec:explanation}
% !TeX root = ./workshop_paper.tex
Ideally, we could generate all explanations of each fact in $max(I_0,\allconstraints)$, and search for the lowest scoring sequence among those explanations.
However, the number of explanations for each fact quickly explodes with the number of constraints, and is hence not feasible to compute.
Instead, we will iteratively construct the sequence, by generating candidates for a given partial interpretation and searching for the smallest one among those (line \ref{alg:min-explanation} in algorithm \ref{alg:nested-main}).

\begin{algorithm}[ht]
  $I_{end} \gets$ propagate$(I_0\land \allconstraints)$\;
  $Seq \gets$ empty sequence\;
  $I \gets I_0$\;
  \While{$I \neq I_{end}$}{
  $(E, S, N) \gets $min-explanation$(I, \allconstraints)$\;\label{alg:min-explanation}
  {\color{gray}
  $nested \gets$ nested-explanations$(E, S, N)$\;\label{alg:nested-explanation}
  append $((E, S, N),nested)$ to $Seq$\;
  }
  $I \gets I \cup N$\;
  }
  \caption{greedy-explain$(I_0,$ $\allconstraints)$}
  \label{alg:nested-main}
\end{algorithm}
\textit{min-explanation$(I,\allconstraints)$} uses an optimistic estmate of the cost function $f$ to compute the next cost-minimal non-redundant explanation $(E \subseteq I, S \subseteq \allconstraints,\{n\})$ that explains $n$ (and possibly explains more).
This means, that whenever one of the facts in $E$ or constraints in $\allconstraints$ is removed, the result is no longer an explanation.
This task is equivalent to finding Minimal Unsat Subset (MUS):
% $ I \wedge \allconstraints \wedge \lnot n. \xLleftRrightarrow{} \lnot (I \wedge T \Rightarrow n)$
\[ I \wedge \allconstraints \wedge \lnot n \xLleftRrightarrow{} \lnot (I \wedge T \Rightarrow n) \]
The nested explanation sequence computed by \textit{nested-explanations$(E,S, N)$} (line \ref{alg:nested-explanation}) exploits \textit{min-explanations}; but it can only use the constraints (and facts) from the original explanation, and the cost of the parent explanation is an upper bound on the acceptable costs at the nested level.

We refer to \cite{ecai/BogaertsGCG20} for further details on the explanation-production algorithm and \cite{bogaerts2020framework} introducing the concept of what we call nested explanation sequences.

% At every iteration, algorithm \ref{alg:nested-main} relies on \textit{min-explanation$(I,\allconstraints)$} (line 5) to get the set of new facts that can be derived from a given partial interpretation $I$ and the constraints T using propagation.
% Then, the next cost-minimal non-redundant explanation $(E \subseteq I, S \subseteq \allconstraints,\{n\})$ is computed that explains $n$ (and possibly explains more).
% This means, that whenever one of the facts in $E$ or constraints in $\allconstraints$ is removed, the result is no longer an explanation.
% This task is equivalent to finding a Minimal Unsat Core (or Minimal Unsat Subset, MUS):
% % To see this, consider the theory
% \[ I \wedge \allconstraints \wedge \lnot n. \xLleftRrightarrow{} \lnot (I \wedge T \Rightarrow n) \]
% We refer to \cite{ecai2020} for further details on the 

% We must point out that MUS algorithms typically find \textit{an} unsatisfiable core that is \textit{subset-minimal}, but not \textit{cardinality-minimal}.
% That is, the unsat core can not be reduced further, but there could be another minimal unsat core whose size is smaller.
% That means that if size is taken as a measure of simplicity of explanations, we do not have the guarantee to find the optimal ones.
% And definitely, when a cost function kicks, optimality is also not guaranteed.


% At line 5 of the algorithm, \textit{min-explanation} uses propagation to get the set of new facts that can be derived from a given partial interpretation $I$ and the constraints T. 

% For each new fact $n$ not in $I$, we wish to find a non-redundant explanation $(E \subseteq I, S \subseteq \allconstraints,\{n\})$ that explains $n$ (and possibly explains more).
% This means, that whenever one of the facts in $E$ or constraints in $\allconstraints$ is removed, the result is no longer an explanation. This task is equivalent to finding a Minimal Unsat Core (or Minimal Unsat Subset, MUS). To see this, consider the theory
% \[ I\wedge \allconstraints \wedge \lnot n. \xLleftRrightarrow{} \lnot (I \wedge T \Rightarrow n) \]
% This theory surely is unsatisfiable since $n$ is a consequence of $I$ and $\allconstraints$. We see that each unsatisifiable subset of this theory is of the form $E \wedge S \wedge \lnot n$ where $(E,S,\{n\})$ is a (not necessarily redundant) explanation of the derivation of $\{n\}$ from $I$.

% In practice, to guide the search to cost-minimal MUS's, we use the observation that typically a small (1 to a few) number of constraints is sufficient to explain the reasoning. A small number of constraints is also preferred in terms of easy to understand explanations, and hence have a lower cost. For this reason, we will don't use the full set of constraints \allconstraints, but we will iteratively grow the number of constraints used.

% We make one further assumption to ensure that we do not have to search for candidates for all possible subsets of constraints. The assumption is that we have an optimistic estimate $g$ that maps a subset $S$ of $\allconstraints$ to a real number such that  $\forall E, N, S: g(S) \leq f(E, S, N)$. This is for example the case if $f$ is an additive function, such as $f(E, S, N) = f_1(E) + f_2(S) + f_3(N)$ where $g(S) = f_2(S)$ assuming $f_1$ and $f_3$ are always positive.

% For each new fact $n$ not in $I$, we wish to find a non-redundant explanation $(E \subseteq I, S \subseteq \allconstraints,\{n\})$ that explains $n$ (and possibly explains more).
% This means, that whenever one of the facts in $E$ or constraints in $\allconstraints$ is removed, the result is no longer an explanation. 
% Algorithm \ref{alg:nested-main} formalizes the greedy construction of the sequence, which determines $I_{end} = max(I_0,\allconstraints)$  through propagation and relies on a \textit{min-explanation$(I,\allconstraints)$} function to find the next cost-minimal explanation.

\section{Discussion and Future work}\label{sec:experiments}
In terms of \emph{efficiency}, the main bottleneck of the current algorithm is the search towards the next cost-minimal explanation
%is the implementation of \textit{min-explanation} 
which requires repeatedly searching for a MUS for increasing constraint sets, which is a hard problem by itself.
Therefore, in future work we want to investigate unsat-core \emph{optimization} with respect to a cost-function as well as exploring other heuristics to construct non-redundant explanation sequences.%either by taking inspiration for instance from the MARCO algorithm~\cite{liffiton2013enumerating} but adapting it to prune based on cost-functions instead of subset-minimality, or alternatively by reduction to quantified Boolean formulas or by using techniques often used there~\cite{QBF,DBLP:journals/constraints/IgnatievJM16}.
% \section{Conclusion}\label{sec:conclusiion}
% \input{05_conclusion}
% ---- Bibliography ----
%
% BibTeX users should specify bibliography style 'splncs04'.
% References will then be sorted and formatted in the correct style.
%
\bibliographystyle{splncs04}
{\small

    \bibliography{ref, mybibfile, krrlib}
}
\end{document}