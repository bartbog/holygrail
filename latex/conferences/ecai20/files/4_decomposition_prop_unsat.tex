In this section, we tackle the goal of searching for a non-redundant explanation sequence that is as simple to understand as possible. 
%, and that ends in the most complete interpretation $I_n$ achievable with the constraints. \tias{terminology of $I_n$, the most complete?} %, e.g. a total interpretation.

Ideally, we could generate all explanations of each fact in $I_n$, and search for the lowest scoring sequence among those explanations. However, the number of explanations for each fact quickly explodes with the number of constraints, and is hence not feasible to compute. Instead, we will iteratively construct the sequence, by generating candidates for a given partial interpretation and searching for the smallest one among those.

\myparagraph{Sequence construction.}
We aim to minimize the cost of the explanations of the sequence, measured with an aggregate over individual explanation costs $f(E_i, S_i, N_i)$ for some aggregate like $max()$ or $average()$. The cost function $f$ could for example be a weighted sum of the cardinalities of $E_i$, $S_i$ and $N_i$; see Section~\ref{sec:cost} for the cost function we will use for logic grid puzzles.

Instead of globally optimizing the aggregated sequence cost, we encode the knowledge that we are seeking a sequence of small explanations in our algorithm. Namely, we will greedily and incrementally build the sequence, each time searching for the lowest scoring next explanation, given the current partial interpretation. Such an explanation always exists since the end point of the explanation process $max(I_0,\allconstraints)$ only contains consequences of $I_0$ and \allconstraints. 
% as all facts in the maximally consistent partial interpretation $I_n$ can by definition be derived without search or failure.
% BART: typically it cannot be derived without search./ You need search to find it. 
%Note that this greedy approach may not find the optimal sequence. 

Algorithm \ref{alg:main} formalizes the greedy construction of the sequence, which determines $I_{end} = max(I_0,\allconstraints)$  through propagation and relies on a \textit{min-explanation$(I,C)$} function to find the next cost-minimal explanation. %We assume that the \textit{propagate} function is optimal in that it computes the intersection of all valid solutions, but our approach is also valid for propagation with other levels of consistency.

\begin{algorithm}
%  \begin{algorithmic}
$I_{end} \gets$ propagate$(I_0\land \allconstraints)$\;
Seq $\gets$ empty sequence\;
$I \gets I_0$\;
\While{$I \neq I_{end}$}{
  $(E, S, N) \gets $min-explanation$(I, \allconstraints)$\;
  append $(E, S, N)$ to Seq\;
  $I \gets I \cup N$\;
}
% \end{algorithmic}
\caption{High-level greedy sequence-generating algorithm.}
\label{alg:main}
\end{algorithm}
\myparagraph{Candidate generation.}
% $ $\tias{Bart, you had a comment on making this part MUS free. I've made the problem definition MUS free so that a future paper can solve the same problem with non-MUS method} Perfect. 
The main challenge is finding the lowest scoring explanation, among all reasoning steps that can be applied for a given partial interpretation $I$. We first look at how to \textit{enumerate} a set of candidate non-redundant explanations given a set of constraints.

For a set of constraints $C$ (later algorithms will not always use \allconstraints for this $C$!), we can first use propagation to get the set of new facts that can be derived from a given partial interpretation $I$ and the constraints $C$. For each new fact $a$ not in $I$, we wish to find a non-redundant explanation $(E \subseteq I, S \subseteq C,\{a\})$ that explains $a$. Recall from Definition~\ref{def:nonred} that this means that whenever one of the facts in $E$ or constraints in $S$ is removed, the result is no longer an explanation. 
% a non-redundant, subset-minimal, explanation $(E, S, N)$ satisfies $S \wedge E \rightarrow N$ with $S \subseteq C$ and $E \subseteq I$ and $\forall s \in S: S \setminus \{s\} \wedge E \nrightarrow N, \forall e \in E: S \wedge E \setminus \{e\} \nrightarrow N$. 
We now show that this task is equivalent to finding a Minimal Unsat Core (or Minimal Unsat Subset, MUS) of a derived this.
To see this, consider the theory 
\[ I\wedge C \wedge \lnot a.\]
This theory surely is unsatisfiable since $a$ is a consequence of $I$ and $C$. 
Furthermore, under the assumption that $I\wedge C$ is consistent (if it were not, there would be nothing left to explain), 
\emph{any} unsatisfiable subset of this theory contains $\lnot a$.
We then see that each unsatisifiable subset of this theory is of the form $E \wedge S \wedge \lnot a$ where $(E,S,\{a\})$ is a (not neccesarily redundant) explanation of the derivation of $\{a\}$ from $I$. 
Vice versa, each explanation of $\{a\}$ corresponds to an unsatisifiable subset. Thus, the \emph{minimal} unsatisifiable subsets (MUS) of the above theory are in one-to-one correspondence with the non-redundant explanations of $a$, allowing us to use existing MUS algorithms to search for non-redundant explanations. 
% Furthermore, each unsatisfiable subset of this theory corresponds to an explanation of $a$. 
% On the other hand, whenever an explanation


% The minimal unsatisfiable subsets of this theory 



% 
% Let us substitute $(S \wedge E \rightarrow N)$ by $X$ resulting in $X$ is \textit{true} and $\forall x \in X: (X \setminus \{x\}$ is false$)$. In comparison, a MUS over a set of constraints $C$ is a subset $C' \subseteq C$ such that $C'$ is UNSAT and $\forall c \in C': (C' \setminus \{c\}$ is SAT$)$. Hence, if we set $C$ to $\neg X$ then, iff $X$ is a subset-minimal explanation, $\neg X$ is a MUS.
% 
% Hence, to find a subset-minimal explanation of $I \wedge C \rightarrow N$ we can search for a MUS of $\neg(I \wedge C \rightarrow N)$. Again through substitution we can rewrite this as $\neg(\neg(I \wedge C) \vee N)$ which is equivalent to $(I \wedge C) \wedge \neg N$. In other words, we search for the smallest subset of $(E \subseteq I, S \subseteq C)$ for which $\neg N$ can not be false, e.g. for which $N$ must be true. Note that because $I \wedge C \rightarrow N$, it will never be the case that the unsat core contains no elements of $\neg N$, as this implies that $I \wedge C$ would be UNSAT but it is not.

%If we know that $N_i$ follows from $S_i$ and $E_i$ then $\neg N_i \wedge S_i \wedge E_i$ must be UNSAT. A Minimal Unsat Subset (MUS) hereof will be a non-redundant subset of $N'_i \subseteq N_i$, $S'_i \subseteq S_I$ and $E'_i \subseteq E_i$ for which we know (because it is UNSAT): $\neg (\neg N'_i \wedge S'_i \wedge E'_i)$ hence through substitution $N'_i \vee \neg (S'_I \wedge E'_i)$ and hence $(S'_I \wedge E'_i) \rightarrow N'_I$. In other words, $(S'_I \wedge E'_i)$ is the smallest explanation of $N'_I$ for which this relation holds, for any smaller explanation it will no longer hold.

We must point out that MUS algorithms typically find \textit{an} unsatisfiable core that is \textit{subset-minimal}, but not \textit{cardinality-minimal}. That is, the unsat core can not be reduced further, but there could be another minimal unsat core whose size is smaller.
That means that if size is taken as a measure of simplicity of explanations, we do not have the guarantee to find the optimal ones. And definitely, when a cost function kicks, optimality is also not guaranteed. 
% 
% We will search for a MUS for each possibly derived fact separately to find the simplest ones. Indeed, when two facts are derivable, the second might need the derivation of the first. 
% In such cases we are interested in fining the derivation for the first one first. 
% As such, we search for a MUS for each new fact separately, and return these MUS's as candidates from which to choose the cost-minimal one.
%\tias{Vvvvreemd, ik zou verwachten dat $N'$ altijd subset minimal gaat zijn en dus enkel de individuele new facts die gepropageerd kunnen worden; maar dat had je eens geprobeerd; mis ik iets of deden we het toen anders?}
%\bart{Bedoel je: volledige $N$ ineens pakken vs apart voor elke $a$? 
%Hoe encodeer je dan $\lnot N$? Er is er een die mis is? (disjucntie)? Werkt niet -> elke uitleg op zich is niet goed aangezien er nog een andere fout kan zijn (ik weet het vaag... maar ... to be discussed later} OK

Algorithm~\ref{alg:cand} shows our proposed algorithm. The key part of the algorithm is on line \ref{line:mus} where we find an explanation of a single new fact $a$ by searching for a MUS that includes $\neg a$.
We search for subset-minimal unsat cores to avoid redundancy in the explanations.
Furthermore, once a good explanation $(E,S,N)$ is found, we immediately explain all implicants of $E$ and $S$. In other words: we take $N$ to be subset-maximal. 
The reason is that we assume that all derivable facts that use the same part of the theory and the same part of the previously derived knowledge probably require similar types of reasoning and it is thus better to consider them at once. 
Thus, we choose to generate candidate explanations at once for all implicants of $(E, S)$ on line~\ref{line:implicants}. 
Note that the other implicants $A \setminus \{a\}$ may have simpler explanations that may be found later in the for loop, hence we do not remove them from $J$.
% \tias{is 'implicants' correct word use?}\bart{yes}

% % \comment{
\begin{algorithm}
% 
\SetKwInOut{Input}{input}\SetKwInOut{Output}{output}
\SetKwComment{command}{/*}{*/}

 \Input{A partial interpretation $I$ and a set of constraints $C$}
% \end{algorithm}
% 
% \Require{$I = $ partial interpretation, $C =$ a set of constraints}
% \Fn(\tcc*[h]{algorithm as a recursive function}){BLA{some args}}{

% \Function{candidate-explanations}{I, C}
  Candidates $\gets \{\}$\;
  $J \gets$ propagate$(I \wedge C)$\;
%   $J \gets J \cap I_n$; \textit{\small // only relevant new facts}\\
  \For{$a \in J \setminus I$\label{line:for}}{ 
  \tcp{\small Minimal expl. of each new fact:}
    $X \gets MUS(\neg a \wedge I \wedge C)$ \label{line:mus}\;
    $E \gets I \cap X$\tcp*{\small facts used}
    $S \gets C \cap X$\tcp*{\small constraints used}
    $A \gets$ propagate$(E \wedge S)$\tcp*{\small all implied facts}\label{line:implicants}
    add $(E, S, A)$ to Candidates
  }
  \Return{Candidates}
% \EndFunction
\caption{candidate-explanations$(I,C)$}

\label{alg:cand}
\end{algorithm}

% }

We assume the use of a standard MUS algorithm, e.g. one that searches for a satisfying solution and if a failure is encountered, the resulting Unsat Core is shrunk to a Minimal one~\cite{marques2010minimal}. While computing a MUS may be computationally demanding, it is far less demanding than enumerating all MUS's (of arbitrary size) as candidates. 
%Hence, the result of the MUS call on line~\ref{line:mus} is \textit{an} unsatisfiable core that is \textit{subset-minimal}, but not \textit{size-minimal}. That is, the unsat core can not be reduced further, but there could be another minimal unsat core whose size is smaller.


\myparagraph{Cost functions and cost-minimal explanations.}
We use Algorithm~\ref{alg:cand} to generate candidate explanations, but in general our goal is to find cost-minimal explanations. In the following, we assume that we have a cost function $f(E, S, N)$ that returns a score for every possible explanation $(E, S, N)$. 

%If we want to find the best ordering (TODO), we need the absolute minimal MUS, which is typically only a few constraints.
%This relates to the objective function...
To guide the search to cost-minimal MUS's, we use the observation that typically a small (1 to a few) number of constraints is sufficient to explain the reasoning. A small number of constraints is also preferred in terms of easy to understand explanations, and hence have a lower cost. For this reason, we will  not call \textit{candidate-explanations} with the full set of constraints \allconstraints, but we will iteratively grow the number of constraints used. 
%\bart{ In practice this means that we are doing some form of prioritized explanatoin. 
%First priority: keep the number of constraints small. Second priority, size (which also takes structure into acccount}

We make one further assumption to ensure that we do not have to search for candidates for all possible subsets of constraints. The assumption is that we have an optimistic estimate $g$ that maps a subset $S$ of $T_P$ to a real number such that  $\forall E, N, S: g(S) \leq f(E, S, N)$. This is for example the case if $f$ is an additive function, such as $f(E, S, N) = f_1(E) + f_2(S) + f_3(N)$ where $g(S) = f_2(S)$ assuming $f_1$ and $f_3$ are always positive.

We can then search for the smallest explanation among the candidates found, by searching among increasingly worse scoring $S$ as shown in Algorithm~\ref{alg:minexpl}. This is the algorithm called by the iterative sequence generation (Algorithm \ref{alg:main}).

\begin{algorithm}
\SetKwInOut{Input}{input}\SetKwInOut{Output}{output}
\SetKwComment{command}{/*}{*/}
\SetKw{Break}{break}


 \Input{A partial interpretation $I$ and a set of constraints $C$}
  Candidates $\gets \{\}$\;
%  $J \gets$ propagate$(I \wedge C)$\tias{is this necessary ?}\bart{$J$ is not even used... So no :p. In algorithm 2 we do need it since that can be called with different S and I}    \; 
  \For{$S \subseteq C$ ordered by $g(S)$}{ \label{alg:min:for}
  	% if the cost of the subset is lower than the most optimistic of the current candidates then it means that any other subset with greater cost g(S) cannot used to generate a new candidate explanation
    \If{$g(S) < min(\{f(cand_i) | cand_i \in $Candidates$\})$}{
        \Break\;}
     cand $\gets$ candidate-explanations$(I, S)$\; \label{alg:min:gets}
     add to Candidates all cand$_i$ with corresp. value $f(cand_i)$\;
     }  
          \Return{$cand_i \in$ Candidates with minimal $f(cand_i)$}
 % \EndFunction
\caption{min-explanation$(I,C)$}
\label{alg:minexpl}
\end{algorithm}

Every time \textit{min-explanation$(I,C)$} is called with an updated partial interpretation $I$ the explanations should be regenerated. The reason is that for some derivable facts $a$, there may now be a much easier and cost-effective explanation of that fact.
There is one benefit in caching the \textit{Candidates} across the different iterations, and that is that in a subsequent call, the cost of the most cost-effective explanation that is still applicable can be used as a lowerbound to start from.
Furthermore, in practice, we cache all candidates and when we (re)compute a MUS for a fact $a$, we only store it if it is more cost-effective than the best one we have previously found for that fact, across the different iterations.
%Without the optimistic estimate $g()$, we would have to search in the worst case for all possible subsets of constraints.
%Note that we can cache the \textit{Candidates} set, and in the next iteration we can update the next best candidate $(E, S, N)$ to $(E, S, N \setminus I)$ where $N \setminus I \neq \emptyset$. This candidate now determines a lowerbound to start from \tias{The effect of this could be tried in experimentation}.
%Caching of \textit{candidate-explanations()} calls across iterations is not advised as new facts can be used to derive other even newer facts, as well as to provide much simpler explanations for facts that already had an explanation before. These can only be generated by recomputing the explanations again. %are possible, but because the found MUS's need not be cost-minimal, we found that in practice it is better to search for explanations from scratch each time in the hope of finding a cheaper one. \tias{This surely has to be evaluated! It feels naive to me; if no new facts are derived, why do we recompute MUS's for each 'old' fact again and again...} 
%\bart{Somewehere (here or in the next section?) we should say something about hte fact that MUS gives us ``just'' a subset-minimal. Without guarantee that it is a good one.
%In practice we *do* keep old results...  Becuase they might ``acccidentally'' be  better than new calls... 
%I think it fits better in the next section (since it is not conceptual but rather about implementation with the tools we have available}
%\tias{OK, will add here}
% \textit{Optimization: MARCO map.} We identify an additional improvement to the above algorithm, inspired by the MARCO algorithm~\cite{liffiton2013enumerating}. In that work, the authors easy the search for new MUS's by pruning the search space for them exploiting the fact that one MUS cannot be a superset of another. 
% 
% the observation is made that given a MUS $M$ involving constraint set $C$, any superset $C''$ of $C$ will also be UNSAT where we already have a corresponding MUS. In our case, given that we search candidate explanations for an increasingly costly constraint set $C$, if we have a candidate explanation involving $C'$ on line \ref{alg:min:gets} we can omit searching for explanations of $C'' \supset C'$ in the for loop on line \ref{alg:min:for}, as an explanation involving $C''$ would have $g(C'') > g(C')$.
% \bart{I removed marco here: I was not convinced: 
% 
% $I$ is not mentioned. Bigger C might make I a lot smaller! 
% 
% Only $g$ was compared, but not $f$ (is also related to $I$ thus)
% 
% Instead, I put something in future work about it. }
%\tias{actually, there could be a smaller one than $C'$ which is found when searching for a MUS of $C''$ but we leave that detail open here}.
%\tias{Future idea: if a constraint is fully satisfied, no need to include it in future calls. I guess because the 'propagate' will return empty the overhead is quite limited anyway}
%\bart{There might be some overhead. THe reason is that in that theory also all the implicit constraints are. In practice they propagate the same for every satisfied clue... } 
%\tias{OK, so we should try empirically but don't have the time now}
%\textbf{That is it for now... more optimisations are possible, basically in the way we search over the C', like, cache Candidates so that its minimum value is maintained (and do an 'update' to it before starting to potentially clean it); do not search for $C'' \supset C'$ where C' has some explanations; perhaps we can find out that at some point we no longer need to optimalprop some $C'$? (e.g. a 'fully used' clue?)}
