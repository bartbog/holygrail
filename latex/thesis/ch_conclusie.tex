\chapter{Conclusie}
\label{ch:conclusie}

\paragraph{}Het is mogelijk om logigrammen automatisch op te lossen door ze te vertalen naar logica. De grootste moeilijkheid daarbij is het opstellen van een algemene grammatica die toepasbaar is op nieuwe logigrammen. Hoewel logigrammen reeds een beperkt aantal grammaticale regels volgen, zijn er toch nog aanpassingen nodig aan de zinnen om ze binnen de grammatica te krijgen.

\paragraph{}Het framework van Blackburn en Bos \cite{Blackburn2005, Blackburn2006} is echter uitermate geschikt voor deze vertaling. Men kan nieuwe grammaticale en lexicale categorieën toevoegen die enkel van toepassing zijn binnen bepaalde domeinen, zoals de onbepaalde woorden dat zijn voor logigrammen. De betekenis van zinnen wordt via het compositionaliteitsprincipe teruggebracht tot de betekenis van woorden. Wanneer alle lexicale categorieën hun betekenis hebben, is het grootste deel van de vertaling naar logica gebeurd. De betekenis van de grammaticale regels bestaat meestal enkel uit één of meerdere lambda-applicaties.

De keuze van deze vertalingen bepaalt ook hoe bepaalde ambiguïteiten in taal worden opgelost. Zo zal in deze thesis een quantifier scope ambiguïteit opgelost worden door de kwantoren te rangschikken volgens hoe ze voorkomen in de zin in natuurlijke taal. Omdat hier een regelgebaseerd systeem is gebruikt, zijn deze keuzes ook deterministisch en kan de gebruiker ze ook leren. Dit maakt het gebruik in mission-critical systemen mogelijk.

\paragraph{}Een probleem met het framework van Blackburn en Bos is dat ze grammaticaal correcte maar betekenisloze zinnen toch toestaan en proberen te vertalen naar logica. We willen echter niet dat in een specificatie zo'n zin kan voorkomen. Daarom hebben we een uitbreiding op het framework voorgesteld dat dit soort zinnen uitsluit. We hebben deze uitbreiding gebruikt om de verschillende domeinen van een logigram automatisch af te leiden. Voor een logigram met een beperkte woordenschat, lukt dit automatisch. Indien een logigram veel verschillende woorden gebruikt om hetzelfde te zeggen, kan het zijn dat de gebruiker moet helpen. Meestal kan dit puur op basis van taalkundige informatie. Indien er echter veel gebruik wordt gemaakt van naamwoordgroepen met een onbekende relatie (zie sectie~\ref{sec:npMissingRelation}), dan kan het zijn dat men de gebruiker moet vragen of twee domeinelementen tot hetzelfde domein behoren.

\paragraph{} Verder onderzoek kan eruit bestaan om de besproken techniek toe te passen op andere domeinen. Bijvoorbeeld op andere soorten puzzels of echte specificaties. Daarnaast kan men ook een echte gestructureerde natuurlijke taal voor logigrammen opstellen die ook makkelijk geleerd kan worden en niet afgeleid is van bestaande logigrammen. Op die manier kan de taal simpel gehouden worden.

Ten slotte zijn ook veel vragen omtrent het type-checken van natuurlijke taal. Bijvoorbeeld of dit het aantal fouten kan verminderen of de productiviteit tijdens het schrijven kan verhogen. Daarnaast is er ook de vraag of er een nood is aan een ingewikkelder type-systeem en hoe dit systeem er dan zou uitzien. Aan het principe van één type per woord is vaak niet voldaan binnen logigrammen. In een systeem met type-checking kan men echter makkelijker meerdere types per woord toelaten. Door middel van type-checking kan men dan namelijk garanderen dat elk voorkomen van een woord slechts één type heeft.

\paragraph{} Er zijn al vele gestructureerde talen opgesteld met heel uiteenlopende doelen. Binnen de (computationele) linguïstiek is er ook al veel werk gedaan rond de betekenis van taal. Deze thesis toont aan dat dit onderzoek ook binnen het domein van kennisrepresentatie nuttig is. Meer zelfs, de computerwetenschappen kunnen dit onderzoek verder stimuleren door een aantal concepten uit de computerwetenschappen, zoals types, toe te voegen aan de bestaande frameworks voor het vertalen van natuurlijke taal. In deze thesis werd een aanzet gegeven hiervoor maar er is nog veel meer onderzoek nodig hieromtrent.
