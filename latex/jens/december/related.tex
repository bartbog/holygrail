In het verleden zijn er al meerdere CNL's gemaakt. Sommige zijn ontwerpen om de leesbaarheid van de specificaties te verhogen en hebben geen formele semantiek. Kuhn \cite{Kuhn2014} heeft een classificatieschema ontwerpen voor CNL's genaamd PENS: \texttt{Precision} (hoe ambigu/formeel is de taal), \texttt{Expressivity} (welke problemen kunnen we uitdrukken), \texttt{Naturalness} (hoe vlot leest de taal), \texttt{Simplicity} (hoe simpel is de taal). In dezelfde paper lijst Kuhn ook een heleboel CNL's op met hun geschiedenis en nut alsook hun classificatie volgens het PENS-schema.

Meer in het algemeen is het omvormen van teksten in natuurlijke taal tot formele modellen al gebeurd in verschillende domeinen: in vereistenanalyse, in het paradigma van business rules, binnen de computationele linguïstiek en ten slotte in het domein van de kennisrepresentatie.

Deze sectie geeft een kort overzicht van wat er al gebeurd is in al deze domeinen. Voor een completer overzicht van CNL's, verwijzen we naar Kuhn \cite{Kuhn2014}.

\subsection{Vereistenanalyse}
\paragraph{Circe} De tool Circe \cite{Ambriola1997} wordt gebruikt in vereistenanalyse. De gebruiker moet een vocabularium, een set van substitutieregels en een specificatie in natuurlijke aanleveren. De tool probeert dan steeds de beste substitutieregel te vinden om zo de tekst geleidelijk aan te transformeren naar een formeel model. Het grote voordeel van de methode is dat er regelsets bestaan voor meerdere soorten modellen: data flow modellen, entity-relationship modellen, ...

Verder is het in Circe mogelijk om in het vocabularium woorden te taggen. De regels kunnen hier dan gebruik van maken om te bepalen of ze van toepassing zijn op bepaalde zinnen. Op die manier introduceert Circe types in het vocabularium.

Een voorbeeld (uit \cite{Ambriola1997}): \example{bron/UIT STUURT data/INFO NAAR doel/IN}. De woorden \example{stuurt} en \example{naar} liggen vast in de regelset. De andere 3 woorden komen uit het vocabularium. Deze moeten een bepaalde tag hebben om te matchen met de regel. Er zijn dus drie types in deze zin: entiteiten die informatie kunnen versturen, entiteiten die informatie kunnen ontvangen en informatie die verzonden kan worden.

Circe is geen CNL. Zinnen die niet matchen met een substitutieregel worden ook niet omgezet in een formeel model. De specificatie kan dus zinnen met en zonder formele betekenis mengen.

\subsection{Business rules}
Binnen het paradigma van business rules spelen zowel natuurlijke talen als formele talen een grote rol. SBVR Structured English (SBVR-SE) \cite{Levy2013} en RuleSpeak \cite{Ross2009a} zijn 2 CNL's die proberen om ambiguïteit in de natuurlijke taal te verminderen. Deze talen focussen echter vooral op het menselijke aspect \cite{Njonko2014}. Het doel van deze talen is om ambiguïteit uit de specificatie te halen en niet zozeer om automatisch natuurlijke taal om te zetten naar een formele voorstelling. RuleCNL \cite{Njonko2014} heeft wel dit doel.

RuleCNL splitst de specificatie op in twee delen: een vocabularium en de regels zelf. Het vocabularium bestaat uit substantieven en werkwoorden alsook hoe deze in verhouding staan tot elkaar. Bijvoorbeeld \example{Auto heeft wiel} geeft aan dat er een relatie kan bestaan tussen een auto en een wiel. Het vocabularium in RuleCNL is dus getypeerd. Voor de regels zelf bestaat er een contextvrije grammatica waaraan de zinnen moeten voldoen.

Om de gebruikers te helpen bij het schrijven van de zinnen in RuleCNL, is er een plug-in voor de Eclipse IDE die automatisch zinnen kan aanvullen en de structuur van bestaande zinnen aangeeft door het kleuren ervan. Bovendien is er een visuele representatie van het domein om de gebruiker te helpen bij het schrijven van het vocabularium.

\subsection{Computationele linguïstiek} Er zijn reeds 2 belangrijke CNL's opgesteld die vertaald kunnen worden naar formele modellen: Attempto Controlled English (ACE) en Processable English (PENG). Beide talen lijken op elkaar en hebben gelijkaardige tools om mee te werken. Ze komen ook allebei uit de computationele linguïstiek en zijn ingebakken in taalkundige frameworks die gemaakt zijn om de semantiek van natuurlijke taal in het algemeen te vatten. In de papers over ACE en PENG wordt er niet zoveel gesproken over deze taalkundige aspecten omdat een achtergrond in de (computationele) linguïstiek verondersteld wordt.

Hierna volgt eerst een bespreking van ACE. Daarna bespreken we PENG. Hierbij ligt de nadruk op de gelijkenissen en verschillen met ACE. We sluiten af met een beschrijving van hoe deze twee talen geïmplementeerd zijn met een nadruk op de verduidelijking van een aantal taalkundige aspecten die amper aan bod komen in de papers over ACE en PENG.

\subsubsection{Attempto Controlled English (ACE)}
\paragraph{} Attempto Controlled English \cite{Fuchs2008} is een gestructureerde natuurlijke taal voor kennisrepresentatie. Het is een subset van Engels die naar een subset van eerste-orde-logica vertaalt. Het is echter wel een formele taal: elke zin in ACE heeft slechts één betekenis, ook al is de zin ambigu in het Engels. Om te bepalen welke van de betekenissen de \textit{correcte} betekenis is, moet men de interpretatieregels volgen. Omdat dit soms nogal ingewikkeld is, kan men ook gebruik maken van de parafraseertool van ACE. Deze tool zet de interne representatie terug om naar één of meerdere zinnen in ACE. Op die manier kan men niet alleen de betekenis van de zin leren, maar ook de taal zelf. Deze parafrasering leunt echter zeer nauw aan bij de formele taal. Hierdoor is ze niet altijd toegankelijk voor iemand zonder een achtergrond in formele talen.

Bijvoorbeeld de zin \example{Everybody is not present.} heeft 2 betekenissen in het Engels: \example{Everybody is absent} en \example{Somebody is absent}. In ACE is de eerste betekenis de \textit{correcte}. Deze zin wordt geparafraseerd als \example{If there is somebody X1 then it is false that X1 is present.} \footnote{De parafrasering komt van de Attempto Parsing Engine (http://attempto.ifi.uzh.ch/ape/)}. Dit is al een vrij moeilijke parafrasering om te begrijpen terwijl de originele zin nog redelijk simpel is.

\paragraph{}ACE is een general purpose CNL: Het bevat een ingebouwd vocabularium. De gebruiker moet dus zelf geen vocabularium opstellen en kan direct beginnen met het schrijven van de specificatie. Het nadeel aan deze aanpak is dat ACE dus ook geen domeinkennis kan gebruiken voor het analyseren van de zinnen. Sommige constructies moeten daarom met een koppelteken geschreven worden. Zo wordt er in \cite{ACEConstructionRules} het voorbeeld gegeven van \example{A student is interested-in a course} en \example{A student is interested in a classroom}.

Op die manier probeert ACE sommige ambiguïteiten op te lossen. Een gelijkaardige truc wordt bijvoorbeeld ook gebruikt om de voorrang van \example{en} en \example{of} op te lossen. Standaard heeft \example{en} voorrang. Maar als de \example{en} voorafgegaan wordt door een komma, dan heeft \example{of} voorrang. \cite{ACEConstructionRules} geeft het voorbeeld \example{A client \{enters a red card or enters a blue card\}, and enters a code.}

In andere gevallen kiest ACE gewoon hoe de zin geïnterpreteerd moet worden op basis van een set van interpretatieregels. Zo slaat de \example{manually} in \example{A customer who {enters a card manually} types a code.} \cite{ACEConstructionRules} op \example{enters} en niet op \example{types} omdat een bijwoord bij voorkeur achter het werkwoord staat. (De parafrasering is in dit geval wel makkelijk te begrijpen maar vrij lang: \example{There is a customer X1. The customer X1 types a code. The customer X1 enters a card manually.}).

\paragraph{} Één van de sterke punten van ACE is haar coreferentie-analyse. Dit is het onderzoeken van welke woordgroepen naar hetzelfde concept verwijzen. Neem bijvoorbeeld de zinnen \example{Een man heeft een vrouw. Hij is gelukkig.}. Hierin is \example{Hij} een anaforische referentie (een referentie naar een woordgroep die eerder komt) naar \example{een man}. ACE kan deze coreferenties correct analyseren. ACE doet dit door haar embedding in Discourse Representation Theory, een taalkundig framework. Men kan hier redelijk ver in gaan. Zo worden de zinnen \example{There is a red house and there is a blue house. The red house is large.} correct geanalyseerd. Doordat zinnen in ACE als één geheel vertaald worden, kan men dus lange zinnen met veel bijzinnen herschrijven in meerdere kortere zinnen. Dit kan de leesbaarheid van een specificatie vergroten.

\paragraph{} Origineel was ACE bedoeld voor het opstellen van specificaties voor software. Ondertussen kent de taal al meerdere toepassingen, in verschillende domeinen. Er zijn ook meerdere tools die overweg kunnen met ACE als input.

Zo is er de Attempto Parsing Engine APE die ACE zinnen omzet naar Discourse Representation Structures. Dit zijn datatypes uit Discourse Representation Theory die de semantiek van de zin bevatten. APE geeft ook een parafrasering van de invoertekst. Zodat de gebruiker kan controleren of de tool de tekst op de juiste manier leest. Bovendien kan APE waarschuwingen geven bij mogelijke problemen. Bijvoorbeeld het gebruik van een anaforische referentie zonder een antecedent waarnaar deze anafoor kan verwijzen.

\paragraph{} Verder is er ook de Attempto Reasoner RACE. Deze tool kan controleren of een specificatie consistent is. Indien niet, zal de tool zeggen welke zinnen met elkaar in conflict zijn. Op die manier weet de gebruiker dat er een fout is en waar deze zich ongeveer bevindt. De tool kan ook vragen in natuurlijke taal beantwoorden. RACE antwoordt niet alleen op de vraag maar geeft ook de zinnen die nodig zijn om te bewijzen dat het gegeven antwoord juist is. Ten slotte kan RACE bewijzen of een bepaalde zin het logische gevolg is van de specificatie. Op die manier kan de gebruiker testen of de specificatie correct is.

APE en RACE zijn de twee belangrijkste tools. Er zijn er echter nog veel meer. Zo is er de ACE View Protégé plug-in. Dit is een plug-in die de vertaling tussen Web Ontology Language (OWL) en ACE doet binnen de Protégé-omgeving (een editor voor het maken van ontologiën). Op die manier ziet de gebruiker enkel ACE zinnen en hoeft dus de formele OWL taal niet te kennen om met bestaande modellen om te gaan of om nieuwe modellen te maken. Ten slotte is er ook nog AceRules. Hiermee kan de gebruiker de zinnen die geïmpliceerd worden door de specificatie te weten komen.

Over het algemeen is ACE een zeer uitgebreide taal. Veel Engelse zinnen zijn ook geldige ACE zinnen, echter niet allemaal. Hierdoor is het moeilijk om de taal te leren. Het is immers niet super duidelijk wat toegestaan is en wat niet. Volgens Fuchs et al.\ \cite{Fuchs2008} heeft een gebruiker 2 dagen nodig om de taal te leren.

\subsubsection{Processable English (PENG)}
\paragraph{} Andere talen zoals PENG \cite{Schwitter2002} zijn makkelijker om te leren. PENG is ook een CNL die vertaalt naar DRS-structuren. In tegenstelling tot ACE bevat PENG geen groot ingebouwd lexicon. De gebruiker moet zelf de woorden aanbrengen die gebruikt worden. De gebruiker kan dit doen tijdens het bewerken en moet dus niet op voorhand aangeven wat het lexicon is. De categorieën voor deze domeinspecifieke woorden zijn substantief, adjectief, werkwoord of bijwoord. PENG biedt ook de mogelijkheid om synoniemen of afkortingen te introduceren. Op die manier kan de specificatie vlotter gemaakt worden. Men kan echter geen types meegeven in het lexicon. Men kan dus niet zeggen dat het werkwoord \example{ademen} enkel toepasbaar is op mensen en dieren en niet op objecten. Naast de domeinspecifieke woorden kent PENG ook een aantal functionele woorden die ingebakken zitten in de taal, zoals lidwoorden en voegwoorden (\example{de man \underline{die}}). Deze woorden helpen PENG om de zinsconstructies te herkennen.

Net zoals ACE kent PENG het principe van constructieregels en interpretatieregels. De constructieregels bepalen welke zinnen deel zijn van de taal. Bij PENG zijn deze regels eenvoudiger omdat de taal zeer simpel gehouden is. Hierdoor is het makkelijker om zinnen te maken in de taal. Over het algemeen lijken de constructieregels van PENG en ACE veel op elkaar.

De interpretatieregels bepalen hoe de zin vertaald wordt naar logica. Zo is er een interpretatieregel voor hoe anaforische referenties opgelost worden. Deze regel is gelijkaardig in ACE en PENG. Andere regels bepalen welke functiewoorden sterker binden. Zowel ACE als PENG gebruiken hiervoor dezelfde volgorde als in eerste-orde-logica. ACE staat wel uitzonderingen toe door het toevoegen van komma's. PENG houdt de taal simpel en staat dit niet toe.

\paragraph{} PENG is makkelijker om te leren dan ACE omwille van ECOLE \cite{Schwitter2003}. Een tool die suggesties geeft over de woordcategorieën die kunnen volgen op een bepaalde zin. Zo geeft Schwitter \cite{Schwitter2003} het voorbeeld van \example{Een} dat gevolgd kan worden door een \texttt{adjectief} of een \texttt{substantief}. Indien de gebruiker de woordcategorie niet kent, kan hij doorklikken op die categorie voor een aantal concrete mogelijkheden. Op die manier moet de gebruiker enkel de woordcategorieën leren en niet de geldige zinsconstructies.

Naast de suggestietool bevat PENG, net zoals ACE, ook een parafraseertool. Deze tool herschrijft de invoer zodat het duidelijker is hoe PENG de zin begrepen heeft. Anaforische referenties worden bijvoorbeeld omgezet in de naamwoordgroep waarnaar ze verwijzen.

Net zoals in ACE is het ook in PENG mogelijk om te controleren of een specificatie consistent is \cite{Schwitter2004b}. Daarnaast is het mogelijk om te controleren op redundantie. Indien een zin al door een andere zin geïmpliceerd wordt, hoeft deze niet expliciet deel uit te maken van de specificatie. Op die manier kan de specificatie kort gehouden worden, wat de leesbaarheid verhoogd.

Naast de general purpose CNL bevat PENG ook een subset specifiek voor het semantische web: PENG-D \cite{Schwitter2004}. Deze subset is, in tegenstelling tot ACE en PENG zelf, wel beslisbaar en kan vertaald worden naar description logic (OWL DL), een subset van eerste-orde-logica. PENG-D kan dus gezien worden als het alternatief voor de ACE View Protégé plug-in.
Schwitter \cite{Schwitter2006} vermeldt drie klassieke manieren van voorstellen van een ontologie (N-Triples, RDF/XML en OWL Abstract) en toont dan verder aan dat PENG een vierde manier is om hetzelfde voor te stellen. De paper toont dit aan door verschillende constructies uit OWL te mappen op zinnen in PENG. Het grote voordeel van PENG t.o.v. de andere voorstellingswijzen is dat PENG ook leesbaar en begrijpbaar is voor de mens.

\subsubsection{Implementatie}
\paragraph{} Zowel ACE als PENG zijn geïmplementeerd in prolog met behulp van Definite Clause Grammars (DCG) en feature structures voor het omzetten van de natuurlijke taal naar DRS-structuren \cite{Fuchs2008, Schwitter2006}.

\paragraph{Definite Clause Grammars} Aangezien zowel ACE als PENG gebruiken maken van prolog voor hun (eerste) implementatie, is de keuze voor DCG's voor de hand liggend. Zoals aangehaald in sectie \ref{sec:DCG} zijn DCG's interessant omdat ze ook als definitie van een grammatica gezien kunnen worden en daardoor gebruikt kunnen worden in chart parsers. Dankzij zo'n chart parser kan men tijdens het schrijven van de zinnen, heel snel voorspellen welke woordcategorieën kunnen volgen. Dit helpt de gebruiker bij het schrijven van grammaticaal correcte zinnen zonder de taal te moeten kennen. PENG gebruikt zo'n chart parsers in ECOLE \cite{Schwitter2003}. ACE heeft in navolging van PENG ook AceWiki \cite{Kuhn2008} gemaakt, een subset van ACE voor semantische wiki's. AceWiki bevat ook een suggestietool. Kuhn et al.\ \cite{Kuhn2008} leggen in detail uit hoe zo'n chart parser gemaakt kan worden voor een CNL en vergelijken de performantie van deze parsers met de gratis parser.

\paragraph{Feature structures} Naast DCG's wordt er in ACE en PENG ook gebruikt gemaakt van feature structures \cite{Shieber2003, NLPCourse}. Een feature structure is een datatype gerelateerd aan een woordgroep. Het is een soort van hashmap met als keys de (grammaticale) features en als values de waarden voor die feature. Een (grammaticale) feature is een eigenschap van een woordgroep. Zo is er de feature \texttt{categorie} die aangeeft welke syntactische rol een woordgroep speelt in een zin (bv. \texttt{np}, \texttt{vp}, ...). Andere features geven bijvoorbeeld de naamval aan van een naamwoordgroep. Grammatica's die gebruik maken van feature structures, gebruiken unificatie voor het samenvoegen van meerdere structuren. Niet alle features moeten namelijk een waarde toegekend krijgen. Zo kan een eigennaam voorkomen als onderwerp en als lijdend voorwerp (en heeft dus geen waarde voor de feature \texttt{naamval}). Terwijl \example{ik} enkel als onderwerp kan voorkomen (en dus wel een waarde heeft voor die feature). Bovendien kan men door unificatie controleren of het onderwerp en werkwoord hetzelfde getal hebben. Blackburn en Striegnitz \cite{NLPCourse} geven de volgende grammaticale regel als voorbeeld:

\[
  \fstructure{
    \feature{CAT}{s}
  }
  \rightarrow
  \fstructure{
    \feature{CAT}{np}
    \feature{NAAMVAL}{nom}
    \feature{NUM}{\fvariable{1}}
  }
  \fstructure{
    \feature{CAT}{vp}
    \feature{NUM}{\fvariable{1}}
  }
\]

Deze regel zegt dat een zin bestaat uit een \texttt{np} gevolgd door een \texttt{vp}. De \texttt{np} moet de naamval \texttt{nom} hebben. Bovendien moet het getal van de \texttt{np} en de \texttt{vp} unificeren (de \framebox{1} kan men zien als een variabele).

\paragraph{} Binnen ACE en PENG zorgen de feature structures voor de syntactische correctheid van de zinnen in de gasttaal, het Engels. Via unificatie kan men namelijk testen of een bepaalde woordgroep voldoet aan de voorwaarden van de context. De unificatie van de feature \texttt{getal} van het onderwerp en het werkwoord zorgt voor de congruentie in getal van het onderwerp met het werkwoord. Op die manier zijn zinnen die in ACE en/of PENG geldig zijn, ook geldig in het Engels. Dit is gelijkaardig aan de rol van \texttt{nom} en de \texttt{acc} in de Definite Clause Grammar van voorbeeld \ref{ex:nom-acc-features}.

\paragraph{} Zoals Shieber et al.\ \cite{Shieber2003} aanhalen, verschillen prolog termen van feature structures enkel in vorm. Zo speelt de volgorde in prolog wel een rol. Bovendien kan men in prolog de features geen naam geven maar worden ze bepaald door de volgorde in de term. Ten slotte moet men steeds alle features vermelden, ook als ze ongebonden zijn. Dit maakt het moeilijker om een extra feature toe te voegen. Feature structures zijn dus handiger om te programmeren. Qua expressiviteit zijn ze echter gelijkwaardig aan DCG's. Zo is bovenstaande grammaticale regel op basis van feature structures equivalent met volgende DCG-regel:

\[
    \texttt{s ---> np(nom, Num), vp(Num).} \\
\]

\paragraph{} Feature structures (en argumenten in DCG's) zijn handig om de explosie van grammatica regels te voorkomen.
\begin{ex}  Een voorbeeld uit \cite{NLPCourse}:
  \label{ex:explosion}
  \begin{quote}
    \texttt{s ---> np\_{singular}, vp\_{singular}.} \\
    \texttt{s ---> np\_{plural}, vp\_{plural}.} \\
    \texttt{np ---> np\_{singular}.} \\
    \texttt{np ---> np\_{plural}.} \\
    \texttt{np\_{singular} ---> det, n\_{singular}.} \\
    \texttt{np\_{plural} ---> det, n\_{plural}.} \\
    \texttt{vp\_{singular} ---> intransitive\_verb\_{singular}.} \\
    \texttt{vp\_{singular} ---> transitive\_verb\_{singular}, np.} \\
    \texttt{vp\_{plural} ---> intransitive\_verb\_{plural}.} \\
    \texttt{vp\_{plural} ---> transitive\_verb\_{plural}, np.} \\
    \texttt{n\_singular ---> [man].} \\
    ...
  \end{quote}
\end{ex} 
Hierbij staat de \texttt{n} voor zelfstandig naamwoord (van het Engelse \texttt{noun}) en \texttt{det} voor determinator. Deze grammatica can veel korter gemaakt worden door gebruik te maken van feature structures:

\begin{ex}  Een equivalente grammatica aan voorbeeld \ref{ex:explosion} (ook uit \cite{NLPCourse})
  \begin{quote}
    \texttt{s ---> np(FS), vp(FS).} \\
    \texttt{np(FS) ---> det, n(FS).} \\
    \texttt{vp(FS) ---> intransitive\_verb(FS).} \\
    \texttt{vp(FS) ---> transitive\_verb(FS), np(\_).} \\
    \texttt{n(singular) ---> [man].} \\
    ...
  \end{quote}
\end{ex} 

Door gebruik te maken van deze feature structures is de grammatica simpeler en leesbaarder. Bovendien hoeft het concept dat een zin bestaat uit een \texttt{np} gevolgd door een \texttt{vp} maar één keer te worden uitgedrukt. De feature structures zorgen voor de congruentie in getal van het onderwerp met het werkwoord.

\paragraph{DRS} Naast feature structures maken ACE en PENG ook gebruik van Discourse Representation Structures. Deze structuren kunnen vertaald worden naar eerste-orde-logica. Ze zijn een onderdeel van Discourse Representation Theory. Dit is een taalkundig framework om de semantiek van natuurlijke taal te vatten. Één van de sterke punten van DRS-structuren is het oplossen van coreferenties. \cite{Fuchs2008drs} bevat meer informatie over hoe DRS-structuren gebruikt worden binnen ACE.

Bos \cite{Bos2011} stelt dat DRS-structuren zowel de rol van semantische inhoud als die van tekstuele context spelen. M.a.w.\ met behulp van deze structuren kan men achterhalen wat de semantiek van een tekst is maar tegelijk bieden ze ook een context aan die helpt bij de coreferentie-analyse. Zo wordt tijdens het parsen de semantiek van een zin opgebouwd en tegelijk de coreferenties opgelost.

Concreet bevat een DRS-structuur een lijst van \textit{discourse referents} (woordgroepen waarnaar andere woordgroepen kunnen verwijzen) en een lijst van bepaling i.v.m. die referenties \cite{Bos2011}. Tabel \ref{table:DRS} geeft een voorbeeld van zo'n DRS-structuur voor de zin \example{There is a movie which every man loves deeply}. Er zijn in totaal 3 \textit{discourse referents}: \texttt{movie}, \texttt{man} en \texttt{loves}. Merk op dat naar deze laatste verwezen wordt door het bijwoord \textit{deeply}.

\begin{savenotes}
\begin{table}[h]
  \centering
  \begin{tabular}{|lcl|}
    \hline
    A & & \\
    \hline
    movie(A) & & \\ 
    \begin{tabular}{|l|}
      \hline
      B \\
      \hline
      man(B) \\
      \hline
    \end{tabular} & $\Rightarrow$ & \begin{tabular}{|l|}
      \hline
      C \\
      \hline
      loves(C, B, A) \\
      deeply(C) \\
      \hline
    \end{tabular} \\
    & & \\
    \hline

  \end{tabular}
  \caption{Een DRS-structuur voor de zin \example{There is a movie which every man loves deeply.} \protect\footnotemark}
  \label{table:DRS}
  \footnotetext{Deze DRS-structuur werd lichtjes aangepast aan de output van APE (http://attempto.ifi.uzh.ch/ape/)}
\end{table}
\end{savenotes}


\paragraph{Conclusie} DCG's zijn een uitbreiding op contextvrije grammatica's uit de wereld van logisch programmeren. Feature structures en Discourse Representation Structures zijn concepten uit de linguïstiek. Ze worden gebruikt om de natuurlijke taal en haar semantiek te modelleren. Zo helpen feature structures om een explosie van het aantal grammatica regels te voorkomen. Discourse Representation Structures worden dan weer vooral gebruikt om de coreferentie-analyse te vergemakkelijken. Beiden concepten passen mooi binnen ACE en PENG omdat deze talen ook ontstaan zijn in het vakgebied van computationele linguïstiek.

\subsection{Kennisrepresentatie}
\label{sec:ASP}
In het domein van kennisrepresentatie hebben Baral et al.\ \cite{Baral2008} natuurlijke taal reeds vertaald naar Answer Set Programs (ASP). Ze maken hiervoor gebruik van Combinatorische Categorische Grammatica (CCG) en $\lambda$-calcalus. Een CCG bestaat uit een aantal basiscategorieën zoals \texttt{S} en \texttt{NP} en afgeleide categorieën zoals \texttt{S/NP} en \texttt{(S/NP)$\backslash$NP} \cite{Baral2008}. Zo wordt een onovergankelijk werkwoord voorgesteld als een \texttt{S$\backslash$NP}. Verder bevat een CCG een aantal regels die bepalen wat de categorie is van een woordgroep. Zo is er een regel die zegt dat $\alpha\beta$ van categorie \texttt{B} is als $\alpha$ categorie \texttt{A} heeft en $\beta$ categorie \texttt{B$\backslash$A} \cite{Baral2008}. Dankzij deze regel kunnen we afleiden dat als we een onovergankelijk werkwoord langs links combineren met een substantief, we dan een zin krijgen. Op deze manier kunnen we een parse tree opstellen waarbij we telkens 2 woordgroepen combineren totdat de zin uiteindelijk categorie \texttt{S} krijgt.

Naast een categorie heeft elk woord ook een betekenis uitgedrukt in een uitbreiding op de $\lambda$-calcalus. In deze uitbreiding kunnen ook ASP-expressies voorkomen. De betekenis van een woordgroep is de combinatie van de betekenis van de woorden. Hierbij gebruiken we de CCG parse tree om de volgorde te bepalen. We verduidelijken met een voorbeeld (grotendeels overgenomen uit Baral et al.\ \cite{Baral2008}). We beschouwen het lexicon zoals weergegeven in tabel \ref{table:CCG} voor de zin \example{Birds fly}. Hierin stelt de \texttt{@} de applicatie voor uit de lambda-calcalus. De combinatie van de woorden \textit{birds} en \textit{fly}, is van de categorie \texttt{S} zoals hierboven reeds uitgelegd. De overeenkomstige lamda-expressies moeten nu op een gelijkaardige manier gecombineerd worden: $\lambda_{birds\ fly}=\lambda_{fly}@\lambda_{birds}$.

\begin{table}
  \centering
  \begin{tabular}{|l|l|l|}
    \hline
    Woord & Categorie & $\lambda$-ASP-expressie \\
    \hline
    \hline
    Birds & NP & $\lambda x.bird(x)$ \\
    Fly & S$\backslash$NP & $\lambda x.fly(X) \leftarrow x@X$ \\
    \hline
  \end{tabular}
  \caption{Een lexicon voor de woorden birds en fly in een CCG grammatica (uit \cite{Baral2008})}
  \label{table:CCG}
\end{table}

\begin{equation}
  \begin{align*}
  \lambda_{birds\ fly} &= \lambda_{fly}@\lambda_{birds} \\
          &= (\lambda x.fly(X) \leftarrow x@X)@(\lambda x.bird(x)) \\
          &= fly(X) \leftarrow (\lambda x.bird(x))@X \\
          &= fly(X) \leftarrow bird(X)
  \label{eq:lambda}
  \end{align*}
\end{equation}

Elk woord heeft een betekenis die men kan zien als een ASP-expressie met gaten erin die opgevuld worden door de combinatie met andere woorden. De manier van combineren van de $\lambda$-expressies hangt af van de categorieën van de woorden. Het nadeel aan deze aanpak is dat elk woord (minstens) één betekenis moeten hebben in het formaat van een $\lambda$-ASP-expressie. Constantini et al.\ \cite{Costantini2010} lossen dit probleem deels op door gebruik te maken van $\lambda$-ASP-expressie-templates. Sommige woorden hebben nog steeds een eigen $\lambda$-ASP-expressie, voor de andere kan er één afgeleid worden uit een $\lambda$-ASP-expressie-template. Zo is $\lambda x. <noun>(x)$ de template voor substantieven. Het gedeelte $<noun>$ moet vervagen worden door een specifieke instantie. Het woord \textit{birds} heeft zo nog steeds dezelfde betekenis.

\paragraph{}Zo'n templates werken echter niet voor alle woorden. Daarom hebben Baral et al.\ \cite{Baral2012} een methode bedacht om van de betekenis van een zin en de betekenis van een aantal woorden, de betekenis van andere woorden af te leiden. Ze doen dit niet langer met $\lambda$-ASP-expressies maar vervangen ASP door eerste-orde-logica: $\lambda$-FOL-expressies. De grammatica en manier van combineren blijft echter hetzelfde. Deze techniek gebruiken Baral et al.\ \cite{Baral2012a} om automatisch een grammatica en semantiek te leren voor logische puzzels op basis van andere logische puzzels en hun vertaling in een ASP programma. Hiervoor gebruiken ze één ASP-ontologie die toepasbaar is voor vele logische puzzels. Ze bewerken hierbij de natuurlijke taal van de logische puzzels een beetje om anaforische referenties te verwijderen. De auteurs benadrukken echter dat dit geen CNL is omdat de grammatica op voorhand niet gedefinieerd is maar geleerd wordt uit de gegeven logische puzzels. Ze maken hiervoor gebruik van een probabilistische CCG. Tot 83\% van de puzzels kunnen ze correct oplossen. De andere puzzels falen o.a.\ omdat bepaalde zinsconstructies niet voorkwamen in de trainingsdata. Belangrijk om op te merken is dat vele woorden (zoals \textit{about}, \textit{on}, \textit{the}) geen betekenis krijgen omdat ze in de logische puzzel geen rol spelen. Er is dus sprake van overfitting op het domein van de logische puzzels.

% \subsubsection{andere papers}
% \begin{enumerate}
%   \item Evaluation CNL\cite{Kuhn2010}
%   \item ------------------------------------------------------------
%   \item A principled approach to CNL's\cite{Kuhn2013}: linguistic
%   \item Model checking\cite{Flake2002, Konrad2005, Nelken, Jak2008}
%   \item Lijst van ambiguïteiten \cite{Berry2003}
%   \item CELT\cite{Pease2010, Dellis2010}
% \end{enumerate}
