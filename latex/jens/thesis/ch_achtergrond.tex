\chapter{Achtergrond}
Om deze thesis te begrijpen worden hier een aantal concepten uitgelegd die niet per se tot de achtergrondkennis behoren van de lezer. Definite Clause Grammars zijn een soort van grammatica die een aantal voordelen biedt voor het parsen van natuurlijke taal. Feature structures zorgen voor een beperking van het aantal grammaticale regels en verhogen de leesbaarheid ervan. Discourse Representation Theory is een theorie uit de taalkunde voor het vatten van de betekenis van taal. Het introduceert Discourse Representation Structures, een logica die dichter aanleunt bij de natuurlijke taal.

\section{Definite Clause Grammars}
\label{sec:DCG}
Definite Clause Grammars \cite{Pereira1980} zijn grammatica's die expressiever zijn dan contextvrije grammatica's. Ze zitten vaak ingebakken in logische talen zoals prolog. Pereira et al.\ \cite{Pereira1980} geven 3 voorbeelden van hoe DCG's kunnen helpen bij het parsen van natuurlijke talen:

\begin{enumerate}
  \item De woordvorm kan afhankelijk gemaakt worden van de context waarin deze verschijnt. Zo kan men eisen dat een werkwoord in de juiste vervoeging voorkomt.
  \item Tijdens het parsen kan men een boom opbouwen die de semantiek van de zin moet vatten. Deze boom hoeft niet isomorf te zijn met de structuur van de grammatica.
  \item Het is mogelijk om (prolog) code toe te voegen die extra restricties oplegt aan de grammatica.
\end{enumerate}

\subsection{Een eerste grammatica}
Een voorbeeld van een DCG is grammatica~\ref{dcg:gram1}.
\begin{dcg}{Een eerste grammatica}{dcg:gram1}
s() --> np(), vp().
np() --> [ik].
np() --> [hem].
vp() --> v(), np().
v() --> [zie].
\end{dcg}
\texttt{s} is het startsymbool en staat voor \texttt{sentence}. \texttt{np} staat voor \texttt{noun phrase} (naamwoordgroep of nominale constituent), \texttt{vp} voor \texttt{verb phrase} (verbale constituent) en \texttt{v} voor \texttt{verb}. Deze grammatica zegt dat een zin bestaat uit een noun phrase gevold door een verb phrase. Een verb phrase is dan weer een werkwoord gevolgd door een noun phrase.

De zin \example{ik zie hem} is onderdeel van deze taal. Maar ook de zin \example{ik zie ik} is deel van de taal. Om dit op te lossen kunnen we argumenten meegeven aan de niet-terminaal \texttt{np}.

\subsection{De woordvorm is afhankelijk van de context}
De verbeterde grammatica~\ref{dcg:nom-acc-features} houdt rekening met welke woordvorm kan voorkomen in welke context. De \texttt{nom} en \texttt{acc} slaan hier op de naamvallen \texttt{nominatief} en \texttt{accusatief}. Ze geven aan in welke functie de naamwoordgroepen gebruikt mogen worden binnen een zin. Namelijk als onderwerp of als lijdend voorwerp.

\begin{dcg}{Een grammatica met argumenten}{dcg:nom-acc-features}
s() --> np(nom), vp().
np(nom) --> [ik].
np(acc) --> [hem].
vp() --> v(), np(acc).
v() --> [zie].
\end{dcg}

\subsection{Een boom als resultaat}
Verder is het ook mogelijk om een boom op te bouwen tijdens het parsen. Dit gebeurt in grammatica~\ref{dcg:treeBuilding}. Bij het parsen van \example{ik zie hem} krijgen we nu volgende boom als waarde voor het eerste argument van \texttt{s}.

\Tree[.\textit{zien} \textit{ik} \textit{hem} ]

Merk op dat deze boom de structuur van de parse tree niet hoeft te volgen. Het werkwoord wordt hier tot belangrijkste woord van de zin gebombardeerd.
\begin{dcg}{Een grammatica die een boom opbouwt}{dcg:treeBuilding}
s(Tree) --> np(NP, nom), vp(Tree, NP).
np(ik, nom) --> [ik].
np(hem, acc) --> [hem].
vp(Tree, Subject) --> v(Tree, Subject, Object), np(Object, acc).
v(zien(Subject, Object), Subject, Object) --> [zie].
\end{dcg} 

\subsection{Prolog-code in de grammatica}
Ten slotte is het mogelijk om prolog restricties te embedden in de grammatica door deze prolog goals tussen accolades te plaatsen.

\begin{dcg}{Een grammatica met embedded prolog-code}{dcg:prologCode}
expression(X) --> factor(X).
expression(X) --> term(X).

factor(X) --> numeral(X).
factor(X) --> numberal(A), [*], factor(B), {X is A * B}.
term(X) --> factor(A), [+], expression(B), {X is A + B}.

numeral(X) --> [X], {number(X)}.
\end{dcg} 

Grammatica~\ref{dcg:prologCode} kan simpele wiskunde expressies omvormen tot hun wiskundige waarde. Zo wordt \texttt{2 + 4 * 5} omgevormd tot \texttt{22} volgens volgende boom. Hierbij wordt de waarde van onder uit de boom naar boven toe gepropageerd via unificatie.

\Tree[.expression(22)
        [.term(22) [.factor(2) [.number(2) 2 ]]
                   +
                   [.expression(20) [.factor(20) [.number(4) 4 ] * [.factor(5) [.number(5) 5 ]]]]]]

De prologcode in de accolades heeft twee functies. Enerzijds berekent die de waarde van een subexpressie zoals een factor of een term. Anderzijds beperkt de prologcode de grammatica. Een \texttt{numeral} bestaat uit 1 token maar enkel als dat token een getal is volgens prolog. Zo'n beperking in prolog kan ook gebruikt worden om uit de beperkingen van een contextvrije grammatica te treden.

\subsection{Conclusie}
\paragraph{} Definite Clause Grammars zijn expressieve grammatica's die uitermate geschikt zijn voor het modelleren van de grammatica van een natuurlijke taal. In deze thesis zullen alle grammatica's dan ook gegeven worden in de vorm van een DCG.

% \paragraph{}Een laatste opmerking bij deze grammatica is de asymmetrische vorm voor factoren en termen. Een factor is bijvoorbeeld niet gedefinieerd als de vermenigvuldiging van 2 factoren. Dit komt omdat DCG's niet enkel definities zijn van grammatica's maar ook een uitvoeringsstrategie hebben. M.a.w. men krijgt er gratis een parser bij. Deze parser werkt, net als prolog, top-down en van links naar rechts. In het geval van links recursieve regels, zou de parser in een oneindige lus kunnen geraken. Het is echter een bekend resultaat dat men een grammatica altijd kan omvormen zodat deze niet langer links recursief is. Dit is dus geen beperking op welke talen voorgesteld kunnen worden.

% \paragraph{} DCG's zonder prolog code zijn zeer declaratief. Men kan ze namelijk ook puur als definitie van een grammatica beschouwen. Zo kan men een chart parser schrijven die gebruik maakt van een DCG als definitie van de grammatica. Chart parsers zijn interessant voor CNL's omdat ze onthouden welke parti\"ele en volledige constituenten ze al gevonden hebben \cite{Kuhn2008}. Daardoor is er geen nood aan backtracking. Men moet zo niet telkens opnieuw bewijzen wat in een andere tak al bewezen was. Een chart parser onthoudt dat \example{een man} een naamwoordgroep is en kijkt hoe het deze woordgroep kan combineren met andere woorden tot parti\"ele of volledige constituenten. Daardoor is een chart parser veel sneller dan de gratis parser van prolog.

% Bovendien kan men uit de parti\"ele constituenten afleiden welke woordcategorie\"en kunnen volgen op een parti\"ele zin. Zo kan de parti\"ele zin \example{Een rode} gevolgd worden door een adjectief of substantief maar niet door een lidwoord of een werkwoord. Op basis hiervan kan men een suggestietool maken die suggesties geeft i.v.m.\ welke woorden kunnen volgen.

% Ten slotte kan men bij het toevoegen van een woord aan een parti\"ele zin, de resultaten van de vorige parse gebruiken. Dit levert een extra performantiewinst op t.o.v.\ de gratis parser in het geval van incrementele parses. Dit is vooral interessant tijdens het schrijven van een zin, waarbij de vorige parti\"ele zin steeds wordt uitgebreid met \'e\'en woord. Een chart parser hoeft in dat geval namelijk enkel te kijken naar dit nieuwe woord en naar wat er in het geheugen is van de vorige parse, niet meer naar de andere woorden in de zin.

\section{Feature structures}
\label{sec:featureStructures}

\paragraph{} Een feature structure is een term uit de taalkunde. Men kan zo'n structuur zien als \textit{named arguments} voor een niet-terminaal symbool die gebruikt worden om een explosie aan grammaticale regels te voorkomen. Zo kan een \texttt{np} en \texttt{vp} een feature \texttt{getal} hebben dat aangeeft of de woordgroep in het enkelvoud of meervoud staat. De grammaticale regel voor een zin kan dan aangeven dat het onderwerp en werkwoord moeten overeenkomen in getal. Andere features geven bijvoorbeeld de naamval aan van een naamwoordgroep. Blackburn en Striegnitz \cite{NLPCourse} geven de volgende grammaticale regel als voorbeeld (hierbij staat \texttt{CAT} voor de grammaticale categorie van een woordgroep, vergelijkbaar met de naam van een niet-terminaal symbool):

\[
  \fstructure{
    \feature{CAT}{s}
  }
  \rightarrow
  \fstructure{
    \feature{CAT}{np}
    \feature{NAAMVAL}{nom}
    \feature{GETAL}{\fvariable{1}}
  }
  \fstructure{
    \feature{CAT}{vp}
    \feature{GETAL}{\fvariable{1}}
  }
\]

Deze regel zegt dat een zin bestaat uit een \texttt{np} gevolgd door een \texttt{vp}. De \texttt{np} moet de naamval \texttt{nom} (nominatief) hebben. Bovendien moet het getal van de \texttt{np} en de \texttt{vp} unificeren (de \framebox{1} is een variabele).

Grammatica's die gebruik maken van feature structures, gebruiken altijd unificatie voor het samenvoegen van meerdere structuren. Niet alle features moeten namelijk altijd een waarde toegekend krijgen. Zo kan een eigennaam voorkomen als onderwerp en als lijdend voorwerp (en heeft dus geen waarde voor de feature \texttt{naamval}). Terwijl \example{ik} enkel als onderwerp kan voorkomen (en dus wel een waarde heeft voor die feature). Zoals in het voorbeeld hierboven kan men door unificatie ook controleren of meerdere woordgroepen dezelfde waarden hebben voor een feature.

\paragraph{} Zoals Shieber et al.\ \cite{Shieber2003} aanhalen, verschillen prolog termen van feature structures enkel in vorm\footnote{Zo speelt de volgorde in prolog wel een rol. Bovendien moet men in een DCG steeds alle argumenten vermelden, ook als ze ongebonden zijn.}. Qua expressiviteit voegen feature structures echter niets toe aan DCG's. Zo is bovenstaande grammaticale regel op basis van feature structures equivalent met grammatica~\ref{dcg:equivalentFS}\footnote{De volgorde van de features \texttt{naamval} en \texttt{getal} is belangrijk. De tijd van het werkwoord moet men vermelden, ook al maakt het niet uit voor deze regel.}.

\begin{dcg}{Een DCG-grammatica equivalent aan de regel met feature structures}{dcg:equivalentFS}
s() --> np([naamval:nom, getal:Getal]), vp([tijd:_, getal:Getal]).
\end{dcg}

In deze thesis zullen we altijd bovenstaande notatie gebruiken om de argumenten van een DCG-grammatica te noteren. De argumenten zijn namelijk taalkundig verantwoord en kunnen aanzien worden als feature structures.

\paragraph{} Feature structures (en dus ook argumenten in DCG's) zijn handig om de explosie van grammatica regels te voorkomen. Grammatica~\ref{dcg:withoutFS} (van Blackburn en Striegnitz \cite{NLPCourse}) is een voorbeeld van een grammatica zonder feature structures.

\begin{dcg}{Een grammatica zonder feature structures (uit \cite{NLPCourse})}{dcg:withoutFS}
s() --> np_singular(), vp_singular().
s() --> np_plural(), vp_plural().
np() --> np_singular().
np() --> np_plural().
np_singular() --> det(), n_singular().
np_plural() --> det(), n_plural().
vp_singular() --> intransitive_verb_singular().
vp_singular() --> transitive_verb_singular(), np_singular().
vp_singular() --> transitive_verb_singular(), np_plural().
vp_plural() --> intransitive_verb_plural().
vp_plural() --> transitive_verb_plural(), np_singular().
vp_plural() --> transitive_verb_plural(), np_plural().
n_singular() --> [man].
...
\end{dcg}
Hierbij staat de \texttt{n} voor zelfstandig naamwoord (van het Engelse \texttt{noun}) en \texttt{det} voor determinator. Deze grammatica can veel korter gemaakt worden door gebruik te maken van feature structures. Grammatica~\ref{dcg:withFS} (ook van Blackburn en Striegnitz \cite{NLPCourse}) is equivalent aan grammatica~\ref{dcg:withoutFS} maar maakt gebruikt van feature structures.

\begin{dcg}{Een equivalente grammatica met feature structures (ook uit \cite{NLPCourse})}{dcg:withFS}
s() --> np([num:Num]), vp([num:Num]).
np([num:Num]) --> det(), n([num:Num]).
vp([num:Num]) --> intransitive_verb([num:Num]).
vp([num:Num]) --> transitive_verb([num:Num]), np([num:_]).
n([num:singular]) --> [man].
...
\end{dcg} 

\paragraph{Conclusie} Door gebruik te maken van feature structures is de grammatica simpeler en leesbaarder. Bovendien hoeft het concept dat een zin bestaat uit een \texttt{np} gevolgd door een \texttt{vp} maar \'e\'en keer te worden uitgedrukt. De feature structures zorgen voor de congruentie in getal van het onderwerp met het werkwoord. 

\section{Discourse Representation Theory}
Voor het vertalen van natuurlijke taal naar logica zouden we graag gebruik maken van Frege's compositionaliteitsprincipe: De betekenis van een zin bestaat uit de combinatie van de betekenissen van de delen ervan. Als we eerste-orde-logica als doeltaal van onze vertaling nemen, komen we echter vrij snel in de problemen. Neem bijvoorbeeld de zin ``If a man lives, he breathes''. De vertaling hiervan in eerste-orde-logica is $\forall x \cdot man(x) \Rightarrow breathes(x)$. De vertaling van ``a man lives'' is echter $\exists x \cdot man(x)$, wat geen deel uit maakt van de betekenis van de hele zin. Blackburn en Bos \cite{Blackburn2006} geven nog een aantal andere problemen met eerste-orde-logica als doeltaal. Zo is eerste-orde-logica ook niet handig voor het oplossen van anaforische referenties.

Ze suggereren Discourse Representation Structures als alternatief. Bos \cite{Bos2011} definieert deze structuren als een lijst van \textit{discourse referents} (woordgroepen waarnaar andere woordgroepen kunnen verwijzen) en een lijst van condities i.v.m. die referenties. Blackburn en Bos \cite{Blackburn2006} geven o.a. een vertaling van deze DRS-structuren naar eerste-orde-logica. DRS-structuren hebben dus ook een formele betekenis. Zoals we verder zullen aantonen, zijn ze echter beter geschikt als doeltaal. Van hieruit kan dan verder vertaald worden naar eerste-orde-logica volgens de vertaling van Blackburn en Bos.

Wij hernemen hier hun vertaling naar eerste-orde-logica als semantiek voor deze structuren. Daarnaast geven we ook een inductieve definitie van zowel een DRS als een DRS-conditie.

Zij $x_i$ variabelen en $\gamma_i$ DRS-condities, dan is \drs{x_1, \ldots, x_n}{\gamma_1 \\ \ldots \\ \gamma_m} een DRS met als vertaling

\[
  \left(\ \drs{x_1, \ldots, x_n}{
      \gamma_1 \\
      \ldots \\
      \gamma_m
    }\ \right)^{fo} = \exists x_{1} \ldots \exists x_n\left((\gamma_{1})^{fo} \land \ldots \land (\gamma_m)^{fo}\right)
\]

Een lege DRS heeft als vertaling $\left( \drs{}{} \right)^{fo} = true$.

Zij $B_1$ en $B_2$ allebei een DRS, dan is ook $\drsMerge{B_1}{B_2}$ een DRS. De $\oplus$-connector kan men zien als de logische conjunctie van twee DRS-structuren. De vertaling bestaat uit het samenvoegen van de twee DRS-structuren:

\[
  \left( \drsMerge{\drs{x_1, \ldots, x_k}{
      \gamma_1 \\
      \ldots \\
      \gamma_l
    }}{\drs{y_1, \ldots, y_m}{
      \delta_1 \\
      \ldots \\
      \delta_n
    }} \right)^{fo} = \left (\drs{x_1, \ldots, x_k, y_1, \ldots, y_m}{
    \gamma_1 \\
    \ldots \\
    \gamma_l \\
    \delta_1 \\
    \ldots \\
    \delta_n
  }
  \right)^{fo}
\]

Dit zijn alle mogelijke DRS-structuren.

\paragraph{} Zij $R$ een predicaat met $n$ argumenten, $x_i$ een variabele, $\tau_i$ een term (uit eerste-orde-logica) en $B_i$ een DRS, dan zijn $R(x_1, \ldots, x_n)$, $\tau_1 = \tau_2$, $\lnot B$, $B_1 \lor B_2$ en $B_1 \Rightarrow B_2$ alle mogelijke DRS-condities. De vertaling naar eerste-orde-logica is:

\[\Big(R(x_1, ..., x_n)\Big)^{fo} = R(x_1, ..., x_n)\]
\[\Big(\tau_1 = \tau_2\Big)^{fo} = \Big(\tau_1 = \tau_2\Big)\]
\[\Big(\lnot B\Big)^{fo} = \lnot\Big(B\Big)^{fo}\]
\[\Big(B_1 \lor B_2\Big)^{fo} = \Big(B_1\Big)^{fo} \lor \Big(B_2\Big)^{fo}\]

\[\left(\ \drs{x_1, ..., x_n}{\gamma_1 \\ ... \\ \gamma_m} \Rightarrow B\right)^{fo} =  \forall x_1...\forall x_n\Bigg(\Big((\gamma_1)^{fo} \land ... \land (\gamma_m)^{fo}\Big) \Rightarrow \Big(B\Big)^{fo} \Bigg)\]

Er zijn twee vertaalstappen van natuurlijke taal naar eerste-orde-logica. Eerst van natuurlijke taal naar DRS en vervolgens van DRS naar eerste-orde-logica. De laatste stap is niet compositioneel. De vertaling van de laatste DRS-conditie is dit namelijk niet. De variabelen van het hoofd van de implicatie zijn universeel gekwantificeerd i.p.v. existentieel.

De eerste stap is echter wel compositioneel. De vertaling van ``If a man lives, he breathes'' naar DRS is \\

\drs{}{\ifdrs{x}{man(x) \\ lives(x)}{}{breathes(x)}}

\paragraph{} De betekenis van het deel ``a man lives'' \drs{x}{man(x) \\ lives(x)} maakt deel uit van de betekenis van de hele zin.

\section{Conclusie} Definite Clause Grammars vormen een expressieve grammatica die natuurlijke taal in al haar facetten makkelijk kan modelleren. Feature Structures kunnen gebruikt worden om grammatica's kort en leesbaar te houden. Ze geven een taalkundige verantwoording voor de argumenten van DCG's. Discourse Representation Structures vormen een representatie die tussen natuurlijke taal en eerste-orde-logica ligt. Ze zijn even expressief als eerste-orde-logica maar een aantal concepten in de natuurlijke taal zijn beter te modelleren met een DRS.
