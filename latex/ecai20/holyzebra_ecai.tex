\documentclass{ecai}
\usepackage{times}
\usepackage{graphicx}
\usepackage{latexsym}

%%\ecaisubmission   % inserts page numbers. Use only for submission of paper.
                  % Do NOT use for camera-ready version of paper.

\begin{document}

\title{Step-wise explanations of logic problems by automated reasoning}

\author{\{Bart, Emilio, Jens, Tias\}; 7 pages +1 with refs}

\maketitle
\bibliographystyle{ecai}

\begin{abstract}
We explore the problem of step-wise explaining how to solve logic problems, for example logic grid puzzle. The main challenge is that automated reasoning can freely combine all knowledge in its knowledge base to make derivations. In contrast, we want explanations to be simple and cognitively easy, so that a human can easily verify the reasoning step, and learn how to make similar reasoning steps.
We identify the problem as that of finding a good ordering of self-contained reasoning steps. Different interpretations of good ordering as well as self-containedness of a reasoning step are discussed. To make the search for a good ordering feasible in reasonable time, we propose an iterative refinement approach where minimal unsat core's over well-defined decompositions of the problem are searched. Our experiments show the feasibility of the approach in terms of the ordering and size of the explanations given as well as the computation time needed.
\end{abstract}

%The page limit for ECAI scientific papers is {\bf 7} pages, plus one ({\bf 1})

\section{Intro}
Need for explanations of reasoning systems

Shallow related work on quickxplain, 'explanations' in search, etc.

HolyGrail challenge and related work

Difference automated reasoning and human reasoning, measuring cognitive load

challenge 1: abstraction in self-contained reasoning step

challenge 2: ordering

challenge 3: computation efficiency

Contributions: (eerste aanzet!!!)
\begin{itemize}
\item Formalize the problem of finding good ordering of self-contained reasoning steps
\item Investigate different interpretations of good ordering and self-contained
\item Propose algorithms to approximate the ???some hardness result??? problem of finding the best ordering
\item Experimentally demonstrate the quality and feasibility of the approach
\end{itemize}

\section{Related work}
TODO

\section{Problem definition}
The larger problem

\section{Decomposition, propagation and unsat cores}
The connections, for a given set of facts and rules

\section{Finding a good ordering}
Some propositions of ordering measures: set of measures

Naive algo

Improved algo

Post-processing steps

\section{Logic Grid Puzzles: a holistic approach}
Description of our pipeline, including the NLP part (though not fully automated)

Can include Jens' table on challenges for BOS even with fully correct lexicon (from his poster).

\section{Experiments}

data, machines

\subsection{Q1: Abstraction level}
Some experiment with different levels of abstraction (e.g. all constraints at once, clues separate but all implicit at once, 2 groups of implicits, full split of implicits)
with 'set of measures', so probably a table

\subsection{Q2: Algorithmic comparison}
Both with the 'set of measures' table and computationally

\subsection{Q3: ablation analysis}
Leaving some parts of the algo out, to see step-wise improvements of the components?

\subsection{Q4: perceived quality}
Compare with the tutorial puzzle of logicgridpuzzles.com?

Ideally, a small human evaluation (e.g. ask people to solve a 3-with-3 puzzle and note the order of derivations and clues used, compare this 'ranking' to our ranking, discuss some differences.



\bibliography{refs}
\end{document}
%%%%%%%%%%%%%%%%%%%%%%%%%%%%%%%%%%%%%%%%%%%%%%%%%%%%%%%%%%%%%%%%%%%%%%
