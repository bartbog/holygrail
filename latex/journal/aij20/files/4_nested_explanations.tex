The generation of the explanation sequence, as formally defined in section 4, is guided by the cost function $f(E, S, N)$, a proxy for the mental-effort of the explanation-step. 
It turns out from preliminary analysis of the generated explanation sequence that some steps are still too hard to understand as they combine different constraints and/or multiple clues. 
In order to cope with this problem, we take a step back, and tackle it from another perspective: \textit{``Can I find another easier way to explain this step using the same facts and constraints by negating the newly derived fact and finding an inconsistency ?''} For the definition of \textit{inconsistency}, we refer back to definition \ref{def:consistent}.


To answer this question, we extend the explanation-production problem with the purpose of refining those explanations that are too complex and taking inspiration from counterfactual reasoning. 
Thus, our problem becomes an explanation-generating problem with 2 levels of abstractions: ``regular'' explanations and ``lower-level'' \textit{nested-explanations}. 
We define the concept of \emph{nested-explanation} as follows:

\begin{definition}
    Given a non-redundant explanation $(E_i, S_i, N_i)$, let $n_i \in N_i$ be a newly derived fact and partial interpretation  $I_0' = \{ \neg n_i \wedge E_i \}$ , the \textbf{nested-explanation} problem consists of finding a non-redundant explanation sequence that leads to an inconsistency 
    \[\langle \ (I_0',(\emptyset,\emptyset,\emptyset)),\ (I_1',(E_1',S_1',N_1')), \dots ,\ (I_n',(E_n',S_n',N_n')) \ \rangle\]
    such that $\forall \ (E_i',S_i',N_i') : f(E_i',S_i',N_i') \leq f(E_i, S_i, N_i)$. 
\end{definition}
In other words, for every newly derived fact, we look for an \emph{nested} explanation-sequence such that every nested explanation-step is easier than the original explanation step.

% Ideas : 
% \begin{itemize}
%     \item During analysis of sequence of reasoning steps too hard/ complex to understand 
%     \item 2 levels of abstraction
%     \item Refine explanations using counterfactual reasoning
%     \item 
% \end{itemize}

% We introduce a second