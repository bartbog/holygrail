\section{Conclusion}
The framework of Blackburn and Bos can be used to translate a CNL into logic. This way, we assign semantics to such a CNL. Therefore, we can use such a CNL as a knowledge representation language within a knowledge base system. This allows multiple types of inference on the (constructed) natural language. For a logic grid puzzle, this could mean automatically solving the puzzle but also helping the user by giving hints or helping the author by giving the possible solutions.

Moreover, we can expand the framework with types. This way we can translate the natural language to a typed logic. We have proven that type inference is possible as well. More research into a typed CNL is definitely necessary. For example, into a more complex type system. Or a type system which allows multiple types per word. The type system could then derive the correct instance of the verb. E.g. it could differentiate between ordering drinks and ordering food.
