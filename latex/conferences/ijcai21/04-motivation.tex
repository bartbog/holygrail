Our work is motivated by the problem of explaining satisfaction problems through a sequence of simple explanation steps. This can be used to teach people problem-solving skills, to compare the difficulty of related satisfaction problems (through the number and complexity of steps needed), and in human-computer solving assistants.

\ignore{

\paragraph{Using Constraints to Encode Domain Knowledge}

% {\color{OliveGreen} OLD TEXT TO BE REWRITTEN
\bart{Not efficient -> Begin paper zeggen. }
The constraints on OCUSs can not only be used to restrict the set of solutions, but also to improve the solver performance by encoding domain knowledge.
Indeed, if we know that all ``good'' OCUSs will satisfy certain constraints, or if we know that it suffices to search for OCUSs satisfying certain constraints (because each OCUS can easily be extended to one such OCUS),  we can also encode that knowledge in $p$, thereby restricting the possible options of the hitting set solver, aiming to improve overall performance of the algorithm. 

In the explanation application, we encountered this phenomenon as follows. 
The clues to be used in explanations were high-level (first-order) constraints. They were translated into clauses, using among other, a Tseitin transformation.
Hence, in the end the transformation of a single high-level clue consists of several clauses, of which some are definitions of newly introduced variables. 
Now, the associated cost function was only concerned with the issue ``\emph{was a certain clue used or not?}'', which translates at the lower level to ``\emph{does the OCUS contain at least one clause from the translation of the clue?}''.
Using such a cost function means that to compute the cost of an OCUS, it does not matter if a single, or if all clauses corresponding to a given clue are used. As such, we might as well include all of them, which can be encoded in $p$ as well.  

An alternative view on the same property is that we can \emph{reify} the high level constraint by considering an indicator variable defining satisfaction of the entire constraint. 
We can then add the property to $p$ that all reified constraints are \emph{hard constraints}, in the sense that they have to be included in each OCUS (and thus each hitting set). With that, only the truth/falsity of the single indicator variable is considered to be a clause of $\formulac$ that can be enabled/disabled by the hitting set algorithm. 
% This variable then represent whether or not the high level constraint is active.
It is easy to see that there is a one to one correspondence between the OCUSs produced by the two approaches. In our implementation, we opted for the latter because of its simplicity. 
%\tias{is this really to $p$? higher up we argued that we push $p$ into the MIP, but all hard clauses are kept outside of the MIP... I guess saying that har dlcauses are 'always included' is somehow doing that? it also means they are 'constant' in the MIP objective and hence can be removed from it, that is perhaps a more pragmatic view on it...}
%\emilio{phrases are loooong.}
}

The recent algorithm of ~\citet{ecai/BogaertsGCG20} starts from a formula $\formulac$ (in the application coming from a high level CSP), a partial interpretation $I$ (a conjunction of literals) and a cost function $f$ quantifying the difficulty of an explanation step, by means of a weight for every clause and literal in \formula. % and a set $U$ of atoms whose literals need explaining (e.g., not containing auxiliary variables). 
% \emilio{not sure why we talk about U and not later in the text. U is especially useful with the call to the optimal propagate}
% U not needed, auxiliaries get zero weight then done...

%; more specifically the cautious consequence of $\formulac \wedge I$ projected onto $U$.
 % Not part of input. It is what the algorithm 
%A single explanation step is an inference step $I' \implies N$ where $I' \subseteq I$ and $atoms(I') \subseteq U$.
\ignore{
\tias{Optional, alg can move to Apdx:} \cref{alg:explainCSPReal} shows the basic explanation sequence generation algorithm of~\cite{ecai/BogaertsGCG20}. 
\bart{If it stays, it needs a lot more explaining. For instance the magic going on in lines such as 
\[       N \gets  (\Iend \setminus I) \cap \optprop(X \cap I).\] 
A reader cannot understand this. Options are: drop or explain, but definitely not keep just like that.

The way it is currently phrased is also assuming that all the $\formulac$ are \textbf{HARD} constraints. But that is not how Bogaerts et al did it. I would definitely avoid phrasing it that way here because then we have to explain *everything*. That also does not match the OneStep description. 
So, if it stays, needs to be cleaned-up and made consistent throughtout hte paper and more details are needed. }





\begin{algorithm}[t]
    %\Input{${\cal C}$,  \textit{a CNF ${\cal C}$ over a vocabulary $V$} }
    %\Input{$U$, \textit{a user vocabulary $U \subseteq V$} }
    %\Input{$f$, \textit{a cost function $f : 2^{lits({U})} \rightarrow  \mathbb{N} $.}}
    %\Input{$I$, \textit{a partial interpretation over $U$}}
    %\Output{$E$, \textit{a sequence of explanation steps as implications $I_{expl} \implies N_{expl}$}}
    % \vspace*{0.01cm}
    \DontPrintSemicolon
    \caption{$\call{ExplainCSP}(\formulac, f, I, U)$}
    \label{alg:explainCSP}\label{alg:explainCSPReal}
    %$\sat \gets \initsat({\cal C}$) \;
        % \tcp{Hyp: f}
    $I_{end} \gets U \cap \optprop(\formulac \wedge I)$ \;
    $E \gets \langle \rangle$\;
    %$U \gets U \cap I_{end}$ \;
    %$I_{expl} = \{i \in I_{end} | f(i) < inf \wedge f(-i) < inf\}$ \;
    % \algemilio{bart: What's a better way to get the initial interpr.?}
    % $I \gets \{l \in I_{end} | f(\lnot l) = inf\}$ \;
    \While{$(I \cap U) \neq I_{end}$}{
        %$E \gets \call{bestStep}({\cal C},U, f,\Iend, I)$\;
        $X \gets \onestep({\cal C},f,I,\Iend)$\;
        %$I_{\mathit{best}} \gets I \cap X$\;
        %$\formulac_{\mathit{best}}\gets \formulac\cap X$\;
        $N \gets  (\Iend \setminus I) \cap \optprop(X \cap I)$\;
        %add $\{I_{\mathit{best}} \wedge \formulac_{\mathit{best}} \implies N_{\mathit{best}}\}$ to $E$\;
        add $\{(X \cap I) \implies N\}$ to $E$\;
        $I \gets I \cup N$\;
    }
    \Return{E}\;
\end{algorithm}
}


\newcommand\onestep{\ensuremath{\call{explain-One-Step}}\xspace}

\begin{algorithm}[t]
  \caption{$\onestep(\formulac,f,I,\Iend)$}
  \label{alg:oneStep}
$X_{best} \gets \mathit{nil}$\;
\For{$l \in \{\Iend \setminus I\}$}{
    $X \gets \call{MUS}{(\formulac \land I \land \neg l)}$\;
    \If{$f(X)<f(X_{best})$}{
        $X_{best} \gets X$\;
    }
}
\Return{$X_{best}$} 
\end{algorithm}


The goal is to find a sequence of \textit{simple} explanation steps, where the simplicity of a step is measured by the total cost of the elements used in the explanation.
An explanation step is an implication $I' \wedge \formulac' \implies N$ where $I'$ is a set of already derived literals, $\formulac'$ is a subset of constraints of the input formula $\formulac$, and $N$ is a set of literals entailed by $I'$ and $\formulac'$ which are not yet explained.

To obtain a sequence of such steps, ~\citet{ecai/BogaertsGCG20} iteratively search for the best (least costly) explanation step and add its consequence to the partial interpretation $I$.
%\bart{In retrospect, I propose to move this back to the constrainedness part. Reason: at this point (and also in prior work) we did not have ``hard constraints'' yet. So even if we would allow indicators, you can still turn their definitions on and off, which does not make sense at this point in the text.}
%\emilio{
%    changed to comment : 
%When aiming to explain satisfaction problems in terms of the subset of constraints and literals needed to derive a new literal, the initial interpretation $I$ should consist of indicators literals for each (group of) constraint(s) as well as already known true literals. }
%
The key part of the algorithm is the search for the next best explanation, given an interpretation $I$ derived so far. 
\cref{alg:oneStep} shows the gist of how this was done.
It takes as input the formula \formulac, a cost function $f$ quantifying the quality of explanations, an interpretation $I$ containing all already derived literals in the sequence so far, and the interpretation-to-explain $\Iend$. 
To compute an explanation, this procedure iterates over the literals that are still to explain, computes for each of them an associated MUS and subsequently selects the lowest cost one from found MUSs.
%\deleted{the best of the found such MUSs}.  
The reason this works is because there is a one-to-one correspondence between MUSs of $\formulac \land I \land \neg l$ and so-called \emph{non-redundant explanation} of $l$ in terms of (subsets of) $\formulac$ and $I$~\cite{ecai/BogaertsGCG20}. 

Experiments have shown that such a MUS-based approach can easily take hours, especially when multiple MUS calls are performed to increase the change of finding a good MUS, and hence that algorithmic improvements are needed to make it more practical. 
We see three main points of improvement, all of which will be tackled by our generic OCUS algorithms presented in the next section. 
% \tias{in the exps, MUS-based does not, also because we don't do the constraint trick. What to do?} \bart{Maybe the experiment section is a good place to discuss this in detail. That we there clearly state: in Bogaerts et al: the MUS based thing from algorithm \onestep is found not sufficient. To get better MUSs they implement this trick that walks a different balance between ``doing it for too many subsets and exploding there'' and ``getting terrible explanations''. What we show is that with our constrained stuff, you can get the best of both worlds: we get optimal explanatiosn (better than the subset stuff) plus we are as efficient as the no-subset approach). 
%  But... in the main text I would not go into detail because it would distract from our main message. }
\begin{inparaenum}
 \item First of all, since the algorithm is based on \call{MUS} calls, there is no guarantee that the explanation found is indeed optimal %don't use ``the best'' -> non-unique 
 (with respect to the given cost function). 
 Performing multiple MUS calls is only a heuristic due to the fact that \textit{there are no algorithms for cost-based unsatisfiable subset \textbf{optimization}}. 
 
 \item Second, this algorithm uses \call{MUS} calls for every literal to explain separately. The goal of all \deleted{of }these calls is to find a single unsatisfiable subset of $\formulac \land I \land \overline{(\Iend\setminus I)}$ that contains exactly one literal from $\overline{(\Iend\setminus I)}$. This begs the questions whether it is possible \textit{to compute a single (optimal) unsatisfiable subset \textbf{subject to constraints}}, where in our case, the constraint is to include exactly one literal from $\overline{(\Iend\setminus I)}$. 
 \item Finally, the algorithm that computes an entire explanation sequence makes use of repeated calls to \onestep and hence will solve many similar problems. This raises the issue of \textit{\textbf{incrementality}: can we re-use the computed data structures to achieve speed-ups in later calls?}
\end{inparaenum}




