\section{Logic Grid Puzzles}
Logic grid puzzles are logical puzzles. The most famous such puzzle is probably the Zebra Puzzle \cite{zebra}, sometimes also known as Einstein's Puzzle. These puzzles consist of a number of sentences or clues in natural language. In the clues, a number of domains appear (e.g. nationality, color, animal, ...), each with a number of domain elements (e.g. Norwegian, Canadian, blue, red, cow, horse, ...). Between each domain there is a bijection. The goal of such a puzzle is to find the value of these bijections, i.e. to find which domain elements belong together. E.g. The Norwegian lives in a blue house and keeps the horse. Every puzzle has an unique solution.
 
Logic grid puzzles can be considered as small specifications. Moreover, they can be expressed in fairly simple logical statements. Finally, it is easy to find numerous examples of these types of puzzles. It is these three properties that make logic grid puzzles the ideal domain to test our assumptions. Namely that the framework of Blackburn and Bos allows to construct a CNL with formal semantics that can be used for knowledge representation (within a small domain).
