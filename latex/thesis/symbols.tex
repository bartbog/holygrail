\chapter{Lijst van afkortingen en symbolen}
\section*{Afkortingen}
\begin{flushleft}
  \renewcommand{\arraystretch}{1.1}
  \begin{tabularx}{\textwidth}{@{}p{12mm}X@{}}
    CNL   & Constructed Natural Language \\
    DCG   & Definite Clause Grammars \\
    DRT   & Discourse Representation Theory \\
    DRS   & Discourse Representation Structures \\
    KBS   & Knowledge Base Systems \\
  \end{tabularx}
\end{flushleft}
\section*{Lexicale en Grammaticale categorieën}
De verschillende soorten constituenten die kunnen voorkomen in een grammatica (terminologie uit de taalkunde). We gebruiken de Engelse afkortingen voor gelijkaardigheid met de literatuur.
\begin{flushleft}
  \renewcommand{\arraystretch}{1.1}
  \begin{tabularx}{\textwidth}{@{}p{12mm}X@{}}
    s     & Sentence (zin) \\
    np    & Noun Phrase (naamwoordgroep) \\
    vp    & Verb Phrase (verbale constituent) \\
    ap    & Adjective Phrase (adjectiefconstituent) \\
    v     & Verb (werkwoord) \\
    iv    & Intransitive Verb (onovergankelijk werkwoord) \\
    tv    & Transitive Verb (overgankelijk werkwoord) \\
    av    & Auxiliary Verb (hulpwerkwoord) \\
    cop   & Copula (koppelwerkwoord) \\
    pn    & Proper Noun (eigennaam) \\
    det   & Determinator \\
    noun  & Noun (zelfstandig naamwoord) \\
    n     & (Modified) Noun (getransformeerd zelfstandig naamwoord) \\
    prep  & Preposition (voorzetsel) \\
    pp    & Prepositional Phrase (voorzetselconstituent) \\
    number& Hoofdtelwoord \\
    relpro& Relative pronoun (betrekkelijk voornaamwoord) \\
    rc    & Relative clause (betrekkelijke bijzin) \\
    comp  & Comparatief (bv. ``lower than'') \\
    some  & Onbepaalde woorden (bv. ``sometime'') \\
    coord & Coordinator (voegwoord) \\
  \end{tabularx}
\end{flushleft}

% \section*{Symbolen} */
% \begin{flushleft} */
%   \renewcommand{\arraystretch}{1.1} */
%   \begin{tabularx}{\textwidth}{@{}p{12mm}X@{}} */
%   \end{tabularx} */
% \end{flushleft} */
