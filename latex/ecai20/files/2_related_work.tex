Our work is inspired by the holy grail challenge at CP2019, which in turn has its roots in earlier work of E. Freuder's work \cite{}AAAI96-048.pdf... In that work, the authors... TODO

A competing approach at the holy grail workshop \cite{} identified 14 categories of constraints that are prevalent in logic grid puzzles, and hand-coded explanations for each constraint. In contrast, our approach searches for a minimal explanation without prior knowledge, and can also explain propagation through combinations of constraints.

Explanations of constraint satisfaction problems has been studied for overconstrained problems, most notably the QuickXplain method \cite{} that ...
\tias{Copied VERBATIM for reference from "Debugging Unsatisfiable Constraint Models": "Several algorithms have been proposed for constraint agnostic MUS enumera-
tion [13]. QuickXplain [8] attempts to discover MUSes using a divide and conquer
approach.Later approaches such as DAA [2] have more powerful techniques for
pruning the search space.DAA’s main drawback was that it has to enumerate
very large hitting sets and as a result had a high memory and time cost. The
MARCO algorithm [12] and the more recent MCS-MUS-BT [1] provide much
more efficient approaches for finding MUSes (see section 4)."}
Marco is interesting... keeps a map, exhaustively enumerates all MUS's.
\tias{Which MUS algo do we use??}

Also... \tias{Is the 'smallest MUS' problem some sort of MinSAT problem? e.g. relation MUS, MCS/MSS, MaxSAT... I'll think it through tomorrow} Ideally we could push down the cost into the MUS/MinSAT... or as constraint for it to terminate quickly.
