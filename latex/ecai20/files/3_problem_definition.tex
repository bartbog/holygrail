
When a human solves a logic grid puzzle, they typically maintain a grid with all the information obtained so far and use it in combination with the clues to derive new conclusions. In logic terminology, the user maintains a partial interpretation.  
With this in mind, the goal underlying our tool is to find a sequence of partial interpretations of the involved relations (e.g., in which it is known that Angie did not choose arrabiata sauce, but it still unknown whether she ordered farfalle) with a 
\emph{justification} (see below) of why each of the steps is correct that is ``simple enough for a person to understand''. 
%Of course the latter is a subjective matter and quite hard to quantify. Therefore, in this paper, we will use a reasonable approximation thereof. 

\tias{I guess you don't like 'explanation' (anymore)?}

\paragraph{Justification of Reasoning Steps.}
The goal is to obtain a sequence of incremental partial interpretations $\langle I_0, I_1, \ldots, I_n \rangle$ where $I_0$ is an initial partial interpretation (potentially empty), and $\forall i>0, I_{i-1} \geqp I_{i}$ with $I_n$ a (full) interpretation.
%\[ I_0 = \emptyset, I_1 = I_0 \cup N_1, \dots , I_n = I_{n-1}\cup N_n\]
%where $I_i$ represents the state of the grid at each point in time and $N_i$ represents the newly derived information. 
Additionally, for each incremental step $\langle I_{i-1}, I_i \rangle$ we want an explanation or \textit{justification} $(E_i,S_i)$ that justifies the newly derived information $N_i = I_i \setminus I_{i-1}$.

\tias{the thing below needs a definition to be referred to later again}
Since we are working in the context of a given fixed finite domain, we identify a partial structure with a consistent set of ground literals (i.e., a set of variable-free literals that do not contain both an atom and its negation), and thus $I_i\geqp I_{i-1}$ iff $I_i\supseteq I_{i-1}$.
With this view (partial) interpretations can also be seen as a theory (having each such literal as a sentence), which will allow us to make claims of the form 
$T\cup I \entails I'$ with $T$ a theory and  $I,I'$ (partial) interpretations, meaning that all literals in $I'$ follow from $T$ and $I$, or stated precisely, that all models of $T$ more precise than $I$ are also more precise than $I'$. 

\tias{Definition using the above? but below E/N are not more precise of eachother?}
We hence define a justification as a tuple $(E_i,S_i)$ such that: 
\begin{compactitem}
	\item $E_i\subseteq I_i$ (i.e., the explaining facts are a subset of what was previously derived),
	\item $S_i \subseteq T_P$ (i.e., a subset of the clues and implicit constraints are used), and 
	\item $S_i \cup E_i \entails N_i$ (i.e., all newly derived information indeed follows from this justification).
\end{compactitem}

The problem of simply checking whether $(E_i,S_i)$ explains\tias{explains or justifies?} $N_i$ is co-NP since this problem can be performed by verifying that $S_i \land \lnot N_i$ has no models more precise than $E_i$. It is hence an instance of the negation of a model expansion problem \cite{ternovskaMXcomplexity}.

Part of our goal of finding easy to interpret explanations is to avoid redundant information in the explanation. That is, we want a non-redundent explanation $(E_i,S_i)$ where none of the facts in $E_i$ or constraints in $S_i$ can be removed or else the implication $S_i \wedge E_i \rightarrow N_i$ would no longer hold. We will use the concept of Minimal Unsat Cores here (or Minimal Unsat Subset, MUS). A MUS over a set of constraints $C$ is a subset $C' \subseteq C$ such that $C' is UNSAT$ and $\forall c \in C': C' \setminus \{c\} is SAT$. If we know that $N_i$ follows from $S_i$ and $E_i$ then $\neg N_i \wedge S_i \wedge E_i$ must be UNSAT. A Minimal Unsat Subset (MUS) hereof will be a non-redundant subset of $N'_i \subseteq N_i$, $S'_i \subseteq S_I$ and $E'_i \subseteq E_i$ for which we know (because it is UNSAT): $\neg (\neg N'_i \wedge S'_i \wedge E'_i)$ hence through substitution $N'_i \vee \neg (S'_I \wedge E'_i)$ and hence $(S'_I \wedge E'_i) \rightarrow N'_I$. In other words, $(S'_I \wedge E'_i)$ is the smallest explanation of $N'_I$ for which this relation holds, for any smaller explanation it will no longer hold.

%Any incremental step $\langle I_{i-1}, I_i \rangle$ has a trivial justification in the form of $(I_i, T_P)$. However, our goal is to find interpretable justifications that a person can understand. We hence wish to remove redundant information from the justification and instead find a \emph{subset-minimal} justification.
After reifying the involved constraints of $T_P$ this can also be cast as a second-order problem of the form
\[\exists S_i\subseteq T_P, E_i\subseteq I_i: (\forall I: I\models S_i\land E_i \Rightarrow N_i) \land \lnot \exists S_i'\subseteq S_i, E_i'\subseteq E_i: \forall ... \]

%Our goal is not to derive any sequence of justifications, but a sequence of easy to interpret justifications. Indeed, while the step from the input to a full solution $\langle I_0, I \rangle$ is easily justified by $(I_0, T_P)$, this will not be interpretable for a user.

\paragraph{Interpretability of reasoning steps.}
While subset-minimality ensures that a justification is non-redundant, it does not quantify how \textit{interpretable} a justification is. 
This quantification is a problem-specific and often subjective manner. In line with Occam's Razor, one would expect that smaller justifications are easier to interpret that justifications that use a larger number of facts or constraints. For example, from a cognitive point of view, the more things a person needs to have in memory simultaneously, the more difficult the task will become. \tias{The cognitive statement is idd risky without scientific support...}
%To approximate of how easy to understand a justification is (i.e., a single transition in the above described sequence), we start from the simple cognitive idea \bart{can we cite this somewhere? } that (in general) the fewer things a human needs to have in memory simultaneously, the easier the task at hand is. 

We assume a cost-function $f(E_i,S_i,N_i)$ that measures the interpretability of a justification $(E_i,S_i)$ of a reasoning step $\langle I_{i-1}, I_i \rangle$ with $N_i = I_i \setminus I_{i-1}$. The goal is to find the sequence $\langle I_0, I_1, \ldots, I_n \rangle$ with minimal $\sum_{i=0}^n f(E_i,S_i,N_i)$ among all sequences for which $I_n$ is a solution of $T_P$.

As finding a single minimal $f(E_i,S_i,N_i)$ is ..., the entire search problem is obviously ...

\tias{Everything below (except the SO remark) is too logic grid puzzle focussed imho, it could come after the general problem description, or perhaps after the algo. Ideally we have: problem, algo, logic-grid case}.

\paragraph{Example} 
Hence, as a measure of difficulty we propose to use a weight function in terms of \textbf{the total number of clues, bijectivity and transitivity constraints, and previously derived facts} needed to derive a new fact. \bart{In fact ,we do not exactly use this measure, since we do not allow combining two clues if there exists some explanation that only uses one}
The reader should be advised that this is quite a rough approximation; for instance certain clues can be easier than others to reason about (e.g., compare ``Claudia did not choose puttanesca sauce'' with ``The person who ordered rotini is either the person who paid \$8 more than Damon or the person who paid \$8 less than Damon'') or it might be that humans perceive reasoning about clues in general as easier or harder than reasoning about the implicit (bijectivity and transitivity) constraints presents in logic grid puzzles. While researching better measures is an interesting topic, it falls out of the scope of the current paper. Moreover, many ideas of the current work remain valid when the actual measure is changed. 
Our observatoin from below, that using second-order logic tools such as e.g. \cite{proB,kr/BogaertsTS16} to find cost-optimal justifications is an option remains valid as long as the cost-functoin is expressible in (a sufficient extension of SO). Yet, due the high complexity, is not a direction we explored. 
\bart{Actually, I would really like to try this sometimes. The complexity argument is valid. But since we have small domains it is actually not so relevant... Maybe it works well. In that case it would yield good explanations!  }

\tias{Moved the below paragraph here, should be part of the cost-optimal imho}
In principle our problem of finding subset-minimal (or even, see the next paragarph, cost-optimal) justifications could be solved by specifying it in such a way in second-order logic and subsequently finding models for it, e.g. by a reduction to QBF \cite{kr/BogaertsTS16,kr/vanderHallenJ18}. 
However, since this is already an $\exists\forall\exists SO$ specification, we expect this approach not be feasible (this expectation remains to be verified in future work). 

\paragraph{Simple sequence vs simple steps}
In the above, we focused on generating and defining understandability of a single step in the reasoning process. 
However, in its most general form, we would like to optimize the understandability of the entire sequence of explanations. 
For this, even defining a measure seems to be a very hard task (for instance, is the understandability of a sequence related to its most difficult step or to its average difficulty? Sometimes even the ordering of the steps might make a difference for humans, even though the actual steps are the same, ...). 
Furthermore, even if we fix a measure, the problem of holistically optimizing a sequence of explanation steps is much harder than optimizing a single step since there are exponentially more sequences. 
Therefore, in the current paper, we take a greedy approach where at each step, the best next possible explanation is chosen, without taking global optimality of the sequence in mind. The results we obtain this way are satisfying. 
\bart{And already say something about postprocessing?}


% The grand goal underlying our tool is to find, given a logic grid puzzle (of which we assume it is given in some logical form for now; we revisit this in Section \ref{sec:holistic}), to find a sequence of partial assignments of variables (e.g., where it is already determined that certain entities are linked (or not linked) to which other entities) that is ``as easy to understand'' as possible.  
% Of course the latter is quite a vague concept and hard to find an objective measure for. However, we 
% The larger problem

