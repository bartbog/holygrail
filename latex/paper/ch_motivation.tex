\section{Motivation}
A knowledge base system is very powerful. It allows multiple inferences based on the same theory. However because knowledge representation languages can be hard to read, write and learn, an expert in formal languages is required. A formal language that is more accessible for non-experts, like a CNL with formal semantics, could solve this issue.

The idea of a knowledge base system is to apply multiple inferences to the same theory. Some examples of inferences that are possible once the clues of a logigram are translated into logic:
\begin{itemize}
  \item Solve the puzzle automatically.
  \item Given a (partial) solution by the user, indicate which clues have already been incorporated in the solution, which clues still hold some new information and which clues have been violated.
  \item Given a partial solution by the user, automatically derive a subset of clues that can be used to expand the solution. This could be part of a ``hint''-system for the user.
  \item Given a (sub)set of clues, indicate which solutions are still possible. This could help the author while writing new logigrams. The possible solutions help to construct a new hint that removes some of these solutions until only one remains.
\end{itemize}
