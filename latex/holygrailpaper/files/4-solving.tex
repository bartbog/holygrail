\paragraph{Translation} 

The framework presented in the previous section translates each clue into (typed) discourse representation theory (DRT) \cite{}. 
In order to build a complete specification of the puzzle, we compute the interpretation of the different types. If the grid is given, nothing needs to be done here. Otherwise we compute an equivalence relation on the set of proper nouns occurring in the clues (two proper nouns are equivalent if they occur in the same position of a verb/preposition; for instance if ``the Englishman smokes cigarettes'' and ``the person who owns a dog does not smoke cigars'' we derive that cigars and cigarettes are of the same type). It might happen that this does not contain enough information to completely determine the types for two reasons. 
First of all, not all proper nouns might occur in the clues (see for instance the zebra in Einsteins famous zebra puzzle). 
However, since the solution of a logic grid puzzle is always unique, there can at most be one such missing entity per type (otherwise by symmetry there would be multiple solutions) and hence, we can then simply add an anonymous element. 
Secondly, there might be a large variation in the verbs used to denote the same relation. In that case, we query the user to ask which verb are -- for the purpose of the puzzle -- synonyms. 

Once the types are completed, we construct a logical vocabulary containing: all the types and a relation for each transitive verb or preposition.
Additionally, we ensure that there is at least one relation between each two types, even if this relation does not occur in the clues. This is not important for solving the puzzle, but it is for explaining it; more on that follows in the next section. 
The interpretation of the types is encoded in \idp by means of a \emph{constructed type}.  A constructed type consist of a set of constants with two extra axioms implied: Domain Closure Axiom (DCA) and Unique Names Axiom (UNA). DCA states that the set of constants are the only possible elements of the domain. UNA states that all constants are different from each other.

% A type that corresponds to a numerical domain is translated to a subset of the natural numbers. In that case, the system asks the user the exact subset as it often cannot be automatically inferred from the clues.

After the vocabulary, we construct logical theories: 
\begin{itemize}
% \item 
 \item we translate each clue into the \idp language (an extension of typed first-order logic), and
 \item we add implicit constraints present in logic grid puzzles.
\end{itemize}
The translation of DRT into first-order logic is well-known. What we added to this is:  \todo{overlap with previous part}
\begin{itemize}
 \item using the type information as a sanity check: if a verb (which is translated into a relation) is used twice, then the two occurrences must have the same typing
 \item type inference: if the grid (the interpretation of the logical types underlying the puzzle) is not given as input, we derive they based on occurrences in the puzzle, e.g., using the fact that if two different entities are used as the subject of the same verb, then they must have the same type,
%  \item 
\end{itemize}

The, there are some implicit constraints stemming from the fact that this is a logic grid puzzle. 
First of all, our translation might generate multiple relations between two types. For instance if there are clues ``The tea drinker is from France'' and ``The person who owns a dog lives in England'', then the translation will create two relations $\mathit{from}$ and $\mathit{livesIn}$ between persons and countries. This happens regularly since logigram designers tend to vary their vocabulary to keep the puzzles interesting. However, we know that there is only one relation between two types, hence we add a theory containing \emph{synonymy} axioms; for this case concretely: 
\[\forall x \forall y \cdot livesIn(x, y) \Leftrightarrow from(x, y).\]
Similarly, if two relations have an inverse signature, they represent the inverse functions (for instance $\mathit{isOwnedBy}$ and $\mathit{likes}$) in the clues ``The Englishman likes cats'' and ``The dog is owned by the Belgian''). In this case we add constraints of the form
\[\forall x \forall y \cdot likes(x, y) \Leftrightarrow isOwnedBy(y,x).\]
Next, we add axioms that state that each relation between two types is actually a \emph{bijection}, e.g. 
\[(\forall x \cdot \exists y \cdot from(x, y)) \land (\forall y \cdot \exists x \cdot from(x, y)).\]
Finally, we add \emph{transitivity} axioms that state how the different relations relate. For instance is the dog is kept in the red house and the Englishman lives in the red house, then the Englishman keeps a dog. This kind of axioms is expressed as:
\[
 \forall x \forall y \forall z: keptIn(x,y) \land livesIn(z,y) \limplies keeps(z,x).
\]


\todo{More on the Usage of types?}


\paragraph{Solving}
The conjunction of all the logical theories created in the previous paragraph completely characterize the constraints underlying a logic grid puzzle.
In order to solve the puzzle, we use \idp's built-in \emph{model expansion} inference, which searches for a model in a given finite domain. Under the hood, \idp uses \minisatid \mycite{minisatid}, a solver using SAT \mycite{SAT} and CP \mycite{CP} technology, in particular lazy clause generation \mycite{lcg} and conflict-driven clause learning \mycite{cdcl}. In our experience, the solving part is often quite trivial since the puzzles are usually crafted in such a way that they are actually easy to solve by a human. However, it deserves to be mentioned that theoretically, logic grid puzzles are NP hard.

\todo{do we wish to say something about this: if there does not have to be a solution and it does not have to be unique, then we can get NP hardness. However, if there has to be a unique solution proving hardness is harder}

REDUCTION OF 3SAT TO LOGIC GRID PUZZLE. 
Create a type with elements ``true'' and ``false''. 
Given a 3CNF over \voc. For each atom in \voc create a type with two elements: $p$ and $\lnot p$.
For each clause $p \lor \lnot q \lor r$, add a clue 
``At least one of the following hold: p is associated with true, $q$ is associated with false or $r$ is associated with 




