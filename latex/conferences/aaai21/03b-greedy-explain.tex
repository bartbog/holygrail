% !TeX root = ./main.tex

We now rephrase, and simplify the explanation algorithm of \citet{ecai/BogaertsGCG20} in terms of OMUS calls. The use of OMUS, rather than a plain MUS call simplifies the algorithm and removes the need for ... EPLAIN

BLA BLA 


% \begin{enumerate}
%     \item OMUS oracle
%     \item Greedy sequence
% \end{enumerate}

\begin{algorithm}
    \DontPrintSemicolon
    \todo{CLEANUP: is it S or T?}
    $\m{I}_{end} \gets$ \textsc{propagate($\m{I}_0$, $\m{T}$)} \;
    $\m{I} \gets \m{I}_0$  \;
    $Seq \gets \emptyset$  \;
    \While{  $\m{I} \neq \m{I}_{end}$ }{
      \For{$i \in \m{I}_{end} \setminus \m{I}$}{
        $X_i \gets$ \textsc{OMUS($\{\neg i\} \wedge \m{I} \wedge \m{S}$)} \;
        $E_i \gets$ $\m{I} \cap X_i$  \;
        $S_i \gets$ $\m{T} \cap X_i$  \;
        $\m{N}_i \gets$ \textsc{propagate($E_i \wedge \m{S}_i$)} \;
        }
        $(E_{best}, S_{best}, N_{best}) \gets (E_i,S_i,N_i)$ with lowest $f(E_i,S_i,N_i)$ \;
        append $( E_{best}, S_{best}, N_{best})$ to $Seq$ \;
        $\m{I} \gets \m{I} \cup \{N_{best}\}$ \;
    }
  \caption{CSP-Explain($\m{T} ,\ f \ [,  \ \m{I}_0 ]$)}
  \todo{present as simple as possible}
  \label{alg:cspExplain}
\end{algorithm}

\todo{explain what the goal is, and what is going on here.}

\bart{I would take the focus away from ``CSP''. This paperi s about SAT-like problems}


When investigating \cref{alg:cspExplain}, we see ample room for improvement. 
First of all, in order to compute an entire explanation sequence, \label{alg:cspExplain} will make use of very many \omus calls that share a large part of the theory. 
This suggests the possibility of developing \emph{incremental} OMUS algorithms that reuse results from previous calls. 
Secondly, the inner loop in \cref{alg:cspExplain} loops over all possibly derivable  literals, searches for each of them an OMUS and subsequently, the best of those is taken. 
In an ideal situation, this could be done in a single call to a solver that exploits all possible information at once. 
In its most general form, this idea can be phrased as the search for a MUS of a given theory that is subject to certain constraints. 
In Section \ref{sec:constrained}, we explore this idea in the generic setting and develop an algorithm that searches for an OMUS satisfying a given set of meta-level constraints. 
In \cref{sec:incrementalExp}, we combine these two ideas to develop a simpler, and more efficient explanation-generation algorithm. 
