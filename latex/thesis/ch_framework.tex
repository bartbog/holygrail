\chapter{Een framework voor semantische analyse}
In dit hoofdstuk bespreken we het framework van Blackburn en Bos \cite{Blackburn2005, Blackburn2006} voor semantische analyse. Het bestaat uit 4 onderdelen: het lexicon (het vocabularium), de grammatica, de semantiek van de grammaticale regels en de semantiek van de woorden in het lexicon. Het hele framework is gebaseerd op lambda-calcalus en Frege's compositionality principe dat stelt dat de betekenis van een woordgroep enkel afhangt van de betekenissen van de woorden waaruit ze bestaat.

\paragraph{} \textit{Dit hele hoofdstuk is een samenvatting van de relevante hoofdstukken van de boeken van Blackburn en Bos \cite{Blackburn2005, Blackburn2006} ``A first course in computational semantics''}

\section{Lexicon}
Het lexicon bestaat uit een opsomming van alle woorden met een aantal (taalkundige) features (zoals de categorie van het woord). Tabel \ref{tbl:lexicon} geeft een voorbeeld van een lexicon. Het lexicon is meer dan een woordenboek. Het bevat alle woordvormen, niet enkel de basisvorm. Zo komt ``love'' drie keer voor in het lexicon. ``love'' zelf komt één keer voor als infinitief en één keer als meervoud van de tegenwoordige tijd. ``loves'' is dan weer het enkelvoud van de tegenwoordige tijd. De meeste woordvormen hebben ook een feature ``Symbool'', dit zal gebruikt worden bij de semantiek van de woorden.

\begin{table}[!]
  \centering
  \begin{tabular}{@{}llll@{}}
    \toprule
    \textbf{Woordvorm} & \textbf{Categorie} & \textbf{Symbool} & \textbf{Andere features} \\ \midrule
    man                & zelfstandig naamwoord     & man     & num=sg            \\
    men                & zelfstandig naamwoord     & man     & num=pl            \\
    woman              & zelfstandig naamwoord     & woman   & num=sg            \\
    women              & zelfstandig naamwoord     & woman   & num=pl            \\
    John               & eigennaam                 & John    &                   \\
    sleep              & onovergankelijk werkwoord & sleeps  & inf=inf           \\
    sleeps             & onovergankelijk werkwoord & sleeps  & inf=fin, num=sg   \\
    sleep              & onovergankelijk werkwoord & sleeps  & inf=fin, num=pl   \\
    love               & overgankelijk werkwoord   & loves   & inf=inf           \\
    loves              & overgankelijk werkwoord   & loves   & inf=fin, num=sg   \\
    love               & overgankelijk werkwoord   & loves   & inf=fin, num=pl   \\
    a                  & determinator              & /       & type=existential  \\
    every              & determinator              & /       & type=universal    \\
    \bottomrule
  \end{tabular}
  \caption{Een voorbeeld van een lexicon}
  \label{tbl:lexicon}
\end{table}

\section{Grammatica}
De grammatica bepaalt welke woorden samen woordgroepen vormen, welke woordgroepen samen andere woordgroepen vormen en welke woordgroepen een zin vormen. Op die manier ontstaat er een boom van woorden.

\paragraph{Een simpele grammatica} \autoref{gramm:simple-gramm} bevat een simpele grammatica. Een simpele zin bestaat uit een \texttt{np} gevolgd door een \texttt{vp}, beiden met hetzelfde getal. Een noun phrase (\texttt{np}) naamwoordgroep of nominale constituent is een woordgroep waar het naamwoord het belangrijkste woord is. Deze woordgroep verwijst altijd naar één of meerdere entiteiten. Een verb phrase (\texttt{vp}) of verbale constituent is een woordgroep waar het werkwoord het belangrijkste woord is. Een verbale constituent drukt een actie uit.

\begin{ex}
  Een simpele grammatica. De lexicale categorieën zijn \texttt{pn}, \texttt{det}, \texttt{n}, \texttt{iv} en \texttt{tv}. Voor deze categorieën wordt het lexicon gebruikt.
  \label{gramm:simple-gramm}
  \begin{quote}
    \texttt{s ---> np([num:Num]), vp([num:Num]).} \\
    \texttt{s ---> [if], s, s.} \\
    \texttt{np([num:sg]) ---> pn.} \\
    \texttt{np([num:Num]) ---> det([num:Num]), n([num:Num]).} \\
    \texttt{vp([num:Num]) ---> iv([num:Num]).} \\
    \texttt{vp([num:Num]) ---> tv([num:Num]), np([num:\_]).} \\
  \end{quote}
\end{ex} 

Een complexe zin bestaat uit het functiewoord ``if'' gevolgd door twee zinnen (bijvoorbeeld ``If a man breathes, he lives''). Een functiewoord is deel van de grammatica en komt niet voor in het lexicon. Ze helpen om de structuur van de zin te herkennen. De betekenis van deze woorden komt via de grammaticale semantiek naar boven.

Een naamwoordgroep (\texttt{np}) kan bestaan uit een eigen naam (proper name of \texttt{pn}) of uit een determinator (\texttt{det}, bijvoorbeeld een lidwoord) en een zelfstandig naamwoord (noun of \texttt{n}) die overeenkomen in getal. Een eigennaam is in deze grammatica altijd in het enkelvoud.

Een verbale constituent (\texttt{vp}) bestaat uit een onovergankelijk werkwoord (intransitive verb of \texttt{iv}) of uit een vergankelijk werkwoord (transitive verb of \texttt{tv}) gevolgd door een nieuwe naamwoordgroep (als lijdend voorwerp). Het werkwoord moet in getal overeenkomen met het getal van de verbale constituent. Daardoor zal het getal van het werkwoord en het onderwerp altijd overeenkomen.

De lexicale categorieën in \autoref{gramm:simple-gramm} zijn \texttt{pn}, \texttt{det}, \texttt{n}, \texttt{iv} en \texttt{tv}. Dat wil zeggen dat men deze niet-terminalen moet gaan opzoeken in het lexicon. Alle woordvormen met de juiste features komen in aanmerking.

Bovenstaande grammatica is nog heel beperkt. De moeilijkheid ligt erin om de grammatica simpel te houden maar toch zoveel mogelijk gewenste zinnen toe te laten. Om logigrammen automatisch te kunnen vertalen moet er dus een grammatica opgesteld worden die de zinnen van deze logigrammen omvat.

\paragraph{Een boom} Op basis van deze grammatica kunnen we ook een parse trees opbouwen voor elke geldige zin. Zo wordt ``Every woman loves john omgezet in de boom

\Tree[.s [.np [.det every ] [.n woman ]] [.vp [.tv loves ] [.np [.pn john ]]]]

Op elke knoop in deze boom zullen we later Frege's compositionality principe toepassen: de betekenis van een woordgroep is gelijk aan een combinatie van de betekenissen van de woord(groep)en waaruit ze bestaat.

\paragraph{Conclusie} De grammatica bepaalt welke combinaties van woorden zinnen vormen. Ze bepaalt dus welke zinnen in de taal liggen en welke er buiten vallen. Bovendien geeft de grammatica ons een boom. Deze boom zullen we gebruiken om de betekenis van onder uit naar boven toe te laten propageren volgens Frege's compositionality principe.

\section{Semantiek}
Het lexicon bepaalt welke woorden gebruikt mogen worden, de grammatica hoe deze woorden een zin kunnen vormen. De vraag die nog rest is welke betekenis de zin heeft. Daarvoor doen we een beroep op de lambda-calcalus. Eerst bespreken we hoe we de betekenis van een woordgroep kunnen afleiden uit de betekenis van de woordgroepen waaruit ze bestaan. Daarna bespreken we de betekenis van de woorden uit het lexicon.

\subsection{Semantiek van de grammaticale regels}
\paragraph{Een getypeerde lambda-calcalus}
Voor de betekenis van de taal zullen we gebruiken maken van een getypeerde lambda-calcalus. We gebruiken het symbool \texttt{@} voor de applicatie uit de lambda-calcalus, naar analogie met Blackburn en Bos. Er zijn drie types in onze lambda-calcalus \texttt{e} voor entiteiten, \texttt{t} voor \textit{truth values} en ten slotte het functie-type die we noteren met $\rightarrow$. Zo is $\lambda x. man(x)$ een lambda-expressie van type $e \rightarrow t$. Elk woord zal een lambda-formule als betekenis krijgen. Deze formules zullen gecombineerd worden tot één formule voor de zin die geen lambda's meer bevat. Die formule vormt de betekenis van de zin. De betekenis van een zin is dus van type $t$.

\paragraph{Frege's compositionality principe} Voor de betekenis van de grammatica berusten we op Frege's compositionality principe: De betekenis van een woordgroep bestaat uit een combinatie van de betekenissen van de woorden of woordgroepen waaruit ze bestaat en de manier waarop deze woorden gecombineerd worden. Op deze manier propageren we de semantiek van de woorden (die de bladeren in de boom vormen) naar boven toe om zo de betekenis van de zin te verkrijgen. We hernemen bovenstaande grammaticale regels nu en voegen de semantiek toe. Later zullen we aantonen dat deze simpele combinatieregels tot een zinnig resultaat leiden.
%We gebruiken de uitgebreide notatie van de feature structures voor de leesbaarheid.

\begin{table}[h]
  \begin{tabular}{@{}ll}
    \hline
    \textbf{Grammaticale regel} & \textbf{Semantiek} \\
    \hline
    \texttt{s ---> np([num:Num]), vp([num:Num]).}              & $\sem{s} = \sem{vp}@\sem{np}$ \\
    \texttt{s ---> [if], s, s.}                                & $\sem{s} = \drs{}{\sem{s1} \Rightarrow \sem{s2}}$ \\
    \texttt{np([num:sg]) ---> pn.}                             & $\sem{np} = \sem{pn}$ \\
    \texttt{np([num:Num]) ---> det([num:Num]), n([num:Num]).}  & $\sem{np} = \sem{det}@\sem{n}$ \\
    \texttt{vp([num:Num]) ---> iv([num:Num]).}                 & $\sem{vp} = \sem{iv}$ \\
    \texttt{vp([num:Num]) ---> tv([num:Num]), np([num:\_]).}   & $\sem{vp} = \app{\sem{tv}}{\sem{np}}$\\
    \hline
  \end{tabular}
  \centering
  \caption{De semantiek van de grammatica uit \autoref{gramm:simple-gramm}}
  \label{tbl:grammar-sem}
\end{table}

\autoref{tbl:grammar-sem} geeft een overzicht van de semantiek van de grammatica uit \autoref{gramm:simple-gramm}. Voor woordgroepen die maar uit één woord bestaan is de betekenis van de woordgroep gelijk aan die van het woord zelf. Voor de meeste andere woordgroepen is de betekenis een simpele lambda-applicatie van de betekenissen van de woordgroepen waaruit ze bestaan. Enkel voor de speciale zinsstructuur van de conditionele zin is er ook een speciale constructie nodig in de semantiek. Indien we het functiewoord ``if'' naar het lexicon zouden verhuizen, dan volstaan twee lambda-applicaties\footnote{$\sem{if} = \lambdaf{S1}{\lambdaf{S2}{\drs{}{S1 \Rightarrow S2}}}$ en $\sem{s} = \app{\left( \app{\sem{if}}{\sem{s1}} \right)}{\sem{s2}}$}.

% \[
%   \fstructure{
%     \feature{CAT}{s}
%     \feature{SEM}{\fvariable{VP}@\fvariable{NP}}
%   }
%   \rightarrow
%   \fstructure{
%     \feature{CAT}{np}
%     \feature{NUM}{\fvariable{Num}}
%     \feature{SEM}{\fvariable{NP}}
%   }
%   \fstructure{
%     \feature{CAT}{vp}
%     \feature{NUM}{\fvariable{Num}}
%     \feature{SEM}{\fvariable{VP}}
%   }
% \]
% De semantiek van een simpele zin is de lambda-applicatie van de semantiek van de naamwoordgroep op die van de verbale constituent.

% \[
%   \fstructure{
%     \feature{CAT}{s}
%     \feature{SEM}{\drs{}{\fvariable{S1} \Rightarrow \fvariable{S2}}}
%   }
%   \rightarrow
%   if
%   \fstructure{
%     \feature{CAT}{s}
%     \feature{SEM}{\fvariable{S1}}
%   }
%   \fstructure{
%     \feature{CAT}{s}
%     \feature{SEM}{\fvariable{S2}}
%   }
% \]
% De semantiek van een conditionele zin wordt vertaald op een implicatie in een DRS.

% \[
%   \fstructure{
%     \feature{CAT}{np}
%     \feature{NUM}{sg}
%     \feature{SEM}{\fvariable{PN}}
%   }
%   \rightarrow
%   \fstructure{
%     \feature{CAT}{pn}
%     \feature{SEM}{\fvariable{PN}}
%   }
% \]
% De semantiek van een naamwoordgroep die bestaat uit een eigennaam is gelijk aan de semantiek van die eigennaam.

% \[
%   \fstructure{
%     \feature{CAT}{np}
%     \feature{NUM}{\fvariable{Num}}
%     \feature{SEM}{\fvariable{DET}@\fvariable{N}}
%   }
%   \rightarrow
%   \fstructure{
%     \feature{CAT}{det}
%     \feature{NUM}{\fvariable{Num}}
%     \feature{SEM}{\fvariable{DET}}
%   }
%   \fstructure{
%     \feature{CAT}{n}
%     \feature{NUM}{\fvariable{Num}}
%     \feature{SEM}{\fvariable{N}}
%   }
% \]
% De semantiek van een naamwoordgroep die bestaat uit een lidwoord en een zelfstandig naamwoord is de applicatie van de semantiek van het zelfstandig naamwoord op die van het lidwoord.

% \[
%   \fstructure{
%     \feature{CAT}{vp}
%     \feature{NUM}{\fvariable{Num}}
%     \feature{SEM}{\fvariable{IV}}
%   }
%   \rightarrow
%   \fstructure{
%     \feature{CAT}{iv}
%     \feature{NUM}{\fvariable{Num}}
%     \feature{SEM}{\fvariable{IV}}
%   }
% \]
% De betekenis van een verbale constituent die bestaat uit een onovergankelijk werkwoord is gelijk aan die van het werkwoord.

% \[
%   \fstructure{
%     \feature{CAT}{vp}
%     \feature{NUM}{\fvariable{Num}}
%     \feature{SEM}{\fvariable{TV}@\fvariable{NP}}
%   }
%   \rightarrow
%   \fstructure{
%     \feature{CAT}{tv}
%     \feature{NUM}{\fvariable{Num}}
%     \feature{SEM}{\fvariable{TV}}
%   }
%   \fstructure{
%     \feature{CAT}{np}
%     \feature{NUM}{\fvariable{1}}
%     \feature{SEM}{\fvariable{NP}}
%   }
% \]
% De betekenis van een verbale constituent die bestaat uit een vergankelijk werkwoord met lijdend voorwerp is gelijk aan die van het lijdend voorwerp toegepast op die van het werkwoord.

% \begin{ex}
%   Een simpele grammatica met semantiek. De lexicale categorieën zijn \texttt{pn}, \texttt{det}, \texttt{n}, \texttt{iv} en \texttt{tv}
%   \label{gramm:simple-gramm}
%   \begin{quote}
%     \texttt{s([sem:VP@NP]) ---> np([num:Num, sem:NP]), vp([num:Num, sem:VP]).} \\
%     \texttt{s([sem:drs([], [S1 => S2])]) ---> [if], s([sem:S1]), s([sem:S2]).} \\
%     \texttt{np([num:sg]) ---> pn.} \\
%     \texttt{np([num:Num]) ---> det([num:Num]), n([num:Num]).} \\
%     \texttt{vp([num:Num]) ---> iv([num:Num]).} \\
%     \texttt{vp([num:Num]) ---> tv([num:Num]), np([num:\_]).} \\
%   \end{quote}
% \end{ex} 

\subsection{Semantiek van het lexicon} We weten nu hoe we de semantiek van de woorden kunnen combineren tot de semantiek van de woordgroepen en bij uitbreiding tot die van een zin. Er ontbreekt enkel nog de semantiek van de woorden zelf.

Voor we de betekenis van woorden kunnen opstellen moeten we eerst de signatuur achterhalen. Cruciaal voor het framework is dat elke grammaticale categorie exact 1 signatuur heeft. Zodanig dat we altijd woordgroepen van dezelfde categorie kunnen uitwisselen voor elkaar. We gebruiken de functie $\tau$ om de signatuur aan te duiden.

We beginnen met de signatuur van een nominale constituent (\texttt{np}) te achterhalen. Daarna bekijken we achtereenvolgens de eigennaam (\texttt{pn}), het zelfstandig naamwoord (\texttt{n}) en het lidwoord (\texttt{det}). Ten slotte bekijken we de hiërarchie van de verbale constituenten: de verbale constituent zelf (\texttt{vp}), onovergankelijk werkwoord (\texttt{iv}) en overgankelijk werkwoord (\texttt{tv}).

\paragraph{De signatuur voor een nominale constituent (np)} Een nominale constituent of naamwoordgroep is een woordgroep waarin het naamwoord het belangrijkste woord is. De betekenis ervan is een verwijzing naar een entiteit (of een groep van entiteiten). Een naïve signatuur voor een \texttt{np} zou dus $\tau(np) = e$ kunnen zijn. Echter, als er sprake is van kwantificatie zoals bij ``a man'' of ``every man'' dan is er ook sprake van een scope. Deze kan niet gevat woorden in een simpele signatuur als $\tau(np) = e$.

Er is dus nood aan een signatuur die een box rond een entiteit bevat. $$\tau(np) = (e \rightarrow t) \rightarrow t$$ voldoet hieraan.

\paragraph{Eigennaam, zelfstandig naamwoord en lidwoord} De signatuur van een eigennaam is gelijk aan die van een naamwoordgroep. Dat volgt uit de semantiek van de grammatica. $$\tau(pn) = \tau(np) = (e \rightarrow t) \rightarrow t$$ Uit de semantiek van de grammatica volgt ook dat de signatuur van een lidwoord gelijk is aan die van een zelfstandig naamwoord naar een naamwoordgroep. $$ \tau(det) = \tau(n) \rightarrow \tau(np) = \tau(n) \rightarrow (e \rightarrow t) \rightarrow t$$ Een zelfstandig naamwoord is een conditie voor een entiteit. Elke entiteit kan ofwel omschreven worden met het naamwoord of niet. Het is dus een functie van een entiteit naar een \textit{truth value}. $$\tau(n) = e \rightarrow t$$ Waardoor de signatuur voor een lidwoord gelijk is aan $$ \tau(det) = \tau(n) \rightarrow (e \rightarrow t) \rightarrow t = (e \rightarrow t) \rightarrow (e \rightarrow t) \rightarrow t $$

\paragraph{Verbale constituent} De signatuur voor een verbale constituent (\texttt{vp}) volgt opnieuw uit de grammatica en de andere signaturen $$\tau(vp) = \tau(np) \rightarrow \tau(s) = ((e \rightarrow t) \rightarrow t) \rightarrow t$$ De signatuur van een onovergankelijk werkwoord is gelijk aan die van een verbale constituent (volgens de grammatica). $$\tau(iv) = \tau(vp) = ((e \rightarrow t) \rightarrow t) \rightarrow t$$ De signatuur van een vergankelijk werkwoord wordt dan $$\tau(tv) = \tau(np) \rightarrow \tau(vp) = ((e \rightarrow t) \rightarrow t) \rightarrow ((e \rightarrow t) \rightarrow t) \rightarrow t$$

\paragraph{} \autoref{tbl:signaturen} vat alle signaturen nog eens samen. Op basis van deze signaturen kunnen we de betekenis van het lexicon opstellen. Blackburn en Bos gebruiken hiervoor \textit{semantische macro's}. Dat wil zeggen dat ze voor elke lexicale categorie een functie hebben die een woordvorm uit het lexicon afbeeldt op een lambda-expressie. Hiervoor wordt enkel de feature ``Symbool'' van de woordvorm in kwestie gebruikt.

\begin{table}[h]
  \begin{tabular}{@{}lll}
    \hline
    \textbf{Grammaticale categorie}             & \textbf{Signatuur} \\
    \hline 
    Zin (\texttt{s})                          & $t$ \\
    Naamwoordgroep (\texttt{np})              & $(e \rightarrow t) \rightarrow t$ \\
    Eigennaam (\texttt{pn})                   & $(e \rightarrow t) \rightarrow t$ \\
    Zelfstandig naamwoord (\texttt{n})        & $(e \rightarrow t)$ \\
    Lidwoord (\texttt{det})                   & $(e \rightarrow t) \rightarrow (e \rightarrow t) \rightarrow t$ \\
    Verbale constituent (\texttt{vp})         & $((e \rightarrow t) \rightarrow t) \rightarrow t$ \\
    Onovergankelijk werkwoord (\texttt{iv})   & $((e \rightarrow t) \rightarrow t) \rightarrow t$ \\
    Vergankelijke werkwoord (\texttt{tv})     & $((e \rightarrow t) \rightarrow t) \rightarrow ((e \rightarrow t) \rightarrow t) \rightarrow t$ \\
    \hline
  \end{tabular}
  \centering
  \caption{De signaturen van de grammaticale categorieën uit \autoref{gramm:simple-gramm}}
  \label{tbl:signaturen}
\end{table}

De grammatica van \autoref{gramm:simple-gramm} telt 5 lexicale categorieën: \texttt{pn}, \texttt{n}, \texttt{det}, \texttt{iv} en \texttt{tv}.
\begin{itemize}
  \item Een eigennaam (\texttt{pn}) kan vertaald worden als een constante met als naam het symbool dat bij die eigennaam hoort. We moeten deze constante enkel nog in de ``box'' steken. $\sem{pn} = \lambdaf{P}{\app{P}{\textit{Symbool}}}$. Bijvoorbeeld voor ``John'': $\sem{John} = \lambdaf{P}{\app{P}{John}}$
  \item Een zelfstandig naamwoord (\texttt{n}) test of een referentie kan omschreven worden met dat naamwoord of niet. $\sem{n} = \lambdaf{x}{\drs{}{\textit{Symbool}(x)}}$. Bijvoorbeeld voor ``man'': $\sem{man} = \lambdaf{x}{\drs{}{man(x)}}$
  \item Een lidwoord of meer algemeen een determinator (\texttt{det}) introduceert een nieuwe referentie die een bepaalde scope heeft. Hier zijn er meerdere mogelijke vertalingen, één voor elk type van determinator. Een determinator heeft 2 argumenten: de \textit{restriction} en de \textit{nuclear scope}. De \textit{restriction} wordt opgevuld door het zelfstandig naamwoord (met eventuele bijzinnen). De \textit{nuclear scope} wordt opgevuld door de verbale constituent.
    \begin{itemize}
      \item Voor een universele determinator (zoals ``every'') moet de variabele met een universele quantor gebonden zijn. In eerste-orde-logica krijgen we dan $\sem{det_{universeel}} = \lambdaf{R}{\lambdaf{S}{\forall x. \left( \app{R}{x} \Rightarrow \app{S}{x} \right)}}$. In DRS wordt dit $$\sem{det_{universeel}} = \lambdaf{R}{\lambdaf{S}{\drs{}{\drsImpl{\drsMerge{\drs{x}{}}{\app{R}{x}}}{\app{S}{x}}}}}$$
      \item De existentiële determinator (zoals ``a'') introduceert een variabele die gebonden is door een existentiële quantor. $$\sem{det_{existentieel}} = \lambdaf{R}{\lambdaf{S}{\left( \drs{x}{} \oplus \app{R}{x} \oplus \app{S}{x} \right)}}$$
      \item De negatieve determinator (zoals ``no''): Deze determinator drukt uit dat er geen entiteit bestaat die aan bepaalde voorwaarden voldoet. Deze introduceert dus een genegeerde existentiële quantor. $$\sem{det_{negatief}} = \lambdaf{R}{\lambdaf{S}{\drs{}{\lnot \left( \drs{x}{} \oplus \app{R}{x} \oplus \app{S}{x} \right)}}}$$
    \end{itemize}
  \item Een onovergankelijk werkwoord (\texttt{iv}) neemt de naamwoordgroep die het onderwerp vormt als argument. Het moet de entiteit van die naamwoordgroep terug uit de box halen (dit kunnen we doen door het argument aan te roepen met een lambda-expressie) en testen of die entiteit de actie van het werkwoord uitvoert $$\sem{iv} = \lambdaf{N}{\app{N}{\lambdaf{x}{\drs{}{\textit{Symbool}(x)}}}}$$
  \item Een overgankelijk werkwoord (\texttt{tv}) is gelijkaardig maar moet twee entiteiten uit hun box halen. Het eerste argument is het lijdend voorwerp, het tweede het onderwerp. We halen eerst het onderwerp uit haar box. Op die manier zullen de quantoren van het onderwerp voor die van het lijdend voorwerp komen. Hierdoor leggen we dus vast hoe quantifier scope ambiguïteiten opgelost worden. In deze semantiek zullen de quantoren namelijk in dezelfde volgorde staan als ze voorkomen in de zin.\footnote{Blackburn en Bos bespreken in hun boek hoe men binnen dit framework de andere lezingen van een quantifier scope ambiguïteit kan verkrijgen} $$\sem{tv} = \lambdaf{N1}{\lambdaf{N2}{\appB{N2}{\lambdaf{x2}{\appB{N1}{\lambdaf{x1}{\drs{}{\textit{Symbool}(x2, x1)}}}}}}}$$
\end{itemize}

Merk op dat al deze lambda-expressies voldoen aan de signaturen van \autoref{tbl:signaturen}

\section{Een voorbeeld}
In deze sectie illustreren we het framework aan de hand van de zin ``If every man sleeps, a woman loves John''. De parse tree die bij deze zin hoort is

\Tree[.s if [.s [.np [.det every ] [.n man ]] [.vp [.iv sleeps ]]] [.s [.np [.det a ] [.n woman ]] [.vp [.tv loves ] [.np [.pn john ]]]]]

Frege's compositionality principe leert ons dat we elke knoop apart mogen behandelen om daarna de resultaten te combineren. In de rest van deze sectie, doorlopen we alle knopen in de boom. We passen de formules die we hierboven hebben afgeleid toe en vereenvoudigen vervolgens met behulp van beta-reductie uit de lambda-calcalus.

\subsection{Every man sleeps}
% \subsection{Every man}
% \Tree[.np:$\app{\sem{det}}{\sem{n}}$ [.det:$\lambdaf{R}{\lambdaf{S}{\drs{}{\drsImpl{\drsMerge{\drs{x}{}}{\app{R}{x}}}{\app{S}{x}}}}}$ every ] [.n:$\lambdaf{x}{\drs{}{man(x)}}$ man ]]
% \Tree[.np [.det every ] [.n man ]]

% \begin{equation*}
  \begin{align*}
    \sem{np_{every\ man}} &= \app{\sem{det}}{\sem{n}} \\
             &= \appB{\lambdaf{R}{\lambdaf{S}{\drs{}{\drsImpl{\drsMerge{\drs{x}{}}{\app{R}{x}}}{\app{S}{x}}}}}}{\lambdaf{x}{\drs{}{man(x)}}} \\
             &= \lambdaf{S}{\drs{}{\drsImpl{\drsMerge{\drs{x}{}}{\app{\left( \lambdaf{x}{\drs{}{man(x)}} \right)}{x}}}{\app{S}{x}}}} \\
             &= \lambdaf{S}{\drs{}{\drsImpl{\drsMerge{\drs{x}{}}{\drs{}{man(x)}}}{\app{S}{x}}}} \\
             &= \lambdaf{S}{\drs{}{\drsImpl{\drs{x}{man(x)}}{\app{S}{x}}}} \\
  \label{eq:np-every-man}
  \end{align*}
% \end{equation*}

% \subsection{Every man sleeps}
% \Tree[.s:$\app{\sem{vp}}{\sem{np}}$ [.np_{every\ man}:$\lambdaf{S}{\drs{}{\drsImpl{\drs{x}{man(x)}}{\app{S}{x}}}}$ ] [.vp [.iv:$\lambdaf{N}{\app{N}{\lambdaf{x}{\drs{}{sleeps(x)}}}}$ sleeps ]]]
% \Tree[.s [.np_{every\ man} ] [.vp [.iv sleeps ]]]

% \begin{equation*}
  \begin{align*}
    \sem{s_{every\ man\ sleeps}} &= \app{\sem{vp}}{\sem{np}} \\
                            &= \appB{\left( \lambdaf{N}{\app{N}{\lambdaf{x}{\drs{}{sleeps(x)}}}} \right)}{\lambdaf{S}{\drs{}{\drsImpl{\drs{x}{man(x)}}{\app{S}{x}}}}} \\
                            &= \app{\left( \lambdaf{S}{\drs{}{\drsImpl{\drs{x}{man(x)}}{\app{S}{x}}}} \right)}{\lambdaf{x}{\drs{}{sleeps(x)}}} \\
                            &= \drs{}{\drsImpl{\drs{x}{man(x)}}{\app{\left( \lambdaf{x}{\drs{}{sleeps(x)}} \right)}{x}}} \\
                            &= \drs{}{\drsImpl{\drs{x}{man(x)}}{\drs{}{sleeps(x)}}} \\
  \label{eq:s-every-man-sleeps}
  \end{align*}
% \end{equation*}

\subsection{A woman loves John}
% \subsection{A woman}
% \Tree[.np:$\app{\sem{det}}{\sem{n}}$ [.det:$\lambdaf{R}{\lambdaf{S}{\left( \drs{x}{} \oplus \app{R}{x} \oplus \app{S}{x} \right)}}$ a ] [.n:$\lambdaf{x}{\drs{}{woman(x)}}$ woman ]]
% \Tree[.np [.det a ] [.n woman ]]

% \begin{equation*}
  \begin{align*}
    \sem{np_{a\ woman}} &= \app{\sem{det}}{\sem{n}} \\
             &= \appB{\lambdaf{R}{\lambdaf{S}{\left( \drs{x}{} \oplus \app{R}{x} \oplus \app{S}{x} \right)}}}{\lambdaf{x}{\drs{}{woman(x)}}} \\
             &= \lambdaf{S}{\left( \drs{x}{} \oplus \drs{}{woman(x)} \oplus \app{S}{x} \right)} \\
             &= \lambdaf{S}{\drsMerge{\drs{x}{woman(x)}}{\app{S}{x}}} \\
  \label{eq:np-a-woman}
  \end{align*}
% \end{equation*}

% \subsection{loves John}
% \Tree[.vp:$\app{\sem{tv}}{\sem{np}}$ [.tv:$\lambdaf{N1}{\lambdaf{N2}{\appB{N2}{\lambdaf{x2}{\appB{N1}{\lambdaf{x1}{\drs{}{loves(x2, x1)}}}}}}}$ loves ] [.np:$\sem{pn}$ [.pn:$\lambdaf{P}{\app{P}{john}}$ john ]]]
% \Tree[.vp [.tv loves ] [.np [.pn john ]]]

% \begin{equation*}
  \begin{align*}
    \sem{vp_{loves\ john}} &= \app{\sem{tv}}{\sem{np}} \\
                        &= \appB{\lambdaf{N1}{\lambdaf{N2}{\appB{N2}{\lambdaf{x2}{\appB{N1}{\lambdaf{x1}{\drs{}{loves(x2, x1)}}}}}}}}{\lambdaf{P}{\app{P}{john}}} \\
                        &= \lambdaf{N2}{\appB{N2}{\lambdaf{x2}{\appB{\left (\lambdaf{P}{\app{P}{john}\right)}}{\lambdaf{x1}{\drs{}{loves(x2, x1)}}}}}} \\
                        &= \lambdaf{N2}{\appB{N2}{\lambdaf{x2}{\app{\left( \lambdaf{x1}{\drs{}{loves(x2, x1)}} \right)}{john}}}} \\
                        &= \lambdaf{N2}{\appB{N2}{\lambdaf{x2}{\drs{}{loves(x2, john)}}}} \\
  \label{eq:vp-loves-john}
  \end{align*}
% \end{equation*}

Niet alleen heeft deze lambda-expressie een signatuur die gelijk is aan die van een onovergankelijk werkwoord, de structuur lijkt er ook sterk op.

% \subsection{A woman loves John}
% \Tree[.s:$\app{\sem{vp}}{\sem{np}}$ [.np_{a\ woman}:$\lambdaf{S}{\drsMerge{\drs{x}{woman(x)}}{\app{S}{x}}}$ ] [.vp_{loves\ john}:$\lambdaf{N2}{\appB{N2}{\lambdaf{x2}{\drs{}{loves(x2, john)}}}}$ ]]
% \Tree[.s [.np_{a\ woman} ] [.vp_{loves\ john} ]]

% \begin{equation*}
  \begin{align*}
    \sem{s_{a\ woman\ loves\ john}} &= \app{\sem{vp}}{\sem{np}} \\
                              &= \appB{\left( \lambdaf{N2}{\appB{N2}{\lambdaf{x2}{\drs{}{loves(x2, john)}}}} \right)}{\lambdaf{S}{\drsMerge{\drs{x}{woman(x)}}{\app{S}{x}}}} \\
                              &= \appB{\lambdaf{S}{\drsMerge{\drs{x}{woman(x)}}{\app{S}{x}}}}{\lambdaf{x2}{\drs{}{loves(x2, john)}}} \\
                              &= \drsMerge{\drs{x}{woman(x)}}{\app{\left( \lambdaf{x2}{\drs{}{loves(x2, john)}} \right)}{x}} \\
                              &= \drsMerge{\drs{x}{woman(x)}}{\drs{}{loves(x, john)}} \\
                              &= \drs{x}{woman(x) \\ loves(x, john)} \\
  \label{eq:s-a-woman-loves-john}
  \end{align*}
% \end{equation*}

\subsection{If every man sleeps, a woman loves John}
% \Tree[.s:$\drs{}{\sem{s1} \Rightarrow \sem{s2}}$ if [.s_{every\ man\ sleeps} ] [.s_{a\ woman\ loves\ john} ]]
% Met $\sem{s1} = \drs{}{\drsImpl{\drs{x}{man(x)}}{\drs{}{sleeps(x)}}}$ en $\sem{s2} = \drs{x}{woman(x) \\ loves(x, john)}$ 
% \Tree[.s if [.s_{every\ man\ sleeps} ] [.s_{a\ woman\ loves\ john} ]]

% \begin{equation*}
  \begin{align*}
    \sem{s} &= \drs{}{\sem{s1} \Rightarrow \sem{s2}} \\
            % &= \drs{}{\drsImpl{\drs{}{\drsImpl{\drs{x}{man(x)}}{\drs{}{sleeps(x)}}}}{\drs{x}{woman(x) \\ loves(x, john)}}} \\
            &= \drs{}{\drsImpl{\drs{}{\drsImpl{\drs{x}{man(x)}}{\drs{}{sleeps(x)}}}}{\drs{y}{woman(y) \\ loves(y, john)}}} \\
            &= \bigg( \forall x \cdot man(x) \Rightarrow sleeps(x) \bigg) \Rightarrow \bigg( \exists y \cdot woman(y) \land loves(y, john) \bigg)
  \label{eq:s-if-a-woman-loves-john-every-man-sleeps}
  \end{align*}
% \end{equation*}

Dit is de vertaling zoals we die zouden verwachten.

\section{Evaluatie}
Het framework dat Blackburn en Bos voorstellen is uitermate geschikt voor semantische analyse van natuurlijke taal. De vier onderdelen staat vrij los van elkaar. Zo kan men de doeltaal vrij kijzen. Blackburn en Bos vertalen in hun eerste boek \cite{Blackburn2005} naar eerste-orde-logica. In hun tweede boek \cite{Blackburn2006} vertalen ze naar DRS-structuren. Ook de vorm van de grammatica is vrij te kiezen. Zowel Blackburn en Bos als deze thesis gebruiken DCG's om de grammatica te specifiëren. Baral et al. \cite{Baral2008} gebruiken een gelijkaardig framework maar met behulp van een Combinatorische Categorische Grammatica.

\paragraph{} Dankzij dit framework, wordt het probleem van semantische analyse herleidt tot het opstellen van een lexicon dat de woorden van de logigram bevat; het opstellen van een grammatica dat de zinsstructuren van een logigram omvat; het opstellen van de semantiek van deze grammaticale regels; en ten slotte het opstellen van semantische macro's voor alle lexicale categorieën die we introduceren.
