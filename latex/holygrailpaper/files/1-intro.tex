Inspired by the 2019 Holy Grail Challenge \cite{}, we present \ourtool,  an end-to-end solution for solving logic grid puzzles (also known as Zebra puzzles) and explaining in human-understandable terms how this solution can be obtained from the clues. 

\ourtool starts from a plain English language representation of the clues, optionally augmented with a list of all the entities present in the puzzle. It translates, using extensions of POSTHINGIE cite{pos} and the framework of BB \cite{bos}, all of these clues to the \idp language \mycite{idp}, an extension of first-order logic. 
It then uses this formal representation of the clues both to solve the puzzle and to explain the solution. 

There are many different ways in which such a system could explain itself. For instance, after finding a solution, it can explain \begin{inparaenum}\item \emph{why that is a solution} or \item \emph{why there are no other solutions}; additionally, it can explain \item \emph{how the system found this solution}, and \item \emph{how a human could find this solution}. \end{inparaenum}
 
Our system implements this last type of explanation. Compared to the third approach, it focuses on simplifying the explanation itself over giving insights in the actual search algorithm used by \idp. As such, our explanations are not to be used for understanding the inner workings of the solver, but rather are to be used by people interested in solving logic puzzles, either for explaining how to obtain an entire solution, or for getting help when they are stuck during the solving process. Indeed, our explanation method will, given a partial solution, find the easiest next derivation to make. 
 
% \todo{A bit more on the working}
% 
% \todo{Possibly something on the tools under the hood? \mycite{idp} \cite{bos} \cite{postagger} ... Can also come later. Maybe better in the next section. }

\todo{A paragraph on the importance of such explanations/the broader use}

\todo{Einstein puzzle as a running example? Adapt all our sentences used to that one maybe. Or a different one since the zebra puzzle is quite simple (quite simple clues)}

\todo{REF to master thesis}
