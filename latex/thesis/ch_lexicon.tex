\chapter{Een lexicon voor logigrammen}
\label{ch:lexicon}
In dit hoofdstuk bespreken we de lexicale categorieën die gebruikt kunnen worden voor het vertalen van logigrammen naar Discourse Representation Structures. We bespreken zowel de categorieën zelf alsook hun vertaling. Er wordt een onderscheid gemaakt tussen open en gesloten lexicale categorieën. De open categorieën zijn open voor uitbreiding. De woorden uit die categorieën zijn verschillend per logigram. De gesloten categorieën bevatten woorden die gemeenschappelijk zijn voor alle logigrammen. Tabel~\ref{tbl:lexiconCategories} geeft een overzicht van de gebruikte lexicale categorieën.

Sommige van deze categorieën zijn gebaseerd op lexicale categorieën uit de code van Blackburn en Bos \cite{Blackburn2006}. Voor deze categorieën werd de semantiek (grotendeels) overgenomen van Blackburn en Bos. We voegen echter ook een aantal nieuwe categorieën toe en passen de semantiek van sommige categorieën aan.

\begin{table}[t]
  \centering
  \begin{tabular}{llll}
    \toprule
    \textbf{Categorie} & \textbf{Afkorting} & \textbf{Open?} & \textbf{Voorbeeld}  \\ \midrule
    Determinator       & det                & gesloten & a, an, the \\
    Hoofdtelwoord      & number             & gesloten & three, 5      \\
    Eigennaam          & pn                 & open     & John, ``the black darts'' \\
    Substantief        & noun               & open     & man, year, \\
    Voorzetsel         & prep               & gesloten & in, to \\
    Betrekkelijk voornaamwoord & relpro     & gesloten & who, which, that \\
    Transitief werkwoord & tv               & open     & loves, ``had a final score of'' \\
    Hulpwerkwoord      & av                 & gesloten & does, ``doesn't'' \\
    Koppelwerkwoord    & cop                & gesloten & is, ``is not'' \\
    Comparatief        & comp               & open     & below, ``older than'' \\
    Onbepaalde woorden & some               & open     & somewhat, sometime \\
    Voegwoord          & coord              & gesloten & and, or, ``neither ... nor ...'' \\
    \bottomrule
  \end{tabular}
  \caption{Een overzicht van de lexicale categorieën}
  \label{tbl:lexiconCategories}
\end{table}

\section{Determinator}
Een determinator kan zowel een lidwoord als een kwantor zijn. In het geval van logigrammen volstaan de lidwoorden ``a'', ``an'' en ``the''. Deze drie determinatoren krijgen alle drie de vertaling van de existentiële determinator uit het vorige hoofdstuk (met $R$ de \textit{restriction} afkomstig van het zelfstandig naamwoord en $S$ de \textit{nuclear scope} of de eigenschap waaraan de naamwoordgroep moet voldoen)$$\sem{det} = \sem{det_{existentieel}} = \lambdaf{R}{\lambdaf{S}{\left( \drs{x}{} \oplus \app{R}{x} \oplus \app{S}{x} \right)}}$$ Er is dus geen nood aan een universele of negatieve determinator voor de logigrammen die wij bestudeerden. Bij logigrammen zijn we namelijk op zoek naar de waarde van bijecties. Er is dus altijd exact één iemand die een bepaalde drank drinkt of een bepaald huisdier heeft. Er is nooit sprake van ``Every man who drinks vodka, ...'' maar altijd van ``The man who drinks vodka, ...''. Ook de negatieve determinator (bv. ``No man drinks vodka'') wordt nooit gebruikt.

\section{Hoofdtelwoord}
\label{sec:lex-number}
Hoofdtelwoorden zijn determinatoren die een aantal uitdrukken. In deze thesis krijgen hoofdtelwoorden echter een eigen lexicale categorie omdat op sommige plaatsen enkel een hoofdtelwoord past en geen andere determinatoren (bijvoorbeeld ``10'' in de zin ``John is 10 years older than Mary''). De hoofdtelwoorden mogen in cijfers voorkomen maar ook in woorden\footnote{In de praktijk zitten enkel de eerste 15 getallen in woorden in het lexicon}.

De signatuur van een hoofdtelwoord is gelijk aan die van een determinator. Er zijn twee mogelijke lezingen voor een hoofdtelwoord: de collectieve en de distributieve lezing. We verduidelijken aan de hand van de voorbeeldzin ``Twee mannen gaan naar zee''. In de collectieve lezing vormen de ``twee mannen'' één geheel. Ze gaan dus samen naar zee. In de distributieve lezing zijn er twee mannen die elk naar zee gaan. In logigrammen gebruiken we enkel de collectieve lezing\footnote{De distributieve lezing wordt heel af en toe gebruikt maar wordt niet ondersteund door onze grammatica}. Wat ons vooral interesseert is het aantal. Meestal gaat het immers om een numerieke eigenschap van iets of iemand. Bijvoorbeeld ``John is 10 years old'' of ``John is 3 years younger than Mary''.

De formule lijkt heel sterk op die van een determinator maar i.p.v. een existentieel gebonden $x$ wordt nu een getal gebruikt.

$$\sem{number} = \lambdaf{R}{\lambdaf{S}{\drsMerge{\app{R}{Number}}{\app{S}{Number}}}}$$

\section{Eigennaam}
\label{sec:lex-pn}
Een eigennaam is een open lexicale categorie. Dat wil zeggen dat de eigennamen verschillend zijn per logigram. De semantiek is identiek aan die in het vorige hoofdstuk. $$\sem{pn} = \lambdaf{E}{\app{E}{\textit{Symbool}}}$$ Een eigennaam voldoet aan een eigenschap $E$ als de constante die geïntroduceerd wordt door de eigennaam, de functiewaarde \textit{waar} heeft voor de functie $E$.

We staan toe dat woordgroepen die taalkundig geen eigennaam zijn, toch gebruikt kunnen worden als een eigennaam. Zo kan ``the black darts'' (wat normaal een determinator + adjectief + substantief is) aanzien worden als een eigennaam. Dit maakt het vertalen van de zinnen makkelijker maar tegelijkertijd wordt het opstellen van het lexicon voor een logigram moeilijker. Het lexicon is niet meer enkel afhankelijk van taalkundige informatie. Alle mogelijke domeinelementen van een (niet-numeriek) domein moeten namelijk als een eigennaam aangegeven worden in het lexicon dat hoort bij het logigram. Bovendien moet dit gebeuren in de vorm zoals het voorkomt in de zinnen van het logigram. Zo is ``black'' een waarde van het concept kleur, toch moet ``the black darts'' ingegeven worden in het lexicon. Deze eigennamen zullen later vertaald worden naar een constante uit een constructed type (zie ook hoofdstuk \ref{ch:specificatie}). Daarmee is het lexicon dus een mengeling van een taalkundig en een formeel vocabularium.

\paragraph{}Een logigram kan 3 soorten eigennamen hebben: een eigennaam in het enkelvoud (bv. ``John''), een eigennaam in het meervoud (bv. ``The Turkey Rolls'') en een \textit{numerieke eigennaam} (bv. ``March''). Bij de eerste twee wordt het symbool afgeleid van de woordvorm. Bij de laatste gebeurt dit door de gebruiker. Numerieke eigennamen worden namelijk gebruikt om woorden om te zetten in getallen. Zo kan ``March'' omgezet worden in 3. Op die manier wordt het zinvol om te spreken over ``1 maand na maart''. Voor een \textit{numerieke eigennaam} is het symbool gelijk aan die numerieke waarde. Deze wordt apart meegegeven in het probleem-specifiek lexicon. De vertaling van woorden naar getallen moet door de gebruiker gebeuren omdat hier achtergrondkennis voor nodig is. Het is opnieuw een voorbeeld van hoe onze vertaling van een logigram deels in het lexicon zit.
%De numerieke eigennamen zullen geen aanleiding geven tot constanten in constructed types. Het is wel een andere voorbeeld van hoe de vertaling van het logigram in het lexicon kruipt.

\section{Substantief}
Ook substantieven zijn een open categorie. Hun semantiek nemen we voorlopig over van het vorige hoofdstuk. $$\sem{noun} = \lambdaf{x}{\drs{}{\textit{Symbool}(x)}}$$ In hoofdstuk~\ref{ch:types} (over types) zullen we DRS uitbreiden met types en zal het predicaat op x verdwijnen en vervangen worden door een echte type-constraint in een DRS. Een substantief in het logigram-specifiek lexicon bevat een enkelvoudsvorm en een meervoudsvorm. Het symbool is gelijk aan de enkelvoudsvorm. Op die manier hebben de enkelvoudsvorm en de meervoudsvorm dezelfde waarde voor de feature \texttt{Symbool}.

\section{Voorzetsel}
De voorzetsels (in het Engels \texttt{prepositions} of \texttt{prep}) vormen een gesloten woordklasse die bestaat uit woorden zoals ``from'', ``in'' en ``with''. Ze worden op twee manieren gebruikt in de zinnen van een logigram. Enerzijds kan het voorzetsel gebruikt worden bij een werkwoord. Dan staat het voorzetsel vlak voor het lijdend voorwerp. Bijvoorbeeld de ``with'' in ``to finish with 500 points''. Anderzijds is een voorzetsel het belangrijkste woord in een voorzetselconstituent (ook wel \texttt{prepositional phrase} of \texttt{pp} genoemd). Bijvoorbeeld de ``from'' in ``the man from France''. Het voorzetsel dat bij een werkwoord hoort, zien we als deel van het werkwoord. In dat geval heeft het voorzetsel geen vertaling.

In het geval van een voorzetselconstituent is er wel een vertaling. We kunnen zo'n voorzetselconstituent zien als een extra beperking op een substantief. Of een transformatie van een substantief naar een nieuw substantief. Met $\tau$ de signatuur van een woordgroep wordt dit $\tau(pp) = \tau(n) \rightarrow \tau(n)$. Een voorzetsel is dan weer een functie van een naamwoordgroep (\texttt{noun phrase} of \texttt{np}) naar een voorzetselconstituent. Of in formulevorm $\tau(prep) = \tau(np) \rightarrow \tau(pp) = \tau(np) \rightarrow \tau(n) \rightarrow \tau(n) = [(e \rightarrow t) \rightarrow t] \rightarrow (e \rightarrow t) \rightarrow (e \rightarrow t)$. De betekenis ziet er uit als
$$\sem{prep} = \lambdaf{N}{\lambdaf{S}{\left( \lambdaf{y}{\drsMerge{\app{S}{y}}{\app{N}{\lambdaf{x}{\drs{}{\textit{Symbool}(y, x)}}}}}} \right)}$$

Een voorzetsel neemt een naamwoordgroep $N$ en een substantief $S$ als argument en geeft een beperking op een entiteit $y$ terug. De beperking op $y$ bestaat enerzijds uit de beperking van het substantief, namelijk $\app{S}{y}$. Anderzijds voegt het voorzetsel zelf ook nog een beperking toe. De naamwoordgroep $N$ moet namelijk voldoen aan de eigenschap $\lambdaf{x}{\drs{}{\textit{Symbool}(y, x)}}$, bijvoorbeeld voor ``with'' $\lambdaf{x}{\drs{}{with(y, x)}}$.

\paragraph{} Daardoor zal een voorzetselconstituent zoals ``with the black darts'' de betekenis $$\sem{pp} = \lambdaf{S}{\left( \lambdaf{y}{\drsMerge{\app{S}{y}}{\drs{}{with(y, TheBlackDarts)}}} \right)}$$ krijgen

\section{Betrekkelijk voornaamwoord}
Een betrekkelijk voornaamwoord (\texttt{relative pronoun} of \texttt{relpro}) is een woord aan het begin van een betrekkelijke bijzin (ook wel \texttt{relative clause} of \texttt{rc}). Voorbeelden zijn ``that'', ``which'' en ``who''. Net als een voorzetselconstituent staat zo'n betrekkelijke bijzin bij een substantief en legt ze een extra beperking op.

$$\sem{relpro} = \lambdaf{V}{\lambdaf{S}{\left( \lambdaf{x}{\drsMerge{\app{S}{x}}{\app{V}{\lambdaf{E}{\app{E}{x}}}}} \right)}}$$
Een betrekkelijk voornaamwoord neemt een verbale constituent $V$ en een substantief $S$ als argument en geeft een beperking op een entiteit $x$ terug. Die beperking op $x$ bestaat uit een conjunctie ($\oplus$-operator) van de beperking van het substantief $S$ (namelijk $\app{S}{x}$) en van een beperking van de verbale constituent. De entiteit $x$ moet namelijk ook voldoen aan de eigenschap $E$ van de verbale constituent $V$.

De $x$ is gebonden door de lambda-functie en dus binnen de lambda-expressie eenduidig bepaald. Men kan $\lambdaf{E}{\app{E}{x}}$ daarom ook zien als een soort van eigennaam $x$ die het onderwerp vormt van de verbale constituent $V$.

\section{Transitief werkwoord}
Een transitief (of overgankelijk) werkwoord is een werkwoord met een lijdend voorwerp. Een logigram heeft enkel transitieve werkwoorden. Er zijn geen intransitieve (zonder lijdend voorwerp) of ditransitieve werkwoorden (met een meewerkend voorwerp). De vertaling van een transitief werkwoord is gelijk aan die van het vorige hoofdstuk $$\sem{tv} = \lambdaf{L}{\lambdaf{O}{\app{O}{\lambdaf{x_o}{\app{L}{\lambdaf{x_l}{\drs{}{\textit{Symbool}(x_o, x_l)}}}}}}}$$

Een entiteit $x_o$ omschreven door het onderwerp $O$ voldoet aan de verbale constituent als voor die $x_o$ het lijdend voorwerp $L$ voldoet aan de eigenschap $\lambdaf{x_l}{\drs{}{\textit{Symbool}(x_o, x_l)}}$. We passen dit toe op ``A man loves every woman''. Een man $x_o$ voldoet aan de verbale constituent (``loves every woman'') als voor die man $x_o$ en elke vrouw $x_l$ geldt dat $\drs{}{\textit{loves}(x_o, x_l)}$ waar is. Het onderwerp (als geheel) voldoet aan de verbale constituent als er zo'n man $x_o$ bestaat.

\paragraph{}Een transitief werkwoord is een open lexicale categorie. Dat wil zeggen dat de woorden verschillend zijn per logigram. In het logigram-specifiek lexicon staat de infinitief, de werkwoordsvorm in de derde persoon enkelvoud alsook een voltooid of onvoltooid deelwoord. De enkelvoudsvorm kan zowel in de verleden tijd als de tegenwoordige tijd zijn, afhankelijk van hoe het werkwoord gebruikt wordt in het logigram. Ten slotte wordt ook nog het voorzetsel gegeven dat voor het lijdend voorwerp wordt gezet (indien van toepassing) en eventueel een achtervoegsel aan het einde van de zin (bv. ``to print'' in ``The design took 8 minutes to print''). Het symbool van het werkwoord (en dus ook de naam van het predicaat) wordt afgeleid uit de enkelvoudsvorm, het voorzetsel en het achtervoegsel.

\paragraph{} Net zoals met de eigennamen wordt er vrij los omgesprongen met de werkwoorden. Zo is ``to be recognized as endangered in'' een werkwoord in één van de logigrammen.

\section{Hulpwerkwoord}
De woordklasse van hulpwerkwoorden is een gesloten woordklasse. Binnen de logigrammen zijn het de werkwoordsvormen van ``to do'' en ``to be'' die deel uitmaken van deze klasse. Alsook het hulpwerkwoord voor de toekomst ``will''. Naast de woordvorm bevat het lexicon ook informatie over de polariteit van die woordvorm: positief of negatief. ``does'' is positief, ``doesn't'' is negatief. De beide polariteiten hebben elk een andere betekenis.

Een hulpwerkwoord is een woord dat een verbale constituent (\texttt{verb phrase} of \texttt{vp}) omvormt tot een nieuwe verbale constituent. Voor een hulpwerkwoord met positieve polariteit is dit de identieke transformatie\footnote{De tijd maakt niet uit voor een logigram. ``John will clean the house'' moet dus niet anders vertaald worden dan ``John cleans the house''}. Dit is bijvoorbeeld het geval voor ``is'' in ``John is cleaning the house.''. $$\sem{av_{pos}} = \lambdaf{V}{V}$$ Voor een hulpwerkwoord met een negatieve polariteit bestaat de transformatie uit een negatie van de verbale constituent. Er is dus geen negatie van het onderwerp. Hiermee wordt de vertaling van ``Everyone doesn't work'' naar logica $\forall x \cdot \lnot work(x)$ i.p.v. $\lnot \forall x \cdot work(x)$

$$\sem{av_{neg}} = \lambdaf{V}{\lambdaf{O}{\app{O}{\lambdaf{x_o}{\app{V}{\lambdaf{E}{\drs{}{\lnot \app{E}{x_o}}}}}}}}$$

Het hulpwerkwoord krijgt een verbale constituent ($V$) en een onderwerp ($O$) als argument. Een entiteit $x_o$ omschreven door het onderwerp voldoet aan de verbale constituent inclusief hulpwerkwoord als het niet voldoet aan de eigenschap $E$ van de verbale constituent $V$ die deel uitmaakt van de gehele verbale constituent.

\section{Koppelwerkwoord}
\label{sec:lex-koppelwerkwoord}
De categorie van koppelwerkwoorden is een gesloten lexicale categorie. Ze bestaat uit verschillende vormen van het werkwoord ``to be'' (bv. is, isn't, is not, was, are, were, ...). Er is een enkelvoud- en meervoudsvorm. Bovendien is er sprake van een positieve of negatieve polariteit. Deze hebben elk een licht andere semantiek. Ten slotte kan een koppelwerkwoord ook op drie verschillende manieren gebruikt worden in een zin van een logigram:

\begin{itemize}
  \item Samen met een \texttt{nominale constituent} (\texttt{noun phrase} of \texttt{np}): Bijvoorbeeld ``John is a man''. Dit type van gebruik heeft dezelfde signatuur als een overgankelijk werkwoord. De semantiek zegt dat de een entiteit $x_o$ van het onderwerp $O$ voldoet als het lijdend voorwerp $L$ voldoet aan $\lambdaf{x_l}{\drs{}{x_o = x_l}}$. Voor ``John is a man'' wilt dit zeggen dat er een man $x_l$ moet zijn die voldoet aan de DRS-conditie $x_l = John$.
  Binnen logigrammen zijn de naamwoordgroep van het onderwerp en het lijdend voorwerp altijd volledig gespecificeerd. De semantiek wil dan zeggen dat deze twee entiteiten gelijk zijn aan elkaar.
  $$\sem{cop_{np, pos}} = \lambdaf{L}{\lambdaf{O}{\app{O}{\lambdaf{x_o}{\app{L}{\lambdaf{x_l}{\drs{}{x_o = x_l}}}}}}}$$
  In de negatieve vorm, is er ook een negatie van het lijdend voorwerp. Zodanig dat er een correcte negatie van de kwantoren is. In de zin ``Mary is not a man'' is het belangrijk dat er geen enkele man is die gelijk is aan Mary. De negatie moet dus voor de existentiële kwantor komen en dus voor het lijdend voorwerp.
  $$\sem{cop_{np, neg}} = \lambdaf{L}{\lambdaf{O}{\app{O}{\lambdaf{x_o}{\drsNot{\app{L}{\lambdaf{x_l}{\drs{}{x_o = x_l}}}}}}}}$$
  \item Samen met een \texttt{adjectiefconstituent} (\texttt{adjective phrase} of \texttt{ap}): Bijvoorbeeld ``John is 30 years old''. De vertaling die wij gebruiken ligt echter vrij ver van de taalkundige structuur. Zo wordt het koppelwerkwoord + adjectief gezien als een soort van transitief werkwoord. De betekenis is gelijk aan die van het transitieve werkwoord en voor de negatie gelijkaardig aan die van het koppelwerkwoord met een nominale constituent. De adjectieven die gebruikt kunnen worden, moeten meegegeven worden via het lexicon van een logigram. Het symbool komt overeen met de woordvorm van het adjectief.
  $$\sem{cop_{ap, pos}} = \sem{tv} = \lambdaf{L}{\lambdaf{O}{\app{x_o}{\lambdaf{x_o}{\app{L}{\lambdaf{x_l}{\drs{}{\textit{Symbool}(x_o, x_l)}}}}}}}$$
  $$\sem{cop_{ap, neg}} = \lambdaf{L}{\lambdaf{O}{\app{O}{\lambdaf{x_o}{\drsNot{\app{L}{\lambdaf{x_l}{\drs{}{\textit{Symbool}(x_o, x_l)}}}}}}}}$$
  \item Samen met een \texttt{voorzetselconstituent} (\texttt{prepositional phrase} of \texttt{pp}): Bijvoorbeeld ``John is from France''. Dit kan aanzien worden als een ellips van ``a person'': ``John is a person from France''. De signatuur is $ \tau(cop_{pp}) = \tau(pp) \rightarrow \tau(vp) = \tau(pp) \rightarrow \tau(np) \rightarrow t = [(e \rightarrow t) \rightarrow e \rightarrow t] \rightarrow [(e \rightarrow t) \rightarrow t] \rightarrow t$.

  $$\sem{cop_{pp, pos}} = \lambdaf{V}{\lambdaf{O}{\app{O}{\lambdaf{x_o}{\app{\app{V}{\lambdaf{y}{\drs{}{}}}}{x_o}}}}}$$
  Het koppelwerkwoord krijgt een voorzetselconstituent $V$ en het onderwerp $O$ als argument. Een entiteit $x_o$ van het onderwerp voldoet aan de verbale constituent als het voldoet aan de eigenschap van de voorzetselconstituent $V$. Die voorzetselconstituent krijgt als eerste argument nog een een lambda-functie mee die een beperking oplegt vanuit het substantief waarbij het hoort. In dit geval is er echter geen substantief en dus geen beperking. Dit vertaalt zich in een lege DRS-structuur. Dit is equivalent aan $\lambdaf{y}{true}$ in eerste-orde-logica.

  De betekenis van de negatieve vorm is $$\sem{cop_{pp, neg}} = \lambdaf{V}{\lambdaf{O}{\app{O}{\lambdaf{x_o}{\drsNot{\app{\app{V}{\lambdaf{y}{\drs{}{}}}}{x_o}}}}}}$$
\end{itemize}

\section{Comparatief en onbepaalde woorden}
\label{sec:lex-some}
Binnen een logigram is er vaak een domein van numerieke aard. In dat geval is er meestal ook een zin die uitdrukt dat iemand een hogere of lagere numerieke waarde heeft dan iemand anders. Bijvoorbeeld ``John scored 15 points less than Mary'' of ``Mary finished sometime after Tom''. De woordgroepen ``less than'' en ``after'' zijn deel van de lexicale categorie \texttt{comparatief}. ``sometime'' is dan weer een voorbeeld van een onbepaald woord. Wat op het eerste zicht niet opvalt aan deze zinnen is dat ze een belangrijke ellips bevatten. Tom is namelijk geen tijdstip maar een persoon. De tweede zin is voluit ``Mary finished sometime after Tom finished''. Deze ellips zal opgelost worden in de grammatica.

%Een comparatief is een functie die twee naamwoordgroepen als argument neemt en een nieuwe naamwoordgroep maakt die als lijdend voorwerp kan dienen. $\tau(comp) = \tau(np) \rightarrow \tau(np) \rightarrow \tau(np) = [(e \rightarrow t) \rightarrow t] \rightarrow [(e \rightarrow t) \rightarrow t] \rightarrow [(e \rightarrow t) \rightarrow t]$. 
\paragraph{} Er zijn opnieuw 2 betekenissen afhankelijk van het woord: één voor ``hoger'' (bv. ``after'') en één voor ``lager'' (bv. ``less than'')
$$\sem{comp_{lager}} = \lambdaf{H}{\lambdaf{N}{\left( \lambdaf{E}{\app{H}{\lambdaf{x_h}{\app{N}{\lambdaf{x_n}{\drsMerge{\drs{x}{x = x_n - x_h}}{\app{E}{x}}}}}}} \right)}}$$
Een comparatief neemt twee naamwoordgroepen als argument. De eerste is een hoeveelheid $H$ en de tweede een numerieke eigenschap $N$. Het resultaat is een nieuwe naamwoordgroep die kan dienen als lijdend voorwerp. Een naamwoordgroep is van type $(e \rightarrow t) \rightarrow t$ en dus krijgen we nog een argument $E$ van type $e \rightarrow t$ mee. Dit is de eigenschap waaraan de nieuwe naamwoordgroep moet voldoen. De semantiek zegt dan dat de naamwoordgroepen $H$ en $N$ respectievelijk een entiteit $x_h$ en $x_n$ moeten omschrijven zodanig zijn dat er een geheel getal $x = x_n - x_h$ bestaat dat voldoet aan de eigenschap $E$. Zo geldt voor ``John scored 15 points less than Mary scored'' dat $x_h = 15$, $x_n$ is de score van Mary en $x = x_n - 15$ moet voldoen aan de eigenschap $\lambdaf{x}{\drs{}{scored(John, x)}}$.

\paragraph{} De andere betekenis van de comparatief, die voor ``hoger'', is gelijkaardig aan de eerste
$$\sem{comp_{hoger}} = \lambdaf{H}{\lambdaf{N}{\left( \lambdaf{E}{\app{H}{\lambdaf{x_h}{\app{N}{\lambdaf{x_n}{\drsMerge{\drs{x}{x = x_n + x_h}}{\app{E}{x}}}}}}} \right)}}$$

De comparatief is een open lexicale categorie. Elke logigram heeft eigen comparatieven. In het logigram-specifiek lexicon bevindt zich naast de woordvorm ook telkens het type (``hoger'' of ``lager'').

\paragraph{} De onbepaalde woorden hebben dezelfde signatuur als een naamwoordgroep $\tau(some) = \tau(np) = (e \rightarrow t) \rightarrow t$. Een onbepaald woord voldoet aan een eigenschap $E$ als er een natuurlijk getal bestaat dat voldoet aan die eigenschap\footnote{Het getal moet strikt positief zijn want anders kan $x=x2-x1$ ook hoger dan $x2$ uitkomen bij een comparatief met als betekenis ``lager dan''}. Meer concreet hebben we $$\sem{some} = \lambdaf{E}{\drsMerge{\drs{x}{x > 0}}{\app{E}{x}}}$$ 

\section{Voegwoord}
\label{sec:lex-coord}
Een voegwoord is een woord dat twee woordgroepen van dezelfde categorie verbindt. Voorbeelden zijn ``and'', ``or'' and ``nor''. Een voegwoorden kan twee zinnen verbinden maar ook twee nominale constituenten. Binnen logigrammen zijn de voegwoorden nevenschikkend. Dat wil zeggen dat beide woordgroepen even belangrijk zijn\footnote{In tegenstelling tot een onderschikkend voegwoord. Zo'n voegwoord verbindt bijvoorbeeld een hoofdzin met een bijzin. Daarbij is de bijzin minder belangrijk dan de hoofdzin.}. Er komen drie soorten voegwoorden voor, elk met een eigen vertaling: een conjunctief voegwoord (``A and B''), een disjunctief voegwoord (``A or B'') en een negatief voegwoord (``A nor B''). Een voegwoord kan uit twee delen bestaan (bv. ``either ... or'' en ``neither ... nor''). Daarom bestaan er twee lexicale categorieën: \texttt{coord} voor het voegwoord zelf (bv. ``or'') en \texttt{coordPrefix} voor de eventuele prefix (bv. ``either'').

De vertaling van het voegwoord is onafhankelijk van de categorie van de woordgroepen die worden verbonden. Belangrijk is wel dat de signatuur van die woordgroepen een functie met als resultaat iets van type $t$ is ($\tau(woordgroep) = \alpha \rightarrow t$). Een signatuur van het voegwoord is dan $\tau(coord) = (\alpha \rightarrow t) \rightarrow (\alpha \rightarrow t) \rightarrow (\alpha \rightarrow t) = (\alpha \rightarrow t) \rightarrow (\alpha \rightarrow t) \rightarrow \alpha \rightarrow t $. De betekenis van een voegwoord bestaat er telkens uit door het derde argument (van type $\alpha$) door te geven aan de eerste twee argumenten en de resultaten op de juiste manier terug te combineren\footnote{$\app{A}{C}$ en $\app{B}{C}$ zijn van type $t$ en dus DRS-structuren}.

$$\sem{coord_{conjunctief}} = \lambdaf{A}{\lambdaf{B}{\left( \lambdaf{C}{\drsMerge{\app{A}{C}}}{\app{B}{C}} \right)}}$$
$$\sem{coord_{disjunctief}} = \lambdaf{A}{\lambdaf{B}{\left( \lambdaf{C}{\drs{}{\app{A}{C} \lor \app{B}{C}}} \right)}}$$
$$\sem{coord_{negatief}} = \lambdaf{A}{\lambdaf{B}{\left( \lambdaf{C}{\drsMerge{\drs{}{\lnot \app{A}{C}}}}{\drs{}{\lnot \app{B}{C}}} \right)}}$$

Deze vertaling is echter niet de enige mogelijke vertaling. In het geval van een disjunctie kan men ook naar een exclusieve of vertalen (zeker voor de constructie ``either ... or ...'' houdt dit steek). Omwille van de bijecties bij logigrammen maakt dit echter geen verschil. Er is immers altijd maar exact één iemand die eraan voldoet dus er is in praktijk geen verschil tussen een inclusieve en exclusieve disjunctie.

\paragraph{} Echter ook voor de conjunctie is een andere vertaling mogelijk, meer bepaald voor de conjunctie van twee naamwoordgroepen. Zo heeft de zin ``John and Mary went to the sea'' twee mogelijke vertalingen. De bovenstaande vertaling wordt de distributieve lezing genoemd en komt overeen met dat beiden (apart) naar de zee zijn gegaan. In de andere vertaling, de collectieve lezing, zijn John en Mary samen naar de zee gegaan. Binnen logigrammen komt de collectieve lezing bijna niet aan bod. De enige plaats waar dat wel gebeurt, is in een soort van \textit{alldifferent}-constraint. Bijvoorbeeld ``John, Bob and Charles are three different politicians''. Bij de vertaling van de grammatica lossen we dit probleem op.

\section{Conclusie}
We hebben twaalf lexicale categorieën opgesteld voor het vertalen van logigrammen naar logica. Een aantal van deze categorieën zijn specifiek aan logigrammen (bijvoorbeeld de onbepaalde woorden). De meeste zijn echter taalkundig verantwoord.

We springen wel vrij los om met welke woorden behoren tot de lexicale categorieën. Daardoor wordt het lexicon een mengeling van een taalkundig en formeel vocabularium. Zo moeten alle niet-numerieke domeinelementen voorkomen als eigennaam in het lexicon. Bovendien moeten numerieke eigennamen (zoals ``March'') een getal meekrijgen dat gebruikt wordt in de vertaling.

Aangezien de betekenis van een zin hoofdzakelijk in de betekenis van de woorden zit, bevat de semantiek van het lexicon ook een aantal keuzes over welke betekenis gebruikt wordt voor het vertalen van logigrammen naar logica. Zo impliceert onze semantiek van een transitief werkwoord dat de kwantoren van het onderwerp voor die van het lijdend voorwerp moeten komen. Binnen logigrammen volstaat deze betekenis. Bij uitbreiding naar andere domeinen moeten deze keuzes opnieuw geëvalueerd worden om te zien of ze nog altijd van toepassing zijn.
