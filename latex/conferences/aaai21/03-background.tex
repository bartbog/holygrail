% !TeX root = ./main.tex
The overarching goal of this paper is to generate a sequence of small reasoning steps, each with an interpretable explanation, and for that we introduce the necessary background.
\begin{itemize}
    \item Explanation sequence : CSP journal paper ?
    \item concept of OMUS
    % \item Logic grid puzzles (depends on experiments)
    % \item Nested explanation still applicable?
\end{itemize}


Let $\m{F}$ be a boolean propositional formula in Conjunctive Normal Form (CNF). A CNF formula $\m{F}$ is a  conjuction of clauses  (ex: $\m{F} = \{ c_1 \wedge c_2  \wedge c_3 \} $) each having a positive integer or infinite weight $w_i$ $\forall i \in \{1.. |C| \}$ with $|C|$ the number of clauses in $\m{F}$.
Each clause is defined as a disjunction of literals (ex: $C_1 = \{ l_1 \vee l_2 \vee \lnot l_3 \}$).
A literal is defined as an atom or its negation (true or false).
A model is a truth assignment to the variables of $\m{F}$ that satisfies $\m{F}$.
$\m{F}$ is said to be unsatisfiable if there does not exist model of $\m{F}$, i.e. a satisfying truth assignment to the variables of $\m{F}$.

Formally, we define a \textit{Minimum Unsatisfiable Subset} (MUS), as a set of clauses that taken together are unsatisfiable, but if any of the clauses are removed then the reduced formula becomes satisfiable.

\begin{definition}
    A subset $\m{S} \subseteq \m{F}$ is a \textbf{Maximal Satisfiable Subset} (MSS) iff $ \m{S}$ is satisfiable and $\forall \ \m{S}  \subset  \m{S}' $, $\m{S}'$ is unsatisfiable.
\end{definition}

\begin{definition}
    % A correction subset of $\m{F}$ is a subset of $\m{F}$ whose complement is satisfiable.
    A subset $\m{C} \subseteq \m{F}$ is a \textbf{Minimal Correction Subset} (MCS) iff $ \forall \ \m{C}' \subset C$ to $\m{F} \setminus \m{C}'$ is unsatisfiable.
\end{definition}

\noindent It is well-known that an MCS $\m{C}$ is the complement of an MSS $\m{S} = \m{F} \setminus \m{C}$ (and vice versa).

\begin{definition}\label{def:minimal-hs}
    Given a collection of sets $\m{K}$, a minimal hitting set h is defined as a cardinality-minimal set such that $\forall \ C \in \m{K}$, $h \cap C \neq \emptyset$.
\end{definition}

\noindent In fact, for a collection of sets $\m{K}$ there exist multiple possible minimal hitting sets on $\m{K}$. Consequently, we define the \textbf{minimum} hitting set as:

\begin{definition}\label{def:minimum-hs}
    Let $\m{K}$ be a collection of sets and $HS(\m{K})$ be the collection of all minimal hitting sets on $\m{K}$.
    A hitting set $hs$ is a \textbf{minimum} hitting set if $\forall \ h' \in HS(\m{K}): |hs| \leq |h'|$.
  \end{definition}

  \noindent We also know from \cite{liffiton2008algorithms,reiter1987theory} that MCSes and MUSes are linked through minimal hitting sets:

\begin{proposition}\label{prop:MCS-MUS-hittingset}
    Given an $ \m{F}$, let MUSes($\m{F}$), be the Minimal Unsatisfiable Subsets of F and MCSes($\m{F}$), be the Minimal Correction Subsets of F:
    
    A subset  $\m{C} \subset \m{F}$ is an MCS of $ \m{F}$ iff  $\m{C}$ is a \emph{minimal hitting set} of MUSes($ \m{F}$);

    A subset  $\m{U} \subset \m{F}$ is a MUS of $ \m{F}$ iff  $\m{U}$ is a \emph{minimal hitting set} of MCSes($ \m{F}$);
\end{proposition}

The duality of proposition \ref{prop:MCS-MUS-hittingset} is fundamental to explaining how we can compute the SMUS in section \ref{sec:smus}. 
% TODO: add something here ?
In section \ref{sec:omus}, we further exploit this proposition for computing the OMUS for a given unsatisfiable CNF formula $\m{F}$.