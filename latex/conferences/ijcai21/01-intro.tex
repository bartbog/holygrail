% !TeX root = ./main.tex

% Research

% \todo{The intro should be updated to reflect our new contributions: incremental and constrained OMUS}
% Generating minimal explanations in case of an unsatisfiable constraint satisfaction problem (CSP) is well-studied~\cite{junker2001quickxplain}. 

Building on old ideas to explain domain-specific propagations performed by constraint solvers  \cite{sqalli1996inference,freuder2001explanation}, we recently introduced a 
method that takes as input a satisfiable constraint program and explains the solution-finding process in a human-understandable way  \cite{ecai/BogaertsGCG20}. 
Explanations in that work are sequences of simple inference steps, involving as few constraints and facts as possible. 
The explanation-generation algorithms presented in that work rely heavily on calls for  \emph{Minimal Unsatisfiable Subsets} (MUS) \cite{marques2010minimal} of a derived program, exploiting a one-to-one correspondence between so-called \emph{non-redundant explanations} and MUSs.
The explanation steps in the seminal work are heuristically optimized with respect to a given cost function that should approximate human-understandability, e.g., taking the number of constraints and facts into account, as well as a valuation of their complexity (or priority). 
The algorithm developed in that work has two main weaknesses: first, it provides no guarantees on the quality of the produced explanations due to internally relying on the computation of $\subseteq$-minimal unsatisfiable subsets, which are often suboptimal with respect to the given cost function. 
Secondly, it suffers from performance problems: the lack of optimality is partly overcome by calling a MUS algorithm on increasingly larger subsets of constraints for each candidate implied fact. %replacing a potential single call for a cost-optimal unsatisfiable subset by multiple calls, 
However, using multiple MUS calls per literal in each iterations quickly causes efficiency problems, causing the explanation generation process to take several hours.


Motivated by these observations, we develop algorithms that aid explaining CSPs and improve the state-of-the-art in the following ways: 
\begin{itemize}
 \item We develop algorithms that compute (cost-)\textbf{Optimal} Unsatisfiable Subsets (from now on called OUSs) based on the well-known hitting-set duality that is also used for computing cardinality-minimal MUSs \cite{ignatiev2015smallest}.
\item We observe that many of the individual calls for MUSs (or OUSs) can actually be replaced by a single call that searches for an optimal unsatisfiable subset \textbf{among subsets satisfying certain structural constraints}. In other words, we introduce the \emph{Optimal \textbf{Constrained} Unsatisfiable Subsets (OCUS)} problem and we show how $O(n^2)$ calls to MUS/OUS can be replaced by $O(n)$ calls to an OCUS oracle, where $n$ denotes the number of facts to explain. 
% \item Finally, building on 
\item Finally, we develop techniques for \textbf{optimizing} the OCUS algorithms further, exploiting domain-specific information coming from the fact that we are in the  \emph{explanation-generation context}. %One such optimization is the development of methods for \textbf{information re-use} between consecutive OCUS calls.
% 
% 
% \tias{This one no longer separate point?}
\end{itemize}



% 
% unsatisfiable subset optimization (i.e., for computing cost-optimal unsatisfiable subsets, from now on called OUSs). The algorithms we develop combine ideas from 
% \hitsetbased algorithms to compute cardinality-minimal MUSs \cite{ignatiev2015smallest} and from \maxsat solving \cite{DBLP:conf/sat/DaviesB13}.
% Furthermore, we show that techniques to speed up \hitsetbased \maxsat solving~\cite{davies} can be readily applied to speed up \hitsetbased OUS solving.
% %\todo{Also say this?}
% %Finally, we investigate how the ideas of \citet{davies} that were originally developed for \hitsetbased algorithms for solving \maxsat can be generalized to the case of constrained OUS generation. 
% 
% As we require computing the OUS of several highly related problem instances, 
% %Furthermore, given the fact that a single call for generation of an explanation sequence internally employs several calls OUS calls for each explanation step, with all OUS calls using approximately the same theory,
% we investigate how to make the OUS algorithm \emph{incremental}, by reusing the information stored in the sets-to-hit of other OUS calls.
% 
% In our setting, finding an optimal explanation step requires finding the OUS for each possible implied fact. Driven by this need, we show how all these OUS calls can be combined into a single OUS search for an optimal unsatisfiable subset \emph{under constraints}. The constraints here are meta-level constraints, that specify which types of subsets are deemed acceptable. 
% % In the case of our explanations, these are subsets involving exactly one unexplained implied fact.
% We show that minimal modifications to the original \hitsetbased OUS algorithm indeed yields a method to find such \emph{constrained optimal unsatisfiable subsets}.
% This allows us to reduce the explanation generation problem into that of finding a sequence of constrained OUSes.
% % Next, we develop, called \emph{unsatis


%We implemented these ideas in the \call{cppy} framework \cite{cppy} and experimentally validate them on CNF problems from SOMEWHERE, as well as the problems considered in \citet{ecai/BogaertsGCG20}.

%Our experimental results confirm that 
%\begin{itemize}
%  \item Furthermore, significant speed-ups in run-time are obtained for the explanation-producing loop. 
% %  \item Both of our extensions (incremental, and constraint OUS generation) are instrumental in achieving this speed-up. 
% %  \item \todo{ confirm Davies's observations ???}
% %  \item The quality of the explanation sequence improves compared to the algorithm of \citet{ecai/BogaertsGCG20}. 
% %\end{itemize}
% % the validity of incremental, as well as constrained generation of optimal 
% 
% Summarized, the main contributions of our paper are:
% \begin{itemize}
%   \item We generalize the algorithm of \cite{ignatiev2015smallest} to allow for arbitrary  optimization functions, resulting in the first algorithm for OUS generation;
%   %\item We translate \maxsat solving techniques to the constrained OUS setting.
%   \item We develop an \emph{incremental} version of our OUS algorithm by exploiting properties of the hitting set dualisation;
%   \item We generalize the OUS problem to a constrained OUS problem by allowing meta-level constraints on the unsatisfiable subsets; % and show how our algorithm generalizes to take those constraints into account;
%   \item We experimentally validate our algorithms in the context of explanation generation for CSPs. %, thereby showing improvements both in terms of the time required to find explanation sequences and the quality thereof. %of the developed explanation.
% \end{itemize}

% DISCUSSION ON : Add if explanations are used in practice or if they are only a recently introdyuced concept that still needs further development.
%OLD: {Ultimately computing an \mm{\call{OCUS}} also benefits a number of applications relying on finding inconsistencies as explanations to solve the problem by being able to formulate explanation-preferences as constraints and using the optimality criterion.}
\emilio{Recently, MUSs have been used to formally define abduction-based explanations for machine learning classifiers \cite{ignatiev2019abduction}.
In that work, from a given classifier, and a concrete classified instance, a propositional theory explains which set of feature values of the instance are responsible for the predicted class.
We see opportunities to also use OCUS in this setting.
For instance, OCUS allows the formulation of structural constraints on the explanations, and (ii) the quantification of explanation difficulty using the optimality criterion.} \tias{This paragraph not appropriate in an introduction because we don't actually do it. Move to discussion/future work or discuss on Teams}.\bart{Not sure I agree. It speaks to motivation and relevance. Everything you can say that can encourage a reader to think ``this is interesting! I would like to read more'' can go in the intro in my opinion. But... if we do leave it here, a better way to introduce it might be
``In this paper, we only apply our OCUS algorithms to generate \emph{step-wise} explanations. However, MUSs have been used in a variety of contexts, and in particular lie at the foundations of several explanation techniques \cite{quickxplain,ignatiev2019abduction,schotten}. We conjecture that OCUS can also prove useful in those settings, to take more fine-grained control over which MUSs, and eventually, which explanations are produced.}  

\ignore{
% the very many MUS calls have as a consequence that 
\emilio{Rephrase Intro: In the last few years, as AI systems employ more advanced reasoning mechanisms and computation power, it becomes increasingly difficult to understand why certain decisions are made.
Explainable (XAI), a subfield of AI, aims to fulfil the need for trustworthy AI systems to understand \emph{how} and \emph{why} the system made a decision, e.g. for verifying correctness of the system, as well as to control for biased or systematically unfair decisions.

Despite the fact that we do not (specifically) aim to explain over-constrained problems, our algorithms will also internally make use of methods to extract a minimal set of conflicting constraints often called a \emph{\underline{M}inimal \underline{U}nsatisfiable \underline{S}ubset} (MUS) or \emph{Minimal Unsatisfiable Core} \cite{marques2010minimal}.

While explainability of constraint optimisation has received little attention so far, in the related field of \textit{planning}, there is the emerging subfield of \textit{eXplainable AI planning} (XAIP)~\cite{fox2017explainable}, which is concerned with building planning systems that can explain their own behaviour.
This includes answering queries such as ``why did the system (not) make a certain decision?'', ``why is this the best decision?'', etc. In contrast to explainable machine learning research~\cite{guidotti2018survey}, in explainable planning one can make use of the explicit \textit{model-based representation} over which the reasoning happens.
Likewise, we will make use of the constraint specification available to constraint solvers, more specifically typed first-order logic~\cite{atcl/Wittocx13}.

This research fits within the general topic of Explainable Agency~\cite{langley2017explainable}, whereby in order for people to trust autonomous agents, the latter must be able to \textit{explain their decisions} and the \textit{reasoning} that produced their choices.
To provide the constraint solver with Explainable Agency~\cite{langley2017explainable}, we first formalize the problem of step-wise explaining the propagation of a constraint solver through a sequence of small inference steps.
Next, we use an optimistic estimate of a given cost function quantifying human interpretability to guide the search to \textit{simple}, low-cost, explanations thereby making use of minimal unsatisfiable subsets.
We extend this approach using \emph{reasoning by contradiction} to produce additional explanations of still difficult-to-understand inference steps.
Finally, we discuss the challenges and some outlooks to explaining how to solve constraint satisfaction problems.


\paragraph*{Publication history} This workshop paper is an extended abstract of previous papers presented at workshops and conferences \cite{claesuser,DBLP:conf/bnaic/ClaesBCGG19,ecai/BogaertsGCG20} and a journal paper under review \cite{bogaerts2020framework}.
}

\begin{compactenum}
    \item XAI
    \item MUS vs Overconstrained/Infeasibility
    \item CSP
    \item XOPT
\end{compactenum}

Contributions : 
\begin{compactenum}
    \item Efficient, greedy algorithm for explaining CSP based on OMUS 
    \item Adapatation smallest MUS adaptation to OMUS (different hs + no maxsat) 
    \item Improving OMUS algorithm from Davies' related delayed MaxSAT algorithm 
    \item Incremental OMUS computation to speed-up explanations
    \begin{itemize}
        \item hoe belangrijk is de incrementaliteit in het algorithme
        \item kunnen  we nog meer incrementeel verder gaan
    \end{itemize}
    \todo{
    \begin{itemize}
        \item Bestaande SMUS/OMUS algorihtmes incrementeel veranderen
        \begin{itemize}
            \item Bredere studie, SMUS => OMUS veralgemenen
            \item Hoe efficient zijn ze om ons probleem op te lossen
        \end{itemize}
    \end{itemize}
    }
\end{compactenum}
}


The rest of this paper is structured as follows.
We discuss background on the hitting-set duality in \cref{sec:background}. \cref{sec:motviation} motivates our work, while \cref{sec:ocus} introduces the OCUS problem and a generic \hitsetbased algorithm for computing OCUSs. In \cref{sec:ocusEx} we show how to optimize this computation in the context of explanations. 
\cref{sec:experiments} experimentally validates the approach, %, we experimentally evaluate all our contributions on the benchmarks of \citet{ecai/BogaertsGCG20}, both in  terms of performance and in quality of the produced explanation. 
with various branches of related work discussed in \cref{sec:related} and conclusions in \cref{sec:conclusion}.


% We discuss related work in \cref{sec:related} and background on the hitting set duality in \cref{sec:background}. \cref{sec:omus} contains our algorithm for OUS generation, as well as insights as to how the ideas of \cite{davies} can be transferred to the OUS setting. \cref{sec:explain} shows how this can be used in a simplification of the explanation algorithm of \cite{ecai/BogaertsGCG20}. 
% Sections \ref{sec:incremental} and \ref{sec:constrained} contain our  conceptual extensions to the OUS problem and algorithm, namely incrementality and constrainedness. 
% In \cref{sec:davies}\tias{maybe subsection of 4}, we discuss how the . 
%We  experimentally validate our  extensions in \cref{sec:experiments}, and conclude in \cref{sec:conclusion}.


