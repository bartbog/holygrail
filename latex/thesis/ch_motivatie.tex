\chapter{Motivatie}

\paragraph{} Vele bedrijfsprocessen worden geregeld door specificaties. Deze worden vaak geschreven in een natuurlijke taal, door een domein expert. Vervolgens worden deze specificaties vertaald naar uitvoerbare programma's. Bij deze vertaling kunnen er fouten insluipen. Bovendien zijn er vaak meerdere programma's die elk opnieuw de specificatie moeten implementeren. Zo ontstaan er niet alleen inconsistenties met de specificatie, maar ook tussen de verschillende programma's onderling. Ten slotte is het moeilijk om de specificatie achteraf nog aan te passen omdat alle programma's dan aangepast moeten worden.

\paragraph{} Het antwoord van de academische wereld op deze problemen, is het \textit{Knowledge Base}-paradigma. In dit paradigma staat een kennisbank centraal. Deze kennisbank bevat de kennis over de wereld en vormt een specificatie van hoe een systeem zich zou moeten gedragen. Deze kennisbank kan dan gebruikt worden in een \textit{Knowledge Base System (KBS)}. Zo'n systeem biedt een aantal inferenties aan die toegepast kunnen worden op deze kennisbanken. De Cat et al. \cite{IDP} geven het voorbeeld van een KBS dat gebruikt wordt om de vakken van een universiteit te beheren. De kennisbank in zo'n systeem bevat regels als ``In elk lokaal en op elk moment mag er maximum één les plaatsvinden''. Het systeem kan dan verschillende inferenties hierop uitvoeren. De Cat et al. geven de voorbeelden van \textit{propagatie} (bijvoorbeeld in het individuele studieprogramma van een student automatisch de vakken toevoegen die een vereiste zijn om andere vakken te volgen die expliciet door de student werden gekozen), \textit{model expansie} (bijvoorbeeld het opstellen van een volledig studieprogramma op basis van een partieel programma) en \textit{bevraging} (bijvoorbeeld het opvragen van een uurrooster voor een specifieke student).

Op basis van de output van de inferenties kunnen de bedrijfsprocessen dan geregeld worden. Om de kennis voor te stellen, kan men gebruik maken van een formele taal (zoals eerste-orde-logica). Zo'n talen hebben een eenduidige semantiek. Er is dus maar \'e\'en mogelijke manier waarop een zin ge\"interpreteerd kan worden. Dit in tegenstelling tot natuurlijke talen waar zelfs simpele zinnen al snel ambigu zijn.

In het Knowledge Base-paradigma moet de specificatie in natuurlijke taal (geschreven door een domein expert) vertaald worden naar een vorm waar het KBS mee overweg kan: de formele specificatie. Deze specificatie wordt slechts eenmaal geschreven en vervolgens wordt ze gebruikt voor alle soorten inferenties. Daardoor is het niet meer mogelijk dat de programma's onderling inconsistent zijn. Ze gebruiken namelijk allemaal dezelfde formele specificatie. Bovendien is ook het aanpassen van de specificatie makkelijker. Enkel de specificatie moet aangepast worden, de programma's blijven hetzelfde.

\paragraph{} Het probleem met deze aanpak is dat er nog steeds een vertaling moet gebeuren van natuurlijke taal naar een formele taal. De specificatie in natuurlijke taal wordt vaak opgesteld door een domein expert die niet vertrouwd is met formele talen. De formele specificatie wordt dan weer opgesteld door een KBS expert. Deze persoon kent formele talen maar heeft een beperkte kennis van het domein. Door deze mismatch van expertise, sluipen er fouten in de vertaling. De domein expert kan namelijk de subtiliteiten van de formele taal niet lezen. Vice versa kent de KBS expert de subtiliteiten van het domein niet.

\paragraph{} De vraag rijst dus of we een formele taal kunnen ontwerpen die toegankelijk is voor domein experten, rijk genoeg is voor praktische problemen en toepasbaar is binnen het KBS paradigma.

Deze thesis onderzoekt of een formele natuurlijke taal het antwoord is op die vraag. Hieronder verstaan we (een subset van) een natuurlijke taal met een formele, eenduidige semantiek. Talen die een subset zijn van een natuurlijke taal worden ook wel gecontroleerde natuurlijke talen of CNL's (naar het Engelse \textit{Controlled Natural Language}) genoemd. Deze talen verschillen van hun gasttaal doordat ze een aantal zinsconstructies niet toelaten. Dit kan bijvoorbeeld de leesbaarheid van een taal verhogen. Voor dit onderzoek zijn deze talen interessant omdat ambigue constructies op die manier verboden kunnen worden. Bovendien is de grammatica van een CNL veel simpeler dan die van de volledige natuurlijke taal, waardoor het makkelijker is om er een parser voor te schrijven. De zinnen die toegestaan zijn in de CNL worden vaak beschreven in een set van \textit{constructieregels}.

Naast constructieregels bevat een formele natuurlijke taal vaak ook interpretatieregels. Deze laatste bepalen hoe een zin die ambigu is in de gasttaal, ge\"interpreteerd moet worden in de nieuwe taal. De moeilijkheid ligt erin om deze regels te beperken in aantal en in complexiteit, zodanig dat de geconstrueerde taal zo dicht mogelijk tegen de gasttaal aanleunt.

% \paragraph{} We zullen deze formele natuurlijke taal opstellen binnen het domein van logigrammen. Hierbij zullen de constructieregels en interpretatieregels zodanig moeten opgesteld worden dat ze de betekenis van de logigrammen juist omzetten naar een formele representatie.

Het grote voordeel van een formele natuurlijke taal is dat natuurlijke taal al gebruikt wordt bij het opstellen van de specificatie. Een voorbeeld van een domein waar specificaties een grote rol spelen is vereistenanalyse. Hier zien we dat specificaties vaak in natuurlijke taal worden opgesteld. Zo toont figuur \ref{fig:natural-language-use} het gebruik van natuurlijke taal in vereistenanalyse in 1999 \cite{Luisa2004}. Slechts 5 procent van de specificaties werd toen in een formele taal opgesteld. Al de rest werd in een natuurlijke taal geformuleerd. 16 procent werd zelfs al in een gecontroleerde natuurlijke taal opgesteld.

\begin{figure}
  \begin{tikzpicture}
      \pie[text = legend, radius = 1.8, explode = {0, 0, 0.5}, color = {blue!60, blue!30, blue!5}]{79/(Gewone) natuurlijke taal, 16/Gecontroleerde natuurlijke taal, 5/Formele taal}
  \end{tikzpicture}
  \caption[Gebruik van natuurlijke taal in vereistenanalyse]{Gebruik van natuurlijke taal in vereistenanalyse in 1999 (van figuur 5 in \cite{Luisa2004})}
  \label{fig:natural-language-use}
\end{figure}

\paragraph{} We voeren ons onderzoek naar zo'n formele natuurlijke taal uit binnen het domein van logigrammen. Ze kunnen namelijk aanzien worden als kleine specificaties. Bovendien kunnen ze uitgedrukt worden in relatief simpele logische zinnen. Ten slotte kan men makkelijk vele voorbeelden vinden van zo'n logigrammen. We zullen hierbij zelf geen grammatica opstellen maar een aantal grammaticale regels afleiden van bestaande logigrammen (uit Puzzle Baron's Logic Puzzles Volume 3 \cite{logigrammen}).
