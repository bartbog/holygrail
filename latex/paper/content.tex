%
% paper title
% Titles are generally capitalized except for words such as a, an, and, as,
% at, but, by, for, in, nor, of, on, or, the, to and up, which are usually
% not capitalized unless they are the first or last word of the title.
% Linebreaks \\ can be used within to get better formatting as desired.
% Do not put math or special symbols in the title.
\title{Automatic Translation of Logigrams into Logic}
%
%
% author names and IEEE memberships
% note positions of commas and nonbreaking spaces ( ~ ) LaTeX will not break
% a structure at a ~ so this keeps an author's name from being broken across
% two lines.
% use \thanks{} to gain access to the first footnote area
% a separate \thanks must be used for each paragraph as LaTeX2e's \thanks
% was not built to handle multiple paragraphs
%

\author{Jens~Claes}
%\thanks{M. Shell was with the Department of Electrical and Computer Engineering, Georgia Institute of Technology, Atlanta, GA, 30332 USA e-mail: (see http://www.michaelshell.org/contact.html).}% <-this % stops a space
%\thanks{J. Doe and J. Doe are with Anonymous University.}% <-this % stops a space
%\thanks{Manuscript received April 19, 2005; revised August 26, 2015.}}

% note the % following the last \IEEEmembership and also \thanks - 
% these prevent an unwanted space from occurring between the last author name
% and the end of the author line. i.e., if you had this:
% 
% \author{....lastname \thanks{...} \thanks{...} }
%                     ^------------^------------^----Do not want these spaces!
%
% a space would be appended to the last name and could cause every name on that
% line to be shifted left slightly. This is one of those "LaTeX things". For
% instance, "\textbf{A} \textbf{B}" will typeset as "A B" not "AB". To get
% "AB" then you have to do: "\textbf{A}\textbf{B}"
% \thanks is no different in this regard, so shield the last } of each \thanks
% that ends a line with a % and do not let a space in before the next \thanks.
% Spaces after \IEEEmembership other than the last one are OK (and needed) as
% you are supposed to have spaces between the names. For what it is worth,
% this is a minor point as most people would not even notice if the said evil
% space somehow managed to creep in.


% The paper headers
%\markboth{Journal of \LaTeX\ Class Files,~Vol.~14, No.~8, August~2015}%
%{Shell \MakeLowercase{\textit{et al.}}: Bare Demo of IEEEtran.cls for IEEE Journals}
% The only time the second header will appear is for the odd numbered pages
% after the title page when using the twoside option.
% 
% *** Note that you probably will NOT want to include the author's ***
% *** name in the headers of peer review papers.                   ***
% You can use \ifCLASSOPTIONpeerreview for conditional compilation here if
% you desire.




% If you want to put a publisher's ID mark on the page you can do it like
% this:
%\IEEEpubid{0000--0000/00\$00.00~\copyright~2015 IEEE}
% Remember, if you use this you must call \IEEEpubidadjcol in the second
% column for its text to clear the IEEEpubid mark.



% use for special paper notices
%\IEEEspecialpapernotice{(Invited Paper)}




% make the title area
\maketitle

% As a general rule, do not put math, special symbols or citations
% in the abstract or keywords.
\begin{abstract}
  This paper presents an application of the semantical framework of Blackburn and Bos to logigrams. We construct a grammar and lexicon adapted for translating logigrams into logic.

Additionally, we introduce types into the framework. Based on these types we could reject sentences that are grammatical but without meaning. In this paper type inference is used to infer the domains of a logigram from the sentences in natural language.

% Specifically, we add types to this framework. Based on types we can reject sentences that are grammatical but without meaning.
% We explore the value of types in the framework by translating 

% Deze thesis stelt een aangepassing voor aan het semantische framework van Blackburn
% en Bos [6, 7]. Specifiek passen we dit framework aan voor het oplossen van logigrammen
% door ze te vertalen naar logica. Deze aanpassingen bestaan enerzijds uit het opstellen van
% een lexicon en een grammatica voor logigrammen. Anderzijds voegen we types toe aan
% het framework. Dankzij types kunnen grammaticaal correcte zinnen zonder betekenis
% toch uitgesloten worden. Bovendien kan men op basis van types achtergrondinformatie
% afleiden. Binnen logigrammen kunnen de verschillende domeinen geleerd worden m.b.v.
% types.

\end{abstract}

% Note that keywords are not normally used for peerreview papers.
% \begin{IEEEkeywords}
% IEEE, IEEEtran, journal, \LaTeX, paper, template.
% \end{IEEEkeywords}






% For peer review papers, you can put extra information on the cover
% page as needed:
% \ifCLASSOPTIONpeerreview
% \begin{center} \bfseries EDICS Category: 3-BBND \end{center}
% \fi
%
% For peerreview papers, this IEEEtran command inserts a page break and
% creates the second title. It will be ignored for other modes.
\IEEEpeerreviewmaketitle

\section{Introduction}

\IEEEPARstart{K}{nowledge} base systems have been around for a while. This has triggered a lot of research into formal languages and their expressivity. These languages are often hard to both read, write and learn.

There has also been a lot of research into Controlled Natural Languages (CNL). These are subsets of natural languages that, for example, allow to write specifications in a more consistent language. Kuhn~\cite{Kuhn2014} made an overview of 100 CNL's. Some of these languages even have formal semantics and can be translated automatically into a logic. They could be used as a knowledge representation language within a knowledge base system. They are often easy to read. However the translations of a CNL into logic are often not well documented which makes it hard to expand the language.

This paper therefor constructs a new CNL based on the language used in a small domain, namely logigrams. It is then tested on unseen logigrams. The first goal of this paper is to show that the framework introduced by Blackburn and Bos \cite{Blackburn2005, Blackburn2006} can be used to automatically translate such a CNL into logic. The second goals is to show that adding types to this framework allows more inference.

\section{Motivation}
A knowledge base system is very powerful. It allows multiple inferences based on the same theory. However because knowledge representation languages can be hard to read, write and learn, an expert in formal languages is required. A formal language that is more accessible for non-experts, like a CNL with formal semantics, could solve this issue.

The idea of a knowledge base system is to apply multiple inferences to the same theory. Some examples of inferences that are possible once the clues of a logigram are translated into logic:
\begin{itemize}
  \item Solve the puzzle automatically.
  \item Given a (partial) solution by the user, indicate which clues have already been incorporated in the solution, which clues still hold some new information and which clues have been violated.
  \item Given a partial solution by the user, automatically derive a subset of clues that can be used to expand the solution. This could be part of a ``hint''-system for the user.
  \item Given a (sub)set of clues, indicate which solutions are still possible. This could help the author while writing new logigrams. The possible solutions help to construct a new hint that removes some of these solutions until only one remains.
\end{itemize}



\section{Related work}

Two of the most advanced controlled natural languages that can be used for knowledge representation are Attempto Controlled English (ACE) and Processable English (PENG). ACE \cite{Fuchs2008} is a general purpose CNL. The language contains a large built-in vocabularium. This has as advantage that the user doesn't have to provide the vocabularium. However, for specifications in small domains, we usually do want to provide the vocabularium to make sure the specification only talks about the modelled domain. PENG \cite{Schwitter2002} does ask the user to provide its own content words. Moreover, there is a tool named ECOLE which gives suggestions while writing PENG sentences, namely which linguistic constructs can follow. This makes it easier to write correct PENG sentences. These languages mainly show that we can translate English into logic, but the literature on both these CNL's lacks a description of this translation process. Therefore, extending these CNL's is hard.

Another important CNL is RuleCNL \cite{Njonko2014}. It is a CNL that translates (a subset of) English into business rules. It's notable because it is the only CNL we could find that uses types. Interestingly, they don't really describe their type system nor the inferences they support. However figure~6 from \cite{Njonko2014} indicates that they do support type checking of business rules in RuleCNL.

Finally, Baral et al. \cite{Baral2008, Costantini2010, Baral2012, Baral2012a} researched translating natural language into ASP programs. They do this based on $\lambda$-calculus and a Combinatorial Categorial Grammar (CCG). Each word gets one $\lambda$-ASP-expression per CCG category it can represent. The CCG then tells how to combine these expressions via $\lambda$-application. In their last paper \cite{Baral2012a}, Baral et al. explain how to learn both these $\lambda$-ASP-expressions and a probabilistic CCG grammar from a set of logigrams. They used a supervised machine learning method for this goal. Based on these methods they can solve unseen logigrams without adapting these logigrams.

However, the goal of this paper is not to solve (unseen) logigrams without adapting them. The goal is to create a CNL with formal semantics that can be applied to logigrams and can be used as a knowledge representation language for these logigrams. This paper illustrates that for a small domain, it is possible to have such a language. For unseen logigrams, slight adaptions are necessary to make sure the clues are grammatically correct according to the constructed CNL. As mentioned earlier, once we have a representation of the clues in a formal logic, inference becomes possible. Solving the logigram is only one of the examples.

\section{Logic Grid Puzzles}
Logic grid puzzles are logical puzzles. The most famous such puzzle is probably the Zebra Puzzle \cite{zebra}, sometimes also known as Einstein's Puzzle. These puzzles consist of a number of sentences or clues in natural language. In the clues, a number of domains appear (e.g. nationality, color, animal, ...), each with a number of domain elements (e.g. Norwegian, Canadian, blue, red, cow, horse, ...). Between each domain there is a bijection. The goal of such a puzzle is to find the value of these bijections, i.e. to find which domain elements belong together. E.g. The Norwegian lives in a blue house and keeps the horse. Every puzzle has an unique solution.
 
Logic grid puzzles can be considered as small specifications. Moreover, they can be expressed in fairly simple logical statements. Finally, it is easy to find numerous examples of these types of puzzles. It is these three properties that make logic grid puzzles the ideal domain to test our assumptions. Namely that the framework of Blackburn and Bos allows to construct a CNL with formal semantics that can be used for knowledge representation (within a small domain).


% \section{Introduction}

\IEEEPARstart{K}{nowledge} base systems have been around for a while. They take a knowledge base as input. This knowledge base describes the world, i.e. it is a specification of how something should work. Based on this knowledge base, different kinds of inference can be applied. De Cat et al. \cite{IDP} give the example of an university course-management system. The knowledge base contains rules like ``In every auditorium at a any time, there should be at most one course.''. The different kinds of inferences that can be applied, include (the examples are from De Cat et al. \cite{IDP}): \textit{propagation} (e.g. automatically selecting required prerequisites when constructing an individual study program), \textit{model expansion} (e.g. getting a full individual study program from a partial one) and \textit{querying} (e.g. getting the schedule for a specific student).

These knowledge base systems have triggered a lot of research into formal languages and their expressivity. These languages are often hard to read, write and learn.

One method to ease the writing of formal theories is to develop languages that are somewhere in between natural language and formal languages. Such languages are called Controlled Natural Languages (CNL). A CNL is a subset of a natural language that, for example, allows to write specifications in a more consistent language. Kuhn~\cite{Kuhn2014} made an overview of 100 CNL's. Some of these languages have formal semantics and can be translated automatically into a formal logic. They can be used as a knowledge representation language within a knowledge base system. They are often easy to read. Unfortunately, the translations of such CNL's into logic are not well documented which makes it hard to expand the language.

This paper therefore constructs a new CNL based on the language used in a small domain, namely logic grid puzzles. It is then tested on unseen puzzles. The first goal of this paper is to show that the framework introduced by Blackburn and Bos \cite{Blackburn2005, Blackburn2006} can be used to automatically translate such a CNL into a formal logic. The second goal is to show that adding types to this framework is useful. It makes it possible to translate to a typed logic and allows more inference. E.g. The system can automatically infer the domains of a logic grid puzzle.

% allows more inference. The system can automatically infer domains of a logic grid puzzle.

% In this paper, we take the next step for bridging the gap between natural language and logic. We use the well-documented framework of Blackburn and Bos and extend it with types. With this extension, we constructed the first typed CNL with a formal semantics in a typed logic. This new CNL is based on the language used in a small domain, namely logic grid puzzles. It is then tested on unseen puzzles. The first goal of this paper is to show that the framework introduced by Blackburn and Bos \cite{Blackburn2005, Blackburn2006} can be used to automatically translate such a CNL into a formal logic. We choose the IDP language \cite{IDP} as our formal logic. The second goal is to show that adding types to this framework is useful. It makes it possible to translate to a typed logic and allows more inference. E.g. The system can automatically infer the domains of a logic grid puzzle.

Therefore, we take the next step for bridging the gap between natural language and logic. We extend the well documented framework of Blackburn and Bos with types. With this extension, we constructed the first typed CNL with a formal semantics in a typed logic.

The result of this research is a fully automated tool that can understand a logic grid puzzle and reason about it (e.g. the system can solve the puzzle automatically).


% \section{Conclusion}
The framework of Blackburn and Bos can be used to translate a CNL into logic. This way, we assign semantics to such a CNL. Therefore, we can use such a CNL as a knowledge representation language within a knowledge base system. This allows multiple types of inference on the (constructed) natural language. For a logic grid puzzle, this can mean automatically solving the puzzle but also helping the user by giving hints or helping the author by giving the possible solutions.

Moreover, we can expand the framework with types. This way we can translate the natural language to a typed logic. We have proven that type inference is possible as well. More research into a typed CNL is definitely necessary. For example, into a more complex type system. Or a type system which allows multiple types per word. The type system can then derive the correct instance of the verb. E.g. it can differentiate between ordering drinks and ordering food.



% if have a single appendix:
%\appendix[Proof of the Zonklar Equations]
% or
%\appendix  % for no appendix heading
% do not use \section anymore after \appendix, only \section*
% is possibly needed

% use appendices with more than one appendix
% then use \section to start each appendix
% you must declare a \section before using any
% \subsection or using \label (\appendices by itself
% starts a section numbered zero.)
%
% \appendices
% \section{Proof of the First Zonklar Equation}
Appendix one text goes here.

% you can choose not to have a title for an appendix
% if you want by leaving the argument blank
\section{}
Appendix two text goes here. \cite{IDP}



% % use section* for acknowledgment
% \section*{Acknowledgment}
% The authors would like to thank...


% Can use something like this to put references on a page
% by themselves when using endfloat and the captionsoff option.
\ifCLASSOPTIONcaptionsoff
  \newpage
\fi



% trigger a \newpage just before the given reference
% number - used to balance the columns on the last page
% adjust value as needed - may need to be readjusted if
% the document is modified later
%\IEEEtriggeratref{8}
% The "triggered" command can be changed if desired:
%\IEEEtriggercmd{\enlargethispage{-5in}}

% references section

% can use a bibliography generated by BibTeX as a .bbl file
% BibTeX documentation can be easily obtained at:
% http://mirror.ctan.org/biblio/bibtex/contrib/doc/
% The IEEEtran BibTeX style support page is at:
% http://www.michaelshell.org/tex/ieeetran/bibtex/
%\bibliographystyle{IEEEtran}
% argument is your BibTeX string definitions and bibliography database(s)
%\bibliography{IEEEabrv,../bib/paper}
%
% <OR> manually copy in the resultant .bbl file
% set second argument of \begin to the number of references
% (used to reserve space for the reference number labels box)
% \begin{thebibliography}{1}

\bibliographystyle{IEEEtran}
\bibliography{../thesis/referenties}
% \bibitem{IEEEhowto:kopka}
% H.~Kopka and P.~W. Daly, \emph{A Guide to \LaTeX}, 3rd~ed.\hskip 1em plus
%   0.5em minus 0.4em\relax Harlow, England: Addison-Wesley, 1999.
% \end{thebibliography}

% biography section
% 
% If you have an EPS/PDF photo (graphicx package needed) extra braces are
% needed around the contents of the optional argument to biography to prevent
% the LaTeX parser from getting confused when it sees the complicated
% \includegraphics command within an optional argument. (You could create
% your own custom macro containing the \includegraphics command to make things
% simpler here.)
%\begin{IEEEbiography}[{\includegraphics[width=1in,height=1.25in,clip,keepaspectratio]{mshell}}]{Michael Shell}
% or if you just want to reserve a space for a photo:

% \begin{IEEEbiography}{Michael Shell}
% Biography text here.
% \end{IEEEbiography}

% % if you will not have a photo at all:
% \begin{IEEEbiographynophoto}{John Doe}
% Biography text here.
% \end{IEEEbiographynophoto}

% % insert where needed to balance the two columns on the last page with
% % biographies
% %\newpage

% \begin{IEEEbiographynophoto}{Jane Doe}
% Biography text here.
% \end{IEEEbiographynophoto}

% You can push biographies down or up by placing
% a \vfill before or after them. The appropriate
% use of \vfill depends on what kind of text is
% on the last page and whether or not the columns
% are being equalized.

%\vfill

% Can be used to pull up biographies so that the bottom of the last one
% is flush with the other column.
%\enlargethispage{-5in}



% that's all folks
