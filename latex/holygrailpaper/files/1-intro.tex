Automated problem solving is central to the research field of Artificial Intelligence. A common assumption here is that the problem formulation is unambiguously specified as a problem specification (sometimes called \textit{model}) in a formal language. However, formulating such a specification is non-trivial, and for human beings it is much easier to specify problems in natural language.

Natural language processing (NLP) is studied as a proper subfield of AI~\cite{manning1999foundations}. Problem solving starting from natural language input has been studied for simple mathematical problems in the subfield of \textit{word problem solving} or natural language mathematical problems~\cite{Mukherjee2008}.

We here aim to study natural language combinatorial problems, both from human-produced input (e.g. natural language) and human-understandable output (in our case, visualisations), as per the Holy Grail 2019 challenge~\footnote{https://freuder.wordpress.com/pthg-19-the-third-workshop-on-progress-towards-the-holy-grail/}. Generating human-understandable output to logic grid puzzles has initially been explored in ~\cite{sqalli1996inference}.

\paragraph{}
We present \ourtool,  an end-to-end solution for solving logic grid puzzles (also known as Zebra puzzles) and explaining in human-understandable terms how this solution can be obtained from the clues. 

\ourtool starts from a plain English language representation of the clues and a list of all the entities present in the puzzle. It then applies the standard NLP techniques to build a puzzle-specific \textit{lexicon}. This lexicon is fed into a type-aware variant of the semantical framework of Blackburn and Bos~\cite{bos}, which translates the clues into logic \todo{REF to master thesis here}. This logic is further transformed to a problem specification in the \idp language \mycite{idp}, an extension of first-order logic. 

It then uses this formal representation of the clues both to solve the puzzle and to explain the solution. 
There are many different ways in which such a system could explain itself. For instance, after finding a solution, it can explain \begin{inparaenum}\item \emph{why that is a solution} or \item \emph{why there are no other solutions}; additionally, it can explain \item \emph{how the system found this solution}, and \item \emph{how a human could find this solution}. \end{inparaenum}
 
Our system implements this last type of explanation. In contrast to how the system found the solution, it focuses on simplifying the explanation itself, rather than giving insights in the actual search algorithm used by \idp. As such, our explanations are not to be used for understanding the inner workings of the solver, but rather are to be used by people interested in solving logic puzzles.

In generating explanations and choosing the \textit{order} in which the reasoning steps are explained, we chose to order by an estimate of mental effort required to follow the reasoning step. Each reasoning step is visualised as the clue(s) involved and the resulting changes on the grid.

When solving such puzzels, it can either be used for explaining how to obtain an entire solution, or for getting help when they are stuck during the solving process. Indeed, our explanation method will, given a partial solution, find the easiest next derivation to make. 

\todo{Perhaps to much of the system overview already in the intro...}
