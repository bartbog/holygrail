% !TeX root = ./workshop_paper.tex
In the last few years, as AI systems employ more advanced reasoning mechanisms and computation power, it becomes increasingly difficult to understand why certain decisions are made.
Explainable (XAI), a subfield of AI, aims to fulfil the need for trustworthy AI systems to understand \emph{how} and \emph{why} the system made a decision, e.g. for verifying correctness of the system, as well as to control for biased or systematically unfair decisions.

Explanations have been investigated in constraint solving before, most notably for explaining over-constrained, and hence unsatisfiable, problems to a user.
The QuickXplain method \cite{junker2001quickxplain} for example, uses a dichotomic approach that recursively partitions the constraints to find a minimal conflict set.
Many other papers consider the same goal and search for explanations of over-constrainedness \cite{leo2017debugging,zeighami2018towards}.

Despite the fact that we do not (specifically) aim to explain over-constrained problems, our algorithms will also internally make use of methods to extract a minimal set of conflicting constraints often called a \emph{minimal unsatisfiable subset} (MUS) or \emph{minimal unsatisfiable core} \cite{marques2010minimal}.

While explainability of constraint optimisation has received little attention so far, in the related field of \textit{planning}, there is the emerging subfield of \textit{eXplainable AI planning} (XAIP)~\cite{fox2017explainable}, which is concerned with building planning systems that can explain their own behaviour.
This includes answering queries such as ``why did the system (not) make a certain decision?'', ``why is this the best decision?'', etc. In contrast to explainable machine learning research~\cite{guidotti2018survey}, in explainable planning one can make use of the explicit \textit{model-based representation} over which the reasoning happens.
Likewise, we will make use of the constraint specification available to constraint solvers, more specifically typed first-order logic~\cite{atcl/Wittocx13}.

This research fits within the general topic of Explainable Agency~\cite{langley2017explainable}, whereby in order for people to trust autonomous agents, the latter must be able to \textit{explain their decisions} and the \textit{reasoning} that produced their choices.
To provide the constraint solver with Explainable Agency~\cite{langley2017explainable}, we first formalize the problem of step-wise explaining the propagation of a constraint solver through a sequence of small inference steps.
Next, we present an algorithm that uses an optimistic estimate of a given cost function quantifying human interpretability to guide the search to \textit{simple}, low-cost, explanations thereby making use of Minimal Unsatisfiable Subsets.
We extend this approach using reasoning by contradiction to produce additional explanations of still difficult-to-understand inference steps.
Finally, we discuss the challenges and some outlooks to explaining how to solve constraint satisfaction problems.

\paragraph*{Publication history} This paper is an extended abstract of previous papers presented at workshops and conferences \cite{claesuser,DBLP:conf/bnaic/ClaesBCGG19,ecai/BogaertsGCG20} and a journal paper under review \cite{bogaerts2020framework}.