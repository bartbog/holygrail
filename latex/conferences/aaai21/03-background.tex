% !TeX root = ./main.tex

In this section, we discuss background on the hitting set duality, which constitutes the core of our algorithm. 
We present all result using propositional logic  as a representation language, as is common in hitting set--based approaches, but the algorithm works equally well for other languages with a model semantics. 
 
%  
% 
% 
% The overarching goal of this paper is to generate a sequence of small reasoning steps, each with an interpretable explanation, and for that we introduce the necessary background.
% \begin{itemize}
%     \item Explanation sequence : CSP journal paper ?
%     \item concept of OMUS
%     % \item Logic grid puzzles (depends on experiments)
%     % \item Nested explanation still applicable?
% \end{itemize}

\newcommand\F{\m{F}}
Let \voc be a set of propositional symbols, also called \emph{atoms}; this set is implicit in the rest of the paper. A \emph{literal} is an atom $p$ or its negation $\lnot p$. A clause is a disjunction of literals. A formula $\m{F}$ is a conjunction of clauses. 
Slightly abusing notation, a clause is sometimes used as a set of literals and a formula as a set of clauses.
% \todo{weight function comes later}
A \emph{weight function} $w$ on $\m{F}$ assigns to each clause in $\m{F}$ either a positive integer, or a $\infty$. 
A \emph{model} is a truth assignment
 to the atoms that satisfies $\m{F}$.
$\m{F}$ is said to be \emph{unsatisfiable} if it has no models. 

\begin{definition}
  A \emph{Minimal Unsatisfiable Subset} (MUS) of \F is a set $\m{S} \subseteq \F$  that is unsatisfiable but such that every strict subset of $\m{S} $ is satisfiable. 
%   
  We write \muses{\F} for the set of MUSs of \F. 
\end{definition}


% Formally, we define a \textit{Minimum Unsatisfiable Subset} (MUS), as a set of clauses that taken together are unsatisfiable, but if any of the clauses are removed then the reduced formula becomes satisfiable.

\begin{definition}
    A set $\m{S} \subseteq \m{F}$ is a \emph{Maximal Satisfiable Subset} (MSS) of $ \m{F}$ if $\m{S}$ is satisfiable and for all $\m{S}'$ with $\m{S}  \subsetneq  \m{S}'\subseteq\m{F} $, $\m{S}'$ is unsatisfiable.
\end{definition}

\begin{definition}
    A subset $\m{C} \subseteq \m{F}$ is a \emph{Minimal Correction Subset} (MCS) if $\m{F}\setminus\m{C}$ is satisfiable, while for all 
    $ \m{C}' \subsetneq C$,  $\m{F} \setminus \m{C}'$ is unsatisfiable.
    We write \mcses{\F} for the set of MCSs of \F. 
\end{definition}

It is well-known that if $\m{F}$ is unsatisfiable, each  MCS is the complement of an MSS and vice versa. 

\begin{definition}\label{def:minimal-hs}
    Given a collection of sets $\m{K}$, a hitting set $h$ is a set such that $\forall \ C \in \m{K}$, $h \cap C \neq \emptyset$. A hitting set is \emph{minimal} if no strict subset of it is also a hitting set. 
\end{definition}

% \noindent In fact, for a collection of sets $\m{K}$ there exist multiple possible minimal hitting sets on $\m{K}$. Consequently, we define the \textbf{minimum} hitting set as:

% \begin{definition}\label{def:minimum-hs}
% %     Let $\m{K}$ be a collection of sets and $HS(\m{K})$ be the collection of all minimal hitting sets on $\m{K}$.
% %     A hitting set $hs$ is a \textbf{minimum} hitting set if $\forall \ h' \in HS(\m{K}): |hs| \leq |h'|$.
%   \end{definition}

The following proposition is the well-known hitting set duality  between MCSs and MUSs that forms the basis of our algorithms \cite{DBLP:journals/jar/LiffitonS08,ai/Reiter87}.

\begin{proposition}\label{prop:MCS-MUS-hittingset}
%     Given an $ \m{F}$, let MUSes($\m{F}$), be the Minimal Unsatisfiable Subsets of F and MCSes($\m{F}$), be the Minimal Correction Subsets of F:
%     
    A set  $\m{C} \subseteq \m{F}$ is an MCS of $ \m{F}$ iff  $\m{C}$ is a \emph{minimal hitting set} of \muses{F}.

    A set  $\m{U} \subseteq \m{F}$ is a MUS of $ \m{F}$ iff  $\m{U}$ is a \emph{minimal hitting set} of \mcses{F}.
\end{proposition}

The duality of proposition \ref{prop:MCS-MUS-hittingset} is fundamental to explaining how we can compute the SMUS in section \ref{sec:smus}. 
% TODO: add something here ?
In section \ref{sec:omus}, we further exploit this proposition for computing the OMUS for a given unsatisfiable CNF formula $\m{F}$.
