\chapter{Lijst van afkortingen en symbolen}
\section*{Afkortingen}
\begin{flushleft}
  \renewcommand{\arraystretch}{1.1}
  \begin{tabularx}{\textwidth}{@{}p{12mm}X@{}}
    CNL   & Constructed Natural Language \\
    DCG   & Definite Clause Grammars \\
    DRT   & Discourse Representation Theory \\
    DRS   & Discourse Representation Structures \\
    KBS   & Knowledge Base Systems \\
  \end{tabularx}
\end{flushleft}
\section*{Constituenten}
De verschillende soorten constituenten die kunnen voorkomen in een grammatica (terminologie uit de taalkunde). We gebruiken de Engelse namen voor gelijkaardigheid met de literatuur.
\begin{flushleft}
  \renewcommand{\arraystretch}{1.1}
  \begin{tabularx}{\textwidth}{@{}p{12mm}X@{}}
    s     & Sentence (zin) \\
    np    & Noun Phrase (naamwoordgroep) \\
    vp    & Verb Phrase (verbale constituent) \\
    v     & Verb (werkwoord) \\
    iv    & Intransitive Verb (onovergankelijk werkwoord) \\
    tv    & Transitive Verb (overgankelijk werkwoord) \\
    pn    & Proper Noun (eigennaam) \\
    n     & Noun (zelfstandig naamwoord) \\
    det   & Determinator \\
  \end{tabularx}
\end{flushleft}

\section*{Symbolen}
\begin{flushleft}
  \renewcommand{\arraystretch}{1.1}
  \begin{tabularx}{\textwidth}{@{}p{12mm}X@{}}
    $@$   & Applicatie uit lambda-calcalus \\
  \end{tabularx}
\end{flushleft}
