While our proposed method is applicable to constraint satisfaction problems in general, we use \textit{logic grid puzzles} as example domain, as it requires no expert knowledge to understand.
Our running example is a puzzle about people having dinner in a restaurant and ordering different types of pasta, which is the hardest logic grid puzzle we encountered; it was sent to us by someone who got stuck solving it and wondered whether it was correct in the first place.    
The entire puzzle can be seen in Figure \ref{fig:zebrascreen}; the full explanation sequence generated for it can be found online at \url{http://bartbog.github.io/zebra/pasta}.

In this section, we first present logic grid puzzles as a use case, and afterwards introduce \emph{typed first-order logic}, which we use as the language to express our constraint programs in. Here, it is important to stress that our definitions and algorithms work for any language with model-theoretic semantics, including typical constraint programming languages~\cite{rossi2006handbook}.

\subsection{Logic grid puzzles}
A logic grid puzzle (also known as ``Zebra puzzle'' or ``Einstein puzzle'') consists of natural language sentences (from hereon referred to as ``clues'') over a set of \emph{entities} occurring in those sentences. 
For instance, our running example in Figure~\ref{fig:zebrascreen} contains as second clue ``The person who chose arrabiata sauce is either Angie or Elisa'' and (among others) the entities ``arrabiata sauce'', ``Angie'' and ``Elisa''. 

The set of entities is sometimes left implicit if it can be derived from the clues, but often it is given in the form of a grid. 
Furthermore, in such a puzzle the set of entities is partitioned into equally-sized groups (corresponding to \emph{types}); in our example, ``person'' and ``sauce'' are two such types. 
% 
The goal of the puzzle is to find relations between each two types such that
\begin{itemize}
	\item \textit{Each clue is respected}; 
	\item \textit{Each entity of one type is matched with exactly one entity of the second type}, e.g., each person chose exactly one sauce and each sauce is linked to one person (this type of constraint will be referred to as \emph{bijectivity}); and 
	\item \textit{The relations are logically linked}, e.g., if Angie chose arrabiata sauce and arrabiata sauce was paired with farfalle, then Angie must also have eaten farfalle (from now on called \emph{transitivity}). 
\end{itemize}

In Section \ref{sec:holistic}, we explain how we obtain a vocabulary and first-order theory in a mostly automated way from the clues using Natural Language Processing. 
The result is a vocabulary with types corresponding to the groups of entities in the clues, and the names and types of the binary relations to find (e.g \textit{chose(person, sauce)}, \textit{paired(sauce, pasta)}, \textit{eaten(person, pasta)});
as well as constraints (first-order sentences) corresponding to the clues, and the bijectivity and transitivity constraints. 

\subsection{Typed first-order logic}

Our constraint solving method is based on \emph{typed first-order logic}. %, with links to \emph{typed second-order logic}.
Part of the input is a logical vocabulary consisting of a set of types, % symbols,
(typed) constant symbols, and (typed) relation symbols with associated type signature (i.e., each relation symbol is typed $T_1\times \dots \times T_n$ with $n$ types $T_i$).
\footnote{We here omit function symbols since they are not used in this paper.} 
For our running example, constant symbol \textit{Angie} of type \textit{person} is linked using relation \textit{chose(.,.)} with signature \textit{person $\times$ sauce} to constant symbol \textit{arrabiata sauce} of type \textit{sauce}.


A \emph{first-order theory} is a set of sentences (well-formed variable-free first-order formulas \cite{enderton} in which each quantified variable has an associated type), also referred to as constraints.
Since we work in a fixed and finite domain, the vocabulary, the interpretation of the types (the domains) and the constants are fixed.
This justifies the following definitions: 
\begin{definition}\label{def:partial-interpretation}
 A \emph{(partial) interpretation} is a finite set of literals, i.e., expressions of the form $P(\ddd)$ or $\lnot P(\ddd)$ where $P$ is a relation symbol typed $T_1\times\dots \times T_n$ and $\ddd$ is a tuple of domain elements where each $d_i$ is of type $T_i$. 
\end{definition}
 \begin{definition}\label{def:consistent}
 A partial interpretation is \emph{consistent} if it does not contain both an atom and its negation, it is called a \emph{full} interpretation if it either contains $P(\ddd)$ or $\lnot P(\ddd)$ for each well-typed atom $P(\ddd)$. 
\end{definition}

For instance, in the partial interpretation $I_1=\{chose(Angie,arrabiata),$ $\lnot chose(Elisa,arrabiata)\}$ it is known that $Angie$ had arrabiata sauce while Elisa did not. This partial interpretation does not specify whether or not Elisa ate Farfalle.  


The syntax and semantics of first-order logic are defined as usual (see e.g.\ \cite{enderton}) by means of a satisfaction relation $I \models T$ between first-order theories $T$ and full interpretations $I$. If $I\models T$, we say that $I$ is a model (or solution) of $T$.


\begin{definition}
	A partial interpretation $I_1$ is \emph{more precise} than partial interpretation $I_2$ (notation $I_1\geqp I_2$) if $I_1\supseteq I_2$.
\end{definition}

Informally, one partial interpretation is more precise than another if it contains more information. For example, the partial interpretation $I_2 =\{chose(Angie,arrabiata)$, $\lnot chose(Elisa,arrabiata)$, $ \lnot chose(damon,arrabiata)\}$ is more precise than $I_1$ ($I_2 \geqp I_1$).

For practical purposes, since variable-free literals are also sentences, we will freely use a partial interpretation as (a part of) a theory in solver calls or in statements of the form $I\land T \models J$, meaning that everything in $J$ is a consequence of $I$ and $T$, or stated differently, that $J$ is less precise than any model $M$ of $T$ satisfying $M\geqp I$. 

In the context of first-order logic, the task of finite-domain constraint solving is better known as \emph{model expansion} \cite{MitchellTHM06}: given a logical theory $T$ (corresponding to the constraint specification) and a partial interpretation $I$ with a finite domain (corresponding to the initial domain of the variables), find a model $M$ more precise than $I$ (a partial solution that satisfies $T$).

If $P$ is a logic grid puzzle, we will use $T_P$ to denote a first-order theory consisting of:
\begin{itemize}
 \item One logical sentence for each clue in $P$;
 \item A logical representation of all possible bijection constraints;
 \item A logical representation of all possible transitivity constraints.
\end{itemize}

For instance, for our running example, some sentences in $T_P$ are: 
\begin{align*}
 &\lnot chose(claudia,puttanesca).\\
 & \forall s\in sauce : \exists p\in pasta : pairedwith(s,p)\\
 & \forall s\in sauce: \forall p1\in pasta, \forall p2\in pasta:  pairedwith(s,p1)\land pairedwith(s,p2)\limplies p1=p2. \\
 & \forall s\in sauce: \forall p\in person, \forall c\in price: chose(s,p)\land payed(p,c)\limplies priced(s,c).  % changed 'cost' to priced to differentiate from constraint cost...
\end{align*}
