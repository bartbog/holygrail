\chapter{Een lexicon voor logigrammen}
\label{ch:lexicon}
In dit hoofdstuk bespreken we de gebruikte lexicale categorieën. We beperken ons tot een set die het makkelijk maakt voor de vertaling van logigrammen naar DRS-structuren. We bespreken zowel de categorieën zelf alsook hun vertaling naar deze DRS-structuren. Er wordt een onderscheid gemaakt tussen open en gesloten lexicale categorieën. De open categorieën zijn open voor uitbreiding. De woorden uit die categorie zijn verschillend per logigram. De gesloten categorieën bevatten woorden die gemeenschappelijk zijn voor alle logigrammen. Tabel~\ref{tbl:lexiconCategories} geeft een overzicht van de gebruikte lexicale categorieën.

\begin{table}[t]
  \centering
  \begin{tabular}{llll}
    \toprule
    \textbf{Categorie} & \textbf{Afkorting} & \textbf{Open?} & \textbf{Voorbeeld}  \\ \midrule
    Determinator       & det                & gesloten & a, an, the \\
    Hoofdtelwoord      & number             & gesloten & three, 5      \\
    Eigennaam          & pn                 & open     & John, ``the black darts'' \\
    Substantief        & noun               & open     & man, year, \\
    Voorzetsel         & prep               & gesloten & in, to \\
    Betrekkelijk voornaamwoord & relpro     & gesloten & who, which, that \\
    Transitief werkwoord & tv               & open     & loves, ``had a final score of'' \\
    Hulpwerkwoord      & av                 & gesloten & does, ``doesn't'' \\
    Koppelwerkwoord    & cop                & gesloten & is, ``is not'' \\
    Comparatief        & comp               & open     & below, ``older than'' \\
    Onbepaalde woorden & some               & open     & somewhat, sometime \\
    Voegwoord          & coord              & gesloten & and, or, ``neither ... nor ...'' \\
    \bottomrule
  \end{tabular}
  \caption{Een overzicht van de lexicale categorieën}
  \label{tbl:lexiconCategories}
\end{table}

\section{Determinator}
Een determinator kan zowel een lidwoord als een kwantor zijn. In het geval van logigrammen volstaan de lidwoorden ``a'', ``an'' en ``the''. Deze drie determinatoren krijgen alle drie de vertaling van de existentiële determinator uit het vorige hoofdstuk $$\sem{det} = \sem{det_{existentieel}} = \lambdaf{R}{\lambdaf{S}{\left( \drs{x}{} \oplus \app{R}{x} \oplus \app{S}{x} \right)}}$$ Er is dus geen nood aan een universele of negatieve determinator. Bij logigrammen zijn we namelijk op zoek naar de waarde van bijecties. Er is dus altijd exact één iemand die een bepaalde drank drinkt of een bepaald huisdier heeft. Er is nooit sprake van ``Every man who drinks vodka, ...'' maar altijd van ``The man who drinks vodka, ...''. Men kan ook nooit gebruik maken van de negatieve determinator (bv. ``No man drinks vodka'') aangezien er altijd één iemand moet zijn die vodka drinkt.

\section{Hoofdtelwoord}
\label{sec:lex-number}
Hoofdtelwoorden kunnen gebruikt worden als determinatoren die een aantal uitdrukken. In deze thesis krijgen hoofdtelwoorden echter een eigen lexicale categorie omdat op sommige plaatsen enkel een hoofdtelwoord past en geen andere determinatoren (bijvoorbeeld ``five'' in de zin ``The five different people are ...''). De hoofdtelwoorden mogen in cijfers voorkomen maar ook in woorden \footnote{In de praktijk zitten enkel de eerste 15 getallen in woorden in het lexicon}.

De signatuur van een hoofdtelwoord is gelijk aan die van een determinator. Er zijn twee mogelijke lezingen voor een hoofdtelwoord: de collectieve en de distributieve lezing. We verduidelijken aan de hand van de voorbeeldzin ``Twee mannen gaan naar zee''. In de collectieve lezing vormen de ``twee mannen'' één geheel. Ze gaan dus samen naar zee. In de distributieve lezing zijn er twee mannen die elk naar zee gaan. In logigrammen komt enkel de collectieve lezing aan bod. Meestal gaat het immers om een numerieke eigenschap van iets of iemand. Bijvoorbeeld ``John is 10 years old'' of ``John is 3 years younger than Mary''.

$$\sem{number} = \lambdaf{R}{\lambdaf{S}{\drsMerge{\app{R}{Number}}{\app{S}{Number}}}}$$

\section{Eigennaam}
Een eigennaam is een open lexicale categorie. Dat wil zeggen dat de eigennamen verschillend zijn per logigram. De semantiek is identiek aan die in het vorige hoofdstuk. $$\sem{pn} = \lambdaf{P}{\app{P}{\textit{Symbool}}}$$ We staan vanaf nu echter wel toe dat woordgroepen die taalkundig geen eigennaam zijn, toch gebruikt kunnen worden als een eigennaam. Zo kan ``the black darts'' (wat normaal een determinator + adjectief + substantief is) aanzien worden als een eigennaam. Dit maakt het vertalen van de zinnen makkelijker maar tegelijkertijd wordt het opstellen van het lexicon voor een logigram moeilijker. Het lexicon is niet meer enkel afhankelijk van taalkundige informatie. Alle mogelijke waarden van een (niet-numeriek) concept moeten namelijk als een eigennaam aangegeven worden in het lexicon dat hoort bij het logigram. Bovendien moet dit gebeuren in de vorm zoals het voorkomt in de zinnen van het logigram. Zo is ``black'' een waarde van het concept kleur, toch moet ``the black darts'' ingegeven worden in het lexicon. Deze eigennamen zullen later vertaald worden naar een constante uit een constructed type (zie ook hoofdstuk \ref{ch:types}). Daarmee is het lexicon dus een mengeling van een taalkundig en een formeel vocabularium.

\paragraph{}Een logigram kan 3 soorten eigennamen hebben: een eigennaam in het enkelvoud (bv. ``John''), een eigennaam in het meervoud (bv. ``The Turkey Rolls'') en een \textit{numerieke eigennaam}. Bij de eerste twee wordt het symbool afgeleid van de woordvorm. Bij de laatste gebeurt dit door de gebruiker. Numerieke eigennamen worden namelijk gebruikt om woorden om te zetten in getallen. Zo kan ``March'' omgezet worden in 3. Op die manier wordt het zinvol om te spreken over ``1 maand na maart''. Voor een \textit{numerieke eigennaam} is het symbool gelijk aan die numerieke waarde. Deze wordt apart meegegeven in het probleem-specifiek lexicon. Deze vertaling van woorden naar getallen moet door de gebruiker gebeuren omdat hier achtergrond kennis voor nodig is. Het is opnieuw een voorbeeld van hoe onze vertaling van een logigram deels in het lexicon zit.
%De numerieke eigennamen zullen geen aanleiding geven tot constanten in constructed types. Het is wel een andere voorbeeld van hoe de vertaling van de logigram in het lexicon kruipt.

\section{Substantief}
Ook substantieven zijn een open categorie. Hun semantiek nemen we voorlopig over van het vorige hoofdstuk. $$\sem{n} = \lambdaf{x}{\drs{}{\textit{Symbool}(x)}}$$ In hoofdstuk~\ref{ch:types} (over types) zullen we DRS uitbreiden met types en zal het predicaat op x verdwijnen en vervangen worden door een echte type-constraint in DRS. Een substantief in het logigram-specifiek lexicon bevat een enkelvoudsvorm en een meervoudsvorm. Het symbool is gelijk aan de enkelvoudsvorm. Op die manier is het symbool voor het enkelvoud en het meervoud gelijk.

\section{Voorzetsel}
De voorzetsels (in het Engels \texttt{prepositions} of \texttt{prep}) vormen een gesloten woordklasse die bestaat uit woorden zoals ``from'', ``in'', ``with''. Ze worden op twee manieren gebruikt in de zinnen van een logigram. Enerzijds bij een werkwoord. Dan staat het vlak voor het lijdend voorwerp. Bijvoorbeeld de ``with'' in ``to finish with 500 points''. Anderzijds als belangrijkste woord in een voorzetselconstituent (ook wel \texttt{prepositional phrase} of \texttt{pp} genoemd). Bijvoorbeeld de ``from'' in ``the man from France''. Het voorzetsel dat bij een werkwoord hoort, zien we als deel van het werkwoord. In dat geval heeft het voorzetsel geen vertaling.

In het geval van een voorzetselconstituent is er wel een vertaling. We kunnen zo'n voorzetselconstituent zien als een extra beperking op een substantief. Of een transformatie van een substantief naar een nieuw substantief ($\tau(pp) = \tau(n) \rightarrow \tau(n)$). Een voorzetsel is het belangrijkste woord in zo'n voorzetselconstituent. Men kan het dus zien als een functie van een naamwoordgroep (\texttt{noun phrase} of \texttt{np}) naar een voorzetselconstituent. Of in formulevorm $\tau(prep) = \tau(np) \rightarrow \tau(pp) = \tau(np) \rightarrow \tau(n) \rightarrow \tau(n) = [(e \rightarrow t) \rightarrow t] \rightarrow (e \rightarrow t) \rightarrow (e \rightarrow t)$. De betekenis ziet er uit als
$$\sem{prep} = \lambdaf{N}{\lambdaf{S}{\left( \lambdaf{y}{\drsMerge{\app{S}{y}}{\app{N}{\lambdaf{x}{\drs{}{\textit{Symbool}(y, x)}}}}}} \right)}$$

Een voorzetsel neemt een naamwoordgroep $N$ en een substantief $S$ als argument en geeft een beperking op een entiteit $y$ terug. De beperking op $y$ bestaat enerzijds uit de beperking van het substantief ($\app{S}{y}$). Anderzijds voegt het voorzetsel zelf ook nog een beperking toe. Hiervoor haalt het de entiteit $x$ uit de box van de naamwoordgroep $N$ en linkt het de entiteiten $x$ en $y$ via een binair predicaat met dezelfde naam als het voorzetsel in kwestie. Het symbool van een voorzetsel is namelijk gelijk aan het voorzetsel zelf.

\section{Betrekkelijk voornaamwoord}
Een betrekkelijk voornaamwoord (\texttt{relative pronoun} of \texttt{relpro}) is een woord aan het begin van een betrekkelijke bijzin (ook wel \texttt{relative clause} of \texttt{rc}). Voorbeelden zijn ``that'', ``which'' en ``who''. Net als een voorzetselconstituent staat zo'n betrekkelijke bijzin bij een substantief en legt ze een extra beperking op.

$$\sem{relpro} = \lambdaf{V}{\lambdaf{S}{\left( \lambdaf{x}{\drsMerge{\app{S}{x}}{\app{V}{\lambdaf{B}{\app{B}{x}}}}} \right)}}$$
Een betrekkelijk voornaamwoord neemt een verbale constituent $V$ en een substantief $S$ als argument en geeft een beperking op een entiteit $x$ terug. De beperkingen op $x$ bestaan enerzijds uit de beperking van het substantief $S$ (namelijk $\app{S}{x}$) en anderzijds uit de beperkingen van de verbale constituent. De verbale constituent heeft $x$ als onderwerp dus $V$ krijgt als argument een lege box rond $x$ mee ($\lambdaf{B}{\app{B}{x}}$)

\section{Transitief werkwoord}
Een transitief werkwoord is een werkwoord met een lijdend voorwerp. Een logigram heeft enkel transitieve werkwoorden. Er zijn geen intransitieve (zonder lijdend voorwerp) of ditransitieve werkwoorden (met een meewerkend voorwerp). De vertaling van een transitief werkwoord is gelijk aan die van het vorige hoofdstuk $$\sem{tv} = \lambdaf{N1}{\lambdaf{N2}{\app{N2}{\lambdaf{x2}{\app{N1}{\lambdaf{x1}{\drs{}{\textit{Symbool}(x2, x1)}}}}}}}$$ Het lijdend voorwerp ($N1$) en het onderwerp ($N2$) zijn de argumenten van het werkwoord. Eerst wordt de box van het onderwerp uitgepakt. Daarbinnen wordt dan weer de box van het lijdende voorwerp uitgepakt. De kwantor van het onderwerp komt dus altijd voor die van het lijdend voorwerp (met deze vertaling). Ten slotte worden de entiteiten van het onderwerp ($x2$) en het lijdend voorwerp ($x1$) gebonden via een predicaat met het symbool van het werkwoord.

\paragraph{}Een transitief werkwoord is een open lexicale categorie. Dat wil zeggen dat de woorden verschillend zijn per logigram. In het logigram-specifiek lexicon staat de infinitief, de werkwoordsvorm in de derde persoon enkelvoud alsook een voltooid of onvoltooid deelwoord. De enkelvoudsvorm kan zowel in de verleden tijd als de tegenwoordige tijd zijn, afhankelijk van hoe het werkwoord gebruikt wordt in de logigram. Ten slotte wordt ook nog het voorzetsel gegeven dat voor het lijdend voorwerp wordt gezet (indien van toepassing) en eventueel een achtervoegsel aan het einde van de zin (bv. ``to print'' in ``The design took 8 minutes to print''). Het symbool van het werkwoord (en dus ook de naam van het predicaat) wordt afgeleid uit de enkelvoudsvorm, het voorzetsel en het achtervoegsel.

\paragraph{} Net zoals met de eigennamen wordt er vrij los omgesprongen met de werkwoorden. Zo is ``to be recognized as endangered in'' een werkwoord in één van de logigrammen.

\section{Hulpwerkwoord}
De woordklasse van hulpwerkwoorden is een gesloten woordklasse. Binnen de logigrammen zijn het de werkwoordsvormen van ``to do'' en ``to be'' die deel uitmaken van deze klasse. Alsook het hulpwerkwoord voor de toekomst ``will''. Naast de woordvorm bevat het lexicon ook informatie over de polariteit van die woordvorm: positief of negatief. ``does'' is positief, ``doesn't'' is negatief. De beide polariteiten hebben elk een andere betekenis.

Een hulpwerkwoord is een woord dat een verbale constituent (\texttt{verb phrase} of \texttt{vp}) omvormt tot een nieuwe verbale constituent. Voor een hulpwerkwoord met positieve polariteit is dit de identieke transformatie \footnote{Merk op dat de tijd niet uitmaakt voor een logigram en ``will clean'' dus niet anders vertaald moet worden dan ``cleans''}. $$\sem{av_{pos}} = \lambdaf{V}{V}$$ Voor een hulpwerkwoord met een negatieve polariteit bestaat de transformatie uit een negatie van de verbale constituent. Merk op dat er dus geen negatie is van het onderwerp. Hiermee wordt de vertaling van ``Everyone doesn't work'' naar logica $\forall x \cdot \lnot work(x)$ i.p.v. $\lnot \forall x \cdot work(x)$

$$\sem{av_{neg}} = \lambdaf{V}{\lambdaf{N}{\app{N}{\lambdaf{x}{\drsNot{\app{V}{\lambdaf{P}{\app{P}{x}}}}}}}}$$

Het hulpwerkwoord krijgt een verbale constituent ($V$) en een onderwerp ($N$) als argument. Het haalt de entiteit van het onderwerp uit de box en geeft dan een negatie van de verbale constituent terug. Die verbale constituent krijgt een lege doos met enkel de entiteit in ($\lambdaf{P}{\app{P}{x}}$) mee als ``onderwerp''.

\section{Koppelwerkwoord}
\label{sec:lex-koppelwerkwoord}
De categorie van koppelwerkwoorden is een gesloten lexicale categorie. Ze bestaat uit verschillende vormen van het werkwoord ``to be'' (bv. is, isn't, is not, was, are, were, ...). Er is een enkelvoud- en meervoudsvorm. Bovendien is er sprake van een positieve of negatieve polariteit. Deze hebben elk een licht andere semantiek. Ten slotte kan een koppelwerkwoord ook op drie verschillende manieren gebruikt worden in een zin van een logigram:

\begin{itemize}
  \item Samen met een \texttt{nominale constituent} (\texttt{noun phrase} of \texttt{np}): Bijvoorbeeld ``John is a man''. Dit type van gebruik heeft dezelfde signatuur als een overgankelijk werkwoord. De semantiek zegt dat de twee referenties die in de box van het onderwerp en het lijdend voorwerp zitten, gelijk zijn.
  $$\sem{cop_{np, pos}} = \lambdaf{N1}{\lambdaf{N2}{\app{N2}{\lambdaf{x2}{\app{N1}{\lambdaf{x1}{\drs{}{x2 = x1}}}}}}}$$
  In negatie wordt dit een ongelijkheid. De negatie bevindt zich echter al voor het ``uitpakken'' van de box van het lijdend voorwerp. Zodanig dat er een correcte negatie van de kwantoren is. In de zin ``Mary is not a man'' is het belangrijk dat er geen enkele man is die gelijk is aan Mary. De negatie moet dus voor de existentiële kwantor komen en dus voor het uitpakken van de box van het lijdend voorwerp.
  $$\sem{cop_{np, neg}} = \lambdaf{N1}{\lambdaf{N2}{\app{N2}{\lambdaf{x2}{\drsNot{\app{N1}{\lambdaf{x1}{\drs{}{x2 = x1}}}}}}}}$$
  \item Samen met een \texttt{adjectiefconstituent} (\texttt{adjective phrase} of \texttt{ap}): Bijvoorbeeld ``John is 30 years old''. De vertaling die wij gebruiken ligt echter vrij ver van de taalkundige structuur. Zo wordt het koppelwerkwoord + adjectief gezien als een soort van transitief werkwoord. De betekenis is gelijkaardig aan die van het koppelwerkwoord met een nominale constituent. De adjectieven die gebruikt kunnen worden, moeten meegegeven worden via het lexicon van een logigram. Het symbool komt overeen met de woordvorm van het adjectief.
  $$\sem{cop_{ap, pos}} = \sem{tv} = \lambdaf{N1}{\lambdaf{N2}{\app{N2}{\lambdaf{x2}{\app{N1}{\lambdaf{x1}{\drs{}{\textit{Symbool}(x2, x1)}}}}}}}$$
  $$\sem{cop_{ap, neg}} = \lambdaf{N1}{\lambdaf{N2}{\app{N2}{\lambdaf{x2}{\drsNot{\app{N1}{\lambdaf{x1}{\drs{}{\textit{Symbool}(x2, x1)}}}}}}}}$$
  \item Samen met een \texttt{voorzetselconstituent} (\texttt{prepositional phrase} of \texttt{pp}): Bijvoorbeeld ``John is from France''. Dit kan aanzien worden als een ellips van ``a person'': ``John is a person from France''. De signatuur is $ \tau(cop_{pp}) = \tau(pp) \rightarrow \tau(vp) = \tau(pp) \rightarrow \tau(np) \rightarrow t = [(e \rightarrow t) \rightarrow e \rightarrow t] \rightarrow [(e \rightarrow t) \rightarrow t] \rightarrow t$.

  $$\sem{cop_{pp, pos}} = \lambdaf{Q}{\lambdaf{N}{\app{N}{\lambdaf{x}{\app{\app{Q}{\lambdaf{y}{\drs{}{}}}}{x}}}}}$$
  De voorzetselconstituent $Q$ heeft als eerste argument een lambda-functie die een beperking oplegt vanuit het substantief waarbij het hoort. In dit geval is er echter geen substantief en dus geen beperking. Dit vertaalt zich in een lege DRS-structuur. Het tweede argument, de entiteit die beperkt wordt door de voorzetselconstituent, is de entiteit uit het onderwerp. Die moeten we dus eerst uitpakken uit de naamwoordgroep $N$.

  De betekenis van de negatieve vorm is $$\sem{cop_{pp, neg}} = \lambdaf{Q}{\lambdaf{N}{\app{N}{\lambdaf{x}{\drsNot{\app{\app{Q}{\lambdaf{y}{\drs{}{}}}}{x}}}}}}$$
\end{itemize}

\section{Comparatief en onbepaalde woorden}
Binnen een logigram is er altijd minstens één concept van numerieke aard. Vaak is er dan ook een zin die uitdrukt dat iemand een hogere of lagere numerieke waarde heeft dan iemand anders. Bijvoorbeeld ``John scored 15 points less than Mary'' of ``Mary finished sometime after Tom''. De woordgroepen ``less than'' en ``after'' zijn deel van de lexicale categorie \texttt{comparatief}. ``sometime'' is dan weer een voorbeeld van een onbepaald woord. Wat op het eerste zicht niet opvalt aan deze zinnen is dat ze een belangrijke ellips bevatten. Tom is namelijk geen tijdstip maar een persoon. De tweede zin is voluit ``Mary finished sometime after Tom finished''. Deze ellips zal opgelost worden in de grammatica.

%Een comparatief is een functie die twee naamwoordgroepen als argument neemt en een nieuwe naamwoordgroep maakt die als lijdend voorwerp kan dienen. $\tau(comp) = \tau(np) \rightarrow \tau(np) \rightarrow \tau(np) = [(e \rightarrow t) \rightarrow t] \rightarrow [(e \rightarrow t) \rightarrow t] \rightarrow [(e \rightarrow t) \rightarrow t]$. 
\paragraph{} Er zijn opnieuw 2 betekenissen afhankelijk van het woord: één voor ``hoger'' (bv. ``after'') en één voor ``lager'' (bv. ``less than'')
$$\sem{comp_{lager}} = \lambdaf{N1}{\lambdaf{N2}{\left( \lambdaf{P}{\app{N1}{\lambdaf{x1}{\app{N2}{\lambdaf{x2}{\drsMerge{\drs{x}{x = x2 - x1}}{\app{P}{x}}}}}}} \right)}}$$
Een comparatief neemt twee naamwoordgroepen ($N1$ en $N2$) als argument. Het resultaat is een nieuwe naamwoordgroep die kan dienen als lijdend voorwerp. Een naamwoordgroep is van type $(e \rightarrow t) \rightarrow t$ en dus krijgen we nog een argument $P$ van type $e \rightarrow t$ mee. We kunnen $\lambda P$ ook zien als het aanmaken van een box waar we later een nieuwe entiteit kunnen insteken. Daarna kunnen we de naamwoordgroepen uitpakken. We doen dit in volgorde van het voorkomen in de zin (dus eerst $N1$ en dan $N2$). Vervolgens maken we een nieuwe entiteit $x$ aan. Deze is gelijk aan het verschil $x2-x1$ (bijvoorbeeld de score van Mary - 15). Ten slotte stoppen we de nieuwe entiteit $x$ in de box van onze resulterende naamwoordgroep.

De andere betekenis van de comparatief, die voor ``hoger'', is gelijkaardig aan de eerste met dat verschil dat $x = x2+x1$ i.p.v. $x = x2-x1$.

De comparatief is een open lexicale categorie. Elke logigram heeft eigen comparatieven. In het logigram-specifiek lexicon bevindt zich naast de woordvorm ook telkens het type (``hoger'' of ``lager'').

\paragraph{} De onbepaalde woorden hebben dezelfde signatuur als een naamwoordgroep $\tau(some) = \tau(np) = (e \rightarrow t) \rightarrow t$. We kunnen het zien als een box rond een existentiële kwantor voor een natuurlijk getal \footnote{Het getal moet strikt positief zijn want anders kan $x=x2-x1$ ook hoger dan $x2$ uitkomen bij een comparatief met als betekenis ``lager dan''}. Meer concreet hebben we $$\sem{some} = \lambdaf{P}{\drsMerge{\drs{x}{x > 0}}{\app{P}{x}}}$$ %We maken een box aan ($\lambda P$). Vervolgens maken we een entiteit $x$ aan dat een positief getal ($x > 0$) voorstelt. (Het getal moet strikt positief zijn want anders kan $x=x2-x1$ ook hoger dan $x2$ uitkomen bij een comparatief met als betekenis ``lager dan''). Ten slotte stoppen we de entiteit in de box ($\app{P}{x}$).

\section{Voegwoord}
Een voegwoord is een woord dat twee woordgroepen van dezelfde categorie verbindt. Voorbeelden zijn ``and'', ``or'' and ``nor''. Een voegwoorden kan twee zinnen verbinden maar ook twee nominale constituenten. Binnen logigrammen zijn de voegwoorden nevenschikkend. Dat wil zeggen dat beide woordgroepen even belangrijk zijn \footnote{In tegenstelling tot een onderschikkend voegwoord. Zo'n voegwoord verbindt bijvoorbeeld een hoofdzin met een bijzin. Daarbij is de bijzin minder belangrijk dan de hoofdzin.}. Er komen drie soorten voegwoorden voor, elk met een eigen vertaling: een conjunctief voegwoord (``A and B''), een disjunctief voegwoord (``A or B'') en een negatief voegwoord (``A nor B''). Een voegwoord kan uit twee delen bestaan (bv. ``either ... or'' en ``neither ... nor''). Daarom bestaan er twee lexicale categorieën: \texttt{coord} voor het voegwoord zelf (bv. ``or'') en \texttt{coordPrefix} voor de eventuele prefix (bv. ``either'').

De vertaling van het voegwoord is onafhankelijk van de categorie van de woordgroepen die worden verbonden. Belangrijk is wel dat de signatuur van die woordgroepen een functie met als resultaat iets van type $t$ is ($\tau(woordgroep) = \alpha \rightarrow t$). Een signatuur van het voegwoord is dan $\tau(coord) = (\alpha \rightarrow t) \rightarrow (\alpha \rightarrow t) \rightarrow (\alpha \rightarrow t) = (\alpha \rightarrow t) \rightarrow (\alpha \rightarrow t) \rightarrow \alpha \rightarrow t $. De betekenis van een voegwoord bestaat er telkens uit door het derde argument (van type $\alpha$) door te geven aan de eerste twee argumenten en de resultaten op de juiste manier terug te combineren

$$\sem{coord_{conjunctief}} = \lambdaf{A}{\lambdaf{B}{\left( \lambdaf{C}{\drsMerge{\app{A}{C}}}{\app{B}{C}} \right)}}$$
$$\sem{coord_{disjunctief}} = \lambdaf{A}{\lambdaf{B}{\left( \lambdaf{C}{\drs{}{\app{A}{C} \lor \app{B}{C}}} \right)}}$$
$$\sem{coord_{negatief}} = \lambdaf{A}{\lambdaf{B}{\left( \lambdaf{C}{\drsMerge{\drsNot{\app{A}{C}}}}{\drsNot{\app{B}{C}}} \right)}}$$

Deze vertaling is echter niet de enigste mogelijke vertaling. In het geval van een disjunctie kan men ook naar een exclusieve of vertalen (zeker voor de constructie ``either ... or ...'' houdt dit steek). Omwille van de bijecties bij logigrammen maakt dit echter geen verschil. Er is immers altijd maar exact één iemand die eraan voldoet dus er is in praktijk geen verschil tussen een inclusieve en exclusieve disjunctie.

\paragraph{} Echter ook voor de conjunctie is een andere vertaling mogelijk, meer bepaald voor de conjunctie van twee naamwoordgroepen. Zo heeft de zin ``John and Mary went to the sea'' twee mogelijke vertalingen. De bovenstaande vertaling wordt de distributieve lezing genoemd en komt overeen met dat beiden (apart) naar de zee zijn gegaan. In de andere vertaling, de collectieve lezing, zijn John en Mary samen naar de zee gegaan. Binnen logigrammen komt de collectieve lezing bijna niet aan bod. De enigste plaats waar dat wel gebeurt, is in een soort van \textit{alldifferent}-constraint. Bijvoorbeeld ``John, Bob and Charles are three different politicians''. Bij de vertaling van de grammatica lossen we dit probleem op.
