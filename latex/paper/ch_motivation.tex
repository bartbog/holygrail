\section{Motivation}
Since knowledge representation languages can be hard to read, write and learn, an expert in formal languages is required to write and maintain the specification. A formal language that is more accessible for non-experts, like a CNL with formal semantics, can solve this issue without compromising on the strength of knowledge base systems. It is still possible to apply different kind of inferences on a specification that is more accessible for non-experts.

For example, when the clues of a logic grid puzzle are expressed in a CNL (with formal semantics), it is possible to
% With a CNL for logic grid puzzles, it is possible to apply different kind of inferences to these puzzles. It is for example possible to
\begin{itemize}
  \item Solve the puzzle automatically.
  \item Given a (partial) solution by the user, indicate which clues have already been incorporated in the solution, which clues still hold some new information and which clues have been violated.
  \item Given a partial solution by the user, automatically derive a subset of clues that can be used to expand the solution. This can be part of a ``hint''-system for the user.
  \item Given a set of clues, indicate which solutions are still possible. This can help the author while writing new puzzles. Based on the solutions that are still possible, the author can construct a new clue to eliminate some of the possibilities until only one solution remains.
\end{itemize}
