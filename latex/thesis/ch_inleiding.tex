\chapter{Inleiding}
\paragraph{} Logigrammen zijn een soort van puzzels waarbij de lezer een aantal zinnen voorgeschoteld krijgt. De zinnen bevatten een aantal domeinen (zoals nationaliteit, dier, kleur, ...) en een aantal domeinelementen (Noor, Brit, kat, hond, rood, blauw, ...). Ze beschrijven één bijectie tussen elk paar van domeinen. Het doel van de puzzel is het achterhalen van de waarde van de bijecties. M.a.w. welke domeinelementen bij elkaar horen. Bijvoorbeeld ``De Noor woont in het blauwe huis en heeft een kat als huisdier''. Elke puzzel heeft exact één oplossing.

De zinnen van logigrammen zijn redelijk gestructureerd waardoor het mogelijk is om ze automatisch om te vormen naar een meer formele representatie waarop inferenties uitgevoerd kunnen worden. Deze thesis onderzoekt hoe haalbaar deze automatische vertaling van logigrammen naar logica is. Dit dient als opstap naar meer algemene vertalingen van een natuurlijke taal naar een formele taal.

\paragraph{} We bestuderen hiervoor het framework van Blackburn en Bos \cite{Blackburn2005, Blackburn2006}. Dit framework is gebaseerd op lambda-calculus en Frege's compositionaliteitsprincipe (de betekenis van een woordgroep is een combinatie van de betekenissen van de woorden of woordgroepen waaruit ze bestaat).

We geven eerst wat achtergrond die kan helpen bij het begrijpen van de thesis. Dan geven we een aantal vertalingen van een natuurlijke taal naar een formele taal uit de literatuur. Vervolgens stellen we het framework van Blackburn en Bos voor. Daarna leggen we uit hoe we dit framework kunnen gebruiken voor het vertalen van logigrammen naar logica. Eerst bespreken we het lexicon, vervolgens de grammatica. Ten slotte breiden we het framework uit met types en illustreren we hoe we deze types kunnen vertalen naar het formele vocabularium. Ter evaluatie passen we dit hele framework toe op een aantal logigrammen uit een puzzelboekje \cite{logigrammen}.
