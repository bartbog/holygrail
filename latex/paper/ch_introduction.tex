\section{Introduction}

\IEEEPARstart{K}{nowledge} base systems have been around for a while. This has triggered a lot of research into formal languages and their expressivity. These languages are often hard to both read, write and learn.

There has also been a lot of research into Controlled Natural Languages (CNL). These are subsets of natural languages that, for example, allow to write specifications in a more consistent language. Kuhn~\cite{Kuhn2014} made an overview of 100 CNL's. Some of these languages even have formal semantics and can be translated automatically into a formal logic. They could be used as a knowledge representation language within a knowledge base system. They are often easy to read. However the translations of a CNL into logic are not well documented which makes it hard to expand the language.

This paper therefor constructs a new CNL based on the language used in a small domain, namely logigrams. It is then tested on unseen logigrams. The first goal of this paper is to show that the framework introduced by Blackburn and Bos \cite{Blackburn2005, Blackburn2006} can be used to automatically translate such a CNL into logic. The second goals is to show that adding types to this framework allows more inference.
