\chapter{Lijst van afkortingen en symbolen}
\section*{Afkortingen}
\begin{flushleft}
  \renewcommand{\arraystretch}{1.1}
  \begin{tabularx}{\textwidth}{@{}p{12mm}X@{}}
    CNL   & Constructed Natural Language \\
    KBS   & Knowledge base paradigma \\
  \end{tabularx}
\end{flushleft}
\section*{Constituenten}
De verschillende soorten constituenten die kunnen voorkomen in een grammatica (terminologie uit de taalkunde). We gebruiken de Engelse namen voor gelijkaardigheid met de literatuur.
\begin{flushleft}
  \renewcommand{\arraystretch}{1.1}
  \begin{tabularx}{\textwidth}{@{}p{12mm}X@{}}
    s     & Sentence (zin) \\
    np    & Noun Phrase (naamwoordgroep) \\
    vp    & Verb Phrase (verbale constituent) \\
    v     & Verb (werkwoord) \\
    n     & Noun (zelfstandig naamwoord) \\
    det   & Determinator \\
  \end{tabularx}
\end{flushleft}

% \section*{Symbolen}
% \begin{flushleft}
%   \renewcommand{\arraystretch}{1.1}
%   \begin{tabularx}{\textwidth}{@{}p{12mm}X@{}}
%     $m$   & Massa \\
%   \end{tabularx}
% \end{flushleft}
