\documentclass{article}
\usepackage{fullpage}
\usepackage{graphicx}
\usepackage{hyperref}
\usepackage{xcolor}
\usepackage{amsmath,amssymb}
\usepackage[ruled,linesnumbered]{algorithm2e}
\usepackage{float}
\usepackage{amsthm}
\usepackage{thmtools} 
\usepackage[parfill]{parskip}
\usepackage[normalem]{ulem}

\definecolor{vuborange}{rgb}{1.0,0.40,0.0}
\newcommand\m[1]{\mathcal{#1}}

\newtheorem{thm}{Theorem}
\newtheorem{definition}[thm]{Definition}
\newtheorem{prop}{Property}
\newtheorem{property}[prop]{Property}
\newtheorem{lem}{Lemma}
\newtheorem{lemma}[lem]{Lemma}
\newtheorem{propo}{Proposition}
\newtheorem{proposition}[propo]{Proposition}
\newtheorem{ex}{Example}
\newtheorem{example}[ex]{Example}

\bibliographystyle{apalike}
\newcommand\comment[1]{\marginpar{\tiny #1}}
\renewcommand\comment[1]{#1}
\newcommand{\tias}[1]{{\comment{\color{blue}\textsc{TG:}#1}}}
\newcommand{\emilio}[1]{{\comment{\color{red}#1}}}
\newcommand{\bart}[1]{{\comment{\color{green}#1}}}
\newcommand{\todo}[1]{{\comment{\color{red}\textsc{TODO:}#1}}}



\begin{document}
% \section{Introduction}
% \subsection{Ideas}
% \begin{itemize}
%   % \item explanations in constraint satisfaction problems \todo{cite: xai special issue} easy, understandable, human-interpretable
%   % \item Explanations can be small, but also difficult (combinations of constraints/clues)
%   % \item \todo{which complexity is extracting an OMUS ?}
%   % \item use a/different proxy(s) to qualify interpretability of an explanation
%   % \item extension SMUS to OMUS based on enumeration of optimal hitting set
%   \item approach tested on Boolean Satisfiability instances and on a high level problem
% \end{itemize}
% \subsection{Todo}
% \begin{itemize}
%   \item \todo{which complexity is extracting an OMUS ?}
% \end{itemize}

% \section{Background}

% The algorithm presented in this section is based on the key ideas and observations of Ignatiev et al. presented in \cite{ignatiev2015smallest}.
% The algorithm is adapted to incorporate an optimality criterion in order to guide the search not in the direction of the SMUS, but towards the OMUS. 
% To do so, we first define the  objective function f:

% \begin{definition}
%   Given a CNF Formula $\m{F}$, let $f : 2^{\m{F}} \rightarrow \mathbb{R}$ be a mapping of a set of clauses to a real number. f is said to be a \textit{consistent} objective function if for any subsets $\m{A}$, $\m{B}$ of $\m{F}$ if $\m{A} \subseteq \m{B}$ then $f(\m{A}) \leq f(\m{B})$.
%   \emilio{function f is said to be set increasing if ...}
% \end{definition}

% For an unsatisfiable CNF formula $\m{F}$, a subset of clauses that are still unsatisfiable, but if any of the clauses are removed then the reduced formula becomes satisfiable. 
% Formally, we define this set as a Minimum unsatisfiable Subset (MUS):

% \begin{definition}
%   A subset $\m{U} \subseteq \m{F}$ is a \textbf{minimal unsatisfiable subset} (MUS) if $\m{U}$ \emph{unsatisfiable} and $\forall \ \m{U}' \subset \m{U}$, $\m{U}'$ is \emph{satisfiable}. An MUS of $\m{F}$ with an optimal value w.r.t an \emph{objective function $f$} is called an \textbf{optimal MUS} (OMUS).
% \end{definition}

% \begin{definition}
%   A subset $\m{C}$ of $\m{F}$ is an \textbf{minimal correction subset} (MCS) if $\m{F}  \setminus \m{C}$ is \emph{satisfiable} and $\forall \m{C}' \subseteq \m{C} \wedge \m{C}' \neq \emptyset$, $(\m{F} \setminus \m{C}) \cup \m{C}'$ is \emph{unsatisfiable}.
% \end{definition}

% \begin{definition}
%   A \emph{satisfiable subset} $\m{S} \subseteq \m{F}$ is a \textbf{Maximal Satisfiable Subset} (MSS) if  $\forall \ \m{S}' \subseteq \m{F}'$ s.t $ \m{S} \subseteq \m{S}'$, $\m{S}'$ is \emph{unsatisfiable}.
% \end{definition}

% An MSS can also be defined as the complement of an MCS (and vice versa). If $\m{C}$ is a MCS then $ \m{S} = \m{F} \setminus \m{C}$ is a MSS. On the other hand, MUSes and MCSes are related by the concept of minimal hitting set.

% \begin{definition}
%   Given a collection of sets $\Gamma$ from a universe $\mathbb{U}$, a hitting set on $\Gamma$ is a set such that $\forall \ \m{S} \in \Gamma, h \cap S \neq \emptyset$.
% \end{definition}

% \begin{proposition}\label{prop:duality_MCS_MUS}
%   Given a CNF formula $\m{F}$, let MUSes($\m{F}$) and MCSes($\m{F}$) be the set of all MUSes and MCSes of $\m{F}$ respectively. Then the following holds:
%   \begin{enumerate}
%     \item A subset $\m{U}$ of $\m{F}$ is an MUS iff $\m{U}$ is a minimal hitting set of MCSes($\m{F}$)
%     \item A subset $\m{C}$ of $\m{F}$ is an MCS iff $\m{U}$ is a minimal hitting set of MUSes($\m{F}$)
%   \end{enumerate}
% \end{proposition}

\section{Explanation Generation}

\begin{algorithm}
  %  \begin{algorithmic}
    \DontPrintSemicolon

  \SetKwInOut{Input}{input}
  \SetKwInOut{OptInput}{optional input}
  \SetKwInOut{Output}{output}
  \SetKwComment{command}{/*}{*/}

  \Input{$\m{T}$  \textit{set of constraints } }
  \Input{$f$ \textit{a consistent objective function} }
  \OptInput{$\m{I}_0$ \textit{a partial interpretation}}
  \Output{\textit{Explanation sequence}}
  \Begin{
   
    $\m{I} \gets \m{I}_0$  \tcp*{Initial partial interpretation}
    $\m{I}_{end} \gets$ \texttt{propagate($\m{I}$, $\m{T}$)}  \tcp*{Goal state}
    $Seq \gets$ \textit{empty set}     \tcp*{explanation sequence}
    $H \gets$ \textit{Empty collection}     \tcp*{Collection of hitting sets}

    \While{  $\m{I} \neq \m{I}_{end}$ }{
      % $\m{F} \gets \m{I}_{end} \setminus \m{I}$ \tcp*{Facts to be derived}
      % $\m{F}' \gets \{\neg \m{F} \}$ \tcp*{Set with all negated literals of $\m{F}$}
      \;
      $X \gets$ \texttt{OMUS($\m{F}' \wedge \m{I} \wedge \m{S}$, $\m{H}$)} \;
      \;
      $E \gets$ $\m{I} \cap X$ \tcp*{Explanation used}
      \;
      $\m{N} \gets$ \texttt{propagate($E \wedge \m{S}$)} \tcp*{Newly derived facts}
      \;
      $\m{I} \gets$ $\m{I} \cup \m{N}$ \tcp*{Update known facts}
      \;

      \For{$n \in \m{N}$}{
        $(E_n$, $\m{S}_n$, $n)$ to $Seq$ \;
      }
      \;

    }

  }
  % \end{algorithmic}
  \caption{CSP-Explain($\m{T} ,\ f \ [,  \ \m{I}_0 ]$)}
  \label{alg:cspExplain}
\end{algorithm}

\clearpage
\section{OMUS Algorithm}

Note that if we assign a unit weight to every element in the subset, we reduce the problem of finding an OMUS to finding a SMUS.

\begin{definition}
  Let $\Gamma$ be a collection of sets and HS($\Gamma$) the set of all hitting sets on $\Gamma$ and let $f$ be an valid objective function. Then a hitting set $ h \in \Gamma$ is said to be an \textbf{optimal} hitting set if $\forall$ $h' \in HS(\Gamma)$ we have that %$|h| \leq |h'|$ and 
  $f(h) \leq f(h')$. %\cite{davies2011solving}.
\end{definition}

\begin{property}
  The \textbf{optimal} hitting set of a collection of sets $\Gamma$ is denoted by $OHS(\Gamma)$.
\end{property}

The algorithm is based on the following observation:

\begin{proposition}\label{prop:optimal-hitting-set}
  A set $\m{U} \subseteq \m{F}$ is an OMUS of $\m{F}$ if and only if $\m{U}$ is an optimal hitting set of MCSes($\m{F}$)
\end{proposition}

\begin{lemma}\label{lemma:K}
  Let $\m{K}  \subseteq$ MCSes($\m{F}$). Then a subset $\m{U}$ of $\m{F}$ is an OMUS if $\m{U}$ is a optimal hitting set on $\m{K}$ and $\m{U}$ is unsatisfiable
\end{lemma}

\begin{algorithm}
  \DontPrintSemicolon
  \SetKwInOut{Input}{input}
  \SetKwInOut{OptInput}{optional input}
  \SetKwInOut{Output}{output}
  \SetKwComment{command}{/*}{*/}
  \SetKwSwitch{Switchy}{Case}{Default}{swtich}{}{case}{otherwhise}{}%

  \Input{$\m{F}$ \textit{a CNF formula } }
  \Input{$cost$ \textit{a cost function} }
  \OptInput{$\m{H}_0$ \textit{initial collection of disjoint Minimum Correction Sets}}
  \Output{$\m{OMUS}(\m{F})$}
  \Begin{
    $\m{H} \gets $ \texttt{DisjointMCS($\m{F}$)} \;
    % \sout{$\m{H} \gets \m{H}_0$} \;
    \While{true}{
      $hs \gets$ \texttt{OptimalHittingSet}($\m{H}, cost$) \tcp*{\small Find \textbf{optimal} solution}
      % \tcp{\small set with all unique clauses from hitting set}
      (sat?, $\mu$) $\gets$ \texttt{SatSolver}($hs$)\;
      \tcp{If SAT, $\mu$ contains the satisfying truth assignment}
      \tcp{IF UNSAT, $hs$ is the OMUS }
      \If{ not sat?}{
        \textbf{break} \;
      }
      $\m{C} \gets \m{F}$ $\setminus $ \texttt{Grow}($hs$) \;
      $\m{H} \gets \m{H} \cup \{ \m{C}\}$ \;
      $nonOptLevel \gets 0$ \;
      \tcp{Find a series of non-optimal solutions}
      \While{true}{
        \Switch{$nonOptLevel$}{
            \Case{0}{
              \tcp{Add/Remove clause (choose clause appears most frequently in the set of hitting sets so far)}
              $hs \gets$ \texttt{FindIncrementalHittingSet}($H$, $\m{C}$, $hs$)\;
              }
              \Case{1}{
                \tcp{Greedy algorithm}
                \tcp{`Approximation algorithms for combinatorial problems' (1973)}
                $hs \gets$ \texttt{FindGreedyHittingSet}($\m{H}$)\;
            }
        }

        (sat?, $\mu$) $\gets$ \texttt{SatSolver}($hs$)\;
        \uIf{ not sat?}{
          \Switch{$nonOptLevel$}{
            \Case{0}{
              $nonOptLevel \gets 1$ \;
              }
              \Case{1}{
                \textbf{break} \;
             }
          }
        }
        \uElse{
          $\m{C} \gets \m{F}$ $\setminus $ \texttt{Grow}($hs$) \;
          $\m{H} \gets \m{H} \cup \{ \m{C}\}$ \;
          $nonOptLevel \gets 0$ \;
        }
      }

  }
  \Return{$(hs', \ cost(hs)) $} \;}
  % \end{algorithmic}
  \caption{OMUS-Delayed($\m{F},\ [ f \ , \m{H}_0]$)}
  % \caption{OMUS($\m{F},\m{H}_0 = \emptyset ,f_o$)}
  \label{alg:omus}
\end{algorithm}



\newpage
\section{MIP hitting set problem specification}
For the set of clauses $ \m{C} = \{c_1, ... c_{|C|}\}$ with weights $\m{W} = \{w_1, ... w_{|C|}\}$ in the collection of sets $\m{H}$. For Example:

\begin{equation} \label{mip:example}
  \begin{split}
    \m{C} &= \{c_1, ... c_6 \}  \\
    \m{W} &= \{w_1 = 20, w_2 = 20,  w_3 = 10,  w_4 = 10,  w_5 = 10,  w_6 = 20\} \\
    \m{H} &= \{c_3 \},\ \{c_2, c_4\},\ \{c_1, c_4\},\ \{c_1, c_5, c_6\} 
  \end{split}
  \end{equation}

The optimal hitting set can be formulated as an integer linear program.
\begin{equation} \label{eq:ILP:objective}
  min \sum_{ i \in \{1..|C|\}} w_i \cdot x_i
\end{equation}
\begin{equation} \label{eq:ILP:hittingset}
  \sum_{i \in \{1..|C|\}} x_i \cdot h_{ij} \geq 1, \  \forall \ j \in \{1..|\m{H}|\}
\end{equation}
\begin{equation} \label{eq:ILP:bool:xi}
  x_i = \{0,1\}
\end{equation}

\begin{itemize}
  \item $w_i$ is the input cost/weight associated with clause i in
  \item $x_i$ is a boolean decision variable if constraint/clause $c_i$ is chosen or not.
  \item Equation \ref{eq:ILP:hittingset}, $h_{ij}$ is a boolean input variable corresponding to if constraint/clause i is in set to hit j.
\end{itemize}

\newpage
\section{Future Work}

\newpage

\bibliography{omusrefs}

\end{document}
