
When a human solves a logic grid puzzle, they typically maintain a grid with all the information obtained so far and use it in combination with the clues to derive new conclusions. In logic terminology, the user maintains a partial interpretation.  
With this in mind, the goal underlying our tool is to find a sequence of partial interpretations of the involved relations (e.g., in which it is known that Angie did not choose arrabiata sauce, but it still unknown whether she ordered farfalle) with an 
\emph{explanation} (see below) of why each of the steps is correct that is ``simple enough for a person to understand''. 

\paragraph{Explanation of reasoning steps.}
The goal is to obtain a sequence of incremental partial interpretations $\langle I_0, I_1, \ldots, I_n \rangle$ where $I_0$ is an initial partial interpretation (potentially empty), and $\forall i>0, I_{i-1} \geqp I_{i}$ with $I_n$ a (full) interpretation.
%\[ I_0 = \emptyset, I_1 = I_0 \cup N_1, \dots , I_n = I_{n-1}\cup N_n\]
%where $I_i$ represents the state of the grid at each point in time and $N_i$ represents the newly derived information. 
Additionally, for each incremental step $\langle I_{i-1}, I_i \rangle$ we want an explanation $(E_i,S_i)$ that justifies the newly derived information $N_i = I_i \setminus I_{i-1}$. When visualized (see e.g. figure \tias{todo}), it will show to the user precisely what information and constraints were used to derive a new piece of information.

\tias{the 'all models more precise' needs a definition to be referred to later again}
Since we are working in the context of a given fixed finite domain, we identify a partial structure with a consistent set of ground literals (i.e., a set of variable-free literals that do not contain both an atom and its negation), and thus $I_i\geqp I_{i-1}$ iff $I_i\supseteq I_{i-1}$.
With this view (partial) interpretations can also be seen as a theory (having each such literal as a sentence), which will allow us to make claims of the form 
$T\cup I \entails I'$ with $T$ a theory and  $I,I'$ (partial) interpretations, meaning that all literals in $I'$ follow from $T$ and $I$, or stated precisely, that all models of $T$ more precise than $I$ are also more precise than $I'$. 

\tias{Definition using the above? but below E/N are not more precise of eachother?}
We hence define a justification as a tuple $(E_i,S_i)$ such that: 
\begin{compactitem}
	\item $E_i\subseteq I_i$ (i.e., the explaining facts are a subset of what was previously derived),
	\item $S_i \subseteq T_P$ (i.e., a subset of the clues and implicit constraints are used), and 
	\item $S_i \cup E_i \entails N_i$ (i.e., all newly derived information indeed follows from this justification).
\end{compactitem}

The problem of simply checking whether $(E_i,S_i)$ explains $N_i$ is in co-NP since this problem can be performed by verifying that $S_i \land \lnot N_i$ has no models more precise than $E_i$. It is hence an instance of the negation of a model expansion problem \cite{ternovskaMXcomplexity}.


Part of our goal of finding easy to interpret explanations is to avoid redundant information in the explanation. That is, we want a non-redundent explanation $(E_i,S_i)$ where none of the facts in $E_i$ or constraints in $S_i$ can be removed or else the implication $S_i \wedge E_i \rightarrow N_i$ would no longer hold. We will use the concept of Minimal Unsat Cores here (or Minimal Unsat Subset, MUS). A MUS over a set of constraints $C$ is a subset $C' \subseteq C$ such that $C' is UNSAT$ and $\forall c \in C': C' \setminus \{c\} is SAT$. If we know that $N_i$ follows from $S_i$ and $E_i$ then $\neg N_i \wedge S_i \wedge E_i$ must be UNSAT. A Minimal Unsat Subset (MUS) hereof will be a non-redundant subset of $N'_i \subseteq N_i$, $S'_i \subseteq S_I$ and $E'_i \subseteq E_i$ for which we know (because it is UNSAT): $\neg (\neg N'_i \wedge S'_i \wedge E'_i)$ hence through substitution $N'_i \vee \neg (S'_I \wedge E'_i)$ and hence $(S'_I \wedge E'_i) \rightarrow N'_I$. In other words, $(S'_I \wedge E'_i)$ is the smallest explanation of $N'_I$ for which this relation holds, for any smaller explanation it will no longer hold.

%Any incremental step $\langle I_{i-1}, I_i \rangle$ has a trivial justification in the form of $(I_i, T_P)$. However, our goal is to find interpretable justifications that a person can understand. We hence wish to remove redundant information from the justification and instead find a \emph{subset-minimal} justification.
After reifying the involved constraints of $T_P$ this can also be cast as a second-order problem of the form
\[\exists S_i\subseteq T_P, E_i\subseteq I_i: (\forall I: I\models S_i\land E_i \Rightarrow N_i) \land \lnot \exists S_i'\subseteq S_i, E_i'\subseteq E_i: \forall ... \]
\tias{TODO bart, extend/explain, e.g. the ...}


%Our goal is not to derive any sequence of justifications, but a sequence of easy to interpret justifications. Indeed, while the step from the input to a full solution $\langle I_0, I \rangle$ is easily justified by $(I_0, T_P)$, this will not be interpretable for a user.

\paragraph{Interpretability of reasoning steps.}
While subset-minimality ensures that a justification is non-redundant, it does not quantify how \textit{interpretable} a justification is. 
This quantification is a problem-specific and often subjective manner. In line with the goal of "simple enough for a person to understand" and Occam's Razor, we reason that smaller justifications are easier to interpret than justifications that use a larger number of facts or constraints. %For example, from a cognitive point of view, the more things a person needs to have in memory simultaneously, the more difficult the task will become. \tias{The cognitive statement is idd risky without scientific support...}
%To approximate of how easy to understand a justification is (i.e., a single transition in the above described sequence), we start from the simple cognitive idea \bart{can we cite this somewhere? } that (in general) the fewer things a human needs to have in memory simultaneously, the easier the task at hand is. 

We assume the existence of a cost-function $f(E_i,S_i,N_i)$ that measures the interpretability of a justification $(E_i,S_i)$ of a reasoning step $\langle I_{i-1}, I_i \rangle$ with $N_i = I_i \setminus I_{i-1}$.

We define \textbf{the explanation-production problem} as that of finding a sequence $\langle I_0, I_1, \ldots, I_n \rangle$ that minimizes an aggregate of the cost vector $f(E_i,S_i,N_i) \forall_i$ and where $I_n$ is the intersection of all total interpretations (solutions) of $T_P$. \tias{TODO: ook hoger naar intersection definitie gaan (mss formele def hier?)}

Both the cost function and the aggregation operator has to be defined, which may be problem specific. For example, the $max()$ aggregation operator will minimize the cost of the most complicated reasoning step, but does not capture whether there is one such step used, or multiple. Likewise, the $average()$ aggregation operator will favor many simple steps, including splitting up trivial steps into many small components if the constraint abstraction allows this.

As finding a single minimal $f(E_i,S_i,N_i)$ is ..., the entire search problem is obviously even more difficult 


\paragraph{Logic Grid Puzzles} In the context of Logic Grid Puzzles, every puzzle has one unique solution. Hence $I_n$ is that total interpretation, and the goal is to find a sequence of simple and interpretable reasoning steps towards that solution. For example, reasoning steps that combine multiple clues can be consider more difficult than reasoning steps over individual clues, and reasoning steps that use few known facts are to be preferred over those using many known facts.

% The grand goal underlying our tool is to find, given a logic grid puzzle (of which we assume it is given in some logical form for now; we revisit this in Section \ref{sec:holistic}), to find a sequence of partial assignments of variables (e.g., where it is already determined that certain entities are linked (or not linked) to which other entities) that is ``as easy to understand'' as possible.  
% Of course the latter is quite a vague concept and hard to find an objective measure for. However, we 
% The larger problem



