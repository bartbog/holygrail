% !TeX root = ./main.tex

In this section, we discuss the background on the hitting set duality, which constitutes the core of our algorithm. 
We present all result using propositional logic but our results easily generalize to richer languages, such as constraint languages, as long as the semantics in terms of a satisfaction relation between expressions in the language and possible states of affairs (assignments of values to variables).
 
%  
% 
% 
% The overarching goal of this paper is to generate a sequence of small reasoning steps, each with an interpretable explanation, and for that we introduce the necessary background.
% \begin{itemize}
%     \item Explanation sequence : CSP journal paper ?
%     \item concept of OMUS
%     % \item Logic grid puzzles (depends on experiments)
%     % \item Nested explanation still applicable?
% \end{itemize}

\newcommand\F{\formula}
Let \voc be a set of propositional symbols, also called \emph{atoms}; this set is implicit in the rest of the paper. A \emph{literal} is an atom $p$ or its negation $\lnot p$.
\marginpar{clause constraint}
%\tias{Can we keep it at general 'constraint' here? the only requirement is that we can check satisfiability of a set of constraints (SAT(F')) and that we can grow the set to a larger set (Grow(F',F)), so nothing clause-specific. The literal is then an assignment or its negation...} \bart{Yes, we can! But it's always a struggle. WE need to be very precise (Ive had reviewers complain the past because of not being explicit about what a formula is or is not allowed to be etc.) Also: we use MAX-sat calls in our algorithm; some care needed there}.
A clause is a disjunction of literals. A formula $\formula$ is a conjunction of clauses. 
Slightly abusing notation, a clause is sometimes used as a set of literals and a formula as a set of clauses.
% \todo{weight function comes later}
A \emph{weight function} $w$ on $\formula$ assigns to each clause in $\formula$ either a positive integer, or a $\infty$. 
A (partial) interpretation is a consistent (not containing both $p$ and $\lnot p$) set of literals. An interpretation is \emph{complete} if it contains either $p$ or $\lnot p$ for all symbols $p\in \voc$. 
Satisfaction of a formula \formula by an interpretation is defined as usual. 
A \emph{model} of \formula is an interpretation that satisfies \formula. 
%  to the atoms that satisfies $\formula$.
$\formula$ is said to be \emph{unsatisfiable} if it has no models.
A literal $l$ is a \emph{consequence} of a formula \formula if $l$ holds in all $\formula$'s models; we denote this $\formula\models l$. 

\begin{definition}
  A \emph{Minimal Unsatisfiable Subset} (MUS) of \F is a set $\m{S} \subseteq \F$  that is unsatisfiable but such that every strict subset of $\m{S} $ is satisfiable. 
%   
  We write \muses{\F} for the set of MUSs of \F. 
\end{definition}


% Formally, we define a \textit{Minimum Unsatisfiable Subset} (MUS), as a set of clauses that taken together are unsatisfiable, but if any of the clauses are removed then the reduced formula becomes satisfiable.

\begin{definition}
    A set $\m{S} \subseteq \formula$ is a \emph{Maximal Satisfiable Subset} (MSS) of $ \formula$ if $\m{S}$ is satisfiable and for all $\m{S}'$ with $\m{S}  \subsetneq  \m{S}'\subseteq\formula $, $\m{S}'$ is unsatisfiable.
\end{definition}

\begin{definition}
    A subset $\m{C} \subseteq \formula$ is a \emph{correction subset} of \formula if $\formula\setminus\m{C}$ is satisfiable. 
    A correction subset \m{C} is a \emph{minimal correction subset} (MCS)  of \formula if no strict subset of \m{C} is also a correction subset. 
%     if $\formula\setminus\m{C}$ is satisfiable, while for all 
%     $ \m{C}' \subsetneq C$,  $\formula \setminus \m{C}'$ is unsatisfiable.
    We write \mcses{\F} for the set of MCSs of \F. 
\end{definition}

It is well-known that if $\formula$ is unsatisfiable, each  MCS is the complement of an MSS and vice versa. 

\begin{definition}\label{def:minimal-hs}
    Given a collection of sets $\m{K}$, a hitting set $h$ is a set such that $\forall \ C \in \m{K}$, $h \cap C \neq \emptyset$. A hitting set is \emph{minimal} if no strict subset of it is also a hitting set. 
\end{definition}

% \noindent In fact, for a collection of sets $\m{K}$ there exist multiple possible minimal hitting sets on $\m{K}$. Consequently, we define the \textbf{minimum} hitting set as:

% \begin{definition}\label{def:minimum-hs}
% %     Let $\m{K}$ be a collection of sets and $HS(\m{K})$ be the collection of all minimal hitting sets on $\m{K}$.
% %     A hitting set $hs$ is a \textbf{minimum} hitting set if $\forall \ h' \in HS(\m{K}): |hs| \leq |h'|$.
%   \end{definition}

The following proposition is the well-known hitting set duality  between MCSs and MUSs that forms the basis of our algorithms \cite{DBLP:journals/jar/LiffitonS08,ai/Reiter87}.

\begin{proposition}\label{prop:MCS-MUS-hittingset}
%     Given an $ \formula$, let MUSes($\formula$), be the Minimal Unsatisfiable Subsets of F and MCSes($\formula$), be the Minimal Correction Subsets of F:
%     
    A set  $\m{C} \subseteq \formula$ is an MCS of $ \formula$ iff  $\m{C}$ is a \emph{minimal hitting set} of \muses{F}.

    \noindent
    A set  $\m{U} \subseteq \formula$ is a MUS of $ \formula$ iff  $\m{U}$ is a \emph{minimal hitting set} of \mcses{F}.
\end{proposition}
 
The duality of \cref{prop:MCS-MUS-hittingset} lies at the basis of various algorithms to compute MUSs \cite{} or MSSs \cite{,}. 
It was also used in an algorithm to find \emph{cardinality-minimal MUSs} (also called \emph{smallest MUS} or SMUS) by \citet{ignatiev2015smallest}, which our algorithm is based on. 
In the next section, we discuss our proposed OUS algorithm. 
%The algorithm of \citet{ignatiev2015smallest} can be obtained from \cref{alg:omus} by using cardinality as a cost function, hence we do not give it explicitly here. They use a maxsat solver for the hitting sets...


% 
% % \newcommand\setstohit{\ensuremath{\m{H} }\xspace}
% \begin{algorithm}[ht]
%   \DontPrintSemicolon
%   $\setstohit  \gets \emptyset$ \; %\label{omus-line1} 
%   \While{true}{
%     $\F' \gets$ FindCardMinHittingSet($\setstohit $) \label{smus-hs} \;%\tcp*{\small Find \textbf{optimal} solution}
%     % \tcp{\small set with all unique clauses from hitting set}
% %     (sat?, $\kappa$) $\gets$ \texttt{SatSolver}($hs$)\;
%     % \tcp{If SAT, $\kappa$ contains the satisfying truth assignment}
%     % \tcp{IF UNSAT, $hs$ is the OMUS }
%     \If{ not sat($\F'$)}{
%       \Return{$\F'$} \;
%     }
%     $\F'' \gets  \texttt{Grow}($\F'$) $\;
%     $\setstohit  \gets \setstohit  \cup \{  \formula$ $\setminus \F''\}$ \;
%   }
%   \caption{\textsc{SMUS($\formula$)}}
%   \label{alg:omus}
% \end{algorithm}
% \todo{notatie van algorithm calls uniform maken. textsc? texttt? plain text? Pick one. }
% 
% Maybe we should not even give it here... 
% 
% 
% % Moreover, it is easy to see that the duality still holds when a cost-function is added. 
% % \begin{proposition}
% %   Let $\cost 
% % \end{proposition}
% 
% % Let \cost 
% Moreover, it is easy to see that to explaining how we can compute the SMUS in section \ref{sec:smus}. 
% % TODO: add something here ?
% In section \ref{sec:omus}, we further exploit this proposition for computing the OMUS for a given unsatisfiable CNF formula $\formula$.
